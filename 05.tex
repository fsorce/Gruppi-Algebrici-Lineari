\chapter{Quozienti}
\section{Realizzazione geometrica di quotienti e insiemi delle classi}
In generale, l'esistenza di quozienti in geometria algebrica \`e un argomento molto complicato. Per evitare questo limite \textit{restringiamo l'attenzione allo studio di insiemi delle classi}. Se $G$ \`e un gruppo algebrico e $H$ \`e un sottogruppo allora possiamo pensare all'insieme delle classi come la variet\`a quoziente $G/H$, cio\`e il quoziente di $G$ tramite l'azione di $H$ indotta in modo ovvio. In questo caso il quoziente esiste grazie ad un lemma che esponiamo nella sezione seguente.

In questa sezione dunque pensiamo a $G$ come una variet\`a della quale cerchiamo un quoziente e ad $H$ come il gruppo algebrico che agisce su questa variet\`a.

\subsection{Costruzione}
Il seguente lemma sar\`a molto utile nella realizzazione geometrica dei quozienti.

\begin{lemma}\label{LmSottogruppoEStabilizzatoreDiUnaRettaInQualcheRappresentazione}
Sia $G$ un gruppo algebrico e $H$ sottogruppo di $G$, allora
\begin{enumerate}
    \item Esistono una rappresentazione di dimensione finita $V$ di $G$ e una retta $L\subseteq V$ tali che $H=\stab_G L=\cpa{g\in G\mid g(L)=L}$
    \item Se $H$ \`e normale allora $V$ si pu\`o scegliere in modo che sia somma dei $V_\al$ per $\al\in X(H)$ e $V_\al=\cpa{v\in V\mid h\cdot v=\al(h)v}$.
\end{enumerate}
\end{lemma}
\begin{proof}
Mostriamo le due affermazioni
\setlength{\leftmargini}{0cm}
\begin{enumerate}
\item Sia\footnote{L'idea di questo punto \`e mostrare che $H$ \`e lo stabilizzatore di $I_H$. Da questo intersecando con una rappresentazione di dimensione finita che contiene i generatori di $I_H$ si ha che $H$ \`e lo stabilizzatore di un particolare sottospazio vettoriale dentro un altro spazio, cio\`e stabilizza un punto in una Grassmanniana. A questo punto basta considerare l'immersione di Pl\"ucker e abbiamo che $H$ \`e lo stabilizzatore di una retta.} $I_H\subseteq \K[G]$ l'ideale che definisce $H$, allora
\[H=\cpa{g\in G\mid g(I_H)=I_H}\]
\setlength{\leftmargini}{0cm}
\begin{itemize}
\item[$\boxed{\subseteq}$] Se $g\in H$ e $f\in I_H$ allora $gf(k)=f(g\ii k)$, quindi se $k\in H$ allora $g\ii k\in H$ e quindi questa funzione vale $0$, cio\`e $gf\in I_H$. L'altra inclusione segue dallo stesso ragionamento fatto su $g\ii$.
\item[$\boxed{\supseteq}$] Se $g\ii(I_H)\subseteq I_H$ allora per ogni $f\in I_H$
\[f(g)=\under{\in I_H}{(g\ii f)}(e)=0\]
cio\`e $g\in H$.
\end{itemize}
\setlength{\leftmargini}{0.5cm}
Consideriamo ora dei generatori $f_1,\cdots, f_m$ per $I_H$ e sia $V_0\subseteq \K[G]$ una $G$-sotto-rappresentazione di dimensione finita che contiene ogni $f_i$. Consideriamo il sottospazio vettoriale $W_0=I_H\cap V_0$ e notiamo che
\[H=\cpa{g\in G\mid g(W_0)= W_0}.\]
\setlength{\leftmargini}{0cm}
\begin{itemize}
\item[$\boxed{\subseteq}$] Ovvio per quanto detto sopra.
\item[$\boxed{\supseteq}$] Se $g(W_0)= W_0$ allora $g(I_H)=I_H$, infatti $g(I_H)\subseteq I_H$ ovvio per costruzione di $W_0$, l'altra inclusione segue dal fatto che $g(W_0)=W_0\coimplies g\ii(W_0)=W_0$.
\end{itemize}
\setlength{\leftmargini}{0.5cm}
Sia $\dim W_0=m$. Poniamo
\[V=\bw^m V_0,\quad L=\bw^m W_0\subseteq \bw^m V_0.\]
Per concludere basta mostrare che
\[H=\cpa{g\in G\mid g(L)=L}\]
\setlength{\leftmargini}{0cm}
\begin{itemize}
\item[$\boxed{\subseteq}$] Se $g(W_0)=W_0$ allora chiaramente 
\[g(L)=g\pa{\bw^mW_0}=\bw^m g(W_0)=\bw^m W_0=L.\] 
\item[$\boxed{\supseteq}$] Notiamo che se $u_1,\cdots, u_m$ \`e una base di $U\subseteq V$ sottospazio vettoriale allora 
\[U=\cpa{u\in V\mid u\wedge u_1\wedge\cdots\wedge u_m=0}.\]
Fissiamo una base $w_1,\cdots, w_m$ di $W_0$ e osserviamo che $g w_1,\cdots, gw_m$ \`e una base di $g(W_0)$. Se $g(L)=L$ allora per definizione
\[\ps{g w_1\wedge\cdots\wedge gw_m}=\ps{w_1\wedge\cdots\wedge w_m},\] 
quindi per il criterio appena citato si ha che
\begin{align*}
    W_0=&\cpa{v\in V\mid v\wedge w_1\wedge\cdots\wedge w_m=0}=\\
    =&\cpa{v\in V\mid u\wedge gw_1\wedge\cdots\wedge gw_m=0}=g(W_0).
\end{align*}
\end{itemize}
\setlength{\leftmargini}{0.5cm}
\item Sia $V'=\bigoplus_{\al\in X(H)} V_\al\subseteq V$ con $V$ di prima. Mostriamo che $V'$ \`e $G$-invariante:
Se $v_\al\in V_\al$ per $\al\in X(H)$, $h\in H$ e $g\in G$ allora
\[h\cdot(g v_\al)=(gg\ii h g)\cdot v_\al\overset{\smat{H\text{ normale}\\\Downarrow\\ g\ii h g\in H}}=g(\al(g\ii hg)v_\al)=\al(g\ii hg)gv_\al,\]
cio\`e $gv_\al$ \`e un autovettore per l'azione di $h$ per un qualsiasi $g\in G$ e $h\in H$, ovvero $V'$ \`e $G$-invariante. 

Per concludere \`e dunque sufficiente mostrare che $L\subseteq V'$, ma abbiamo gi\`a visto che 
\[g(w_1\wedge\cdots \wedge w_m)=gw_1\wedge\cdots \wedge gw_m=\la(g) w_1\wedge\cdots\wedge w_m\] 
per qualche $\la(g)$ per ogni $g\in H$, quindi $L\subseteq V_\la$.
\end{enumerate}
\setlength{\leftmargini}{0.5cm}
\end{proof}

\begin{remark}
Se $H$ è un sottogruppo di $G$ e $\pi\colon G \to X=G\cdot L$ \`e la mappa ovvia, allora $\pi$ è $G$-equivariante e induce una bigezione tra $G/H$ e $X$.
\end{remark}

Questa mappa risulta in realt\`a essere un valido quoziente categorico. Mostreremo nel resto della sezione i risultati coinvolti
\begin{proposition}[]\label{PrInsiemeDelleClassiEVarietaQuasiProiettiva}
Sia $G$ un gruppo algebrico e $H$ un sottogruppo, allora l'insieme delle classi $G/H$ pu\`o essere identificato con una variet\`a quasiproiettiva.
\end{proposition}
\begin{proof}
Siano $H< G$ e $L,V$ come nel lemma. Dato che $G$ agisce su $V$ si ha che esso agisce anche su $\Pj(V)$. Interpretando $L\in \Pj(V)$ abbiamo che l'insieme delle classi coincide con l'orbita $X=G\cdot L=G/\stab_G(L)=G/H$. Definiamo la variet\`a proiettiva
\[Y=\ol{G\cdot L}\subseteq \Pj(V).\]
Concludiamo mostrando che $X$ \`e aperto in essa: data la mappa
\[\funcDef{G}{Y}{g}{gL},\]
per Chevalley (\ref{ThChevalley}) si ha che $X$ contiene un aperto $U$ di $Y$. Se $x\in U\subseteq X$ allora $gx\in gU\subseteq X$ quindi
\[X=\bigcup_{g\in G}gU\text{ \`e aperto.}\]
\end{proof}


Nel caso in cui il sottogruppo \`e normale sappiamo che l'insieme delle classi \`e un gruppo. Quello che si scopre \`e che effettivamente in questo caso $G/H$ \`e un gruppo algebrico (anche se la costruzione che diamo sotto \`e diversa da quella sopra):
\begin{proposition}\label{PrGruppoQuozienteEGruppoAlgebrico}
Se $H$ \`e normale, $G/H$ \`e un gruppo algebrico affine.
\end{proposition}
\begin{proof}
Costruiamo $L$ e $V=\bigoplus_{\al\in X(H)} V_\al$ come nel punto 2. del lemma (\ref{LmSottogruppoEStabilizzatoreDiUnaRettaInQualcheRappresentazione}). Sia
\[W=\cpa{T:V\to V\mid \forall \al\in X(H),\ T(V_\al)\subseteq V_\al}\]
e notiamo che $G$ agisce su $W$ come $gT=g\circ T\circ g\ii$. Per rendere valido quanto detto dobbiamo verificare che $g T g\ii(V_\al)\subseteq V_\al$, ma questo segue dal fatto che se $g\ii(V_\al)=V_\beta$ allora
\[g T g\ii(V_\al)=gT(V_\beta)\subseteq gV_\beta=V_\al.\]
Notiamo ora che
\[\cpa{g\in G\mid g\res W=id_W}\overset{(\star)}=\cpa{g\in G\mid g\res{V_\al}=\la_\al id_{V_\al}\ \forall \al\in X(H)}\overset{(\star\star)}=H.\]
\setlength{\leftmargini}{0cm}
\begin{itemize}
\item[$\boxed{\;(\star)\;}$] Notiamo che $g\res W=id_W$ significa che per ogni $T\in W$ abbiamo $gTg\ii=T$, cio\`e $gT=Tg$. L'uguaglianza segue scrivendo le matrici associate a $T$ e $g$ visti come elementi di $\GL(V)$:  Le matrici in $W$ sono le diagonali a blocchi per la decomposizione $\bigoplus_{\al\in X(H)}V_\al$ e una matrice commuta con tutte le diagonali a blocchi per una data decomposizione se e solo se \`e a blocchi per la stessa decomposizione e in ogni blocco \`e multiplo dell'identit\`a.
\item[$\boxed{(\star\star)}$] Segue per doppia inclusione: Per definizione di carattere, gli elementi di $h$ sono tali che $h\res{V_\al}=\la_\al id_{V_\al}$. Se invece $g$ \`e tale che $g\res{V_\al}=\la_\al id_{V_\al}$ allora in particolare $g\res{V_\la}$ \`e un multiplo dell'identit\`a per il blocco $V_\la$ che contiene $L$, quindi $g\cdot L=L$ ovvero $g\in \stab_G(L)=H$.
\end{itemize}
\setlength{\leftmargini}{0.5cm}

Abbiamo dunque costruito un omomorfismo di gruppi algebrici $\vp:G\to \GL(W)$ il cui nucleo \`e $H$. Poich\'e $\vp(G)$ \`e un sottogruppo chiuso di $\GL(W)$ per Chevalley (\ref{PrMorfismoTraGruppiAlgebriciAffiniHaImmagineChiusa}) si ha che $G/H\cong \vp(G)$ eredita la struttura di gruppo algebrico lineare da $\vp(G)$.
\end{proof}

Diamo due esempi dove il sottogruppo NON \`e normale:
\begin{exercise}
Sia $G=\GL(2)$ e $H=B(2)=\cpa{\mat{\ast&\ast\\0&\ast}}$. Sia $V=\K^2$ e $L=\K e_1$, allora effettivamente $H=\stab_G(L)$ e $G\cdot L=\Pj(V)$ (ogni retta si ottiene da $L$ applicando una trasformazione lineare), quindi\footnote{Intuitivamente, questo risultato semplicemente afferma che l'insieme delle rette in $V$ corrisponde all'orbita di una retta fissata tramite l'azione di $\GL$.} $G/H\leftrightarrow \Pj(V)$.
\end{exercise}

\begin{exercise}
Sia $G=\GL(n,\C)$ e $H=O(n,\C)$. Sia $V=\Sym(n,\C)$ lo spazio delle matrici simmetriche. Se $A\in V$ e $g\in G$ agisce su $A$ tramite 
\[g\cdot A=gA g^\top\]
allora $H=\stab_G(I_n)$ (stabilizzatore della matrice identit\`a).
\[X=G\cdot I_n=\cpa{gg^\top\mid g\in \GL(n)}=\cpa{A\in \GL(n)\mid A=A^\top}\subseteq\ol{G\cdot I_n}=Y\subseteq \Pj(V).\]
%Siano $V'=V\oplus \K$ e $L=\ps{(I_n,1)}$, allora \[\stab_G(L)=H\text{ e }\Pj(V')=V\oplus \Pj(V).\]
\end{exercise}















\subsection{Variet\`a omogenee}

\begin{center}
    \textit{Nel seguito supporremo $\cha(\K)=0$.}
\end{center}


\begin{definition}
Una variet\`a \`e \textbf{omogenea} rispetto al gruppo $G$ se l'azione di $G$ su $X$ \`e transitiva.
\end{definition}

\begin{corollary}
    Se $X$ è una varietà omogenea rispetto a $G$, allora $X$ è liscia.
\end{corollary}
\begin{proof}
    Poiché $X$ ha un aperto $U$ di punti lisci, la tesi segue dal fatto che $X=\bigcup_{g\in G} gU$.
\end{proof}


\begin{corollary}\label{CorEquivarianteTraVarietaOmogeneeELiscia}
    Se $X$ e $Y$ sono varietà omogenee per $G$ e $\varphi\colon X \to Y$ è $G$-equivariante, allora $\varphi$ è liscia.
\end{corollary}
\begin{proof}
    Sicuramente $\varphi$ è surgettiva, perché $X$ e $Y$ sono varietà omogenee per $G$ e $\varphi$ è $G$-equivariante. Per il teorema (\ref{ThMorfismoDominanteTraLisceHaRestrizioneLisciaSuAperto}) esiste un aperto non vuoto $U$ di $Y$ tale che $\varphi\res{\varphi^{-1}(U)}\colon \varphi^{-1}(U) \to U$ è liscia. Allora si ha un diagramma commutativo % https://q.uiver.app/#q=WzAsNCxbMCwwLCJcXHZhcnBoaV57LTF9KGdVKSJdLFsyLDAsImdVIl0sWzAsMiwiXFx2YXJwaGleey0xfShVKSJdLFsyLDIsIlUiXSxbMCwxLCJcXHZhcnBoaVxccmVze2dcXHZhcnBoaV57LTF9KFUpfSJdLFsyLDMsIlxcdmFycGhpXFxyZXN7XFx2YXJwaGleey0xfShVKX0iXSxbMiwwLCJcXHNpbSJdLFszLDEsIlxcc2ltIiwyXV0=
\[\begin{tikzcd}
	{\varphi^{-1}(gU)} && gU \\
	\\
	{\varphi^{-1}(U)} && U
	\arrow["{\varphi\res{g\varphi^{-1}(U)}}", from=1-1, to=1-3]
	\arrow["\sim", from=3-1, to=1-1]
	\arrow["{\varphi\res{\varphi^{-1}(U)}}", from=3-1, to=3-3]
	\arrow["\sim"', from=3-3, to=1-3]
\end{tikzcd}\] 
Poiché $Y=\bigcup gU$, $\varphi$ è liscia in ogni punto e quindi è liscia.

%\[\varphi\res{g\varphi^{-1}(U)}\colon \varphi^{-1}(gU) \to gU\]
\end{proof}




\subsection{Quoziente categorico}

Sfruttando quanto detto sulle variet\`a omogenee possiamo dimostrare il seguente lemma, il quale ci permette di mostrare l'universalit\`a del quoziente \[\pi:G\to X=G\cdot L.\]
\begin{lemma}[]\label{LmMappaEquivariantePerUnSottogruppoEUniversalmenteAperta}
Se $H$ è un sottogruppo di $G$ e $\pi\colon G \to X$ \`e la proiezione sull'insieme delle classi come in (\ref{PrInsiemeDelleClassiEVarietaQuasiProiettiva}) allora per ogni varietà $Z$, la mappa \[G\times Z \xrightarrow{(\pi,id)}X\times Z\] è aperta.
\end{lemma}
\begin{proof}
Notiamo che $\pi$ \`e $G$-equivariante e che sia $G$ che $X$ sono variet\`a $G$-omogenee. Segue che $\pi$ \`e liscia. Segue dunque che per ogni varietà $Z$ la mappa $(\pi,id)\colon G\times Z \to X\times Z$ è liscia e quindi tale mappa è anche aperta (\ref{ThMappaLisciaEAperta}).
\end{proof}



\begin{proposition}\label{PrFascioDiUnQuozienteDatoDagliInvarianti}
    Se $H$ è un sottogruppo di $G$ e $\pi\colon G \to X$ \`e la proiezione sull'insieme delle classi come in (\ref{PrInsiemeDelleClassiEVarietaQuasiProiettiva}), si ha che per ogni aperto $U$ di $X$ vale\footnote{Questo \`e in realt\`a un risultato generale sui fasci di struttura quando facciamo un quoziente di spazi localmente anulati.} 
    \[\pi^\ast:\Oc_X(U)\xrightarrow{\sim}(\Oc_G(\pi^{-1}(U)))^H.\]
\end{proposition}

\begin{proof}
Sia $U$ un aperto di $X$, poniamo $V=\pi\ii(U)$, e consideriamo la mappa
\[\pi^\ast:\Oc_X(U)\to \Oc_G(V)^H.\]
Osserviamo che tale mappa è iniettiva, infatti se $f(\pi(x))=0$ per ogni $x$ in $V=\pi\ii (U)$, si ha $f(y)=0$ per ogni $y$ in $U$ per surgettivit\`a di $\pi$, per cui $f=0$. L'immagine di $\pi^\ast$ \`e contenuta negli $H$-invarianti perch\'e
\[(h\cdot (\pi^\ast(f)))(g)=f(\pi(h\ii g))\overset{X=G/H}=f(\pi(g))=\pi^\ast(f).\]
Resta solo da mostrare che la mappa $\pi^\ast:\Oc_X(U)\to \Oc_G(V)^H$ è surgettiva.

\medskip

\noindent
Come primo passo ci riduciamo al caso di $U$ irriducibile:
\begin{mdframed}[topline=false,rightline=false,bottomline=false] 
In $G$ le componenti connesse coincidono con le componenti irriducibili (\ref{ReInGruppoComponentiConnesseSonoIrriducibili}). Lo stesso vale per $X$, infatti se $G^0$ \`e la componente connessa di $1_G$ allora da $X=G/H$ troviamo che $X^0=G^0/H\cap G^0$ \`e aperto e $X$ \`e unione finita disgiunta dei traslati di $X^0$ ($G$ ha finite componenti irriducibili per Noetherianit\`a).
\smallskip

\noindent
Se $U$ \`e un aperto di $X$ allora 
\[X=X^0\sqcup g_1X^0\sqcup\cdots\sqcup g_nX^0\implies U=U\cap X^0\sqcup \cdots\sqcup U\cap g_n X^0.\]
Poich\'e $X^0$ irriducibile i suoi aperti sono irriducibili. Se verifichiamo la tesi per ogni $U\cap g_iX^0$ allora data $f\in \Oc_G(\pi\ii(U))^H$ possiamo trovare una preimmagine considerando l'insieme delle preimmagini per le varie restrizioni $f\res{\pi\ii(U\cap g_i X^0)}\in \Oc_G(\pi\ii(U\cap g_i X^0))^H$, infatti una tale collezione da proprio un elemento di $\prod_{i=1}^n \Oc_X(U\cap g_i X^0)=\Oc_X(U)$.
\end{mdframed}

Sia $f:V\to \K$ regolare $H$-invariante e consideriamo il grafico $\Gamma(f)\subseteq V\times \K$. Sia $g$ la fattorizzazione (insiemistica per ora)
% https://q.uiver.app/#q=WzAsMyxbMCwwLCJWIl0sWzIsMCwiVT1WL0giXSxbMCwyLCJcXEsiXSxbMCwyLCJmIiwyXSxbMCwxLCJcXHBpIl0sWzEsMiwiZyIsMCx7InN0eWxlIjp7ImJvZHkiOnsibmFtZSI6ImRhc2hlZCJ9fX1dXQ==
\[\begin{tikzcd}
	V && {U=V/H} \\
	\\
	\K
	\arrow["\pi", from=1-1, to=1-3]
	\arrow["f"', from=1-1, to=3-1]
	\arrow["g", dashed, from=1-3, to=3-1]
\end{tikzcd}\]
Decomponiamo $V$ in irriducibili $V=V_1\cup\cdots\cup V_n$ e per irriducibilit\`a di $U$ deve essere il caso che $H$ agisce transitivamente su $\cpa{V_1,\cdots, V_n}$.

Per il lemma (\ref{LmMappaEquivariantePerUnSottogruppoEUniversalmenteAperta}) sappiamo che $(\pi,id):G\times \K\to X\times\K$ \`e una mappa aperta, quindi la restrizione $\psi:V\times \K\to U\times \K$ resta aperta perch\'e $V\times \K$ \`e un aperto di $G\times \K$. Segue che\footnote{Se $(v,f(v))\in\Gamma(f)$ allora $\psi(v,f(v))=(\pi(v),f(v))=(\pi(v),g(\pi(v)))$.} $\psi(\Gamma(f))=\Gamma(g)$ \`e un chiuso di $U\times \K$. In realt\`a notando che $\Gamma(g)$ \`e $H$ invariante e che esso \`e immagine di $\Gamma(f)=\bigcup \Gamma(f)\cap V_i$ si ha che in realt\`a possiamo scrivere $\Gamma(g)$ solo come immagine di un singolo $\Gamma(f)\cap V_i$, in particolare \`e irriducible (altrimenti potremmo decomporre $V$ ulteriormente).

Consideriamo ora il diagramma
% https://q.uiver.app/#q=WzAsMyxbMCwwLCJcXEdhbW1hKGcpIl0sWzIsMCwiVVxcdGltZXMgXFxLIl0sWzIsMiwiVSJdLFswLDEsIiIsMCx7InN0eWxlIjp7InRhaWwiOnsibmFtZSI6Imhvb2siLCJzaWRlIjoidG9wIn19fV0sWzEsMiwicF9VIl0sWzAsMiwicSIsMl1d
\[\begin{tikzcd}
	{\Gamma(g)} && {U\times \K} \\
	\\
	&& U
	\arrow[hook, from=1-1, to=1-3]
	\arrow["q"', from=1-1, to=3-3]
	\arrow["{p_U}", from=1-3, to=3-3]
\end{tikzcd}\]
dove $p_U$ \`e la proiezione su $U$ e $q$ \`e la restrizione a $\Gamma(g)$. Chiaramente $q$ \`e regolare in quanto composizione di regolari. $q$ \`e anche bigettiva perch\'e $(u,g(u))\mapsto u$ pu\`o essere facilmente invertita. Poich\'e $U$ \`e liscio e $\Gamma(g)$ irriducibile, per il teorema di Zariski (\ref{ThZariski}) si ha che $q$ \`e un isomorfismo, quindi la mappa $u\mapsto (u,g(u))$ un morfismo e in particolare $g$ stesso \`e un morfismo. Questo mostra che $\pi^\ast$ effettivamente \`e surgettiva perch\'e abbiamo trovato $g\in \Oc_X(U)$ tale che $\pi^\ast(g)=g\circ \pi=f$.
\end{proof}


Siamo pronti per mostrare la propriet\`a universale dei quozienti

\begin{theorem}
    Per ogni $G$-varietà $Y$ e per ogni $y_0$ in $Y$ tale che $H$ è contenuto in $\stab_G(y_0)$, vale la seguente proprietà: se $\varphi\colon G \to Y$ è la mappa definita da $\varphi(g)=gy_0$, allora esiste un'unica $\psi\colon X \to Y$  tale che $\psi\circ\pi=\varphi$.
    % https://q.uiver.app/#q=WzAsMyxbMCwwLCJHIl0sWzAsMSwiWCJdLFsyLDAsIlkiXSxbMCwxLCJcXHBpIl0sWzAsMiwiXFx2YXJwaGkiXSxbMSwyLCJcXGV4aXN0cyAhXFxwc2kiLDIseyJzdHlsZSI6eyJib2R5Ijp7Im5hbWUiOiJkYXNoZWQifX19XV0=
\[\begin{tikzcd}
	G && Y \\
	X
	\arrow["\varphi", from=1-1, to=1-3]
	\arrow["\pi", from=1-1, to=2-1]
	\arrow["{\exists !\psi}"', dashed, from=2-1, to=1-3]
\end{tikzcd}\]
dove $X\cong G/H$ \`e la variet\`a quasiproiettiva di (\ref{PrInsiemeDelleClassiEVarietaQuasiProiettiva}).
\end{theorem}

\begin{proof}
    Insiemisticamente, la mappa $\psi\colon gH \mapsto gy_0$ è definita. Inoltre $\psi$ è continua: se $U$ è un aperto di $Y$, allora \[\psi^{-1}(U)=\pi(\pi^{-1}(\psi^{-1}(U)))=\pi(\varphi^{-1}(U)),\] che è aperto per il lemma (\ref{LmMappaEquivariantePerUnSottogruppoEUniversalmenteAperta}). Dato un aperto $U$ di $Y$, verifichiamo che l'immagine della mappa 
    \[\psi^\ast:\Oc_Y(U)\to\Hom_{(\mathrm{Set})}(\psi^{-1}(U),\K)\]
    è contenuta nell'insieme delle funzioni regolari su $X$. Consideriamo una funzione $f\colon U\to \K$. Allora $\wt{f}:= f\circ \psi\circ\pi =f \circ \varphi$ è in $\Oc_G(\varphi^{-1}(U))$ in quanto $\varphi$ è un morfismo di varietà. Mostriamo che in realtà è in $\Oc_G(\varphi^{-1}(U))^H$. Sia $\lambda$ in $\Oc_G(\varphi^{-1}(U))$ e sia $h$ in $H$. Allora per ogni $x$ in $G$ si ha \[(h\widetilde{f})(x)=\widetilde{f}(xh)=f(\varphi(xh))=f(xhy_0)=f(x y_0)=f(\varphi(x))=\widetilde{f}(x).\]
    Dato che $\Oc_G(\varphi^{-1}(U))^H=\Oc_X(\psi\ii(U))$ per la proposizione (\ref{PrFascioDiUnQuozienteDatoDagliInvarianti}) questo conclude.
\end{proof}















\section{Sottogruppi generati e ``algebre di Lie"}
\begin{lemma}\label{LmSottogruppoGenerato}
Sia $G$ un gruppo algebrico e siano $X_i$ delle variet\`a irriducibili. Siano $\vp_i:X_i\to G$ tali che $1_G\in \imm\vp_i$ per ogni $i$. Poniamo $Y_i=\vp_i(X_i)$.

Sia $H=\ps{\cpa{Y_i}_i}$ il sottogruppo generato dalle immagini delle $\vp_i$. Allora

\begin{enumerate}
	\item $H$ \`e chiuso
	\item esistono $i_1,\cdots, i_k$, $\e_1,\cdots, \e_k$ tali che
	\[H=Y_{i_1}^{\e_1}\cdots Y_{i_k}^{\e_k},\qquad \e_j\in\cpa{1,-1}\]
\end{enumerate}
\end{lemma}
\begin{proof}
Senza perdita di generalit\`a supponiamo $X_i=Y_i\subseteq G$ con $\vp_i$ date dall'inclusione\footnote{l'immagine di una varit\`a irriducibile \`e irriducible.}. Supponiamo inoltre che tra le $X_i$ compaiano anche le $X_i\ii$, in modo da poter semplificare la notazione evitiando gli esponenti $\e_j$.
\noindent
Definiamo iterativamente
\[\begin{cases}
    Z_0=\cpa{1_G}&\\
    Z_{kn+i+1}=Z_{kn+i}\cdot X_{i+1} &\text{se }k\in \N,\ i\in \cpa{0,\cdots, n-1}
\end{cases}\]
e poniamo $W_i=\ol{Z_i}$ per ogni $i\in \N$.

\noindent
Notiamo che $Z_i$ \`e l'immagine di $X_1\times\cdots\times X_i\to G$, quindi $Z_i$ \`e irriducible per ogni $i$, dunque anche $W_i=\ol{Z_i}$ \`e irriducibile.
Segue dunque che 
\[W_1\subseteq W_2\subseteq\cdots\subseteq G\] 
\`e una catena di chiusi irriducibili e, dato che $G$ ha dimensione finita, essa stabilizza, cio\`e esiste $N$ tale che $W_N=W_{N+i}$


%, ovvero $W_N\cdot X_i\subseteq W_N$ per ogni $i$. Notiamo che segue anche che $X_i\cdot W_N\subseteq W_N$ per ogni $i$, infatti moltiplicando per $1_G$ nelle posizioni prima di $W_N$ nella scrittura di un elemento di $X_iW_N$ in modo appropriato possiamo vedere un elemento di $X_iW_N$ come un elemento di $Z_nW_N\subseteq W_{N+n}=W_N$.

Un qualsiasi elemento di $H$ \`e contenuto in qualche $Z_i$ per costruzione, quindi in particolare $H\subseteq W_N$.
%Allora $W_NZ_N\subseteq W_N$ cio\`e $W_N\cdot W_N\subseteq W_N$ perch\'e $W_n$ \`e chiuso. Quindi\footnote{$Z_N\ii W_N\subseteq W_N\implies Z_N\ii\subseteq W_N\implies \ol{Z_N\ii}\subseteq W_N$ e $\ol{Z_N\ii}=\ol{Z_N}\ii=W_N\ii$.} $W_N\ii\subseteq W_N$ e in particolare $Z_N\ii\subseteq W_N$, ma $Z_N\ii=X_N\ii\cdots X_1\ii=X_{i_1}\cdots X_{i_N}$, cio\`e mettendo tutto insieme $H\subseteq W_N$
Mostriamo che $W_N=Z_N\cdot Z_N$: abbiamo una mappa
\[\under{irrid.}{X_1\times\cdots\times X_{N}}\to Z_N\to W_N,\]
quindi per Chevalley (\ref{ThChevalley}) $Z_N\supseteq U$ per $U$ aperto non vuoto di $W_N$ (e quindi denso per irriducibilit\`a), dunque\footnote{Per ogni $g\in W_N$ fissiamo $f:W_N\to W_N$ data da $f(a)=a\ii g$. Questa mappa \`e ben definita perch\'e $W_N\cdot W_N\subseteq W_N$ ed \`e evidentemente continua. Dato che $U$ \`e aperto, $f\ii(U)$ \`e aperto e quindi interseca $U$ in quanto $U$ \`e denso. Segue dunque che esiste $u\in U\cap f\ii(U)$, cio\`e $u\ii g=v\in U$, ovvero $g=u\cdot v\in U\cdot U$.} $U\cdot U=W_N$.

Questo mostra che $W_N\subseteq H$ ma ci sono tutti gli elementi di $H$ quindi effettivamente $H=W_N$.
\end{proof}


\begin{remark}[Dimensione del quoziente]\label{ReDimensioneDelQuoziente}
Sia $\vp:G\to G/H$ e ricordiamo che una mappa di questo tipo \`e liscia per omogeneit\`a. Allora per il teorema (\ref{ThFibrePerMappeLiscie}) abbiamo una successione esatta
\[0\to T_{e}H\to T_{e}G\xrightarrow{d\vp_{e}}T_{e}(G/H)\to 0\]
da cui $\dim (G/H)=\dim G=\dim H$.
\end{remark}

\begin{proposition}\label{PrContenimentoTraSottogruppiSiControllaSuiTangenti}
Supponiamo $\cha\K=0$. Siano $H,K\subseteq G$ sottogruppi connessi, allora\footnote{Questo risultato \`e analogo alla corrispondenza tra sottogruppi di Lie connessi di un gruppo di Lie e sottoalgebre di Lie.}
\[H\subseteq K\coimplies T_e H\subseteq T_eK\text{ in }T_eG.\]
\end{proposition}
\begin{proof}
L'implicazione $\implies$ \`e ovvia quindi basta mostrare l'altra.

Sia $X=H\cdot K\subseteq G$ e notiamo che \`e omogeneo rispetto all'azione di $H\times K$ data da $(h,k)\cdot g=hgk\ii$.
Osserviamo che $X$ \`e aperto in $\ol X\subseteq G$ per Chevalley (\ref{ThChevalley}) (contiene un aperto della chiusura ed \`e omogeneo in quanto orbita di $e$).
Si ha
\[X\cong \frac{H\times K}{\stab_{H\times K}(e)}.\]
Studiando lo stabilizzatore notiamo che
\[\stab_{H\times K}(e)=\cpa{(h,k)\mid hk\ii=e}=\cpa{(h,k)\mid h=k}\cong H\cap K,\]
dunque per il teorema (\ref{ThFibrePerMappeLiscie}) abbiamo $\dim X=\dim H+\dim K-\dim H\cap K$ e una successione esatta
\[0\to T_e(H\cap K)\to T_e(H\times K)\to T_e X\to 0\]
da cui segue $\dim T_e X=\dim T_eH+\dim T_eK-\dim T_e(H\cap K)$.
\medskip

Se $T_e X=T_eH+T_eK$ allora 
\[\dim (T_e X)=\dim (T_eH+T_eK)\overset{T_e H\subseteq T_eK}=\dim T_eK,\]
dunque 
\[\dim X=\dim T_eX=\dim T_eK=\dim K\]
e quindi $K\subseteq X\subseteq \ol X$ con $K$ e $\ol X$ irriducibili chiusi della stessa dimensione, dunque $K=X=\ol X$, e poich\'e $X=H\cdot K$ questo mostra 
\[K=X=H\cdot K\implies H\subseteq K\]
che \`e la tesi.

Per concludere basta dunque dimostrare che $T_e X=T_eH+T_eK$.
Consideriamo la composizione
\[\begin{array}{ccccc}
	H\times K &\xrightarrow{}& H\times K & \to &X \\
	(h,k) & \mapsto & (h,k\ii) &  & \\
	       &        & (h',k') &\mapsto & h'(k'{\ii}) 
\end{array}\] 
che \`e la mappa di moltiplicazione $\mu\res{H\times K}$. Dal teorema (\ref{ThFibrePerMappeLiscie}) ricaviamo una successione esatta
\[0\to T_e(\stab_{H\times K}(e))\to \under{T_eH\times T_eK}{T_e(H\times K)}\to T_eX\to 0\]
In particolare $d\pa{\mu\res{H\times K}}_{(e,e)}$ \`e surgettiva, quindi per concludere basta mostrare che 
\[d\pa{\mu\res{H\times K}}_{(e,e)}(\al,\beta)=\al+\beta.\]
Dato che $G\subseteq \GL(n)$ \`e sufficiente che la tesi valga per $\GL(n)$ e $d\mu_{(e,e)}$. Notiamo che $\Mat_{n\times n}=T_I\GL(n)$ dove l'identificazione \`e data da
\[(a_{i,j})\longleftrightarrow \pa{\rbar{a_{i,j}\pp{x_{ij}}{}}_{I}}\]
Traduciamo allora $d\mu_{(e,e)}$ in queste coordinate: se
\[d\mu_{(e,e)}\funcDef{\Mat_{n\times n}\times \Mat_{n\times n}}{\Mat_{n\times n}}{((a_{i,j}),(b_{i,j}))}{(c_{i,j})}\]
allora
\begin{align*}
	c_{i,j}=&\rbar{\pp{x_{i,j}}{}(x_{i,j}\circ \mu)}_I=\rbar{\pp{x_{i,j}}{}\pa{\sum_\ell y_{i,\ell}z_{\ell,j}}}_I=\\
	=&\sum_\ell \rbar{\pp{x_{i,\ell}}{}(y_{i,\ell})z_{\ell,j}}_I + \sum_\ell y_{i,\ell}\rbar{\pp{x_{i,j}}{}(z_{\ell,j})}_I\overset{(\star)}=\\
	=&\sum_\ell \rbar{\pp{y_{i,j}}{}(y_{i,\ell})}_I\delta_{\ell,j} + \sum_\ell \delta_{i,\ell}\rbar{\pp{z_{i,j}}{}(z_{\ell,j})}_I=\\
	=&\rbar{\pp{y_{i,j}}{}(y_{i,j})}_I+\rbar{\pp{z_{i,j}}{}(z_{i,j})}_I=\\
	=&a_{i,j}+b_{i,j}
\end{align*}
dove il passaggio $(\star)$ segue perch\'e stiamo interpretando quei differenziali come derivazioni.
\end{proof}



\section{Quando quozienti di gruppi sono completi}
\subsection{Gruppi risolubili e commutatori}
Ricordiamo la seguente definizione
\begin{definition}[]
Un gruppo $G$ \`e \textbf{risolubile} se la catena dei suoi sottogruppi derivati termina nel gruppo banale, cio\`e, se $G=G^{(0)}$ e $G^{(n+1)}=[G^{(n)},G^{(n)}]$ allora chiediamo che $G^{(n)}=\cpa{id_G}$ per $n\gg0$.
\end{definition}

\begin{lemma}[]\label{LmDerivatoEChiusoEConnesso}
Sia $G$ un gruppo algebrico connesso. Allora $[G,G]$ \`e chiuso e connesso.
\end{lemma}
\begin{proof}
Per ogni $i$ definiamo 
\[\phi_i:\funcDef{G^i\times G^i}G{(x_1,\cdots, x_i),(y_1,\cdots, y_i)}{[x_1,y_1]\cdots[x_i,y_i]}.\]
Per il lemma (\ref{LmSottogruppoGenerato}) segue che il sottogruppo di $G$ generato $\imm \phi_i$ (cio\`e $[G,G]$ per definizione) \`e chiuso. Notiamo per\`o che $\ol{\imm \phi_i}$ \`e una catena di chiusi irriducibili contenuta in $[G,G]$ che stabilizza, quindi in particolare $[G,G]=\ol{\imm \phi_n}$ per qualche $n$ e dunque $[G,G]$ \`e anche irriducibile.
\end{proof}

\subsection{Punto fisso di Borel}
\begin{proposition}[Esiste orbita chiusa]\label{PrEsisteOrbitaChiusa}
Se $G$ \`e un gruppo che agisce su una variet\`a $Y$ allora $G$ ha un'orbita chiusa in $Y$.
\end{proposition}
\begin{proof}
Consideriamo un'orbita che ha dimensione minima $Z=G\cdot y$. Notiamo che $Gy$ \`e un aperto denso di $\ol Z$ per Chevalley (\ref{ThChevalley}) (contiene un aperto ed \`e omogeneo), quindi $\dim \ol Z\bs Z<\dim Z$, ma se questa differenza non fosse vuota allora $G$ agirebbe su essa e quindi esisterebbe un'orbita di dimensione pi\`u piccola.

Dunque $\ol Z\bs Z=\emptyset$, cio\`e $\ol Z=Z$.
\end{proof}

Quindi orbite di dimensione minima sono chiuse. Sotto quali condizioni questa orbita \`e la pi\`u piccola possibile? Sotto quali condizioni abbiamo un punto fisso?

\begin{lemma}[]\label{LmPuntiFissiSonoUnChiuso}
Se $G$ \`e un gruppo algebrico che agisce su una variet\`a separata $X$ allora $X^G$ \`e chiuso in $X$.
\end{lemma}
\begin{proof}
Per ogni $g\in G$ sia $X^g$ il sottoinsieme dei punti di $X$ fissati dall'azione di $G$. Dato che $X^G=\bigcap_{g\in G}X^g$ basta mostrare che $X^g$ \`e chiuso.

$X^g$ \`e la preimmagine della diagonale $\Delta(X)$ tramite il morfismo $(1,g):X\to X\times X$, quindi per separatezza di $X$ \`e chiusa.
\end{proof}

\begin{theorem}[Punto fisso di Borel]\label{ThPuntoFissoBorel}
Se $G$ \`e un gruppo risolubile connesso che agisce su una variet\`a completa allora esiste un punto fisso per l'azione.
\end{theorem}
\begin{proof}
Se $\dim G=0$ allora $G=\cpa{id}$ per connessione, quindi ogni punto \`e fisso.


Supponiamo $\dim G>0$. Sia $H=[G,G]$ il sottogruppo derivato. Questo \`e un sottogruppo algebrico connesso per il lemma (\ref{LmDerivatoEChiusoEConnesso}). Dato che $G$ \`e risolubile, $H\subsetneq G$, quindi per ipotesi induttiva $X^H\neq \emptyset$.

Notiamo ora che su $X^H$ agisce $A=G/H$, quindi su $X^H$ abbiamo un'orbita chiusa (\ref{PrEsisteOrbitaChiusa}) $A\cdot x\subseteq X^H$. Dato che $X^H$ \`e chiuso in $X$ per il lemma (\ref{LmPuntiFissiSonoUnChiuso}), $A\cdot x$ \`e chiusa in $X$ e quindi \`e completa perch\'e chiuso in $X$ che \`e completa. 
Dato che $x\in X^H$ si ha che $A\cdot x=G\cdot x$. Dato che $[G,G]=H\subseteq \stab_G x$, si ha che $\stab_G x$ \`e un sottogruppo normale di $G$, quindi $G\cdot x=G/\stab_G x$ \`e una variet\`a affine per (\ref{PrGruppoQuozienteEGruppoAlgebrico}). Notiamo che $A\cdot x$ \`e anche connessa perch\'e $A$ \`e connesso in quanto $G$ lo \`e.

Dunque $A\cdot x$ \`e affine, completa e connessa, quindi \`e un punto, cio\`e $x$ \`e un punto fisso per $A$ in $X^H$, ma allora $x$ \`e un punto fisso per $G$.
\end{proof}


\begin{example}
Consideriamo $\C^\ast\acts \Pj^n$ come segue:
\[\la\cdot [x_0:\cdots:x_n]=[x_0:\la x_1:\cdots:\la^n x_n]\]
Questa azione ha come punti fissi quelli della forma $[0:\cdots:0:1:0:\cdots:0]$.
\end{example}



Ricordiamo che $B_n$ è il sottogruppo di $\GL(n)$ costituito dalle matrici triangolari superiori.

\begin{corollary}\label{CorCaratterizzazioneConnessiRisolubili}
    Se $G$ è connesso, sono equivalenti 
    \begin{enumerate}
        \item $G$ è risolubile.
        \item Le uniche rappresentazioni irriducibili di $G$ sono di dimensione $1$.
        \item Esiste un intero positivo $n$ tale che $G$ è contenuto in $B_n$. 
    \end{enumerate}    
\end{corollary}
\begin{proof}
Mostriamo le tre implicazioni
\setlength{\leftmargini}{0cm}
\begin{itemize}
\item[$\boxed{1.\implies2.}$] Sia $V$ una rappresentazione irriducibile di $G$. Allora $\Pj(V)^G$ è non vuoto per il teorema del punto fisso (\ref{ThPuntoFissoBorel}), cio\`e esiste una retta $\ell$ in $V$ tale che $G\cdot\ell=\ell$. Quindi $\ell=V$, da cui $\dim V=1$.
\item[$\boxed{2.\implies3.}$] Assumiamo $G$ contenuto in $\GL(V)$ per qualche $V$. Sia $F_1$ una sottorappresentazione irriducibile di dimensione $1$. Notiamo che $G$ agisce su $V/F_1$ quindi, invocando nuovamente l'ipotesi, esiste una retta $F_2/F_1$ in $V/F_1$ tale che $G F_2/F_1 \subseteq F_2/F_1$. Procedendo induttivamente, troviamo una bandiera \[0\subseteq F_1\subseteq F_2 \subseteq \cdots \subseteq F_n =V\] tale che $G F_i=F_i$ e $\dim F_i=i$. Se $v_1,\cdots,v_n$ è una base compatibile con tale bandiera (cioè $v_i\in F_i$ per ogni indice $i$), allora abbiamo un'immersione $G \hookrightarrow B_n$ definita da $g\mapsto [g]_{\underline{v}}^{\underline{v}}$. 
\item[$\boxed{3.\implies1.}$] Segue dal fatto che $B_n$ è risolubile.
\end{itemize}
\setlength{\leftmargini}{0.5cm}
\end{proof}



\subsection{Sottogruppi parabolici e di Borel}
Abbiamo visto che i sottogruppi normali danno come quozienti gruppi algebrici (e quindi affini) e che in generale i quozienti di gruppi algebrici sono variet\`a quasi-proiettive. 

\`E lecito chiedersi quando il quoziente \`e proiettivo.

\begin{definition}
Sia $G$ un gruppo algebrico e $P\subseteq G$ sottogruppo chiuso.
\begin{enumerate}
\item $P$ si dice \textbf{parabolico} se $G/P$ \`e completo
\item $B\subseteq G$ si dice \textbf{sottogruppo di Borel} se \`e un sottogruppo risolubile connesso massimale.
\end{enumerate}
\end{definition}

\begin{proposition}
Se $P$ \`e un sottogruppo parabolico, $G/P$ \`e proiettiva.
\end{proposition}
\begin{proof}
Ricordiamo che esistono una rappresentazione $V$ di $G$ e una retta $L\subseteq V$ tali che $\stab_G L=P$ per il lemma (\ref{LmSottogruppoEStabilizzatoreDiUnaRettaInQualcheRappresentazione}). Per questa rappresentazione $G/P=G\cdot L\subseteq \Pj(V)$.

Mostriamo che l'immagine di $G/P$ in $\Pj(V)$ \`e chiusa. Se $\Gamma$ \`e il grafico di $G/P\inj \Pj(V)$ allora esso \`e un chiuso (per separatezza) di $G/P \times \Pj(V)$. Dato che $G/P$ \`e completa per definizione di sottogruppo parabolico, $G/P\times \Pj(V)\to \Pj(V)$ \`e una mappa chiusa, quindi l'immagine di $\Gamma$ in $\Pj(V)$ \`e chiusa, ma questa immagine \`e per definizione anche l'immagine di $G/P\to \Pj(V)$.
% https://q.uiver.app/#q=WzAsNCxbMSwwLCJHL1BcXHRpbWVzXFxQaihWKSJdLFsyLDAsIlxcUGooVikiXSxbMCwwLCJcXEdhbW1hIl0sWzEsMSwiRy9QIl0sWzAsMV0sWzIsMCwiXFxzdWJzZXRlcSIsMyx7InN0eWxlIjp7ImJvZHkiOnsibmFtZSI6Im5vbmUifSwiaGVhZCI6eyJuYW1lIjoibm9uZSJ9fX1dLFswLDNdLFszLDFdLFszLDJdLFszLDIsIlxcc2ltIiwzLHsib2Zmc2V0IjoxLCJzdHlsZSI6eyJib2R5Ijp7Im5hbWUiOiJub25lIn0sImhlYWQiOnsibmFtZSI6Im5vbmUifX19XV0=
\[\begin{tikzcd}
	\Gamma & {G/P\times\Pj(V)} & {\Pj(V)} \\
	& {G/P}
	\arrow["\subseteq"{marking, allow upside down}, draw=none, from=1-1, to=1-2]
	\arrow[from=1-2, to=1-3]
	\arrow[from=1-2, to=2-2]
	\arrow[from=2-2, to=1-1]
	\arrow["\sim"{marking, allow upside down}, shift right, draw=none, from=2-2, to=1-1]
	\arrow[from=2-2, to=1-3]
\end{tikzcd}\]
\end{proof}


\begin{example}
Sia $G=\GL(n)$, $B\subseteq G$ le matrici triangolari superiori, diagonale inclusa.
\[G/B=\cpa{0\subseteq F_1\subseteq \cdots\subseteq F_n=\K^n\mid \dim F_i=i}=:\Fl(\K^n)\]
\end{example}

\begin{lemma}\label{LmConnessoRisolubileSSENoSottogruppiParabolici}
$G$ \`e un gruppo connesso e risolubile se e solo se $G$ non ha sottogruppi parabolici propri.
\end{lemma}
\begin{proof}
Diamo le due implicazioni
\setlength{\leftmargini}{0cm}
\begin{itemize}
\item[$\boxed{\implies}$] Sia $P\subseteq G$ parabolico. Per il teorema del punto fisso di Borel (\ref{ThPuntoFissoBorel}) si ha che $G/P$ ha un punto fisso per $G$, ma per omogeneit\`a questo significa che $G/P$ consiste di un solo punto, cio\`e $P=G$.
\item[$\boxed{\impliedby}$] Osserviamo che $G^0$ \`e un sottogruppo parabolico, infatti $G/G^0$ \`e un inisieme finito di punti (uno per componente irriducibile di $G$). Per ipotesi dunque $G^0=G$ e quindi il gruppo \`e connesso.

Per il corollario (\ref{CorCaratterizzazioneConnessiRisolubili}) possiamo concludere verificando che le rappresentazioni irriducibili hanno dimensione 1: se $V$ irriducibile, $G$ agisce su $\Pj(V)$ e ha un'orbita chiusa $G\cdot \ell\subseteq\Pj(V)$ per (\ref{PrEsisteOrbitaChiusa}). Sia $P=\stab_G(\ell)$, questo gruppo deve essere parabolico perch\'e il quoziente \`e $G\cdot \ell$ orbita chiusa in $\Pj(V)$, ma allora $P=G$, cio\`e $G\cdot\ell=\cpa{\ell}$ e quindi per irriducibilit\`a questo mostra $V=\ell$.
\end{itemize}
\setlength{\leftmargini}{0.5cm}
\end{proof}



\begin{lemma}[Transitivit\`a dei sottogruppi parabolici]\label{LmSottogruppoParabolicoDiParabolicoEParabolico}
    Siano $G$ un gruppo algebrico, $P\subseteq Q\subseteq G$ due sottogruppi tali che $Q$ è parabolico in $G$ e $P$ è parabolico in $Q$. Allora $P$ è parabolico in $G$.
\end{lemma}

\begin{proof}
    Data una varietà arbitraria $Z$, mostriamo che la mappa $\pi_Z\colon G/P\times Z \to Z$ è chiusa. Sia $A$ un chiuso in $G/P\times Z$. Sia $A'$ l'immagine inversa di $A$ rispetto la proiezione $\pi_1\colon G\times Z \to G/P\times Z$. Allora $A'$ è chiuso e $A' \cdot P=A'$. Se mostriamo che $A'\cdot Q=A'$, abbiamo concluso, infatti, considerando la composizione \[\begin{array}{ccccc}
         G\times Z &\xrightarrow{}& G/Q \times Z& \to &Z \\
          A' & \to & A'' & \to & \pi_Z(A) 
    \end{array}\] 
    l'immagine di $A'$ tramite la prima mappa è un certo $A''$, che viene mandato in $\pi_Z(A)$. \\
    Sia $A'''=A'\cdot Q$. Mostriamo che $A'''$ è chiuso (ciò permette di concludere per quanto appena detto). Consideriamo la mappa 
	\[\vp:\funcDef{Q\times G\times Z}{G\times Z}{(q,g,x)}{(gq,x)}\] 
    Allora $A^{IV}\coloneqq \varphi^{-1}(A')$ è chiuso ed è stabile per l'azione a destra di $P$ su $Q$, cioè se $(q,g,x)$ è in $A^{IV}$, allora $(gp,g,x)$ è in $A^{IV}$. Infatti se $(gq,x)$ è in $A'$, allora anche $(gqp,x)$ è in $A'$. Ragionando come prima, abbiamo 
	\[\begin{array}{ccccc}
         Q\times G\times Z &\xrightarrow{}& Q/P \times G \times Z& \to &G\times Z \\
          A^{IV} & \to & B^V & \to & A^V \text{chiuso} 
    \end{array}\] 
    dove 
	\[A^V=\left\{(g,x)\in G\times Z \colon \, \exists\,q\in Q, \ (gq,x)\in A'\right\}=A'\cdot Q.\] 
	A questo punto $A' \cdot Q$ è chiuso e concludiamo come prima.
\smallbreak

\noindent Per seguire meglio i passaggi pu\`o essere utile questo diagramma. Le frecce ondulate indicano che il target \`e stato ottenuto come preimmagine tramite la mappa tra i due spazi.
    % https://q.uiver.app/#q=WzAsMTUsWzIsMSwiRy9QXFx0aW1lcyBaIl0sWzMsMSwiWiJdLFsyLDMsIkdcXHRpbWVzIFoiXSxbMywzLCJHL1FcXHRpbWVzIFoiXSxbNCw3LCJRXFx0aW1lcyBHXFx0aW1lcyBaIl0sWzIsNiwiUS9QXFx0aW1lcyBHXFx0aW1lcyBaIl0sWzEsMCwiQSJdLFsxLDIsIkEnIl0sWzQsMCwiXFxwaV9aKEEpIl0sWzEsNCwiUVxcY2RvdCBBJyJdLFs0LDgsIkFee0lWfSJdLFsxLDcsIkJeViJdLFs0LDMsIkEnJyJdLFszLDcsIlBcXGNkb3Rfe1xcdGV4dHsxbW8gZmF0dG9yZX19IEFee0lWfSJdLFswLDhdLFsyLDBdLFswLDEsIlxccGlfWiJdLFszLDEsImNoaXVzYSIsMyx7Im9mZnNldCI6LTMsInN0eWxlIjp7ImJvZHkiOnsibmFtZSI6Im5vbmUifSwiaGVhZCI6eyJuYW1lIjoibm9uZSJ9fX1dLFs1LDIsImNoaXVzYSIsMyx7Im9mZnNldCI6Mywic3R5bGUiOnsiYm9keSI6eyJuYW1lIjoibm9uZSJ9LCJoZWFkIjp7Im5hbWUiOiJub25lIn19fV0sWzYsNywiIiwzLHsic3R5bGUiOnsiYm9keSI6eyJuYW1lIjoic3F1aWdnbHkifX19XSxbMiwzLCJcXHNtYXR7bWFudGllbmVcXFxcIGNoaXVzaVxcXFwgaW52YXJpYW50aX0iXSxbNiw4XSxbNiwwLCJcXHN1YnNldGVxIiwzLHsic3R5bGUiOnsiYm9keSI6eyJuYW1lIjoibm9uZSJ9LCJoZWFkIjp7Im5hbWUiOiJub25lIn19fV0sWzEsOCwiXFxzdXBzZXRlcSIsMyx7InN0eWxlIjp7ImJvZHkiOnsibmFtZSI6Im5vbmUifSwiaGVhZCI6eyJuYW1lIjoibm9uZSJ9fX1dLFszLDFdLFs5LDIsIlxcc3Vic2V0ZXEiLDMseyJzdHlsZSI6eyJib2R5Ijp7Im5hbWUiOiJub25lIn0sImhlYWQiOnsibmFtZSI6Im5vbmUifX19XSxbMTAsNCwiXFxzdWJzZXRlcSIsMyx7InN0eWxlIjp7ImJvZHkiOnsibmFtZSI6Im5vbmUifSwiaGVhZCI6eyJuYW1lIjoibm9uZSJ9fX1dLFs1LDJdLFs5LDEyLCIiLDIseyJsYWJlbF9wb3NpdGlvbiI6NjAsImN1cnZlIjoxfV0sWzEzLDQsIlxcc3Vic2V0ZXEiLDMseyJzdHlsZSI6eyJib2R5Ijp7Im5hbWUiOiJub25lIn0sImhlYWQiOnsibmFtZSI6Im5vbmUifX19XSxbMTMsMTFdLFsxMSw1LCJcXHN1YnNldGVxIiwzLHsic3R5bGUiOnsiYm9keSI6eyJuYW1lIjoibm9uZSJ9LCJoZWFkIjp7Im5hbWUiOiJub25lIn19fV0sWzMsMTIsIlxcc3Vwc2V0ZXEiLDMseyJzdHlsZSI6eyJib2R5Ijp7Im5hbWUiOiJub25lIn0sImhlYWQiOnsibmFtZSI6Im5vbmUifX19XSxbMiw0LCJwcm9kb3R0byIsMyx7Im9mZnNldCI6LTMsInN0eWxlIjp7ImJvZHkiOnsibmFtZSI6Im5vbmUifSwiaGVhZCI6eyJuYW1lIjoibm9uZSJ9fX1dLFs0LDJdLFs0LDUsIlxcc21hdHttYW50aWVuZVxcXFwgY2hpdXNpXFxcXCBpbnZhcmlhbnRpfSIsMix7ImxhYmVsX3Bvc2l0aW9uIjo3MH1dLFsxMSw5XSxbNywyLCJcXHN1YnNldGVxIiwzLHsic3R5bGUiOnsiYm9keSI6eyJuYW1lIjoibm9uZSJ9LCJoZWFkIjp7Im5hbWUiOiJub25lIn19fV0sWzEyLDhdLFs3LDksIlxcc3Vic2V0ZXEiLDMseyJzdHlsZSI6eyJib2R5Ijp7Im5hbWUiOiJub25lIn0sImhlYWQiOnsibmFtZSI6Im5vbmUifX19XSxbMTAsMTMsIj0iLDMseyJzdHlsZSI6eyJib2R5Ijp7Im5hbWUiOiJub25lIn0sImhlYWQiOnsibmFtZSI6Im5vbmUifX19XSxbNywxNCwiIiwzLHsiY3VydmUiOjMsInN0eWxlIjp7ImJvZHkiOnsibmFtZSI6InNxdWlnZ2x5In0sImhlYWQiOnsibmFtZSI6Im5vbmUifX19XSxbMTQsMTAsIiIsMyx7ImN1cnZlIjoyLCJzdHlsZSI6eyJib2R5Ijp7Im5hbWUiOiJzcXVpZ2dseSJ9fX1dLFs3LDEyLCIiLDMseyJjdXJ2ZSI6LTF9XV0=
\[\begin{tikzcd}
	& A &&& {\pi_Z(A)} \\
	&& {G/P\times Z} & Z \\
	& {A'} \\
	&& {G\times Z} & {G/Q\times Z} & {A''} \\
	& {Q\cdot A'} \\
	\\
	&& {Q/P\times G\times Z} \\
	& {B^V} && {P\cdot_{\text{1mo fattore}} A^{IV}} & {Q\times G\times Z} \\
	{} &&&& {A^{IV}}
	\arrow[from=1-2, to=1-5]
	\arrow["\subseteq"{marking, allow upside down}, draw=none, from=1-2, to=2-3]
	\arrow[squiggly, from=1-2, to=3-2]
	\arrow["{\pi_Z}", from=2-3, to=2-4]
	\arrow["\supseteq"{marking, allow upside down}, draw=none, from=2-4, to=1-5]
	\arrow["\subseteq"{marking, allow upside down}, draw=none, from=3-2, to=4-3]
	\arrow[curve={height=-6pt}, from=3-2, to=4-5]
	\arrow["\subseteq"{marking, allow upside down}, draw=none, from=3-2, to=5-2]
	\arrow[curve={height=18pt}, squiggly, no head, from=3-2, to=9-1]
	\arrow[from=4-3, to=2-3]
	\arrow["\begin{array}{c} \smat{mantiene\\ chiusi\\ invarianti} \end{array}", from=4-3, to=4-4]
	\arrow["prodotto"{marking, allow upside down}, shift left=3, draw=none, from=4-3, to=8-5]
	\arrow["chiusa"{marking, allow upside down}, shift left=3, draw=none, from=4-4, to=2-4]
	\arrow[from=4-4, to=2-4]
	\arrow["\supseteq"{marking, allow upside down}, draw=none, from=4-4, to=4-5]
	\arrow[from=4-5, to=1-5]
	\arrow["\subseteq"{marking, allow upside down}, draw=none, from=5-2, to=4-3]
	\arrow[curve={height=6pt}, from=5-2, to=4-5]
	\arrow["chiusa"{marking, allow upside down}, shift right=3, draw=none, from=7-3, to=4-3]
	\arrow[from=7-3, to=4-3]
	\arrow[from=8-2, to=5-2]
	\arrow["\subseteq"{marking, allow upside down}, draw=none, from=8-2, to=7-3]
	\arrow[from=8-4, to=8-2]
	\arrow["\subseteq"{marking, allow upside down}, draw=none, from=8-4, to=8-5]
	\arrow[from=8-5, to=4-3]
	\arrow["\begin{array}{c} \smat{mantiene\\ chiusi\\ invarianti} \end{array}"'{pos=0.7}, from=8-5, to=7-3]
	\arrow[curve={height=12pt}, squiggly, from=9-1, to=9-5]
	\arrow["{=}"{marking, allow upside down}, draw=none, from=9-5, to=8-4]
	\arrow["\subseteq"{marking, allow upside down}, draw=none, from=9-5, to=8-5]
\end{tikzcd}\]
\end{proof}

Con tecniche simili, è possibile mostrare il seguente risultato.

\begin{theorem}\label{ThNormalizzatoreDiBorelEDiParabolico}
    Se $G$ è un gruppo algebrico connesso, $B$ un sottogruppo di Borel di $G$ e $P$ un sottogruppo parabolico di $G$, allora \begin{enumerate}
        \item $N_G(B)=B$ e $N_G(P)=P$.
        \item $P$ è connesso.
    \end{enumerate}
\end{theorem}


\begin{theorem}\label{ThProprietaSottogruppiDiBorel}
Sia $G$ un gruppo algebrico. 
\begin{enumerate}
    \item Se $B$ è un sottogruppo di Borel di $G$, allora $B$ è parabolico.
    \item Se $B$ è un sottogruppo di Borel di $G$ e $P$ è un sottogruppo parabolico di $G$, allora esiste $g$ in $G$ tale che $gBg^{-1}$ è contenuto in $P$.
    \item Tutti i sottogruppi di Borel sono coniugati.
\end{enumerate}
\end{theorem}

\begin{proof}
Mostriamo contemporaneamente il primo e il secondo punto. Per i punti 1. e 2. possiamo supporre $G$ connesso. Dato che $B$ \`e connesso, $B\subseteq G^0$ quindi $G/B=\coprod_i g_i G^0/B$ e $gBg\ii\subseteq G^0$. 
\setlength{\leftmargini}{0cm}
\begin{itemize}
\item[$\boxed{2.}$] Se $G$ è risolubile, allora per definizione di sottogruppo di Borel $G=B$ e la tesi vale perch\'e un gruppo connesso e risolubile non ha sottogruppi parabolici propri (\ref{LmConnessoRisolubileSSENoSottogruppiParabolici}). Se $G$ non è risolubile, allora per il lemma (\ref{LmConnessoRisolubileSSENoSottogruppiParabolici}) esiste un sottogruppo parabolico proprio $P$ di $G$. Possiamo quindi considerare l'azione di $B$ su $G/P$, la quale ha un punto fisso (\ref{ThPuntoFissoBorel}), cioè esiste un $g$ in $G$ tale che $gBg^{-1}P=P$, ovvero $gBg^{-1}\subseteq P$.
\item[$\boxed{1.}$] Possiamo assumere $B\subseteq P \subsetneq G$ grazie al punto 2., per cui $B$ è un sottogruppo di Borel di $P$. A questo punto procediamo per induzione sulla dimensione di $G$. Poiché $\dim P <\dim G$, abbiamo che $B$ è un sottogruppo parabolico di $P$ per ipotesi induttiva, dunque per il lemma (\ref{LmSottogruppoParabolicoDiParabolicoEParabolico}) si ha che $B$ è parabolico in $G$.
\item[$\boxed{3.}$] Siano $B,B'$ due sottogruppi di Borel di $G$. Allora esiste $g$ in $G$ tale che $gBg^{-1} \subseteq B'$ per il punto 2. Poiché $gBg^{-1}$ e $B'$ sono sottogruppi connessi, risolubili e massimali, concludiamo che $gBg^{-1} = B'$.
\end{itemize}
\setlength{\leftmargini}{0.5cm}
\end{proof}

\begin{corollary}\label{CorPrabolicoRisolubileConnessoEBorel}
Se $B\subseteq G$ \`e parabolico, connesso e risolubile allora \`e di Borel.
\end{corollary}
\begin{proof}
Per il punto 2. del teorema (\ref{ThProprietaSottogruppiDiBorel}) si ha che $B$ contiene un sottogruppo di Borel, ma essendo connesso e risolubile deve essere gi\`a un sottogruppo di Borel per massimalit\`a.
\end{proof}



\begin{corollary}\label{CorSottogruppoCheContieneBorelEParabolico}
Se $B\subseteq H\subseteq G$ sono sottogruppi e $B$ \`e di Borel allora $H$ \`e parabolico.
\end{corollary}
\begin{proof}
Osserviamo che la mappa di proiezione $G/B\to G/H$ \`e surgettiva, $G/B$ \`e una variet\`a completa e $G/H$ \`e separata, quindi $G/H$ \`e completa, cio\`e $H$ \`e parabolico.
\end{proof}


Sia $\mathscr{B}$ la famiglia dei sottogruppi di Borel di un gruppo algebrico $G$. Allora $G$ agisce per coniugio (e transitivamente) su $\mathscr{B}$. Per il Teorema (\ref{ThNormalizzatoreDiBorelEDiParabolico}) abbiamo $\stab_G(B)=N_G(B)=B$, da cui deduciamo il seguente risultato.
\begin{corollary}
    Sia $B$ un sottogruppo di Borel di un gruppo algebrico $G$. Esiste una corrispondenza biunivoca tra la famiglia $\Bs$ e la variet\`a $G/B$, data da \[\begin{array}{ccc}
        G/B &\longrightarrow& \mathscr{B} \\
         gB & \longmapsto & gBg^{-1}.
    \end{array}\]
\end{corollary}





\section{Esempi di sottogruppi di Borel e Parabolici}
Citiamo il seguente teorema per la sua utilit\`a in alcuni degli esempi:

\begin{theorem}[Witt]\label{ThWitt}
Se $V$ spazio vettoriale, $b$ forma simmetrica non degenere, $W_1,W_2$ sottospazi di $V$ e $\vp:W_1\to W_2$ isomorfismo che preserva $b$, allora $\vp$ si estende a $V$ e continua a preservare $b$.
\end{theorem}

\subsection{Matrici invertibili}
Fissiamo $G=\GL(n)$. 

\begin{lemma}[]
Il sottogruppo $B_n\subseteq \GL(n)$ delle matrici triangolari superiori \`e di Borel.
\end{lemma}
\begin{proof}
Sappiamo che $B_n$ è connesso e risolubile. Mostriamo che è massimale. 

Sia $H$ un sottogruppo risolubile connesso contenente $B_n$. Essendo risolubile e connesso, $H$ si triangolarizza rispetto a una bandiera $F_1\subset \cdots \subset F_n$ per la caratterizzazione (\ref{CorCaratterizzazioneConnessiRisolubili}). Poiché $H\cdot F_i=F_i$ per ogni $i=1,\cdots,n$, si ha anche $B_n\cdot F_i=F_i$ per ogni indice $i$. Poiché $B_n$ è lo stabilizzatore in $G$ di ogni bandiera $F_1\subset \cdots \subset F_n$, si ottiene $H=B_n$. 
\end{proof}



\begin{example}[Grassmanniane come quoziente di $GL(n)$]
Fissiamo $1\leq h<n$ e sia $P$ il sottogruppo di $G$ dato dalle matrici triangolari superiori a blocchi con due blocchi, il primo dei due di taglia $h$. Segue che per questo gruppo $G/P\cong \Gr(h,\K^n)$. Considerando l'azione transitiva di $G$ segue che $P\cong \stab_G \langle e_1,\cdots,e_h\rangle$\footnote{o ogni altro sottospazio di dimensione $h$ fissato.}. Dato che $P$ contiene $B_n$, $P$ \`e parabolico per (\ref{CorSottogruppoCheContieneBorelEParabolico}), quindi $G/P\cong \Gr(h,\K^n)$ \`e proiettiva. 

\medskip
Descriviamo adesso una immersione esplicita in $\Pj(V)$ per qualche spazio vettoriale. Per farlo possiamo seguire il procedimento indicato dalla dimostrazione del punto 1. del lemma (\ref{LmSottogruppoEStabilizzatoreDiUnaRettaInQualcheRappresentazione}) e dalla proposizione (\ref{PrInsiemeDelleClassiEVarietaQuasiProiettiva}).

Cerchiamo uno spazio vettoriale $V$ contenente una retta $L$ fissata da $P$. Consideriamo $V=\bigwedge^h(\K^n)$ e $L=\bigwedge^h W$, dove $W$ è il sottospazio generato da $e_1,\cdots,e_h$. Abbiamo già visto che, fissato $g$ in $G$, si ha $gL=L$ se e solo se $gW=W$. Quindi $\stab_G L = \stab_G W=P$. La mappa che immerge $G/P$ in $\Pj(V)$ secondo questa procedura \`e $[g]\mapsto gL$. Notiamo per\`o che
\[gL =\left(g\bigwedge^hW\right)=\left(\bigwedge^h gW\right)\]
quindi la mappa $\Gr(k,\K^n)\to \Pj(V)$ corrispondente \`e
% https://q.uiver.app/#q=WzAsMyxbMCwwLCJcXEdyKGgsXFxLXm4pIl0sWzIsMSwiXFxQaihWKSJdLFswLDIsIkcvUCJdLFswLDIsImdXXFxsZWZ0cmlnaHRhcnJvdyBbZ10iLDIseyJzdHlsZSI6eyJ0YWlsIjp7Im5hbWUiOiJhcnJvd2hlYWQifX19XSxbMCwxLCJVXFxtYXBzdG8gXFxid15oIFUiXSxbMiwxLCJbZ11cXG1hcHN0byBnTCIsMl1d
\[\begin{tikzcd}
	{\Gr(h,\K^n)} \\
	&& {\Pj(V)} \\
	{G/P}
	\arrow["{U\mapsto \bw^h U}", from=1-1, to=2-3]
	\arrow["{gW\leftrightarrow [g]}"', tail reversed, from=1-1, to=3-1]
	\arrow["{[g]\mapsto gL}"', from=3-1, to=2-3]
\end{tikzcd}\]
Questa \`e la ben nota \textbf{immersione di Pl\"ucker}.
Descriviamo la mappa in coordinate: considerando un sottospazio $U\in \Gr(h,\K^n)$ abbiamo
\[U=\ps{v_1 , \cdots ,v_h} \mapsto v_1\wedge \cdots \wedge v_h =\sum_{i_1<\cdots<i_h} p_{i_1,\cdots,i_h}e_{i_1}\wedge\cdots\wedge e_{i_h}\]
dove $p_{i_1,\cdots,i_h}$ è il determinante del minore di $M=(v_1 \lvert \cdots \lvert v_h)$ corrispondente alle righe $i_1,\cdots,i_h$.
Quindi il punto di $\Pj(V)$ scritto nelle coordinate omogenee corrispondenti alla base $\cpa{e_{I}}=\cpa{e_{i_1}\wedge\cdots\wedge e_{i_h}}$ di $V$ \`e
\[U=\langle v_1,\cdots,v_h\rangle \longmapsto [p_I(v_1,\cdots,v_h)]_{I}\]
Se cambiamo base, $U=\langle w_1,\cdots,w_h\rangle$, allora esiste una matrice invertibile $A$ di taglia $h$ tale che $MA=N$, dove $N=(w_1\lvert\cdots\lvert w_h)$. Si ha dunque
\begin{align*}
    p_I(M)&=\det (M_{i_1,\cdots,i_h}) \\
    p_I(N)&=\det(N_{i_1,\cdots,i_h})=\det (M_{i_1,\cdots,i_h}\cdot A)=p_I(M) \det(A).
\end{align*}
Quindi $(v)\mapsto [p_I(M)]$ e $(w)\mapsto [p_I(N)]=[p_I(M)\det (A)]=[p_I(M)]$ e la mappa \`e ben definita.
\end{example}



\begin{example}[Parabolici di $\GL(n)$]
Sia $P\supseteq B$ con $P$ parabolico e $B$ di Borel, $P$ connesso. Notiamo che $T_e P\supseteq T_e B=\cpa{\text{triang.sup.}}$. Sia $A=(a_{i,j})\in T_e P$. Si ha che $B$ agisce per coniugio su $P$. In particolare se fissiamo $b\in B$ allora definiamo
\[AD_b:\funcDef{P}{P}{p}{bpb\ii}\quad\leadsto\quad Ad_b:\funcDef{T_eP}{T_eP}{C}{bCb\ii}\]
quindi $T_e P\subseteq T_e\GL(n)=\Mat_{n\times n}$ \`e stabile per $Ad_b$ per ogni $b\in B$. Se $b=diag(\la_1,\cdots,\la_n)$ allora
\[Ad_b(A)=\pa{\la_i\la_j\ii a_{i,j}}\]
dunque se $a_{i,j}\neq 0$ allora $E_{i,j}\in T_eP$ 
\end{example}


\subsection{Matrici ortogonali speciali}
Consideriamo ora $G=\SO(n)$. Ricordiamo che per un campo algebricamente chiuso (di caratteristica diversa da 2) tutte le forme quadratiche inducono lo stesso prodotto scalare a meno di isometria, quindi in quanto segue cambiamo liberamente la forma in esame in base alla convenienza.

\begin{remark}
$\SO(1)=\cpa{1}$.
\end{remark}

\begin{remark}
$\SO(2)\cong \G_m$.
\end{remark}
\begin{proof}
Scegliamo la forma quadratica $q(x,y)=xy$. Lo stabilizzatore di tale forma è dato da 
\[\left\{A\in \SO(2)\sep A \begin{pmatrix}
    0 & 1 \\ 1& 0
\end{pmatrix}A^\top=\begin{pmatrix}
    0 & 1 \\ 1& 0
\end{pmatrix}\right\}.\] 
Scriviamo 
\[A=\begin{pmatrix}
    a & b \\ c& d
\end{pmatrix}.\] 
Imponendo che $A$ sia nello stabilizzatore della forma $q$, si trova $b=c=0$. Poiché $\det(A)=1$, troviamo $d=a^{-1}$ e quindi $\SO(2)$ è isomorfo a $\K^\times$.
\end{proof}


\begin{proposition}[]\label{PrSO(n)ConnessoECalcoloDimensione}
$\SO(n)$ \`e connesso e ha dimensione $\binom{n}2$. 
\end{proposition}
\begin{proof}
Consideriamo la mappa 
\[\vp:\funcDef{\Mat_{n\times n}}{\Mat_{n\times n}}{A}{A^\top A}.\] 
Per costruzione $\operatorname{O}(n)=\varphi^{-1}(I)$. Abbiamo una mappa indotta tra gli spazi tangenti a $I$ 
\[d\vp_I:\funcDef{T_I\Mat_{n\times n}}{T_I\Mat_{n\times n}}{B}{B+B^\top}\]
e per il teorema (\ref{ThFibrePerMappeLiscie}) si ha $T_I\SO(n)=T_I\operatorname{O}(n)=\ker d\vp_I$. In particolare \[\dim\SO(n)=\binom{n}{2}.\]
Mostriamo ora che $\SO(n)$ \`e connesso per induzione su $n\ge 1$. Per $n=1$ e $n=2$ sappiamo già che ciò è vero. Assumiamo $n>2$. Consideriamo la forma quadratica standard rappresentata dalla matrice identità, consideriamo $e_1$ in $\K^n$ e 
\[Y\coloneqq \SO(n)\cdot e_1=\left\{ \begin{pmatrix}
        x_1 \\ \vdots \\ x_n 
    \end{pmatrix} \sep x_1^2+\cdots+x_n^2=1\right\}\] 
Poiché il polinomio $x_1^2+\cdots+x_n^2-1$ è irriducibile, la varietà $Y$ è connessa. Inoltre si ha $Y\cong \SO(n)/\SO(n-1)$, che, in quanto connesso, ci permette di concludere.
\end{proof}

 





\begin{proposition}
$\SO(3)\cong \SL(2)/\{\pm I\}.$
\end{proposition}
\begin{proof}
Sappiamo che $\SL(2)$ agisce su $W\coloneqq \K^2$. Consideriamo la rappresentazione $W\otimes W=S^2 W \oplus \bigwedge^2 V$. 
Consideriamo la forma bilineare e simmetrica 
\[b(u\otimes u', v \otimes v')=\det(u,v)\det(u',v').\] 
Tale forma è $\SL(2)$-invariante, infatti, se $g$ è in $\SL(2)$, allora $\det(g)=1$ e quindi
\begin{align*}
	b(gu\otimes gu', gv \otimes gv')=&\det(gu,gv)\det(gu',gv')=\\
	=&\det(g)^2b(u\otimes u', v \otimes v')=\\
	=&b(u\otimes u', v \otimes v').
\end{align*}
Fissiamo una base $e_{11}=e_1\otimes e_1$, $e_{12}=e_1\otimes e_2$, $e_{21}=e_2\otimes e_1$, $e_{22}=e_2\otimes e_2$. Allora \[[b]_{e_{ij}}=\begin{pmatrix}
    0 & 0 & 0 & 1 \\
    0 &0 & -1 & 0 \\
    0 & -1 & 0&0 \\
    1 &0&0&0
\end{pmatrix}.\]
In particolare $b$ è non degenere. Inoltre, anche la restrizione $\beta \coloneqq b\res{S^2W}$ è non degenere, in quanto $S^2W=\langle e_{11},e_{22},e_{12}+e_{21}\rangle$. Abbiamo quindi definito una mappa 
\[\SL(2)\to \SO(\beta).\] 
Imponendo $g e_{11}=e_{11}$ e $g e_{22}=e_{22}$, troviamo $g=\pm I$, quindi abbiamo un'immersione $\SL(2)/\{\pm I\}\hookrightarrow \SO(\beta)$. Mostriamo che è suriettiva. Notiamo che la seguente mappa \`e un isomorfismo
\[\funcDef{\SL(2)\times\K^\times}{\SO(3)}{(A,\la)}{A \mat{\lambda & 0 \\ 0 & 1}}\]
Quindi $\dim\SL(2)=3$. 
Per la proposizione (\ref{PrSO(n)ConnessoECalcoloDimensione}) abbiamo che
\[\dim\SO(3)=3=\dim\SL(2)\] 
e che $\SO(3)$ \`e connesso, dunque il quoziente $\SL(2)/\{\pm I\}$ è $\SO(3)$.
\end{proof}

\begin{remark}
Consideriamo il gruppo 
\[B=\cpa{\mat{a&b\\0&a\ii}}\subseteq \SL(2)\]
Sia $V=\K^2$ e $L=[e_1]\in\Pj(V)$. $\SL(2)$ agisce su $\Pj^1=\Pj(V)$ transitivamente e lo stabilizzatore di $L$ \`e $B$.

Se poniamo
\[W=S^n\K^2=S^nV\]
allora, notando che $e_1^n\in W$, si ha che $\stab_G[e_1^n]=B$.
Possiamo dunque descrivere il quoziente anche come
\[\Pj(V)=\Pj^1\cong G/B=G[e_1^n]\subseteq \Pj(W)=\Pj(S^n V)\]
dove l'immersione \`e data da $v\mapsto v^n$. Questa mappa \`e un morfismo equivariante e una immersione chiusa $\Pj(V)\inj \Pj(W)$.
\end{remark}




\begin{remark}
Se $n=2$ allora
\[\SO(3)/B_{\SO(3)}=\frac{\SL(2)/\pm id}{B_{\SL}/\pm id}=\frac{\SL(2)}{B_{\SL}}.\]
Lo spazio $V=\K^2$ non \`e una rappresentazione di $\SO(3)$ perch\'e $-id$ non agisce banalmente, ma $S^2V$ \`e una rappresentazione di $\SO(3)$.
\end{remark}



\begin{example}
Consideriamo la forma quadratica $x_1x_3=x_2^2$, lo spazio $V=\K^3$ e la retta $L=\K(1\ 0\ 0)^\top$. In $\Pj(V)$ si ha che $G\cdot L$ \`e la quadrica $Q:x_1x_3=x_2^2$ per il teorema di Witt (\ref{ThWitt}).
Notiamo anche che $\stab(L)=B$, quindi abbiamo una mappa
\[\funcDef{\Pj^1}{Q}{[a:b]}{[a^2:ab:b^2]}.\]
\end{example}






\subsection{Quoziente per sottogruppo di Borel}
Sia $G=\SL(n)$ e ricordiamo che il sottogruppo delle matrici triangolari superiori $B$ \`e un sottogruppo di Borel.
\[G/B\cong\cpa{F_1\subseteq\cdots\subseteq F_n\mid \dim F_i=i}\]

\begin{definition}
Definiamo le \textbf{variet\`a delle bandiere} (\textbf{flag variety}) come
\[\GL(n)/B_{\GL(n)}=\SL(n)/B_{\SL(n)}=\Fl(\K^n)=\cpa{F_1\subseteq\cdots\subseteq F_n\mid \dim F_i=i}.\]
\end{definition}

Vogliamo un risultato analogo per $\SO(n)$
\begin{remark}
Se consideriamo la forma quadratica standard $q=x_1^2+\cdots+x_n^2$ e interseco $\SO(n)=\cpa{A\mid A^\top A=I,\det A=1}$ con $B_{\SL(n)}$ troviamo
\[T^\top T=I\quad e \quad T\text{ triangolare superiore}\implies T\text{ diagonale con entrate }\pm1,\]
ovvero 
\[B_{\SO(n)}=\cpa{\emat{T\text{ diagonale con entrate sulla diagonale 1}\\\text{ eccetto un numero pari di $-1$.}}}\]
\end{remark}

\bigskip

\noindent Consideriamo ora la forma quadratica associata a
\[J=\mat{
0&\cdots&0&1\\
\vdots&0&\iddots&0\\
0 &\iddots& 0&\vdots\\
1&0&\cdots&0
}\]
da cui $\SO(J)=\cpa{A\mid AJ A^\top=J,\ \det A=1}$. Osserviamo che $J=J\ii$.

\begin{proposition}
Il sottogruppo di $\SO(J)$ dato da
\[B=\cpa{\mat{A&B\\0&D}\sep D=(A^{at})\ii,\ (BA^{at})^{at}=-(BA^{at}),\ A\text{ triang. sup.}},\]
dove $M^{at}$ indica la trasposta di $M$ rispetto all'altra diagonale, \`e un sottogruppo di Borel.
\end{proposition}
\begin{proof}
Assumiamo $n=2m$, il caso dispari \`e simile.

Notiamo che $Jg^\top J\ii=Jg^\top J=g^{at}$, cio\`e la trasposta di $g$ rispetto all'altra diagonale. Scrivendo $g$ triangolare superiore evidenziando blocchi di dimensione $m$ troviamo
\[g=\mat{A&B\\0&D}.\]
Poich\'e $g\in \SO(J)$ significa che $gg^{at}=id$ con determinante 1, $g$ triangolare superiore appartiene a $\SO(J)$ se
\[\mat{I_m &0\\0&I_m}=\mat{A&B\\0&D}\mat{D^{at}&B^{at}\\0&A^{at}}=\mat{AD^{at}&AB^{at}+BA^{at}\\0&DA^{at}}\]
ovvero se $D=(A^{at})\ii$ e $(BA^{at})^{at}=-(BA^{at})$. Il determinante di $g$ \`e automaticamente $1$ perch\'e\footnote{$\det M^{at}=\det(JM^\top J)=1\cdot \det M^\top\cdot 1=\det M$.} $\det D=(\det A)\ii$.
Dunque l'intersezione tra $\SO(J)$ e le triangolari superiori \`e
\[B=\cpa{\mat{A&B\\0&D}\sep D=(A^{at})\ii,\ (BA^{at})^{at}=-(BA^{at}),\ A\text{ triang. sup.}}.\]
Notiamo che $B$ \`e connesso\footnote{connesso perch\'e immagine di $\cpa{\text{triang.sup.invertibili}}\times \cpa{\text{antisimmetriche per l'altra diagonale}}$} e risolubile\footnote{per quanto appena detto \`e un sottogruppo di $B_n$ e quindi invochiamo (\ref{CorCaratterizzazioneConnessiRisolubili})}. Se dimostriamo che $B$ \`e parabolico allora \`e un sottogruppo di Borel per (\ref{CorPrabolicoRisolubileConnessoEBorel}). 



Sia $\Es$ la bandiera data dai vettori di base 
\[E_1=\ps{e_1}\subseteq E_2=\ps{e_1,e_2}\subseteq\cdots\subseteq E_n=\ps{e_1,\cdots, e_{2m}}=\K^n.\] 
Notiamo che $\stab_{\GL(n)}(\Es)=B_{\GL(n)}$ e che $\stab_{O(J)}(\Es)=B_{\GL(n)}\cap O(J)=B_{O(J)}=B_{\SO(J)}=B$, dove la penultima uguaglianza segue perch\'e tutte le matrici ortogonali triangolari superiori per la forma associata a $J$ hanno gi\`a determinante 1.
Notiamo che 
\[O(J)\cdot\Es\subseteq \cpa{0=F_0\subseteq\cdots\subseteq F_{2m}\mid \dim F_i=i,\ F_{m-i}^\perp=F_{m+i}\ \forall i\in\cpa{0,\cdots,m}}\subseteq \Fl.\]
Per il teorema di Estensione di Witt (\ref{ThWitt}) si ha che $O(J)\cdot\Es$ \`e esattamente lo spazio delle bandiere di quella forma, che chiamiamo $\Fs$. Vogliamo dimostrare che $\Fs$ \`e chiuso, cos\`i per proiettivit\`a di $\Fl$ avremo mostrato che $O(J)/B=O(J)/\stab_{O(J)} \Es=O(J)\cdot \Es$ \`e proiettivo. Sia
\[\Gs=\cpa{g\in \GL(n)\mid gB_{\GL}\in \Fs}\]
Poich\'e $\Fl=\GL(n)/B_{\GL}$ basta\footnote{$\Fs=\Gs/B_{\GL}$ e $\Gs$ \`e $B_{\GL}$-invariante.} mostrare che $\Gs$ \`e chiuso. Sia $g=\pa{v_1\mid\cdots\mid v_{2m}}$ e notiamo che $g\in\Gs$ se e solo se $b(v_i,v_j)=0$ per $i,j\leq m$, $b(v_{m+\al},v_i)=0$ per $i\leq m-\al$ e $b(v_i,v_j)=0$ per $i+j\leq 2m$. Queste sono condizioni chiuse quindi $\Gs$ \`e chiuso.

Concludiamo notando che
\[O(J)=\frac{\SO(J)\sqcup g\SO(J)}{B}\]
per una qualsiasi $g$ tale che $\det g=-1$, $g\in O(J)$, dunque
\[\frac{O(J)}{B}=\frac{\SO(J)}{B}\sqcup g\frac{\SO(J)}{B}\]
e quindi $\frac{\SO(J)}{B}$ \`e proiettivo come volevasi dimostrare.
\end{proof}



Vogliamo descrivere la componente data da $\SO(2m)/B$.
\begin{theorem}
Sia $V=\K^{2m}$ munito di una forma bilineare, simmetrica e non degenere $b(\cdot,\cdot)$. Siano $L,M$ 
sottospazi isotropi (lagrangiani) di $V$ tali che $\dim L=\dim M$. Allora \[\exists\,g\in \SO(b)\colon \ gL=M\quad \iff \quad \dim(L\cap M)\equiv m \pmod{2}.\] 
\end{theorem}

%\[\SO(2m)\cdot \xi =\{F_1\subset \ldots\subset F_{2m} \colon F_i^\perp=F_{2m-i}, \ \dim(F_m\cap \langle e_1,\ldots,e_m\rangle )\equiv m \pmod{2}\}.\] 
\begin{proof}
Scegliamo una base $e_1,\cdots,e_m,e_{m+1},\cdots,e_{2m}$ di $V$ tale che $e_1,\cdots,e_m$ è base di $L$ e tale che $b$ è rappresentata in questa base dalla matrice 
\[\begin{pmatrix}
    0 & I_m \\ I_m & 0
\end{pmatrix}.\] 
Sia inoltre $L'=\langle e_1,\cdots,e_{m-1},e_{2m}\rangle$. Notiamo che $L'$ è isotropo. Mostriamo che se $g$ è un elemento di $\SO(2m)$, allora 
\begin{align}
    \label{cong1}\dim(gL\cap L)&\equiv m \pmod{2} \\
    \label{cong2}\dim(gL\cap L')&\not\equiv m \pmod{2}
\end{align}
Iniziamo mostrando la \eqref{cong1}. Sia $P$ il sottogruppo di $\SO(b)$ costituito dalle matrici del tipo 
\[\begin{pmatrix}
    A & B \\ 0 & C
\end{pmatrix}.\] 
Notiamo che una matrice di questa forma è in $\SO(b)$ se e solo se $A^\top=C^{-1}$ e $B^\top C+C^\top B=0$, per cui
\[\dim P= \dim \GL(m)+\binom{m}{2}=m^2+\binom{m}{2}.\] 
Sia $U$ il sottogruppo di $\SO(b)$ costituito dalle matrici del tipo \[\begin{pmatrix}
    I_m & 0 \\ X& I_m 
\end{pmatrix}\]
Osserviamo che una matrice di questo tipo appartiene a $\SO(b)$ se e solo se $X^\top=-X$, cioè se e solo $X$ è antisimmetrica, quindi $\dim U=\binom{m}{2}$. Ne consegue che $\dim P +\dim U= 2m^2-m$. Ricordiamo che $\dim \SO(2m)=\dim \operatorname{O}(2m)=2m^2-m$, infatti la mappa 
\[\begin{array}{ccc}
    \GL(2m)& \longrightarrow & \operatorname{Sym}(2m) \\
     A&\longmapsto& AA^\top
\end{array}\] 
è surgettiva e ha nucleo $\operatorname{O}(2m)$, che quindi ha dimensione $4m^2-\binom{2m+1}{2}=2m^2-m$. Consideriamo la mappa $U\times P \to \SO(b)$ data da


\[\begin{array}{cccccc}
    U\times P & \xrightarrow{\sim} & U\times P &\longrightarrow & \SO(b) \\
     (u,p^{-1}) &\longmapsto& (u,p)&\longmapsto& u I p \\
     (u,p) &\longmapsto& (u,p^{-1})&\longmapsto& u I p^{-1}
\end{array}\]
Ora $UP$ è aperto in $\overline{UP}$ e poiché $U\cap P=\{I_{2m}\}$, si ha 
\[\dim(UP)=\dim (U \times P)= \dim \SO(2m),\] 
da cui $\overline{UP}=\SO(2m)$. Consideriamo gli aperti $V_1=UP$ e
$V_2=PU$ in $\SO(2m)$. Mostriamo che $V_2V_1=\SO(2m)$. %e quindi che $g$ è in $V_2 V_1$. Infatti s
Sia $g$ in $\SO(2m)$ e sia $v_2$ nell'intersezione $V_2\cap gV_2$ (che è non vuota in quanto $V_1$ e $V_2$ sono aperti in una varietà irriducibile). Allora $v_2=gv_2'$ per qualche $v_2'$ in $V_2$. Poiché $(v_2')^{-1}\in UP=V_1$, si ha $g=v_2(v_2')^{-1}\in V_2V_1$. Ne deduciamo che 
\[\SO(2m)= P U\cdot U P=PUP\] 
e quindi, posto $g=p_1 u p_2$, si ha 
\[\dim(gL\cap L)=\dim(p_1 u p_2L \cap L)=\dim(u p_2 L \cap p_1^{-1}L)=\dim(uL \cap L).\] 
Quindi ci siamo ridotti a considerare il caso di un elemento 
\[u=\begin{pmatrix}
    I_m & 0 \\ X & I_m
\end{pmatrix}\in U.\]
Osserviamo che, poiché $L=\operatorname{span}\{e_1,\cdots,e_m\}$, si ha 
\[uL=\operatorname{span}\left\{\begin{pmatrix}
    I_m & 0\\ X & I_m
\end{pmatrix} e_i,\  i=1,\cdots,m \right\}, %\quad L=\operatorname{span}\left\{\begin{pmatrix}
    %I_m & 0 \\ 0
%\end{pmatrix}\right\}
\] 
quindi 
\[uL\cap L=\operatorname{span}\left\{\begin{pmatrix}
    v \\ Xv
\end{pmatrix}, \  v\in \K^m \colon  Xv=0\right\} \cong \ker X,\] 
da cui $\dim(uL\cap L)=\dim(\ker X)=m-\rnk(X) \equiv m \pmod{2}$ perché le matrici antisimmetriche hanno rango pari. 

Mostriamo la \eqref{cong2}. Sia $g$ un elemento in $\operatorname{O}(2m)$. Sappiamo che se $g$ è in $\SO(2m)$, allora $\dim(gL\cap L)\equiv m \pmod{2}$; se invece $g$ non è in $\SO(2m)$, allora $\dim(gL\cap L)\not\equiv m \pmod{2}$. Sia $h$ in $\operatorname{O}(2m)\setminus \SO(2m)$ che scambia $e_m$ con $e_{2m}$ e lascia fissi gli altri $e_i$. Allora $L'=hL$  %Supponiamo $g\in \operatorname{O}(2m)\setminus \SO(2m)$. 
ed esiste $\widetilde{g}$ in $\SO(2m)$ tale che $g=h\widetilde{g}$. Dunque \[\dim(gL\cap L)=\dim(h\widetilde{g}L\cap L)=\dim(\widetilde{g}L \cap L')\not\equiv m \pmod{2}\] e ciò conclude la dimostrazione. 
\end{proof}

\begin{example}[Caso $m=1$] Consideriamo $V=\K^2$ munito della forma bilineare la cui forma quadratica associata è data da $b(x,y)=xy$. Allora i sottospazi isotropi $\K e_1$ e $\K e_2$ si intersecano banalmente e quindi non vengono scambiati dagli elementi di $\SO(b)$.  In effetti 
    \[\SO(b)=\left\{\begin{pmatrix}
    a & 0 \\
    0 & a^{-1}
\end{pmatrix}, \ a\in \K^\times\right\}.\]
\end{example}
\begin{example}
    [Caso $m=2$] Consideriamo $\K^4$ con la forma quadratica 
	\[b(x_1,x_2,x_3,x_4)=x_1x_4-x_2x_3.\] 
	Cerchiamo i sottospazi di dimensione $2$ contenuti nel radicale della forma: cerchiamo quindi rette in $\Pj(\K^4)$ contenute nella quadrica $x_1x_4=x_2x_3$. Questa quadrica corrisponde a $\Pj^1 \times \Pj^1$ tramite la mappa $([a:b],[c:d])\mapsto [ac:ad:bc:bd]$. Il gruppo $\SO(b)$ agisce su ciascun fattore $\Pj^1$, ma non scambia i due fattori tra loro. %Concretamente, la quadrica è composta da due famiglie di rette ...
\end{example}