\section{Gruppi unipotenti}
\begin{definition}
    Un gruppo $G$ si dice \textbf{unipotente} se ogni suo elemento è unipotente.
\end{definition}

\begin{lemma}\label{LmUnicaRappresentazioneIrriducibileBanaleImplicaUpperTriangularCon1SuDiagonale}
Se $G$ \`e tale che l'unica rappresentazione irriducibile di $G$ \`e banale allora $G$ si immerge nel gruppo delle matrici triangolari superiori aventi $1$ sulla diagonale, cio\`e
\[G\subseteq U(n)=\cpa{\mat{1 & \ast & \ast\\ &\ddots &\ast\\ &&1}}\]
\end{lemma}
\begin{proof}
    Sia $V$ un $\K$-spazio vettoriale di dimensione finita per cui $G$ è un sottogruppo di $\GL(V)$. Sia $W$ una sottorappresentazione di $V$ non banale. Allora $W=\K v_1$ e $g v_1=v_1$ per ogni $g$ in $G$. Consideriamo il quoziente $V_1=V/\langle v_1\rangle$. Allora esiste $v_2$ in $V$ tale che per ogni $g$ in $G$ si ha $gv_2\equiv v_2 \pmod{V_1}$. Procedendo in questo modo, possiamo scegliere una base $v_1,\ldots,v_n$ di $V$, rispetto alla quale risulta chiaramente $G\subseteq U_n$.
\end{proof}

\begin{theorem}\label{ThSottoalgebraAssociativaDiEndomorfismiDiSempliceETuttiGliEndomorfismi}
Sia $\K$ algebricamente chiuso, $V$ spazio vettoriale di dimensione finita, $A\subseteq \End_\K(V)$ che sia una $\K$-algebra associativa. Se $V$ \`e un $A$-modulo semplice allora $A=\End_\K(V)$.
\end{theorem}
\begin{proof}
Sia $v_1,\cdots, v_n$ una base di $V$. Se $v=(v_1,\cdots, v_n)\in V^n$ allora vogliamo mostrare che $Av=V^n$.

Poich\'e $V$ \`e semplice, $V^n$ \`e somma di rappresentazioni semplici, quindi per la proposizione (\ref{PrSommaSempliciESommaDirettaSemplici}) si ha che $V^n$ \`e semisemplice e quindi completamente riducibile. Possiamo allora scrivere $V^n=Av\oplus P$ per $P$ un $A$-sottomodulo. Sia $\pi_P:V^n\to V^n$ la proiezione su $P$ e notiamo che essa ammette una decomposizione a blocchi
\[\pi_P=(\al_{ij})_{i,j}\quad\text{ per degli endomorfismi }\al_{ij}:V\to V\]
dove il dominio di $\al_{ij}$ \`e la $j$-esima copia di $V$ in $V^n$ e il codominio \`e l'$i$-esima copia.

Quindi $\al_{ij}\in \End_A(V)$ con $V$ spazio vettoriale su $\K$ algebricamente chiuso. Per il lemma di Schur (\ref{LmSchur}) questi endomorfismi sono le costanti.

Ricordando ricordando che $V^n=Av\oplus P$, si ha necessariamente $\pi_P(v)=0$ per definizione di $\pi_P$, quindi per ogni $i$ abbiamo $\sum \al_{ij}v_j=0$. Poich\'e gli $v_j$ sono una base e gli endomorfismi $\al_{ij}$ sono costanti, per indipendenza lineare questo significa che per ogni $i$ e ogni $j$ si ha $\al_{ij}=0$. Segue dunque che $\pi_P=0$ e quindi $P=\imm \pi_P=\cpa{0}$, cio\`e $V^n=Av$.
\end{proof}

\begin{exercise}
Trova un controesempio per $\K$ non algebricamente chiuso.
\end{exercise}

\begin{theorem}\label{ThUnipotenteSeESoloSeUnicaRappresentazioneIrriducibileEBanale}
    Un gruppo $G$ è unipotente se e solo se l'unica rappresentazione irriducibile di $G$ è quella banale.
\end{theorem}
\begin{proof}
Diamo le due implicazioni
\setlength{\leftmargini}{0cm}
\begin{itemize}
\item[$\boxed{\impliedby}$] Se $G$ ha questa propriet\`a allora per il lemma (\ref{LmUnicaRappresentazioneIrriducibileBanaleImplicaUpperTriangularCon1SuDiagonale}) abbiamo
\[G\subseteq U(n)=\cpa{\mat{1 & \ast & \ast\\ &\ddots &\ast\\ &&1}}\]
e chiaramente un gruppo di questa forma \`e unipotente.
\item[$\boxed{\implies}$] Sia $V$ una rappresentazione semplice di $G$ di dimensione $n$. Allora $\tr(g_V)=n$ per ogni $g\in G$ (perch\'e si immerge nelle triangolari superiori), quindi
\[\forall g_V,h_V\in G,\qquad \tr((g_V-1)h_V)=\tr(g_Vh_V-h_V)=0.\]
Se $A$ \`e il $\K$-spazio vettoriale generato da $G$, $A\subseteq \End_\K(V)$ ed \`e un'algebra associativa. Poich\'e $V$ \`e semplice per $A$ (in quanto semplice per $G$ e $A=\Span_\K(G)$), si ha per il teorema (\ref{ThSottoalgebraAssociativaDiEndomorfismiDiSempliceETuttiGliEndomorfismi}) si ha $A=\End_\K(V)$.

Segue per linearit\`a della traccia che $\tr((g_V-1)a)=0$ per ogni $a\in \End_\K(V)=A$, da cui $(g_v-1)=0$, cio\`e $G$ agisce banalmente.
\end{itemize}
\setlength{\leftmargini}{0.5cm}
\end{proof}

\begin{corollary}
    Se $G$ è unipotente, allora è contenuto nel gruppo $U_n$ delle matrici triangolari superiori aventi $1$ sulla diagonale principale.
\end{corollary}

\begin{remark}
    Consideriamo il gruppo $(\C,+)$. Questo è un gruppo unipotente perché possiamo vederlo in $\GL(2)$ tramite la rappresentazione \[x\mapsto \begin{pmatrix}
        1 & x \\ 0 & 1 
    \end{pmatrix}\]
    Abbiamo anche una rappresentazione data da \[ \C\to \GL(1), \quad x\mapsto \text{e}^{\alpha x}.\]
    La differenza tra i due casi è che da una rappresentazione $V$ non possiamo costruire tutte le funzioni di $C^{\infty}(\C)$, ma solo quelle di $\C[G]$ (con $G=\C$).
\end{remark}


\begin{corollary}\label{CorFissatoDaUnipotenteNonEBanale}
    Se $V$ \`e una rappresentazione di $G$ non nulla allora $V^G\neq 0$.
\end{corollary}

\begin{corollary}
    Se $G$ \`e unipotente allora $G$ \`e nilpotente come gruppo, cio\`e definiamo iterativamente $G^{(0)}=G$ e $G^{(k+1)}=[G^{(k)},G]$ allora esiste $n$ tale che $G^{(n)}=\cpa{id_G}$.
\end{corollary}
\begin{proof}
Basta immergere $G$ in $U(n)$ e notare che sottogruppi di triangolari superiori con 1 sulla diagonale hanno questa propriet\`a.
\end{proof}


\begin{example}
Il gruppo $G=\znz p=C_p$ con $\cha \K=p$ \`e un gruppo unipotente, perch\'e si pu\`o scrivere come
\[G=\cpa{\mat{1&\al\\0&1}\mid \al\in\znz p}\]
\end{example}
\begin{remark}
Se $G=U(n)$ con $\cha \K=p$ allora $g^{p^n}=id_G$.
\end{remark}

\subsection{Esponenziale e logaritmo}

\begin{notation}
Definiamo lo spazio vettoriale
\[N(n)=\cpa{\mat{0&\ast&\ast\\&\ddots&\ast\\&&0}}.\]
\end{notation}

\begin{definition}[Esponenziale]
Definiamo la mappa \textbf{esponenziale}
\[\exp:\funcDef{N(n)}{U(n)}{M}{\sum_{i=0}^n\frac1{i!}M^i}.\]
\end{definition}
\begin{remark}
La mappa esponenziale appena definita \`e algebrica. Inoltre rispetta le usuali propriet\`a:
\begin{itemize}
    \item $\exp((\la+\mu)M)=\exp(\la M)\exp(\mu M)$
    \item Se $M_1M_2=M_2M_1$ allora $\exp(M_1+M_2)=\exp(M_1)\exp(M_2)$.
\end{itemize}
\end{remark}

\begin{definition}[Logaritmo]
Definiamo la mappa \textbf{logaritmo}
\[\log:\funcDef{U(n)}{N(n)}{B}{\sum_{i=1}^n\frac{(-1)^{i+1}}{i}(M-I_n)^i}.\]
\end{definition}

\begin{remark}
$\log$ e $\exp$ sono mappe algebriche e inverse. Sono anche in realt\`a la definizione usuale, solo che per matrici in $N(n)$ e $U(n)$ queste somme finite coincidono con la definizione in serie.
\end{remark}

\begin{notation}
Se $G\subseteq U(n)$ definiamo $X=\log(G)$ e notiamo che $X$ \`e isomorfo a $G$ come variet\`a.
\end{notation}

\begin{proposition}\label{PrChiusuraDelGeneratoDaPotenzeDiElemento}
Se $g\in G\bs\cpa{id_G}$ allora, ponendo
\[\ol{\cpa{g^n\mid n\in\Z}}=H\subseteq G,\]
si ha $H\cong \G_a=(\K,+)$.
\end{proposition}
\begin{proof}
Poich\'e $g\neq id_G$, $\log g=x\neq 0$, inoltre $\log(g^n)=nx$. Notiamo dunque che da $nx\neq 0$ per ogni $n$ ricaviamo $g^n\neq 1_G$ per ogni $n$. Se $Y=\K x$ allora
\[\log(H)=\log(\ol{\cpa{g^n}})=\ol{\cpa{nx}}\subseteq Y\]
Poich\'e $Y$ \`e una retta e $\cpa{nx}$ sono infiniti, $\ol{nx}= Y$ per come sono fatti i chiusi di Zariski di $\A^1$, quindi 
\[\funcDef{\K}{H=\exp(Y)}{\la}{\exp(\la x)}\]
\`e un isomorfismo di gruppi tra $\K$ e $H$.
\end{proof}

\begin{exercise}
Se $\cha \K=0$ e $G$ unipotente abeliano allora $G\cong \K^n$.
\end{exercise}

\begin{fact}
Se $\cha \K=p$ e $g^p=id$ per ogni $g\in G$ abeliano connesso allora $G$ \`e unipotente e $G\cong \K^n$.
\end{fact}

\begin{example}
Se $\cha \K=p$ poniamo $\wt \al_i=\frac1p\binom pi$ per $i\in\cpa{0,\cdots, p-1}$ e definiamo $\al_i$ come l'immagine di $\wt \al_i$ in $\K$.

Definiamo\footnote{moralmente $c(a,b)=\frac{(a+b)^p-a^p-b^p}p$, che non potremmo fare direttamente in $\K$ per l'identit\`a del binomio ingenuo}
\[c:\funcDef{\K\times \K}{\K}{(a,b)}{\sum_{i=1}^{p-1} \al_i a^ib^{p-1}}\]
e notiamo che
\[c(a,b)c(a+b,c)=c(b,c)c(a,b+c)\implies c(a,0)=c(a,b)=0,\]
quindi se $G=\K\times\K$ con prodotto
\[(a,b)\cdot(a',b')=(a+a',b+b'+c(a,a'))\]
allora $G$ \`e unipotente e $G\cong \K^2$ come gruppo.

(QUESTO DOVREBBE ESSERE UN CONTROESEMPIO DI QUALCOSA???)
\end{example}

\section{Gruppi completamente riducibili}
\begin{definition}
Un gruppo \`e \textbf{completamente riducibile} se ogni sua rappresentazione regolare \`e semisemplice.
\end{definition}
\begin{remark}
Basta anche chiedere ``ogni rappresentazione regolare \emph{finita} \`e semisemplice".
\end{remark}

\begin{remark}
$G$ \`e completamente riducibile se e solo se $\K[G]$ \`e semisemplice.
\end{remark}
\begin{proof}
$\K[G]$ \`e una rappresentazione regolare di $G$ quindi una implicazione \`e ovvia.
Se $V$ ha dimensione $n$ e $\K[G]$ \`e semisemplice allora per l'immersione (\ref{LmIniezioneRappresentazioniFiniteInPotenzaAnelloCoordinateDiG}) $V\inj \K[G]^m$ si ha che $V$ \`e semisemplice (\ref{LmPassaggioSemisempliceUnipotenteNilpotenteAdOperazioniVettoriali}).
\end{proof}

\begin{definition}
Un gruppo $G$ \`e un \textbf{toro (algebrico)} se $G\cong (\G_m)^n$.
\end{definition}

\begin{remark}
Ricordando che $\K[\G_m]=\K[t^{\pm1}]$ notiamo che
\[\K[\G_m^n]=\K[t_1^{\pm1},\cdots,t_n^{\pm1}].\]
Questo spazio ha una base data da $t^\al=t_1^{\al_1}\cdots t_n^{\al_n}$ per $\al\in \Z^n$.
\end{remark}

\begin{proposition}[Tori algebrici sono semisemplici]\label{PrToriAlgebriciSonoSemisemplici}
Se $G$ toro algebrico allora $G$ \`e semisemplice.
\end{proposition}
\begin{proof}
Se $g=(\la_1,\cdots, \la_n)$ e $h=(\mu_1,\cdots, \mu_n)$ sono elementi di $\G_m^n=(\K^\times)^n$ si ha che
\begin{align*}
(gt^\al)(h)=&t^\al(g\ii h)=t^\al(\la_1\ii \mu_1,\cdots, \la_n\ii\mu_n)=\\
=&(\la_1\ii\mu_1)^{\al_1}\cdots (\la_n\ii\mu_n)^{\al_n}=\\
=&\la^{-\al}\mu^\al=\\
=&(\la^{-\al}t^\al)h.
\end{align*}
Segue che $gt^\al=\la^{-\al}t^\al$ e che quindi i $t^\al$ sono autovettori per ogni $g\in G$. Poich\'e
\[\K[G]=\bigoplus_{\al\in\Z^n} \K t^\al\]
e ogni $\K t^\al$ \`e semisemplice, segue (\ref{LmPassaggioSemisempliceUnipotenteNilpotenteAdOperazioniVettoriali}) che $\K[G]$ \`e semisemplice.

Quindi le rappresentazioni semplici di $G$ sono di dimensione 1 e sono date da $\K_\al=\K t^{-\al}$ con $gz=t^\al(g)z$
\end{proof}


\begin{lemma}
    Se $G\subseteq \GL(V)$ allora $G$ \`e completamente riducibile se e solo se $V^{\otimes n}$ \`e semisemplice per ogni $n$
\end{lemma}
\begin{proof}
Diamo le implicazioni
\setlength{\leftmargini}{0cm}
\begin{itemize}
\item[$\boxed{\implies}$] Ovvio
\item[$\boxed{\impliedby}$] La dimostrazione \`e del tutto analoga a quella esposta per il lemma (\ref{LmCriterioSemisempliceUnipotentePerGruppiLineari}). Dimostriamo che $\K[G]$ \`e semisemplice. Dato il morfismo
$\K[\GL(V)]\onto \K[G]$
basta mostrare che $\K[\GL(V)]$ \`e semisemplice. Scriviamo \[\K[\GL(V)]=\K[\End(V)][{\det}\ii].\]
$\K[\End(V)]$ \`e un quoziente di somme di rappresentazioni della forma $(V^\ast)^{\otimes m}$ e quindi \`e quoziente di $(V^\ast\oplus\cdots, \oplus V^\ast)^{\otimes m}$ e dato che $V^\ast$ \`e semisemplice ho finito.
\end{itemize}
\setlength{\leftmargini}{0.5cm}
\end{proof}

\begin{corollary}
    Se $\K=\C$ e $G\subseteq \GL(n,\C)$ \`e tale che se $g\in G$ allora $\ol g\in G$, allora $G$ \`e completamente riducibile.
\end{corollary}
\begin{proof}
Sia $V=\C^n$ e dimostriamo che $V^{\otimes m}$ \`e semisemplice per ogni $m$, cio\`e che per ogni $U\subseteq V^{\otimes m}$ che sia $G$-stabile esiste $W$ $G$-stabile tale che $V^{\otimes m}=U\oplus W$.

Consideriamo il caso $m=1$. Definiamo la forma hermitiana
\[h((x_i)(y_i))=\sum_{i=1}^n \ol{x_i}y_i.\]
Se $U\subseteq V$, poniamo $W=U^\perp$ rispetto a questa forma. Chiaramente $V=U\oplus U^\perp$, quindi vogliamo mostrare che $U$ $G$-stabile implica $U^\perp$ $G$-stabile.
\[h(gw,u)=\ol w^\top \ol g^\top u=h(w,\under{\in U}{\ol g^\top u})\pasgnlmath={w\in U^\perp}0.\]


Per il caso generale l'idea \`e la stessa ma usiamo
\[h_m(v_1\otimes\cdots\otimes v_m,u_1\otimes \cdots\otimes u_m)=\prod_{i=1}^m h(v_i,u_i).\]
Si conclude usando la tesi per $h$.
\end{proof}

\begin{example}
Sia $G\in\cpa{\GL(n,\C), \SL(n), O(n)}$, allora $G$ \`e completamente riducibile. Per $\GL(n,\C)$ e $\SL(n,\C)$ questo \`e ovvio. Per $O(n)=\cpa{g^\top g=id}$ basta mostrare che se $g^\top g=id$ allora $\ol g^\top \ol g=id$, ma questo \`e chiaro.


Anche $SO(n)$ e $S_p(2n)$ hanno questa propriet\`a, dove
\[S_p(2n)=\cpa{g\mid gJg^\top=J},\qquad J=\mat{0 &I\\ -I &0}.\]
\end{example}



\begin{example}
Sia $\cha \K=p=2$ e consideriamo $G=\SL(2,\K)$. Esso ammette una rappresentazione semplice $V=\K^2$. Sia $x,y$ una base di $V$ e consideriamo l'azione data da
\[\mat{a&b\\c&d}x=ax+by,\qquad \mat{a&b\\c&d}y=cx+dy.\]
Consideriamo ora $S^2V=\ps{X^2,xy,y^2}$. Per questioni di caratteristica 2, $gx^2=(gx)^2$. Notiamo allora che $W=\ps{x^2,y^2}$ \`e una sottorappresentazione di $S^2V$ che non ammette un complementare.
\end{example}

\begin{exercise}
Se $\cha \K=p$ allora $\SL(n,\K)$ e $\GL(n,\K)$ non sono completamente riducibili.
\end{exercise}




\begin{proposition}\label{PrCompletamenteRiducibileNonHaSottogruppiNormaliUnipotenti}
Se $G$ \`e completamente riducibile allora $G$ non ha sotto-gruppi unipotenti normali non banali.
\end{proposition}
\begin{proof}
Sia $W$ tale che $G\inj \GL(W)$ (definizione di gruppo algebrico lineare).

Sia $V$ una rappresentazione irriducibile di $G$ e sia $U\subseteq G$ un sottogruppo normale unipotente. Poich\'e $V\neq (0)$ e $U$ unipotente, $V^U\neq 0$ per il corollario (\ref{CorFissatoDaUnipotenteNonEBanale}). Poich\'e $U$ \`e normale $V^U$ \`e stabile per $G$ e quindi $V^U=V$, cio\`e $U$ agisce banalmente su tutte le rappresentazioni irriducibili. Questo mostra che per la mappa $G\to \GL(W)$ si ha che $U$ finisce in $\cpa{id_W}$ perch\'e $G$ \`e completamente riducibile, ma questa mappa \`e iniettiva e quindi $U=\cpa{1_G}$.
\end{proof}



\begin{definition}[Caratteri di un gruppo]
Dato un gruppo algebrico $G$ definiamo un \textbf{carattere} di $G$ come un omomorfismo di gruppi
\[\al:G\to \GL(1)=\K^\times.\]
L'insieme dei caratteri $X(G)$ forma un gruppo abeliano:
\[(\al\beta)(g)=\al(g)\beta(g)=\beta(g)\al(g)=(\beta\al)(g).\]
\end{definition}

\begin{theorem}\label{ThAbelianoConnessoCompletamenteRiducibileImplicaToro}
Sia $G$ un gruppo abeliano connesso completamente riducibile, allora $G$ \`e un toro algebrico.
\end{theorem}
\begin{proof}
Notiamo che ogni elemento di $G$ \`e semisemplice: se $g\in G$ allora per la decomposizione di Jordan (\ref{PrDecomposizioneSemisempliceUnipotente}) si ha $g=su$ con $u$ unipotente in $G$. Poich\'e $G$ \`e abeliano, $\ps{u}$ \`e un suo sottogruppo normale, quindi per la proposizione sopra (\ref{PrCompletamenteRiducibileNonHaSottogruppiNormaliUnipotenti}) si ha che $\ps{u}=\cpa{1_G}$, cio\`e $u=1_G$. Dunque $g=s1_G=s$, cio\`e $g$ \`e semisemplice.

Poich\'e $G$ \`e abeliano, gli elementi commutano. Dato che ogni elemento \`e semisemplice (cio\`e in ogni rappresentazione \`e diagonalizzabile), si ha che per ogni rappresentazione esiste una base di autovettori dove ogni elemento di $G$ \`e \emph{simultaneamente} diagonalizzabile. In particolare le rappresentazioni irriducibili hanno dimensione $1$.
\smallskip

Consideriamo allora una decomposizione di $\K[G]$ che rende ogni elemento di $G$ diagonalizzabile
\[\K[G]=\bigoplus \K^{n_\al}_\al\]
dove $\K_\al^{n_\al}$ sono le funzioni regolari $f$ tali che $gf=\al(g)f$. Notiamo che 
\[\al(gh)f=(gh)f=g(hf)=g(\al(h)f)=\al(h)gf=\al(h)\al(g)f\]
quindi $\al(gh)=\al(g)\al(h)$. Questo ci permette di identificare questa decomposizione con
\[\K[G]=\bigoplus_{\al\in X(G)}V_\al,\qquad\text{dove }V_\al=\cpa{h\in \K[G]\mid gh=\al(g)h}.\]

Per ogni carattere $\al\in X(G)$ definiamo $f_\al=\al\ii\in \Hom(G,\K^\times)\subseteq\Hom(G,\K)=\K[G]$. Sfruttando il fatto che $\al$ \`e un omomorfismo si ha $f_\al\in V_\al$, infatti
\begin{align*}
    (gf_\al)(x)=&f_\al(g\ii x)=\al\ii(g\ii x)=(\al(g\ii x))\ii=\\
    =&(\al(g)\ii \al(x))\ii=(\al(x))\ii\al(g)=\\
    =&\al(g) \al\ii(x)=\\
    =&\al(g)f_\al(x).
\end{align*}
Se $h$ ha carattere $\al$, cio\`e $gh=\al(g)h$, allora per ogni $g\in G$
\[\frac{h(1_G)}{f_\al(1_G)}=\frac{\al(g\ii)h(1_G)}{\al(g\ii)f_\al(1_G)}=\frac{g\ii h(1_G)}{g\ii f_\al(1_G)}=\frac{h(g1_G)}{f_\al(g1_G)}=\frac{h(g)}{f_\al(g)},\]
cio\`e $h$ \`e un multiplo di $f_\al$ (in particolare $n_\al=1$ nella scrittura sopra).
\smallskip

Mostriamo che $X(G)\cong \Z^n$ per qualche $n$ mostrando che \`e un gruppo abeliano finitamente generato libero da torsione:
\setlength{\leftmargini}{0cm}
\begin{itemize}
\item[$\boxed{\text{fin.gen.}}$] Sappiamo che $\K[G]$ \`e una $\K$-algebra finitamente generata quindi consideriamo dei generatori $f_{\al_1},\cdots, f_{\al_m}$. Come $\K$-spazio vettoriale, $\K[G]$ \`e generato da elementi della forma
\[f_{\al_1}^{n_1}\cdots f_{\al_m}^{n_m},\]
il quale ha carattere $\prod \al_i^{n_i}$. Questo mostra che i caratteri $\al_1,\cdots, \al_m$ sono dei generatori di $X(G)$.
\item[$\boxed{\text{tor.free}}$] EPer assurdo supponiamo $\al^N=1$ ma $\al\neq 1$, cio\`e $\al(g)^N=1$ per ogni $g\in G$. Notiamo che
\[G=\coprod_{\omega\ t.c.\ \omega^N=1}\cpa{g\mid \al(g)=\omega},\]
ma poich\'e $G$ \`e connesso, questa unione disgiunta deve consistere di un solo termine, mostrando che $\al$ assume solo il valore $1$ contraddicendo le ipotesi.
\end{itemize}
\setlength{\leftmargini}{0.5cm}
Abbiamo quindi mostrato che $X(G)\cong \Z^n$. Sia $\al_1,\cdots, \al_n$ una sua base. Se scriviamo $x_i=f_{\al_i}$ troviamo
\[\K[G]=\bigoplus \ps{f_\al}_\K=\bigoplus \ps{x_1^{m_1}\cdots x_n^{m_n}}_\K\]
e in questa decomposizione il prodotto \`e esattamente quello che ci aspetteremmo.
Se scriviamo $\al=\al_1^{m_1}\cdots\al_n^{m_n}$ allora $f_\al=x_1^{m_1}\cdots x_n^{m_n}$, dunque abbiamo proprio mostrato che
\[\K[G]=\K[x_1^{\pm1},\cdots, x_n^{\pm 1}]=\K[(\K^\times)^n].\]
Essendo sia $G$ che $(\K^\times)^n$ affini questo mostra che sono isomorfi.
\end{proof}




