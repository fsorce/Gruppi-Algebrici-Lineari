\chapter*{Introduzione}

\section*{Di cosa stiamo parlando?}
Un \textbf{gruppo algebrico lineare} \`e un sottogruppo di $\GL(n)$ definito dall'annullarsi di equazioni polinomiali.

Vogliamo studiare questi gruppi e le loro rappresentazioni.


\begin{example}
Il gruppo $\SL(2,\C)$ agisce su $\C^2$, e quindi anche sulle funzioni definite su $\C^2$, infatti se $f:\C^2\to X$ abbiamo una azione
\[g(f)(v)=f(g\ii v)\]
In particolare notiamo che porta funzioni polinomiali in funzioni polinomiali, in quanto
\[\mat{a&b\\c&d}(x)=dx-by,\quad\mat{a&b\\c&d}(y)=-cx+ay.\]
Notiamo anche che preserva il grado in quanto manda polinomi lineari in lineari.

\textbf{Domanda:} come funzionano le orbite di questa azione sugli spazi omogenei?
\[\C[x,y]_d=\ps{x^d,x^{d-1}y,\cdots,y^d}_\C\]

Consideriamo per esempio $V_2=\C[x,y]_2=\ps{x^2,xy,y^2}$. I suoi elementi sono $\alpha x^2+\beta xy+\gamma y^2$. Segue che $\C[\al,\beta,\gamma]$ sono le funzioni polinomiali su $V_2$. Possiamo classificare le orbite in termini dell'invariante $\Delta=\al\gamma-4\beta^2$.
\end{example}