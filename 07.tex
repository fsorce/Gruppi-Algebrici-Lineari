\chapter{Fibrati vettoriali}

\begin{definition}
    Un \textbf{fibrato vettoriale} di rango $m$ su una varietà $X$ è una mappa $p\colon E \to X$ che soddisfa le seguenti proprietà. Lo spazio 
    \[E \times_X E \coloneqq \left\{(a,b)\in E \times E \mid p(a)=p(b)\right\}\] 
    munito di una mappa somma $+\colon E\times_X E \to E$ e di un prodotto per scalare $\cdot \colon \K\times_X E \to E$ tali che, detta $q\colon E \times_X E \to X$ la proiezione $q(a,b)\coloneqq p(a)=p(b)$, i seguenti diagrammi commutano 
    % https://q.uiver.app/#q=WzAsNixbMCwwLCIgRVxcdGltZXNfWCBFIl0sWzIsMCwiRSJdLFsxLDEsIlgiXSxbNCwwLCIgXFxLXFx0aW1lc19YIEUiXSxbNiwwLCJFIl0sWzUsMSwiWCJdLFsxLDIsInAiXSxbMCwxLCIrIl0sWzAsMiwicSIsMl0sWzMsNCwiXFxjZG90Il0sWzMsNSwicCIsMl0sWzQsNSwicCJdXQ==
\[\begin{tikzcd}
	{ E\times_X E} && E && { \K\times_X E} && E \\
	& X &&&& X
	\arrow["{+}", from=1-1, to=1-3]
	\arrow["q"', from=1-1, to=2-2]
	\arrow["p", from=1-3, to=2-2]
	\arrow["\cdot", from=1-5, to=1-7]
	\arrow["p"', from=1-5, to=2-6]
	\arrow["p", from=1-7, to=2-6]
\end{tikzcd}\]

Inoltre vale una condizione di trivializzazione locale, cio\`e per ogni $x$ in $X$ esiste un intorno $U$ di $x$ e una mappa lineare sulle fibre tale che 
% https://q.uiver.app/#q=WzAsMyxbMCwwLCJwXnstMX0oVSkiXSxbMiwwLCJVXFx0aW1lcyBcXEtebSJdLFsxLDEsIlUiXSxbMCwxLCJcXHNpbSIsMl0sWzAsMiwicCIsMl0sWzEsMiwiXFxwaV9VIl0sWzAsMSwiXFx2YXJwaGkiXV0=
\[\begin{tikzcd}
	{p^{-1}(U)} && {U\times \K^m} \\
	& U
	\arrow["\sim"', from=1-1, to=1-3]
	\arrow["\varphi", from=1-1, to=1-3]
	\arrow["p"', from=1-1, to=2-2]
	\arrow["{\pi_U}", from=1-3, to=2-2]
\end{tikzcd}\]
\end{definition}

\begin{example}
    $X\times \K^m$ con la mappa di proiezione su $X$ \`e un fibrato vettoriale, detto \textbf{fibrato banale}. 
\end{example}

\begin{example}
    Sia $X=\Pj^1=\Pj(\K^2)$ e consideriamo 
    \[E=\{(\ell,v)\in \Pj^1\times \K^2 \colon v\in \ell \} \xrightarrow{p}\Pj^1\] 
    dove la somma è data da $(\ell,v)+(\ell,w)=(\ell,v+w)$. Questo \`e detto il \textbf{fibrato tautologico}.
\end{example}

\begin{exercise} Consideriamo la coppia $(E,p)$ dell'esempio precedente. Mostrare che
    \begin{enumerate}
        \item È un fibrato vettoriale. 
        \item Non è il fibrato banale\footnote{Una intuizione geometrica importante (seppur lontana da essere una dimostrazione) che giustifica la non banalit\`a \`e che il fibrato banale di rango 1 ``topologicamente" \`e un cilindro mentre il fibrato tautologico \`e un nastro di M\"obius.}.
    \end{enumerate}
\end{exercise}

\section{Cocicli e banalizzazioni}
\begin{remark}
Se $p:E\to X$ \`e un fibrato vettoriale su $X$ di rango $m$ allora esiste un ricoprimento (finito) $\cpa{U_i}$ di $X$ tale che per ogni $i$
% https://q.uiver.app/#q=WzAsMyxbMCwwLCJwXnstMX0oVV9pKSJdLFsyLDAsIlVfaVxcdGltZXMgXFxLXm0iXSxbMSwxLCJVX2kiXSxbMCwxLCJcXHNpbSIsMl0sWzAsMiwicCIsMl0sWzEsMiwiXFxwaV97VV9pfSJdLFswLDEsIlxcdmFycGhpX2kiXV0=
\[\begin{tikzcd}
	{p^{-1}(U_i)} && {U_i\times \K^m} \\
	& {U_i}
	\arrow["\sim"', from=1-1, to=1-3]
	\arrow["{\varphi_i}", from=1-1, to=1-3]
	\arrow["p"', from=1-1, to=2-2]
	\arrow["{\pi_{U_i}}", from=1-3, to=2-2]
\end{tikzcd}\]
\end{remark}

\begin{definition}[Cocicli]
Dato un ricoprimento $\Uc=\cpa{U_i}$ di $X$ definiamo gli \textbf{1-cocicli} relativi al ricoprimento come
\[Z^1(\Uc,\GL(m))=\cpa{\cpa{\al_{i,j}}\mid \al_{i,j}:U_{i,j}\to\GL(m),\ \al_{i,h}=\al_{j,h}\al_{i,j},\ \al_{i,i}(u)=I_m}.\]
\end{definition}
\begin{proposition}
Se $\Uc$ \`e un ricoprimento banalizzante per il fibrato $p:E\to X$ allora esso definisce dei cocili indotti dalle mappe di transizione.
\end{proposition}
\begin{proof}
Se $U_{i,j}=U_i\cap U_j$ allora abbiamo degli isomorfismi\footnote{da ora in poi eviteremo di scrivere ogni volta le restrizioni}
\[\vp_{i,j}=\vp_j\res{U_{i,j}}\circ \vp_i\res{U_{i,j}}\ii:U_{i,j}\times\K^m\to U_{i,j}\times \K^m\]
Sia $\vp_{i,j}(u,v)=(u,\al_{i,j}(u)(v))$. Questo definisce delle mappe regolari
\[\al_{i,j}:U_{i,j}\to \GL(m).\]
Notiamo inoltre che su $U_{i,j,h}\times \K^m$
\[\vp_{j,h}\circ \vp_{i,j}=\vp_h\circ \vp_{j}\ii\circ \vp_j\circ \vp_i\ii=\vp_{i,h}.\]
Vedendo queste identit\`a in termini delle $\al_{i,j}$ abbiamo mostrato
\[\al_{i,h}(u)=\al_{j,h}(u)\al_{i,j}(u).\]
Chiaramente $\vp_i\circ \vp_i\ii=id_{U_i\times\K^m}$ quindi $\al_{i,i}(u)=I_m$.
\end{proof}

\begin{proposition}[]\label{PrCocicliDeterminanoFibrato}
Sia $\Uc$ un ricoprimento di $X$. Una collezione di cocicli $\al\in Z^1(\Uc,\GL(m))$ determina un fibrato vettoriale su $X$.
\end{proposition}
\begin{proof}
Definiamo lo spazio totale come il seguente coprodotto fibrato
\[E_\al:=\frac{\coprod(U_i\times \K^m)}{(u,v)\sim (u,\al_{i,j}(u)(v))}.\]
Siano $\psi_i:U_i\times\K^m\to E_\al$ le mappe ovvie verso il coprodotto.
\smallskip

Su $E_\al$ imponiamo la seguente topologia: $V$ \`e aperto in $E_\al$ se e solo se $\psi_i\ii(V)$ \`e aperto in $U_i\times\K^m$ per ogni $i$.

\smallskip

Se $V$ \`e aperto in $E_\al$ allora, poich\'e la regolarit\`a \`e una condizione locale, si ha
\[f\in \Oc_{E_\al}(V)\coimplies f\circ \psi_i\in \Oc_{U_i}(\psi_i\ii(V)) \quad\forall i.\]
Affermiamo che $\psi_i$ \`e iniettiva, aperta e regolare per ogni $i$:
\setlength{\leftmargini}{0cm}
\begin{itemize}
\item[$\boxed{\text{Iniettiva}}$] Supponiamo $\psi_i(u,v)=\psi_i(u',v')$, allora
\[U_i\times \K^m\ni(u,v)\sim (u',v')\in U_i\times \K^m\overset{\al_{i,i}=id_{U_i}}\implies u=u',\ v=v'.\]
\item[$\boxed{\text{Aperta}}$] Se $W\subseteq U_i\times\K^m$ aperto e $\wt W=\psi_i(W)\subseteq E_\al$ allora dobbiamo verificare che $\psi_j\ii(\wt W)$ \`e aperto per ogni $j$. Consideriamo la seguente catena di equivalenze:
\begin{gather*}
(u,v)\in \psi_j\ii(\wt W)\subseteq U_j\times\K^m\\
\psi_j(u,v)\in \wt W=\psi_i(W)\\
\exists (u',v')\in W\subseteq U_i\times\K^m\ t.c.\ \psi_j(u,v)=\psi_i(u',v')\text{ in }E_\al\\
\exists (u',v')\in W\ t.c.\ U_j\times \K^m\ni(u,v)\sim (u',v')\in U_i\times \K^m\\
(u,\al_{j,i}(u)(v))\in W
\end{gather*}
cio\`e $\psi_j\ii(\wt W)$ \`e la preimmagine di $W$ tramite la mappa $U_{i,j}\times \K^m\to U_i\times \K^m$ che \`e continua.
\item[$\boxed{\text{Regolare}}$] $\psi_i$ \`e continua per definizione della topologia di $E_\al$. Essendo la regolarit\`a una condizione locale, per come abbiamo definito la topologia su $E_\al$ anche la condizione sui pullback \`e chiara.
\end{itemize}
\setlength{\leftmargini}{0.5cm}

Abbiamo dunque mostrato che $\psi_i(U_i\times \K^m)\cong U_i\times\K^m$.
\medskip

\noindent
La struttura vettoriale sulle fibre di $E_\al$ \`e quella indotta dal coprodotto in modo ovvio.
\end{proof}


\begin{remark}
Se $U_i$ \`e un aperto di $X$ che banalizza $p:E\to X$ notiamo che possono essere definiti diversi isomorfismi tra $p\ii(U_i)$ e $U_i\times \K^m$
% https://q.uiver.app/#q=WzAsMyxbMCwwLCJwXnstMX0oVV9pKSJdLFsyLDAsIlVfaVxcdGltZXMgXFxLXm0iXSxbMSwxLCJVX2kiXSxbMCwxLCJcXHd0XFx2cF9pIiwyLHsiY3VydmUiOjF9XSxbMCwyLCJwIiwyXSxbMSwyLCJcXHBpX3tVX2l9Il0sWzAsMSwiXFx2YXJwaGlfaSIsMCx7ImN1cnZlIjotMX1dXQ==
\[\begin{tikzcd}
	{p^{-1}(U_i)} && {U_i\times \K^m} \\
	& {U_i}
	\arrow["{\wt\vp_i}"', curve={height=6pt}, from=1-1, to=1-3]
	\arrow["{\varphi_i}", curve={height=-6pt}, from=1-1, to=1-3]
	\arrow["p"', from=1-1, to=2-2]
	\arrow["{\pi_{U_i}}", from=1-3, to=2-2]
\end{tikzcd}\]
In particolare questi definiscono diverse mappe di transizione e diversi cocicli.


Studiamo la ``differenza" tra le mappe di transizione 
\[\chi_i:=\wt\vp_i\circ \vp_i\ii:U_i\times\K^m\to U_i\times\K^m\] 
Come notazione scriviamo $\chi_i(u,v)=(u,\beta_i(u)(v))$.
Notiamo che
\[\wt\vp_{i,j}=\wt\vp_j\circ \wt\vp_i\ii=\wt\vp_j\circ\vp_j\ii\circ\vp_j\circ\vp_i\ii\circ\vp_i\circ \wt\vp_i\ii=\chi_j\circ\vp_{i,j}\circ\chi_i\ii,\]
quindi in termini di cocicli $\wt \al_{i,j}=\beta_j\al_{i,j}\beta_i\ii$ o equivalentemente $\wt\al_{i,j}\beta_i=\beta_j\al_{i,j}$.
\end{remark}

Questo ci spinge a dare la seguente definizione:
\begin{definition}[Coomologia]
Definiamo due cocicli $\al_{i,j}$ e $\wt\al_{i,j}$ \textit{equivalenti} se esiste $\beta_i:U_i\to \GL(m)$ tale che $\wt \al_{i,j}=\beta_j\al_{i,j}\beta_i\ii$. Definiamo il \textbf{primo insieme di coomologia} come
\[H^1(\Uc,\GL(m))=\quot{Z^1(\Uc,\GL(m))}\sim.\]
\end{definition}


\begin{theorem}[]\label{ThCorrispondenzaFibratiECoomologia}
Se $\Uc$ \`e un ricoprimento aperto di $X$ allora abbiamo una corrispondenza biunivoca
\[\quot{\cpa{\text{Fibrati vett. che si banalizzano su $\Uc$}}}{\text{iso. di fibr.}}\longleftarrow H^1(\Uc,\GL(m)).\]
\end{theorem}
\begin{proof}
Ad un fibrato associo una collezione di cocicli e quindi una classe in coomologia. Questa associazione \`e ben definita perch\'e cambiando cocicli associati alla stessa banalizzazione essi sono equivalenti.

Cambiare la classe di isomorfismo del fibrato non cambia la classe di coomologia perch\'e l'isomorfismo sugli aperti banalizzanti induce un automorfismo che corrisponde a cambiare cocicli.

\bigskip
\noindent Viceversa, dati dei coclicli sappiamo costruire un fibrato come in (\ref{PrCocicliDeterminanoFibrato}). Se $\al$ e $\wt \al$ sono cocicli equivalenti allora mostriamo che $E_\al\cong E_{\wt\al}$ come fibrati. Se $\beta_i$ sono tali che $\wt \al_{i,j}=\beta_i\al_{i,j}\beta_i\ii$ allora abbiamo una ovvia mappa tra i fibrati
% https://q.uiver.app/#q=WzAsMixbMCwwLCJcXGRmcmFje1xcY29wcm9kKFVfaVxcdGltZXMgXFxLXm0pfXtcXHNpbV9cXGFsfSJdLFszLDAsIlxcZGZyYWN7XFxjb3Byb2QoVV9pXFx0aW1lcyBcXEtebSl9e1xcc2ltX3tcXHd0XFxhbH19Il0sWzAsMSwiXFxjb3Byb2QgaWRfe1VfaX1cXHRpbWVzIFxcYmV0YV9pKHUpKHYpIl1d
\[\begin{tikzcd}
	{\dfrac{\coprod(U_i\times \K^m)}{\sim_\al}} &&& {\dfrac{\coprod(U_i\times \K^m)}{\sim_{\wt\al}}}
	\arrow["{\coprod id_{U_i}\times \beta_i(u)(v)}", from=1-1, to=1-4]
\end{tikzcd}\]
che si verifica essere un isomorfismo (Esercizio).
\end{proof}

\section{Sezioni di un fibrato}
\begin{definition}[Sezioni di un fibrato]
Dato un fibrato vettoriale $p\colon E \to X$ e dato un aperto $U$ di $X$, definiamo le \textbf{sezioni} di $U$ come 
\[\Gamma(U,E)\coloneqq \{\sigma\colon U \to E \mid p\circ \sigma = id_U\}.\]
\end{definition}
\begin{remark}
    $\Gamma(U,E)$ eredita la somma dal fibrato:
    \[(\sigma+\tau)(u)=\sigma(u)+\tau(u)\]
    dove il membro di destra \`e l'immagine della coppia $(\sigma(u),\tau(u))\in E\times_X E$ tramite la mappa $+$.
    Se $f$ è in $\Oc_X(U)$ e $\sigma\in\Gamma(U,E)$, allora è definito $f\cdot\sigma$ in modo analogo.
    \medskip

    \noindent Insieme questi fatti mostrano che $\Gamma(U,E)$ è un $\Oc_X(U)$-modulo.
\end{remark}

\begin{example}[Sezioni del fibrato banale]
    % https://q.uiver.app/#q=WzAsNixbMCwwLCJYXFx0aW1lcyBcXEtebSJdLFswLDIsIlgiXSxbMiwwLCJVXFx0aW1lcyBcXEtebSJdLFsyLDIsIlUiXSxbMSwyLCJcXHN1cHNldGVxIl0sWzEsMCwiXFxzdXBzZXRlcSJdLFswLDFdLFsyLDMsInAiLDJdLFszLDIsIiIsMCx7ImN1cnZlIjozfV1d
\[\begin{tikzcd}
	{X\times \K^m} & \supseteq & {U\times \K^m} \\
	\\
	X & \supseteq & U
	\arrow[from=1-1, to=3-1]
	\arrow["p"', from=1-3, to=3-3]
	\arrow[curve={height=18pt}, from=3-3, to=1-3]
\end{tikzcd}\]
    Consideriamo $\sigma(u)=(u,\alpha_1(u),\ldots,\alpha_m(u))$, dove $\alpha_i\colon U \to \K$. Allora 
    \[\Gamma(U,U\times \K^m)\cong \Oc_X(U)^m.\]
\end{example}

Studiamo ora le sezioni di un fibrato qualsiasi.

\begin{proposition}\label{PrSezioniDiFibrato}
Siano $p:E\to X$ un fibrato vettoriale e $\Uc$ un ricoprimento banalizzante per questo.
Dare $\sigma\in \Gamma(X,E)$ \`e equivalente a dare delle $\sigma_i:U_i\to \K^m$ per ogni aperto del ricoprimento che verificano
\begin{center}
	\begin{tabular}[width=\pagewidth]{ccr}
		~\hspace{4.3cm}~&$\sigma_j=\al_{i,j}\sigma_i$ &\hspace{1cm}(\textbf{condizione di cociclo})
	\end{tabular} 
\end{center}
dove $\al_{i,j}$ sono dei cocicli per il fibrato.
\end{proposition}
\begin{proof}
Fissiamo $\vp_i:E\res{U_i}\to U_i\times \K^m$ e $\al_{i,j}$ come prima. Se $\sigma\in\Gamma(X,E)$ allora
\[\wt \sigma_i=\sigma\res{U_i}:U_i\to E\res{U_i},\quad \sigma_i:=\pi_{\K^m}\circ\vp_i\circ\wt\sigma_i\]
\[\vp_i\circ \wt\sigma_i:\funcDef{U_i}{U_i\times\K^m}{u}{(u,\sigma_i(u))}\]
Su $U_{i,j}$ abbiamo
\[\wt \sigma_j=\vp_{i,j}\circ \wt \sigma_i,\quad \sigma_j=\al_{i,j}\sigma_i\]
quindi da $\sigma$ abbiamo trovato delle $\sigma_i:U_i\to \K^m$ che verificano $\sigma_j=\al_{i,j}\sigma_i$.

Viceversa dati questi dati essi si possono incollare a $\sigma:X\to E$ perch\'e la condizione di cocliclo garantisce che le $\sigma_i$ coincidano sulle intersezioni degli aperti del ricoprimento.
\end{proof}


\subsection{Fibrati lineari sulla retta proiettiva}
\begin{notation}
Scriviamo $\A^1\cong U_0=\Pj^1\bs\cpa\infty$ e $\A^1\cong U_\infty=\Pj^1\nz$, da cui
\[\A^1\nz\cong U_{0,\infty}=\Pj^1\bs\cpa{0,\infty}.\]
Scriviamo dunque $\K[U_0]=\K[z]$, $\K[U_\infty]=\K[w]$ e $\K[U_{0,\infty}]=\K[z,z\ii]$ con $w=z\ii$.
\end{notation}

\begin{remark}
Rispetto a questo ricoprimento i coclicli sono
\[Z^1(\Uc)=\cpa{\al_{0,\infty}:U_{0,\infty}\to\GL(1)}=\cpa{\funcDef{U_{0,\infty}}{\GL(1)}{z}{\la z^n}\mid n\in\Z,\ \la\in\K^\times}\]
A priori sono mappe della forma $z\mapsto p(z)$ ma se comparissero pi\`u monomi allora ci sarebbero radici diverse da $0$, quindi esisterebbe $a\in U_{0,\infty}$ tale che $p(a)=0\notin \GL(1)$.
\end{remark}



\begin{remark}
Si ha
\[H^1(\Uc,\GL(1))=\cpa{z^n}_{n\in\Z}.\]
\end{remark}
\begin{proof}
Se $\al=\la z^n$ e $\wt \al=\mu z^m$ sono cocicli equivalenti allora esistono $\beta_0:U_0\to \GL(1)$ e $\beta_\infty:U_{\infty}\to\GL(1)$ tali che
\[\mu z^m=\wt\al=\beta_\infty\al\beta_0\ii=\la\beta_\infty\beta_0\ii z^n\implies n=m\]
Se $n=m$ allora possiamo sempre trovare $\beta_0$ e $\beta_\infty$, quindi abbiamo mostrato
\[H^1(\Uc,\GL(1))=\cpa{z^n}_{n\in\Z}.\]
\end{proof}

\begin{proposition}
Le sezioni globali dei fibrati lineari su $\Pj^1$ sono
\[\Gamma(\Pj^1,E_{z^{-n}})=\begin{cases}
	\K[z]_{n} & n\geq 0\\
	0 &n<0
\end{cases}\]
\end{proposition}
\begin{proof}
Consideriamo due sezioni $\sigma_0=f(z)$ e $\sigma_\infty=g(w)$ tali che $\sigma_\infty=\al_{0,\infty}\sigma_0$. Per quanto detto possiamo scegliere $\al_{0,\infty}=z^{-n}$, da cui sull'intersezione $U_{0,\infty}$ abbiamo 
\[g(z\ii)=z^{-n}f(z).\]
Scrivendo le espansioni troviamo
\[g(z\ii)=a_0+a_1 z\ii+\cdots\qquad z^{-n}f(z)=b_0 z^{-n}+b_1 z^{1-n}+\cdots\]
dunque:
\setlength{\leftmargini}{0cm}
\begin{itemize}
\item[$\boxed{n<0}$] In questo caso imporre uguaglianza impone che ogni coefficiente sia nullo, cio\`e $\sigma_0=\sigma_\infty=0$ e quindi il $\sigma$ a cui corrispondono \`e $0$.
\item[$\boxed{n\geq0}$] Imponendo l'uguaglianza troviamo
\[f(z)=a_0z^n+\cdots+a_{n-1}z+a_n\]
cio\`e un generico polinomio di grado $n$.
\end{itemize}
\setlength{\leftmargini}{0.5cm}
\end{proof}



\subsection{Fibrati vettoriali su variet\`a affini}

\begin{remark}
Se $X$ \`e una variet\`a affine, $A=\K[X]$ e $E$ \`e un fibrato vettoriale di rango $m$ su $X$ allora $M=\Gamma(X,E)$ \`e un $A$-modulo.
\end{remark}

\begin{remark}
Supponiamo che $E$ si banalizzi su aperti affini $X_i$ tali che\footnote{ricordiamo che aperti di questo tipo formano una base per la topologia di Zariski.} $\K[X_i]=A_{f_i}$. Siano $\al_{i,j}$ i cocicli e consideriamo la mappa
\[J:\funcDef{\bigoplus_i (A_{f_i})^m}{\bigoplus_{i,j} (A_{f_if_j})^m}{(\sigma_i)}{(\sigma_j-\al_{i,j}\sigma_i)}\]
Per quanto detto $\Gamma(X,E)=\ker J$.
\end{remark}

\begin{proposition}
$\Gamma(X_f,E)=\Gamma(X,E)_f$
\end{proposition}
\begin{proof}
Su $X_f=U\subseteq X$ abbiamo 
% https://q.uiver.app/#q=WzAsMTIsWzAsMCwiMCJdLFsxLDAsIlxcR2FtbWEoWCxFKSJdLFsxLDEsIlxcR2FtbWEoVSxFKSJdLFswLDEsIjAiXSxbMiwwLCJcXGJpZ29wbHVzXFxLW1VfaV1ebSJdLFsyLDEsIlxcYmlnb3BsdXNcXEtbVV9pXFxjYXAgVV1ebSJdLFszLDAsIlxcYmlnb3BsdXNcXEtbVV97aSxqfV1ebSJdLFszLDEsIlxcYmlnb3BsdXNcXEtbVV97aSxqfVxcY2FwIFVdXm0iXSxbMiwyLCJcXGJpZ29wbHVzXFxLW1VfaV1ebV9mIl0sWzMsMiwiXFxiaWdvcGx1c1xcS1tVX3tpLGp9XV5tX2YiXSxbMSwyLCJcXEdhbW1hKFgsRSlfZiJdLFswLDIsIjAiXSxbMTEsMTBdLFsxMCw4XSxbOCw5XSxbMywyXSxbMiw1XSxbNSw3XSxbMCwxXSxbMSw0XSxbNCw2XSxbNCw1XSxbNiw3XSxbNSw4LCI9IiwzLHsic3R5bGUiOnsiYm9keSI6eyJuYW1lIjoibm9uZSJ9LCJoZWFkIjp7Im5hbWUiOiJub25lIn19fV0sWzcsOSwiPSIsMyx7InN0eWxlIjp7ImJvZHkiOnsibmFtZSI6Im5vbmUifSwiaGVhZCI6eyJuYW1lIjoibm9uZSJ9fX1dLFsxMCwyLCI9IiwzLHsic3R5bGUiOnsiYm9keSI6eyJuYW1lIjoibm9uZSJ9LCJoZWFkIjp7Im5hbWUiOiJub25lIn19fV1d
\[\begin{tikzcd}
	0 & {\Gamma(X,E)} & {\bigoplus\K[U_i]^m} & {\bigoplus\K[U_{i,j}]^m} \\
	0 & {\Gamma(U,E)} & {\bigoplus\K[U_i\cap U]^m} & {\bigoplus\K[U_{i,j}\cap U]^m} \\
	0 & {\Gamma(X,E)_f} & {\bigoplus\K[U_i]^m_f} & {\bigoplus\K[U_{i,j}]^m_f}
	\arrow[from=1-1, to=1-2]
	\arrow[from=1-2, to=1-3]
	\arrow[from=1-3, to=1-4]
	\arrow[from=1-3, to=2-3]
	\arrow[from=1-4, to=2-4]
	\arrow[from=2-1, to=2-2]
	\arrow[from=2-2, to=2-3]
	\arrow[from=2-3, to=2-4]
	\arrow["{=}"{marking, allow upside down}, draw=none, from=2-3, to=3-3]
	\arrow["{=}"{marking, allow upside down}, draw=none, from=2-4, to=3-4]
	\arrow[from=3-1, to=3-2]
	\arrow["{=}"{marking, allow upside down}, draw=none, from=3-2, to=2-2]
	\arrow[from=3-2, to=3-3]
	\arrow[from=3-3, to=3-4]
\end{tikzcd}\]
\end{proof}
\begin{corollary}
Se $E$ su $X_f$ \`e banale allora
\[\Gamma(X,E)_f\cong \K[X_f]^m=(A_f)^m.\]
\end{corollary}


\begin{remark}
Il modulo $M=\Gamma(X,E)$ \`e localmente libero, cio\`e per ogni $x\in X$ esiste $f\in A$ tale che $f(x)\neq 0$ e $M_f\cong (A_f)^m$.
\end{remark}




