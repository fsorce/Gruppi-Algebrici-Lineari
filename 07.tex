\chapter{Fibrati vettoriali}

\section{Fibrati vettoriali}
\begin{definition}
    Un \textbf{fibrato vettoriale} di rango $m$ su una varietà $X$ è una mappa $p\colon E \to X$ che soddisfa le seguenti proprietà. Lo spazio 
    \[E \times_X E \coloneqq \left\{(a,b)\in E \times E \mid p(a)=p(b)\right\}\] 
    munito di una mappa somma $+\colon E\times_X E \to E$ e di un prodotto per scalare $\cdot \colon \K\times_X E \to E$ tali che, detta $q\colon E \times_X E \to X$ la proiezione $q(a,b)\coloneqq p(a)=p(b)$, i seguenti diagrammi commutano 
    % https://q.uiver.app/#q=WzAsNixbMCwwLCIgRVxcdGltZXNfWCBFIl0sWzIsMCwiRSJdLFsxLDEsIlgiXSxbNCwwLCIgXFxLXFx0aW1lc19YIEUiXSxbNiwwLCJFIl0sWzUsMSwiWCJdLFsxLDIsInAiXSxbMCwxLCIrIl0sWzAsMiwicSIsMl0sWzMsNCwiXFxjZG90Il0sWzMsNSwicCIsMl0sWzQsNSwicCJdXQ==
\[\begin{tikzcd}
	{ E\times_X E} && E && { \K\times_X E} && E \\
	& X &&&& X
	\arrow["{+}", from=1-1, to=1-3]
	\arrow["q"', from=1-1, to=2-2]
	\arrow["p", from=1-3, to=2-2]
	\arrow["\cdot", from=1-5, to=1-7]
	\arrow["p"', from=1-5, to=2-6]
	\arrow["p", from=1-7, to=2-6]
\end{tikzcd}\]

Inoltre vale una condizione di trivializzazione locale, cio\`e per ogni $x$ in $X$ esiste un intorno $U$ di $x$ e una mappa lineare sulle fibre tale che 
% https://q.uiver.app/#q=WzAsMyxbMCwwLCJwXnstMX0oVSkiXSxbMiwwLCJVXFx0aW1lcyBcXEtebSJdLFsxLDEsIlUiXSxbMCwxLCJcXHNpbSIsMl0sWzAsMiwicCIsMl0sWzEsMiwiXFxwaV9VIl0sWzAsMSwiXFx2YXJwaGkiXV0=
\[\begin{tikzcd}
	{p^{-1}(U)} && {U\times \K^m} \\
	& U
	\arrow["\sim"', from=1-1, to=1-3]
	\arrow["\varphi", from=1-1, to=1-3]
	\arrow["p"', from=1-1, to=2-2]
	\arrow["{\pi_U}", from=1-3, to=2-2]
\end{tikzcd}\]
\end{definition}

\begin{example}
    $X\times \K^m$ con la mappa di proiezione su $X$ \`e un fibrato vettoriale, detto \textbf{fibrato banale}. 
\end{example}

\begin{example}
    Sia $X=\Pj^1=\Pj(\K^2)$ e consideriamo 
    \[E=\{(\ell,v)\in \Pj^1\times \K^2 \colon v\in \ell \} \xrightarrow{p}\Pj^1\] 
    dove la somma è data da $(\ell,v)+(\ell,w)=(\ell,v+w)$. Questo \`e detto il \textbf{fibrato tautologico}.
\end{example}

\begin{exercise} Consideriamo la coppia $(E,p)$ dell'esempio precedente. Mostrare che
    \begin{enumerate}
        \item È un fibrato vettoriale. 
        \item Non è il fibrato banale\footnote{Una intuizione geometrica importante (seppur lontana da essere una dimostrazione) che giustifica la non banalit\`a \`e che il fibrato banale di rango 1 ``topologicamente" \`e un cilindro mentre il fibrato tautologico \`e un nastro di M\"obius.}.
    \end{enumerate}
\end{exercise}

\begin{definition}[Sezioni di un fibrato]
Dato un fibrato vettoriale $p\colon E \to X$ e dato un aperto $U$ di $X$, definiamo le \textbf{sezioni} di $U$ come 
\[\Gamma(U,E)\coloneqq \{\sigma\colon U \to E \mid p\circ \sigma = id_U\}.\]
\end{definition}
\begin{remark}
    $\Gamma(U,E)$ eredita la somma dal fibrato:
    \[(\sigma+\tau)(u)=\sigma(u)+\tau(u)\]
    dove il membro di destra \`e l'immagine della coppia $(\sigma(u),\tau(u))\in E\times_X E$ tramite la mappa $+$.
    Se $f$ è in $\Oc_X(U)$ e $\sigma\in\Gamma(U,E)$, allora è definito $f\cdot\sigma$ in modo analogo.
    \medskip

    \noindent Insieme questi fatti mostrano che $\Gamma(U,E)$ è un $\Oc_X(U)$-modulo.
\end{remark}

\begin{example}[Sezioni del fibrato banale]
    % https://q.uiver.app/#q=WzAsNixbMCwwLCJYXFx0aW1lcyBcXEtebSJdLFswLDIsIlgiXSxbMiwwLCJVXFx0aW1lcyBcXEtebSJdLFsyLDIsIlUiXSxbMSwyLCJcXHN1cHNldGVxIl0sWzEsMCwiXFxzdXBzZXRlcSJdLFswLDFdLFsyLDMsInAiLDJdLFszLDIsIiIsMCx7ImN1cnZlIjozfV1d
\[\begin{tikzcd}
	{X\times \K^m} & \supseteq & {U\times \K^m} \\
	\\
	X & \supseteq & U
	\arrow[from=1-1, to=3-1]
	\arrow["p"', from=1-3, to=3-3]
	\arrow[curve={height=18pt}, from=3-3, to=1-3]
\end{tikzcd}\]
    Consideriamo $\sigma(u)=(u,\alpha_1(u),\ldots,\alpha_m(u))$, dove $\alpha_i\colon U \to \K$. Allora 
    \[\Gamma(U,U\times \K^m)\cong \Oc_X(U)^m.\]
\end{example}

\section{Fibrati con azione equivariante di un gruppo}

\begin{definition}
    Se $X$ è una $G$-varietà, un \textbf{$G$-fibrato} su $X$ è un fibrato vettoriale $p\colon E \to X$ su $X$ per cui esiste un'azione lineare di $G$ su $E$ tale che $p$ è $G$-equivariante.
\end{definition}
\begin{remark}
In questo caso, $\Gamma(X,E)$ ha un'azione lineare di $G$, data da 
\[(g\cdot \sigma)(x)=g(\sigma(g^{-1}x)).\] 
\end{remark}

Si potrebbe dimostrare che $G$ agisce in modo regolare su $\Gamma(X,E)$, e quindi che $\Gamma(X,E)$ è una rappresentazione algebrica di $G$. (Non lo faremo.) 

\begin{definition}
    Dati due fibrati vettoriali $p_1\colon E_1 \to X$ e $p_2\colon E_2 \to X$ su $X$, un \textbf{morfismo di fibrati} è una mappa $\varphi\colon E_1 \to E_2$ che preserva la somma e il prodotto per scalare e tale che $p_2\circ\varphi=p_1$. Diciamo che $\varphi$ è \textbf{equivariante} se commuta con $G$.
\end{definition}

\begin{remark}
Per ogni $x_0$ in $X$, un morfismo $\varphi$ definisce una mappa lineare $\varphi_{x_0}\colon (E_1)_{x_0}\to (E_2)_{x_0}$. 
\end{remark}

\subsection{Classificazione dei fibrati vettoriali equivarianti}
\begin{proposition}\label{PrClassificazioneFibratiSuGruppoQuoziente}
Le classi di isomorfismo fibrati vettoriali $G$-equivarianti di rango $m$ su $X=G/H$ sono in corrispondenza con le classi di isomorfismo delle rappresentazioni $m$-dimensionali di $H$.
\end{proposition}
\begin{proof}
Costruiamo esplicitamente questa corrispondenza:
\smallskip

\noindent
Dato un fibrato vettoriale $G$-equivariante $p\colon E\to G/H$ e posto $x_0=H=e_{G/H}$, possiamo considerare la fibra $E_{x_0}$ e chiaramente $H$ agisce su $E_{x_0}$. 
\smallskip

\noindent
Data una rappresentazione $V$ di $H$, consideriamo $G\times V$ con l'azione di $H$ data da 
\[h(g,v)=(gh^{-1},hv).\]
Passando al quoziente in $H$ abbiamo
% https://q.uiver.app/#q=WzAsNCxbMCwwLCJFPVxcZGZyYWN7R1xcdGltZXMgVn17SH0iXSxbMCwxLCJbZyx2XSJdLFsxLDAsIlg9e0d9L3tIfSJdLFsxLDEsIltnXSJdLFswLDJdLFsxLDMsIiIsMCx7InN0eWxlIjp7InRhaWwiOnsibmFtZSI6Im1hcHMgdG8ifX19XSxbMSwwLCJcXGluIiwzLHsic3R5bGUiOnsiYm9keSI6eyJuYW1lIjoibm9uZSJ9LCJoZWFkIjp7Im5hbWUiOiJub25lIn19fV0sWzMsMiwiXFxpbiIsMyx7InN0eWxlIjp7ImJvZHkiOnsibmFtZSI6Im5vbmUifSwiaGVhZCI6eyJuYW1lIjoibm9uZSJ9fX1dXQ==
\[\begin{tikzcd}
	{E=\dfrac{G\times V}{H}} & {X={G}/{H}} \\
	{[g,v]} & {[g]}
	\arrow[from=1-1, to=1-2]
	\arrow["\in"{marking, allow upside down}, draw=none, from=2-1, to=1-1]
	\arrow[maps to, from=2-1, to=2-2]
	\arrow["\in"{marking, allow upside down}, draw=none, from=2-2, to=1-2]
\end{tikzcd}\]
Notiamo che 
\[[g,v]+[gh,u]=[g,v]+[g,hu]=[g,v+hu]\] 
e che 
\[\lambda[g,v]=[g,\lambda v],\]
quindi le fibre hanno una struttura vettoriale.

Per concludere andrebbe mostrato che $E$ è una varietà algebrica, che la mappa costruita è regolare e che abbiamo la trivializzazione locale.
\end{proof}



\begin{example}
Se vogliamo costruire fibrati $G$-equivarianti lineari (cioè di rango $1$) su $\GL(n)/B_n$, sappiamo che questi corrispondono alle rappresentazioni $1$-dimensionali di $B_n$ 
% https://q.uiver.app/#q=WzAsMyxbMCwwLCJCX24iXSxbMiwwLCJcXEteXFx0aW1lcyJdLFsxLDEsIlQ9Ql9uL1VfbiJdLFswLDJdLFsyLDFdLFswLDFdXQ==
\[\begin{tikzcd}
	{B_n} && {\K^\times} \\
	& {T=B_n/U_n}
	\arrow[from=1-1, to=1-3]
	\arrow[from=1-1, to=2-2]
	\arrow[from=2-2, to=1-3]
\end{tikzcd}\]
Quindi queste possono essere caratterizzate attraverso i caratteri del toro.
\end{example}


