\chapter{Fibrati vettoriali}

\begin{definition}
    Un \textbf{fibrato vettoriale} di rango $m$ su una varietà $X$ è una mappa $p\colon E \to X$ che soddisfa le seguenti proprietà. Lo spazio 
    \[E \times_X E \coloneqq \left\{(a,b)\in E \times E \mid p(a)=p(b)\right\}\] 
    munito di una mappa somma $+\colon E\times_X E \to E$ e di un prodotto per scalare $\cdot \colon \K\times_X E \to E$ tali che, detta $q\colon E \times_X E \to X$ la proiezione $q(a,b)\coloneqq p(a)=p(b)$, i seguenti diagrammi commutano 
    % https://q.uiver.app/#q=WzAsNixbMCwwLCIgRVxcdGltZXNfWCBFIl0sWzIsMCwiRSJdLFsxLDEsIlgiXSxbNCwwLCIgXFxLXFx0aW1lc19YIEUiXSxbNiwwLCJFIl0sWzUsMSwiWCJdLFsxLDIsInAiXSxbMCwxLCIrIl0sWzAsMiwicSIsMl0sWzMsNCwiXFxjZG90Il0sWzMsNSwicCIsMl0sWzQsNSwicCJdXQ==
\[\begin{tikzcd}
	{ E\times_X E} && E && { \K\times_X E} && E \\
	& X &&&& X
	\arrow["{+}", from=1-1, to=1-3]
	\arrow["q"', from=1-1, to=2-2]
	\arrow["p", from=1-3, to=2-2]
	\arrow["\cdot", from=1-5, to=1-7]
	\arrow["p"', from=1-5, to=2-6]
	\arrow["p", from=1-7, to=2-6]
\end{tikzcd}\]

Inoltre vale una condizione di trivializzazione locale, cio\`e per ogni $x$ in $X$ esiste un intorno $U$ di $x$ e una mappa lineare sulle fibre tale che 
% https://q.uiver.app/#q=WzAsMyxbMCwwLCJwXnstMX0oVSkiXSxbMiwwLCJVXFx0aW1lcyBcXEtebSJdLFsxLDEsIlUiXSxbMCwxLCJcXHNpbSIsMl0sWzAsMiwicCIsMl0sWzEsMiwiXFxwaV9VIl0sWzAsMSwiXFx2YXJwaGkiXV0=
\[\begin{tikzcd}
	{p^{-1}(U)} && {U\times \K^m} \\
	& U
	\arrow["\sim"', from=1-1, to=1-3]
	\arrow["\varphi", from=1-1, to=1-3]
	\arrow["p"', from=1-1, to=2-2]
	\arrow["{\pi_U}", from=1-3, to=2-2]
\end{tikzcd}\]
\end{definition}

\begin{example}
    $X\times \K^m$ con la mappa di proiezione su $X$ \`e un fibrato vettoriale, detto \textbf{fibrato banale}. 
\end{example}

\begin{example}
    Sia $X=\Pj^1=\Pj(\K^2)$ e consideriamo 
    \[E=\{(\ell,v)\in \Pj^1\times \K^2 \colon v\in \ell \} \xrightarrow{p}\Pj^1\] 
    dove la somma è data da $(\ell,v)+(\ell,w)=(\ell,v+w)$. Questo \`e detto il \textbf{fibrato tautologico}.
\end{example}

\begin{exercise} Consideriamo la coppia $(E,p)$ dell'esempio precedente. Mostrare che
    \begin{enumerate}
        \item È un fibrato vettoriale. 
        \item Non è il fibrato banale\footnote{Una intuizione geometrica importante (seppur lontana da essere una dimostrazione) che giustifica la non banalit\`a \`e che il fibrato banale di rango 1 ``topologicamente" \`e un cilindro mentre il fibrato tautologico \`e un nastro di M\"obius.}.
    \end{enumerate}
\end{exercise}

\section{Cocicli e banalizzazioni}
\begin{remark}
Se $p:E\to X$ \`e un fibrato vettoriale su $X$ di rango $m$ allora esiste un ricoprimento (finito) $\cpa{U_i}$ di $X$ tale che per ogni $i$
% https://q.uiver.app/#q=WzAsMyxbMCwwLCJwXnstMX0oVV9pKSJdLFsyLDAsIlVfaVxcdGltZXMgXFxLXm0iXSxbMSwxLCJVX2kiXSxbMCwxLCJcXHNpbSIsMl0sWzAsMiwicCIsMl0sWzEsMiwiXFxwaV97VV9pfSJdLFswLDEsIlxcdmFycGhpX2kiXV0=
\[\begin{tikzcd}
	{p^{-1}(U_i)} && {U_i\times \K^m} \\
	& {U_i}
	\arrow["\sim"', from=1-1, to=1-3]
	\arrow["{\varphi_i}", from=1-1, to=1-3]
	\arrow["p"', from=1-1, to=2-2]
	\arrow["{\pi_{U_i}}", from=1-3, to=2-2]
\end{tikzcd}\]
\end{remark}

\begin{definition}[Cocicli]
Dato un ricoprimento $\Uc=\cpa{U_i}$ di $X$ definiamo gli \textbf{1-cocicli} relativi al ricoprimento come
\[Z^1(\Uc,\GL(m))=\cpa{\cpa{\al_{i,j}}\mid \al_{i,j}:U_{i,j}\to\GL(m),\ \al_{i,h}=\al_{j,h}\al_{i,j},\ \al_{i,i}(u)=I_m}.\]
\end{definition}
\begin{proposition}
Se $\Uc$ \`e un ricoprimento banalizzante per il fibrato $p:E\to X$ allora esso definisce dei cocili indotti dalle mappe di transizione.
\end{proposition}
\begin{proof}
Se $U_{i,j}=U_i\cap U_j$ allora abbiamo degli isomorfismi\footnote{da ora in poi eviteremo di scrivere ogni volta le restrizioni}
\[\vp_{i,j}=\vp_j\res{U_{i,j}}\circ \vp_i\res{U_{i,j}}\ii:U_{i,j}\times\K^m\to U_{i,j}\times \K^m\]
Sia $\vp_{i,j}(u,v)=(u,\al_{i,j}(u)(v))$. Questo definisce delle mappe regolari
\[\al_{i,j}:U_{i,j}\to \GL(m).\]
Notiamo inoltre che su $U_{i,j,h}\times \K^m$
\[\vp_{j,h}\circ \vp_{i,j}=\vp_h\circ \vp_{j}\ii\circ \vp_j\circ \vp_i\ii=\vp_{i,h}.\]
Vedendo queste identit\`a in termini delle $\al_{i,j}$ abbiamo mostrato
\[\al_{i,h}(u)=\al_{j,h}(u)\al_{i,j}(u).\]
Chiaramente $\vp_i\circ \vp_i\ii=id_{U_i\times\K^m}$ quindi $\al_{i,i}(u)=I_m$.
\end{proof}

\begin{proposition}[]\label{PrCocicliDeterminanoFibrato}
Sia $\Uc$ un ricoprimento di $X$. Una collezione di cocicli $\al\in Z^1(\Uc,\GL(m))$ determina un fibrato vettoriale su $X$.
\end{proposition}
\begin{proof}
Definiamo lo spazio totale come il seguente coprodotto fibrato
\[E_\al:=\frac{\coprod(U_i\times \K^m)}{(u,v)\sim (u,\al_{i,j}(u)(v))}.\]
Siano $\psi_i:U_i\times\K^m\to E_\al$ le mappe ovvie verso il coprodotto.
\smallskip

Su $E_\al$ imponiamo la seguente topologia: $V$ \`e aperto in $E_\al$ se e solo se $\psi_i\ii(V)$ \`e aperto in $U_i\times\K^m$ per ogni $i$.

\smallskip

Se $V$ \`e aperto in $E_\al$ allora, poich\'e la regolarit\`a \`e una condizione locale, si ha
\[f\in \Oc_{E_\al}(V)\coimplies f\circ \psi_i\in \Oc_{U_i}(\psi_i\ii(V)) \quad\forall i.\]
Affermiamo che $\psi_i$ \`e iniettiva, aperta e regolare per ogni $i$:
\setlength{\leftmargini}{0cm}
\begin{itemize}
\item[$\boxed{\text{Iniettiva}}$] Supponiamo $\psi_i(u,v)=\psi_i(u',v')$, allora
\[U_i\times \K^m\ni(u,v)\sim (u',v')\in U_i\times \K^m\overset{\al_{i,i}=id_{U_i}}\implies u=u',\ v=v'.\]
\item[$\boxed{\text{Aperta}}$] Se $W\subseteq U_i\times\K^m$ aperto e $\wt W=\psi_i(W)\subseteq E_\al$ allora dobbiamo verificare che $\psi_j\ii(\wt W)$ \`e aperto per ogni $j$. Consideriamo la seguente catena di equivalenze:
\begin{gather*}
(u,v)\in \psi_j\ii(\wt W)\subseteq U_j\times\K^m\\
\psi_j(u,v)\in \wt W=\psi_i(W)\\
\exists (u',v')\in W\subseteq U_i\times\K^m\ t.c.\ \psi_j(u,v)=\psi_i(u',v')\text{ in }E_\al\\
\exists (u',v')\in W\ t.c.\ U_j\times \K^m\ni(u,v)\sim (u',v')\in U_i\times \K^m\\
(u,\al_{j,i}(u)(v))\in W
\end{gather*}
cio\`e $\psi_j\ii(\wt W)$ \`e la preimmagine di $W$ tramite la mappa $U_{i,j}\times \K^m\to U_i\times \K^m$ che \`e continua.
\item[$\boxed{\text{Regolare}}$] $\psi_i$ \`e continua per definizione della topologia di $E_\al$. Essendo la regolarit\`a una condizione locale, per come abbiamo definito la topologia su $E_\al$ anche la condizione sui pullback \`e chiara.
\end{itemize}
\setlength{\leftmargini}{0.5cm}

Abbiamo dunque mostrato che $\psi_i(U_i\times \K^m)\cong U_i\times\K^m$.
\medskip

\noindent
La struttura vettoriale sulle fibre di $E_\al$ \`e quella indotta dal coprodotto in modo ovvio.
\end{proof}


\begin{remark}
Se $U_i$ \`e un aperto di $X$ che banalizza $p:E\to X$ notiamo che possono essere definiti diversi isomorfismi tra $p\ii(U_i)$ e $U_i\times \K^m$
% https://q.uiver.app/#q=WzAsMyxbMCwwLCJwXnstMX0oVV9pKSJdLFsyLDAsIlVfaVxcdGltZXMgXFxLXm0iXSxbMSwxLCJVX2kiXSxbMCwxLCJcXHd0XFx2cF9pIiwyLHsiY3VydmUiOjF9XSxbMCwyLCJwIiwyXSxbMSwyLCJcXHBpX3tVX2l9Il0sWzAsMSwiXFx2YXJwaGlfaSIsMCx7ImN1cnZlIjotMX1dXQ==
\[\begin{tikzcd}
	{p^{-1}(U_i)} && {U_i\times \K^m} \\
	& {U_i}
	\arrow["{\wt\vp_i}"', curve={height=6pt}, from=1-1, to=1-3]
	\arrow["{\varphi_i}", curve={height=-6pt}, from=1-1, to=1-3]
	\arrow["p"', from=1-1, to=2-2]
	\arrow["{\pi_{U_i}}", from=1-3, to=2-2]
\end{tikzcd}\]
In particolare questi definiscono diverse mappe di transizione e diversi cocicli.


Studiamo la ``differenza" tra le mappe di transizione 
\[\chi_i:=\wt\vp_i\circ \vp_i\ii:U_i\times\K^m\to U_i\times\K^m\] 
Come notazione scriviamo $\chi_i(u,v)=(u,\beta_i(u)(v))$.
Notiamo che
\[\wt\vp_{i,j}=\wt\vp_j\circ \wt\vp_i\ii=\wt\vp_j\circ\vp_j\ii\circ\vp_j\circ\vp_i\ii\circ\vp_i\circ \wt\vp_i\ii=\chi_j\circ\vp_{i,j}\circ\chi_i\ii,\]
quindi in termini di cocicli $\wt \al_{i,j}=\beta_j\al_{i,j}\beta_i\ii$ o equivalentemente $\wt\al_{i,j}\beta_i=\beta_j\al_{i,j}$.
\end{remark}

Questo ci spinge a dare la seguente definizione:
\begin{definition}[Coomologia]
Definiamo due cocicli $\al_{i,j}$ e $\wt\al_{i,j}$ \textit{equivalenti} se esiste $\beta_i:U_i\to \GL(m)$ tale che $\wt \al_{i,j}=\beta_j\al_{i,j}\beta_i\ii$. Definiamo il \textbf{primo insieme di coomologia} come
\[H^1(\Uc,\GL(m))=\quot{Z^1(\Uc,\GL(m))}\sim.\]
\end{definition}


\begin{theorem}[]\label{ThCorrispondenzaFibratiECoomologia}
Se $\Uc$ \`e un ricoprimento aperto di $X$ allora abbiamo una corrispondenza biunivoca
\[\quot{\cpa{\text{Fibrati vett. che si banalizzano su $\Uc$}}}{\text{iso. di fibr.}}\longleftrightarrow H^1(\Uc,\GL(m)).\]
\end{theorem}
\begin{proof}
Ad un fibrato associo una collezione di cocicli e quindi una classe in coomologia. Questa associazione \`e ben definita perch\'e cambiando cocicli associati alla stessa banalizzazione essi sono equivalenti.

Cambiare la classe di isomorfismo del fibrato non cambia la classe di coomologia perch\'e l'isomorfismo sugli aperti banalizzanti induce un automorfismo che corrisponde a cambiare cocicli.

\bigskip
\noindent Viceversa, dati dei coclicli sappiamo costruire un fibrato come in (\ref{PrCocicliDeterminanoFibrato}). Se $\al$ e $\wt \al$ sono cocicli equivalenti allora mostriamo che $E_\al\cong E_{\wt\al}$ come fibrati. Se $\beta_i$ sono tali che $\wt \al_{i,j}=\beta_i\al_{i,j}\beta_i\ii$ allora abbiamo una ovvia mappa tra i fibrati
% https://q.uiver.app/#q=WzAsMixbMCwwLCJcXGRmcmFje1xcY29wcm9kKFVfaVxcdGltZXMgXFxLXm0pfXtcXHNpbV9cXGFsfSJdLFszLDAsIlxcZGZyYWN7XFxjb3Byb2QoVV9pXFx0aW1lcyBcXEtebSl9e1xcc2ltX3tcXHd0XFxhbH19Il0sWzAsMSwiXFxjb3Byb2QgaWRfe1VfaX1cXHRpbWVzIFxcYmV0YV9pKHUpKHYpIl1d
\[\begin{tikzcd}
	{\dfrac{\coprod(U_i\times \K^m)}{\sim_\al}} &&& {\dfrac{\coprod(U_i\times \K^m)}{\sim_{\wt\al}}}
	\arrow["{\coprod id_{U_i}\times \beta_i(u)(v)}", from=1-1, to=1-4]
\end{tikzcd}\]
che si verifica essere un isomorfismo (Esercizio).
\end{proof}

\section{Operazioni sui fibrati}

\begin{definition}[Morfismo di fibrati]
Se $E\to X$ e $F\to X$ sono fibrati, un \textbf{morfismo di fibrati} $T:E\to F$ \`e un morfismo di variet\`a $E\to X$ che commuta con i cocicli. Pi\`u precisamente, definite $T_i$
% https://q.uiver.app/#q=WzAsNCxbMCwwLCJFXFxyZXN7VV9pfSJdLFsxLDAsIkZcXHJlc3tVX2l9Il0sWzAsMSwiXFxLXm5cXHRpbWVzIFVfaSJdLFsxLDEsIlxcS15tXFx0aW1lcyBVX2kiXSxbMCwxLCJUXFxyZXN7RVxccmVze1VfaX19Il0sWzIsMywiVF9pIl0sWzAsMiwiXFxlX2kiLDJdLFsxLDMsIlxcdnBfaSJdXQ==
\[\begin{tikzcd}
	{E\res{U_i}} & {F\res{U_i}} \\
	{\K^n\times U_i} & {\K^m\times U_i}
	\arrow["{T\res{E\res{U_i}}}", from=1-1, to=1-2]
	\arrow["{\e_i}"', from=1-1, to=2-1]
	\arrow["{\vp_i}", from=1-2, to=2-2]
	\arrow["{T_i}", from=2-1, to=2-2]
\end{tikzcd}\]
dove $\e_i$ e $\vp_i$ sono isomorfismi fissati, che inducono i cocicli $\e_{i,j}=\e_j\e_i\ii$ e $\vp_{i,j}=\vp_j\vp_i\ii$, chiediamo che commuti il diagramma 
% https://q.uiver.app/#q=WzAsNixbMCwxLCJFXFxyZXN7VV97aSxqfX0iXSxbMSwxLCJGXFxyZXN7VV97aSxqfX0iXSxbMCwyLCJcXEteblxcdGltZXMgVV97aSxqfSJdLFsxLDIsIlxcS15tXFx0aW1lcyBVX3tpLGp9Il0sWzAsMCwiXFxLXm5cXHRpbWVzIFVfe2ksan0iXSxbMSwwLCJcXEtebVxcdGltZXMgVV97aSxqfSJdLFswLDEsIlQiXSxbMiwzLCJUX2kiXSxbMCwyLCJcXGVfaSIsMl0sWzEsMywiXFx2cF9pIl0sWzEsNSwiXFx2cF9qIiwyXSxbMCw0LCJcXGVfaiJdLFs0LDUsIlRfaiJdLFsyLDQsIlxcZV97aSxqfSIsMCx7ImN1cnZlIjotNX1dLFszLDUsIlxcdnBfe2ksan0iLDIseyJjdXJ2ZSI6NX1dXQ==
\[\begin{tikzcd}
	{\K^n\times U_{i,j}} & {\K^m\times U_{i,j}} \\
	{E\res{U_{i,j}}} & {F\res{U_{i,j}}} \\
	{\K^n\times U_{i,j}} & {\K^m\times U_{i,j}}
	\arrow["{T_j}", from=1-1, to=1-2]
	\arrow["{\e_j}", from=2-1, to=1-1]
	\arrow["T", from=2-1, to=2-2]
	\arrow["{\e_i}"', from=2-1, to=3-1]
	\arrow["{\vp_j}"', from=2-2, to=1-2]
	\arrow["{\vp_i}", from=2-2, to=3-2]
	\arrow["{\e_{i,j}}", curve={height=-30pt}, from=3-1, to=1-1]
	\arrow["{T_i}", from=3-1, to=3-2]
	\arrow["{\vp_{i,j}}"', curve={height=30pt}, from=3-2, to=1-2]
\end{tikzcd}\]
cio\`e che valga la relazione 
\[\vp_{i,j}\circ T_i=T_j\circ \e_{i,j}.\]
\end{definition}


\begin{definition}[Somma diretta]
Dati due fibrati $p_E:E\to X$ e $p_F\to X$ possiamo definire la loro \textbf{somma diretta} come
\[E\oplus F=\cpa{(e,f)\in E\times F\mid p_E(e)=p_F(f)}\to X\]
dove la proiezione \`e definita mandando $(e,f)$ in $x=p_E(e)=p_F(f)$.
\end{definition}
\begin{remark}
$E\oplus F$ \`e il prodotto fibrato su $X$ dei due fibrati.
% https://q.uiver.app/#q=WzAsNCxbMCwxLCJFIl0sWzEsMCwiRiJdLFsxLDEsIlgiXSxbMCwwLCJFXFxvcGx1cyBGIl0sWzAsMiwicF9FIiwyXSxbMSwyLCJwX0YiXSxbMywwXSxbMywxXSxbMywyLCIiLDAseyJzdHlsZSI6eyJuYW1lIjoiY29ybmVyIn19XV0=
\[\begin{tikzcd}
	{E\oplus F} & F \\
	E & X
	\arrow[from=1-1, to=1-2]
	\arrow[from=1-1, to=2-1]
	\arrow["\lrcorner"{anchor=center, pos=0.125}, draw=none, from=1-1, to=2-2]
	\arrow["{p_F}", from=1-2, to=2-2]
	\arrow["{p_E}"', from=2-1, to=2-2]
\end{tikzcd}\]
\end{remark}

\begin{remark}
Se $\e_{i,j}$ \`e un cociclo per $E$ e $\vp_{i,j}$ per $F$ allora $\e_{i,j}+\vp_{i,j}$ \`e un cociclo di $E\oplus F$ (stiamo considerando un ricoprimento che banalizza sia $E$ che $F$).
\end{remark}

\begin{definition}[Prodotto tensore]
Dati due fibrati $p_E:E\to X$ e $p_F\to X$ possiamo definire il loro \textbf{prodotto tensore} $E\otimes F$ come il fibrato associato al cociclo
\[\e_{i,j}\otimes \vp_{i,j}:U_{i,j}\to \GL(\K^n\otimes \K^m)\]
dove $n=\rnk E$ e $m=\rnk F$.
\end{definition}

\begin{definition}[Fibrato duale]
Dato un fibrato $p:E\to X$ definiamo il \textbf{fibrato duale} come quello associato al cociclo $(\e_{i,j}\ii)^\top:U_{i,j}\to \GL(n)$.
\end{definition}



\begin{remark}
Abbiamo un morfismo 
\[E^\ast\otimes E\to \K\times X\]
dato mandando il cociclo $\wt\e_{i,j}:=(\e_{i,j}\ii)^\top\otimes \e_{i,j}$ a $h_{i,j}=1$.
\end{remark}
\begin{proof}
Se ci restringiamo ad un ricorprimento $\cpa{U_i}$ che banalizza $E$ definiamo
\[T_i:\funcDef{(\K^n\otimes \K^n)\times U_i}{\K\times U_i}{((\la\otimes v),u)}{(\la^\top v,u)}\]
e notiamo che queste $T_i$ si incollano ad un mosrfismo di fibrati:

\noindent
Data la definizione delle mappe di incollamento sui fibrati, verificare
\[h_{i,j}\circ T_i\circ \wt\e_{i,j}\ii=T_j\]
mostra anche che le $T_i$ si incollano come funzioni e questo incollamento sar\`a un morfismo di fibrati.
La formula vale perch\'e $h_{i,j}=1$ e
\[T_i(\wt\e_{i,j}\ii((\la,v),u))=T_i((\e_{i,j}^\top\la,\e_{i,j}\ii v),u)=(\la^\top\e_{i,j}\e_{i,j}\ii v,u)=(\la^\top v,u)=T_j((\la\otimes v),u).\]
\end{proof}

\section{Sezioni di un fibrato}
\begin{definition}[Sezioni di un fibrato]
Dato un fibrato vettoriale $p\colon E \to X$ e dato un aperto $U$ di $X$, definiamo le \textbf{sezioni} di $U$ come 
\[\Gamma(U,E)\coloneqq \{\sigma\colon U \to E \mid p\circ \sigma = id_U\}.\]
\end{definition}
\begin{remark}
    $\Gamma(U,E)$ eredita la somma dal fibrato:
    \[(\sigma+\tau)(u)=\sigma(u)+\tau(u)\]
    dove il membro di destra \`e l'immagine della coppia $(\sigma(u),\tau(u))\in E\times_X E$ tramite la mappa $+$.
    Se $f$ è in $\Oc_X(U)$ e $\sigma\in\Gamma(U,E)$, allora è definito $f\cdot\sigma$ in modo analogo.
    \medskip

    \noindent Insieme questi fatti mostrano che $\Gamma(U,E)$ è un $\Oc_X(U)$-modulo.
\end{remark}

\begin{example}[Sezioni del fibrato banale]
    % https://q.uiver.app/#q=WzAsNixbMCwwLCJYXFx0aW1lcyBcXEtebSJdLFswLDIsIlgiXSxbMiwwLCJVXFx0aW1lcyBcXEtebSJdLFsyLDIsIlUiXSxbMSwyLCJcXHN1cHNldGVxIl0sWzEsMCwiXFxzdXBzZXRlcSJdLFswLDFdLFsyLDMsInAiLDJdLFszLDIsIiIsMCx7ImN1cnZlIjozfV1d
\[\begin{tikzcd}
	{X\times \K^m} & \supseteq & {U\times \K^m} \\
	\\
	X & \supseteq & U
	\arrow[from=1-1, to=3-1]
	\arrow["p"', from=1-3, to=3-3]
	\arrow[curve={height=18pt}, from=3-3, to=1-3]
\end{tikzcd}\]
    Consideriamo $\sigma(u)=(u,\alpha_1(u),\ldots,\alpha_m(u))$, dove $\alpha_i\colon U \to \K$. Allora 
    \[\Gamma(U,U\times \K^m)\cong \Oc_X(U)^m.\]
\end{example}

Studiamo ora le sezioni di un fibrato qualsiasi.

\begin{proposition}\label{PrSezioniDiFibrato}
Siano $p:E\to X$ un fibrato vettoriale e $\Uc$ un ricoprimento banalizzante per questo.
Dare $\sigma\in \Gamma(X,E)$ \`e equivalente a dare delle $\sigma_i:U_i\to \K^m$ per ogni aperto del ricoprimento che verificano
\begin{center}
	\begin{tabular}[width=\pagewidth]{ccr}
		~\hspace{4.3cm}~&$\sigma_j=\al_{i,j}\sigma_i$ &\hspace{1cm}(\textbf{condizione di cociclo})
	\end{tabular} 
\end{center}
dove $\al_{i,j}$ sono dei cocicli per il fibrato.
\end{proposition}
\begin{proof}
Fissiamo $\vp_i:E\res{U_i}\to U_i\times \K^m$ e $\al_{i,j}$ come prima. Se $\sigma\in\Gamma(X,E)$ allora
\[\wt \sigma_i=\sigma\res{U_i}:U_i\to E\res{U_i},\quad \sigma_i:=\pi_{\K^m}\circ\vp_i\circ\wt\sigma_i\]
\[\vp_i\circ \wt\sigma_i:\funcDef{U_i}{U_i\times\K^m}{u}{(u,\sigma_i(u))}\]
Su $U_{i,j}$ abbiamo
\[\wt \sigma_j=\vp_{i,j}\circ \wt \sigma_i,\quad \sigma_j=\al_{i,j}\sigma_i\]
quindi da $\sigma$ abbiamo trovato delle $\sigma_i:U_i\to \K^m$ che verificano $\sigma_j=\al_{i,j}\sigma_i$.

Viceversa dati questi dati essi si possono incollare a $\sigma:X\to E$ perch\'e la condizione di cocliclo garantisce che le $\sigma_i$ coincidano sulle intersezioni degli aperti del ricoprimento.
\end{proof}

\section{Fibrati vettoriali su variet\`a affini}

\begin{remark}
Se $X$ \`e una variet\`a affine, $A=\K[X]$ e $E$ \`e un fibrato vettoriale di rango $m$ su $X$ allora $M=\Gamma(X,E)$ \`e un $A$-modulo.
\end{remark}

\begin{remark}
Supponiamo che $E$ si banalizzi su aperti affini $X_i$ tali che\footnote{ricordiamo che aperti di questo tipo formano una base per la topologia di Zariski.} $\K[X_i]=A_{f_i}$. Siano $\al_{i,j}$ i cocicli e consideriamo la mappa
\[J:\funcDef{\bigoplus_i (A_{f_i})^m}{\bigoplus_{i,j} (A_{f_if_j})^m}{(\sigma_i)}{(\sigma_j-\al_{i,j}\sigma_i)}\]
Per quanto detto $\Gamma(X,E)=\ker J$.
\end{remark}

\begin{proposition}
$\Gamma(X_f,E)=\Gamma(X,E)_f$
\end{proposition}
\begin{proof}
Su $X_f=U\subseteq X$ abbiamo 
% https://q.uiver.app/#q=WzAsMTIsWzAsMCwiMCJdLFsxLDAsIlxcR2FtbWEoWCxFKSJdLFsxLDEsIlxcR2FtbWEoVSxFKSJdLFswLDEsIjAiXSxbMiwwLCJcXGJpZ29wbHVzXFxLW1VfaV1ebSJdLFsyLDEsIlxcYmlnb3BsdXNcXEtbVV9pXFxjYXAgVV1ebSJdLFszLDAsIlxcYmlnb3BsdXNcXEtbVV97aSxqfV1ebSJdLFszLDEsIlxcYmlnb3BsdXNcXEtbVV97aSxqfVxcY2FwIFVdXm0iXSxbMiwyLCJcXGJpZ29wbHVzXFxLW1VfaV1ebV9mIl0sWzMsMiwiXFxiaWdvcGx1c1xcS1tVX3tpLGp9XV5tX2YiXSxbMSwyLCJcXEdhbW1hKFgsRSlfZiJdLFswLDIsIjAiXSxbMTEsMTBdLFsxMCw4XSxbOCw5XSxbMywyXSxbMiw1XSxbNSw3XSxbMCwxXSxbMSw0XSxbNCw2XSxbNCw1XSxbNiw3XSxbNSw4LCI9IiwzLHsic3R5bGUiOnsiYm9keSI6eyJuYW1lIjoibm9uZSJ9LCJoZWFkIjp7Im5hbWUiOiJub25lIn19fV0sWzcsOSwiPSIsMyx7InN0eWxlIjp7ImJvZHkiOnsibmFtZSI6Im5vbmUifSwiaGVhZCI6eyJuYW1lIjoibm9uZSJ9fX1dLFsxMCwyLCI9IiwzLHsic3R5bGUiOnsiYm9keSI6eyJuYW1lIjoibm9uZSJ9LCJoZWFkIjp7Im5hbWUiOiJub25lIn19fV1d
\[\begin{tikzcd}
	0 & {\Gamma(X,E)} & {\bigoplus\K[U_i]^m} & {\bigoplus\K[U_{i,j}]^m} \\
	0 & {\Gamma(U,E)} & {\bigoplus\K[U_i\cap U]^m} & {\bigoplus\K[U_{i,j}\cap U]^m} \\
	0 & {\Gamma(X,E)_f} & {\bigoplus\K[U_i]^m_f} & {\bigoplus\K[U_{i,j}]^m_f}
	\arrow[from=1-1, to=1-2]
	\arrow[from=1-2, to=1-3]
	\arrow[from=1-3, to=1-4]
	\arrow[from=1-3, to=2-3]
	\arrow[from=1-4, to=2-4]
	\arrow[from=2-1, to=2-2]
	\arrow[from=2-2, to=2-3]
	\arrow[from=2-3, to=2-4]
	\arrow["{=}"{marking, allow upside down}, draw=none, from=2-3, to=3-3]
	\arrow["{=}"{marking, allow upside down}, draw=none, from=2-4, to=3-4]
	\arrow[from=3-1, to=3-2]
	\arrow["{=}"{marking, allow upside down}, draw=none, from=3-2, to=2-2]
	\arrow[from=3-2, to=3-3]
	\arrow[from=3-3, to=3-4]
\end{tikzcd}\]
\end{proof}
\begin{corollary}\label{CorLocalizzazioneSezioniFibratoSuAffine}
Se $E$ su $X_f$ \`e banale allora
\[\Gamma(X,E)_f\cong \K[X_f]^m=(A_f)^m.\]
\end{corollary}


\subsection{Caratterizzazione con i moduli proiettivi}

Ricordiamo il seguente 
\begin{theorem}[]\label{ThSuLocaleNoetherianoUnFinitamenteGeneratoEPiattoImplicaLibero}
$A$ locale e Noetheriano, $M$ finitamente generato, allora
\begin{center}
	$M$ \`e piatto $\coimplies$ $M$ \`e proiettivo $\coimplies$ $M$ \`e libero.
\end{center}
\end{theorem}

\begin{lemma}[]\label{LmLocalizzazioneDiHom}
Sia $A$ un anello, $X$ un $A$ modulo e $M$ un $A$-modulo finitamente presentato, allora per $\mf$ massimale di $A$ abbiamo un isomorfismo
\[\funcDef{(\Hom_A(M,X))_\mf}{\Hom_{A_\mf}(M_\mf,X_\mf)}{\vp/s}{\frac ut\mapsto \frac{\vp(u)}{st}}\]
\end{lemma}
\begin{proof}
Se $M$ \`e libero allora la tesi \`e vera in quanto
\[(\Hom_A(M,X))_\mf\cong (X^n)_\mf=X_\mf^n\cong \Hom_{A_\mf}(A_\mf^n,X_\mf)=\Hom_{A_\mf}(M_\mf,X_\mf)\]
e la mappa data \`e questo isomorfismo.

Se ora $A^h\to A^k\to M\to 0$ \`e una presentazione finita di $M$ costruiamo il diagramma commutativo con righe esatte\footnote{le righe sono esatte per esattezza della localizzazione e esattezza a sinistra degli $\Hom$}
% https://q.uiver.app/#q=WzAsOCxbMCwwLCIwIl0sWzEsMCwiXFxIb20oTSxYKV9cXG1mIl0sWzIsMCwiXFxIb20oQV5rLFgpX1xcbWYiXSxbMywwLCJcXEhvbShBXmgsWClfXFxtZiJdLFsxLDEsIlxcSG9tKE1fXFxtZixYX1xcbWYpIl0sWzIsMSwiXFxIb20oQV5rX1xcbWYsWF9cXG1mKSJdLFszLDEsIlxcSG9tKEFeaF9cXG1mLFhfXFxtZikiXSxbMCwxLCIwIl0sWzAsMV0sWzEsMl0sWzIsM10sWzcsNF0sWzQsNV0sWzUsNl0sWzMsNl0sWzIsNV0sWzEsNF1d
\[\begin{tikzcd}
	0 & {\Hom(M,X)_\mf} & {\Hom(A^k,X)_\mf} & {\Hom(A^h,X)_\mf} \\
	0 & {\Hom(M_\mf,X_\mf)} & {\Hom(A^k_\mf,X_\mf)} & {\Hom(A^h_\mf,X_\mf)}
	\arrow[from=1-1, to=1-2]
	\arrow[from=1-2, to=1-3]
	\arrow[from=1-2, to=2-2]
	\arrow[from=1-3, to=1-4]
	\arrow[from=1-3, to=2-3]
	\arrow[from=1-4, to=2-4]
	\arrow[from=2-1, to=2-2]
	\arrow[from=2-2, to=2-3]
	\arrow[from=2-3, to=2-4]
\end{tikzcd}\]
Poich\'e le due mappe a destra sono isomorfismi anche la prima lo \`e per diagram chasing.
\end{proof}

\begin{theorem}[]\label{ThCaratterizzazioneProiettivoNoetheriano}
Sia $A$ Noetheriano e $M$ finitamente generato. Le seguenti sono equivalenti
\begin{enumerate}
	\item $M$ \`e proiettivo
	\item $M$ \`e piatto
	\item $M_\mf$ \`e libero per ogni $\mf$ massimale in $A$
	\item $M$ \`e localmente libero, cio\`e esistono $f_1,\cdots, f_n\in A$ tali che $(f_1,\cdots, f_n)=A$ e $M_{f_i}$ \`e libero.
\end{enumerate}
\end{theorem}
\begin{proof}
Diamo le implicazioni
\setlength{\leftmargini}{0cm}
\begin{itemize}
\item[$\boxed{1.\implies2.}$] ovvio. 
\item[$\boxed{2.\implies3.}$] ovvio per il teorema (\ref{ThSuLocaleNoetherianoUnFinitamenteGeneratoEPiattoImplicaLibero}) dato che $A_\mf$ \`e locale e $M_\mf$ sarebbe piatto finitamente generato su locale noetheriano, quindi libero.
\item[$\boxed{4.\implies3.}$] Se $\mf$ \`e massimale esiste $f_i$ che non appartiene a $\mf$. Allora $M_\mf=(M_{f_i})_\mf$ ma $M_{f_i}$ \`e libero.
\item[$\boxed{3.\implies1.}$] Mostriamo che $\Hom_A(M,\bullet)$ \`e esatto. Consideriamo una successione esatta corta di moduli
\[0\to X\to Y\to Z\to 0\]
e mostriamo che
\[0\to \Hom_A(M,X)\to \Hom_A(M,Y)\to \Hom_A(M,Z)\to 0\]
\`e esatta. Per fare ci\`o basta mostrare che per ogni $\mf$ massimale \`e esatta
\[0\to (\Hom_A(M,X))_\mf\to (\Hom_A(M,Y))_\mf\to (\Hom_A(M,Z))_\mf\to 0\]
Per il lemma (\ref{LmLocalizzazioneDiHom}) la mappa
\[\funcDef{(\Hom_A(M,X))_\mf}{\Hom_{A_\mf}(M_\mf,X_\mf)}{\vp/s}{\frac ut\mapsto \frac{\vp(u)}{st}}\]
\`e un isomorfismo (poich\'e $A$ noetheriano si ha che $M$ \`e finitamente generato esattamente se \`e finitamente presentato). Otteniamo dunque un diagramma commutativo
% https://q.uiver.app/#q=WzAsMTAsWzAsMCwiMCJdLFsxLDAsIlxcSG9tKE0sWClfXFxtZiJdLFsyLDAsIlxcSG9tKE0sWSlfXFxtZiJdLFszLDAsIlxcSG9tKE0sWilfXFxtZiJdLFsxLDEsIlxcSG9tKE1fXFxtZixYX1xcbWYpIl0sWzIsMSwiXFxIb20oTV9cXG1mLFlfXFxtZikiXSxbMywxLCJcXEhvbShNX1xcbWYsWl9cXG1mKSJdLFswLDEsIjAiXSxbNCwwLCIwIl0sWzQsMSwiMCJdLFswLDFdLFsxLDJdLFsyLDNdLFs3LDRdLFs0LDVdLFs1LDZdLFszLDZdLFsyLDVdLFsxLDRdLFs2LDldLFszLDhdXQ==
\[\begin{tikzcd}
	0 & {\Hom(M,X)_\mf} & {\Hom(M,Y)_\mf} & {\Hom(M,Z)_\mf} & 0 \\
	0 & {\Hom(M_\mf,X_\mf)} & {\Hom(M_\mf,Y_\mf)} & {\Hom(M_\mf,Z_\mf)} & 0
	\arrow[from=1-1, to=1-2]
	\arrow[from=1-2, to=1-3]
	\arrow[from=1-2, to=2-2]
	\arrow[from=1-3, to=1-4]
	\arrow[from=1-3, to=2-3]
	\arrow[from=1-4, to=1-5]
	\arrow[from=1-4, to=2-4]
	\arrow[from=2-1, to=2-2]
	\arrow[from=2-2, to=2-3]
	\arrow[from=2-3, to=2-4]
	\arrow[from=2-4, to=2-5]
\end{tikzcd}\]
con mappe verticali date da isomorfismi, dunque la riga sopra \`e esatta se e solo se la riga sotto lo \`e, ma la riga sotto \`e esatta perch\'e $M_\mf$ \`e libero (e quindi proiettivo).
\item[$\boxed{3.\implies4.}$] Mostriamo che per ogni $\mf$ esiste $f\notin \mf$ tale che $M_f$ \`e libero come $A_f$ modulo. Questo \`e sufficiente perch\'e $X=\bigcup_{\mf\in\Spec_{max}(A)} X_f$ per queste $f$ e per compattezza estraiamo un sottoricoprimento finito. Le $f_i$ corrispondenti a questo ricoprimento per costruzione sono della forma cercata.

Mostriamo il claim: Poich\'e $M_\mf$ \`e libero esso ha una base $m_1,\cdots,m_n$. Consideriamo la mappa
\[0\to A_\mf^n\xrightarrow{\al}M_\mf\to 0\]
che manda $v_i$ in $m_i$, dove $v_1,\cdots, v_n$ \`e la base canonica di $A_\mf^n$. Per definizione della localizzazione esiste $f\notin \mf$ tale che
\[v_i=\frac{u_i}f,\ m_i=\frac{x_i}{f},\qquad \text{per degli }u_i\in A^n,\ x_i\in M.\]
Segue dunque che abbiamo una successione esatta
\[0\to K\to A_f^n\to M_f\to 0\]
dato che l'unico elemento che appare al denominatore \`e $f$.

Notiamo che $K$ \`e finitamente generato per Noetherianit\`a, per esempio da $w_1,\cdots, w_t$, e che $K_\mf=0$, dunque esiste $g\notin \mf$ tale che $gw_i=0$ per ogni $i$, cio\`e $K_g=0$. Se allora localizziamo la successione di prima troviamo una nuova successione esatta
\[0\to K_g\to A_{fg}^n\to M_{fg}\to 0\]
ma dato che $K_g=0$ abbiamo dimostrato che $M_{fg}$ \`e libero. Per costruzione $fg\notin \mf$ in quanto il complementare di $\mf$ \`e una parte moltiplicativa.
\end{itemize}
\setlength{\leftmargini}{0.5cm}
\end{proof}

\begin{corollary}
Se $p:E\to X$ \`e un fibrato vettoriale e $M=\Gamma(X,E)$ allora $M$ \`e proiettivo.
\end{corollary}
\begin{proof}
Ricordando (\ref{CorLocalizzazioneSezioniFibratoSuAffine}) che $\Gamma(X_f,E)=M_f$, se scegliamo un ricoprimento $\cpa{X_f}$ che banalizza $E\to X$ si ha che $M_f$ \`e libero, dunque abbiamo mostrato che $M$ \`e localmente libero, che implica proiettivo per il teorema (\ref{ThCaratterizzazioneProiettivoNoetheriano}) se $M$ \`e finitamente generato.

Poich\'e ogni $M_{f_i}$ (dove $f_1,\cdots, f_n$ determinano il ricoprimento banalizzante) \`e finitamente generato (libero di rango pari al rango del fibrato), esistono $m_1,\cdots, m_T\in M$ tali che $m_1,\cdots, m_T$ generano $M_{f_i}$ per ogni $i$. Se $N=\ps{m_1,\cdots, m_T}\subseteq M$ si ha che $N_{f_i}=M_{f_i}$ per ogni $f_i$ per costruzione, dunque localizzando ulteriormente $N_\mf=M_\mf$ per ogni $\mf$ massimale e quindi $N=M$.
\end{proof}

\begin{definition}[Rango di un modulo proiettivo]
Sia $P$ un modulo proiettivo. Affermiamo $n$ \`e il \textbf{rango} di $P$ se per ogni $\mf$ massimale $P_\mf\cong A_\mf^n$.
\end{definition}

\begin{proposition}[]\label{PrModuloProiettivoDefinisceFibrato}
A un $A$-modulo proiettivo $P$ di rango $n$ corrisponde un fibrato vettoriale di rango $n$ sulla variet\`a associata ad $A$.
\end{proposition}
\begin{proof}
Poich\'e $P$ \`e proiettivo esistono $f_1,\cdots, f_N$ tali che $(f_1,\cdots, f_N)=A$ e $P_{f_i}\cong A_{f_i}^n$. Sia $\al_i:P_{f_i}\to A_{f_i}^n$ un fissato isomorfismo.
Questi isomorfismi si restringono a isomorfismi $P_{f_if_j}\to A_{f_if_j}^n$, quindi possiamo definire 
\[\al_{i,j}=\al_j\circ \al_i\ii\in \GL(A_{f_i f_j}^n).\] 
Per definizione $\al_{ih}=\al_{jh}\al_{ij}$.
Ricordiamo che
\[A_{f_if_j}=\Oc(X_{f_if_j})=\Oc(X_{f_i}\cap X_{f_j})\]
quindi effettivamente $\al_{i,j}:X_{f_if_j}\to \GL_n(\Oc(X_{f_if_j}))$. Avendo verificato la condizione di cociclo si ha che questo definisce un fibrato.
\end{proof}

Possiamo dare una descrizione pi\`u concreta di questo fibrato:

\begin{remark}[Costruzione intrinseca del fibrato]
~

\noindent
Sia $P$ un modulo proiettivo e consideriamo il duale
\[P^\ast=\Hom_A(P,A)\]
e poi l'algebra simmetrica
\[\Sym_A P^\ast=\frac{A\oplus (P^\ast)\oplus (P^\ast\otimes_A P^\ast)\oplus\cdots}{(u\otimes v-v\otimes u)}.\]
Chiaramente $\Sym_AP^\ast$ \`e un anello. Si pu\`o verificare che esso \`e finitamente generato e ridotto, quindi \`e l'anello di coordinate di una variet\`a affine $E$.

Notiamo che se $P$ \`e libero allora $\Sym_AP^\ast$ \`e un anello di polinomi:
chiamiamo $x_1,\cdots, x_n$ i generatori di $P^\ast$ duali alla base di $P$, allora \`e evidente che 
\[\Sym_AP^\ast=A[x_1,\cdots, x_n]=A\otimes \K[x_1,\cdots, x_n]=\Oc(X\times \K^n).\]
Quindi $E$ \`e coperto da intorni della forma $X\times\K^n$ e la mappa di proiezione \`e quella ovvia.
\end{remark}


\begin{theorem}[]\label{ThCorrispondenzaFibratiVettorialiModuliProiettivi}
C'\`e una corrispondenza biunivoca
\[\quot{\cpa{\emat{\text{Moduli proiettivi su $A$}\\\text{di rango $n$}}}}{iso.}\longleftrightarrow \quot{\cpa{\emat{\text{Fibrati vettoriali su $X$}\\\text{di rango $n$}}}}{iso.}\]
e la mappa dal fibrato $E$ ai moduli \`e $E\mapsto \Gamma(X,E)$.
\end{theorem}



\section{Fibrati vettoriali sulla retta proiettiva}
\subsection{Fibrati lineari}
\begin{notation}
Scriviamo $\A^1\cong U_0=\Pj^1\bs\cpa\infty$ e $\A^1\cong U_\infty=\Pj^1\nz$, da cui
\[\A^1\nz\cong U_{0,\infty}=\Pj^1\bs\cpa{0,\infty}.\]
Scriviamo dunque $\K[U_0]=\K[z]$, $\K[U_\infty]=\K[w]$ e $\K[U_{0,\infty}]=\K[z,z\ii]$ con $w=z\ii$.
\end{notation}

\begin{remark}
Rispetto a questo ricoprimento i coclicli sono
\[Z^1(\Uc)=\cpa{\al_{0,\infty}:U_{0,\infty}\to\GL(1)}=\cpa{\funcDef{U_{0,\infty}}{\GL(1)}{z}{\la z^n}\mid n\in\Z,\ \la\in\K^\times}\]
A priori sono mappe della forma $z\mapsto p(z)$ ma se comparissero pi\`u monomi allora ci sarebbero radici diverse da $0$, quindi esisterebbe $a\in U_{0,\infty}$ tale che $p(a)=0\notin \GL(1)$.
\end{remark}



\begin{remark}\label{RmCoclicliFibratiLineariSuRettaProiettiva}
Si ha
\[H^1(\Uc,\GL(1))=\cpa{z^n}_{n\in\Z}.\]
\end{remark}
\begin{proof}
Se $\al=\la z^n$ e $\wt \al=\mu z^m$ sono cocicli equivalenti allora esistono $\beta_0:U_0\to \GL(1)$ e $\beta_\infty:U_{\infty}\to\GL(1)$ tali che
\[\mu z^m=\wt\al=\beta_\infty\al\beta_0\ii=\la\beta_\infty\beta_0\ii z^n\implies n=m\]
Se $n=m$ allora possiamo sempre trovare $\beta_0$ e $\beta_\infty$, quindi abbiamo mostrato
\[H^1(\Uc,\GL(1))=\cpa{z^n}_{n\in\Z}.\]
\end{proof}

\begin{notation}
Indichiamo il fibrato lineare $E_{z^{-n}}\to \Pj^1_\K$ con $\Oc_{\Pj^1_\K}(n)$ (o solo $\Oc(n)$ se chiaro dal contesto).
\end{notation}

\begin{proposition}\label{PrSezioniGlobaliFibratiLineariSuP1}
Le sezioni globali dei fibrati lineari su $\Pj^1$ sono
\[\Gamma(\Pj^1,\Oc(n))=\begin{cases}
	\K[z]_{n} & n\geq 0\\
	0 &n<0
\end{cases}\]
\end{proposition}
\begin{proof}
Consideriamo due sezioni $\sigma_0=f(z)$ e $\sigma_\infty=g(w)$ tali che $\sigma_\infty=\al_{0,\infty}\sigma_0$. Per quanto detto possiamo scegliere $\al_{0,\infty}=z^{-n}$, da cui sull'intersezione $U_{0,\infty}$ abbiamo 
\[g(z\ii)=z^{-n}f(z).\]
Scrivendo le espansioni troviamo
\[g(z\ii)=a_0+a_1 z\ii+\cdots\qquad z^{-n}f(z)=b_0 z^{-n}+b_1 z^{1-n}+\cdots\]
dunque:
\setlength{\leftmargini}{0cm}
\begin{itemize}
\item[$\boxed{n<0}$] In questo caso imporre uguaglianza impone che ogni coefficiente sia nullo, cio\`e $\sigma_0=\sigma_\infty=0$ e quindi il $\sigma$ a cui corrispondono \`e $0$.
\item[$\boxed{n\geq0}$] Imponendo l'uguaglianza troviamo
\[f(z)=a_0z^n+\cdots+a_{n-1}z+a_n\]
cio\`e un generico polinomio di grado $n$.
\end{itemize}
\setlength{\leftmargini}{0.5cm}
\end{proof}

\begin{remark}\label{ReRangoSezioniGlobaliFibratiLineariSuP1}
Segue immediatamente che
\[\dim\Gamma(\Pj^1,\Oc(n))=\begin{cases}
	n+1&\text{se }n\geq 0\\
	0 &\text{se }n<0
\end{cases}.\]
\end{remark}

\subsection{Fibrati vettoriali}

\begin{lemma}[]\label{LmMatriciDiCocicliPerFibrativettorialiSuRettaProiettiva}
Per ogni $C\in \GL_n(\K[t^{\pm1}])$ possiamo trovare $\al$ e $\beta$ matrici in $\GL_n(\K[t])$ tali che
\[\beta C\al=\mat{t^{m_1} & &\\
&\ddots&\\&&t^{m_k}}\]
con $m_1\geq\cdots\geq m_k$. Se $c_{i,j}\in \K[t]$ per ogni $i,j$ vogliamo anche poter scegliere $m_\ell\geq 0$ per ogni $\ell$.
\end{lemma}
\begin{proof}
Per $k=1$ la tesi \`e esattamente il contenuto dei conti nell'osservazione (\ref{RmCoclicliFibratiLineariSuRettaProiettiva}).

Mostriamo ora il caso di $k>1$. A meno di moltiplicare per una potenza alta di $t$ supponiamo $c_{i,j}\in \K[t]$ per ogni $i$ e $j$, finito il conto in questo caso possiamo moltiplicare per $t\ii$ alla stessa potenza e questo non influisce sulla forma diagonale che vorremmo ottenere.


Mostriamo che possiamo intanto trovare una $\al\in \GL_k(\K[t])$ tale che
\[C\al=\mat{t^m &0 &\cdots&0\\
\ast &\ast&\cdots&\ast}\]
Osserviamo che se abbiamo $c_{1,1}$ e $c_{1,j}$ sulla prima riga possiamo moltiplicare a destra per una matrice in modo da ottenere $m=\gcd(c_{1,1},c_{1,j})$ e $0$ al loro posto: se
\[m=c_{1,1}a+c_{1,j}c,\quad 1=da-bc\]
dove $a,c$ esistono per l'identit\`a di B\'ezout e $d,b$ esistono perch\'e $a$ e $c$ devono essere coprimi affinch\'e $m$ sia il massimo comune divisore, possiamo definire la matrice 
\[M=\mat{a&b\\c&d}\mat{1 & -\frac{c_{1,1}b+c_{1,j}d}{m}\\ 0&1}\] 
e ``inserirla" in una matrice identit\`a di taglia $k$ in modo da fare l'operazione richiesta, per esempio se $j=2$ allora
\[C\mat{M & &&&\\
&1&&\\
&&\ddots&\\
&&&1}=\mat{m &0 &\ast &\cdots&\ast\\
\ast &\ast&\ast &\cdots&\ast\\
\vdots &\vdots&\vdots &\ddots&\vdots\\
\ast &\ast&\ast &\cdots&\ast}\]
Reiterando questo procedimento prendendo la prima e seconda colonna, poi prima e terza, prima e quarta eccetera troviamo $\al$ tale che
\[C\al=\mat{\gcd(\cpa{c_{1,j}\mid 1\leq j\leq k}) &0 &\cdots&0\\
\ast &\ast&\cdots&\ast}\]
ma questo massimo comune divisore deve essere della forma $t^\ell$ perch\'e deve dividere $\det C$ e questo deve essere un elemento invertibile di $\K[t^{\pm1}]$ (ogni eventuale coefficiente di $t^\ell$ pu\`o essere inglobato da $\al$). Notiamo che $\ell\geq 0$ perch\'e i singoli $c_{i,j}$ sono elementi di $\K[t]$.

Per ipotesi induttiva esistono $\al'$ e $\beta'$ che riducono la sottomatrice sud-est, cio\`e
\[\mat{1&0\\0&\beta'}C\al\mat{1&0\\0&\al'}=\mat{t^{\ell_1} & 0 & \cdots &0\\
\ast & t^{\ell_2} & \cdots & 0\\
\vdots &\vdots&\ddots &\vdots\\
\ast & 0 &\cdots &t^{\ell_k}}\]
con $\ell_i\geq 0$ per ogni $1\leq i\leq k$. In particolare $\det C=t^n$ con $n=\sum_{i=1}^k\ell_i$. Poich\'e dunque $\ell_1$ \`e limitato da $n$, esiste un $\ell_1$ massimo che possiamo ottenere a meno di moltiplicare a destra o sinistra per matrici in $\GL_k(\K[t])$.

Affermiamo che per un tale $\ell_1$ massimale $\ell_1\geq \ell_2$, infatti se $\ell_1<\ell_2$, poich\'e
\[\mat{t^{\ell_1}&0\\c_{2,1}&t^{\ell_2}}\text{ \`e equivalente a }\mat{t^{\ell_2} & c_{2,1}\\ 0 &t^{\ell_1}},\]
se $c_{2,1}=0$ allora evidentemente $\ell_1$ non era massimale, se $c_{2,1}\neq 0$ allora riusciamo a sostituire questo pezzo $2\times 2$ con $\pa{\smat{t^{\ell_1+1}&0\\\ast&t^{m}}}$, negando ancora la massimalit\`a di $\ell_1$.


Quindi $\ell_1\geq \ell_2$ (e analogamente $\ell_1\geq \ell_i$ per ogni $2\leq i\leq k$) quindi possiamo agire sulle colonne e sulle righe con elementi della forma $\pa{\smat{1&0\\f(t)&1}}$ per rindurre $c_{2,1}(t)$ as un polinomio che contemporaneamente dovrebbe avere grado minore di $\ell_2$ ma maggiore di $\ell_1$, che \`e assurdo e quindi $c_{1,2}=0$.
\end{proof}

\begin{theorem}[]\label{ThClassificazioneFibratiVettorialiSuP1}
Su $\Pj^1$ gli unici fibrati vettoriali sono $E\cong \Oc(n_1)\oplus\cdots\oplus \Oc(n_k)$ per degli interi $n_i$.
\end{theorem}
\begin{proof}
Scriviamo $\Pj^1=U_0\cup U_\infty$ e procediamo per induzione su $n$ il rango del fibrato.
\setlength{\leftmargini}{0cm}
\begin{itemize}
\item[$\boxed{n=1}$] Notiamo che $E\res {U_0}$ e $E\res{U_\infty}$ sono banali perch\'e $U_0=\Pj^1\bs V(x_1)$, $U_\infty=\Pj^1\bs V(x_0)$ e vale (\ref{ThCaratterizzazioneProiettivoNoetheriano}).

Sia $\Uc=\cpa{U_0,U_\infty}$ e consideriamo $H^1(\Uc,\GL(1))$, che per il teorema (\ref{ThCorrispondenzaFibratiECoomologia}) e quanto appena osservato determina tutti i fibrati di rango 1 che si banalizzano su $\Uc$ a meno di isomorfismo. Per quanto osservato nella sottosezione precedente questo conclude.
\item[$\boxed{n>1}$] In modo del tutto analogo vogliamo considerare $H^1(\Uc,\GL(n))$. Ricordando che i cocicli per questo ricoprimento sono
\[Z^1(\Uc,\GL(n))=\cpa{C:U_{0,\infty}\to \GL(k)}\]
e che la relazione per cui quozientiamo per trovare la coomologia \'e
\[C\sim D\coimplies \exists \al:U_\infty\to\GL(n),\ \beta:U_0\to\GL(n)\ t.c.\ D=\al C\beta\ii.\]
Scriviamo 
\begin{align*}
	C(t)=&(c_{i,j}(t))\in \GL_n(\K[t^{\pm}])\\
	\al(t)=& (a_{i,j}(t))\in \GL_n(\K[t])\\
	\beta(t)=&(b_{i,j}(t))\in \GL_n(\K[t\ii])
\end{align*}
Per definizione di somma diretta di fibrati vista dal punto di vista dei cocicli il lemma (\ref{LmMatriciDiCocicliPerFibrativettorialiSuRettaProiettiva}) conclude.
\end{itemize}
\setlength{\leftmargini}{0.5cm}
\end{proof}

\begin{proposition}[]\label{PrUnicitaDecomposizioneFibratiVettorialiInLineariPerP1}
Se $E$ \`e un fibrato vettoriale su $\Pj^1$ e (per il teorema (\ref{ThClassificazioneFibratiVettorialiSuP1})) esso \`e isomorfo a $\Oc(n_1)\oplus\cdots\oplus \Oc(n_k)$ per $n_1\geq\cdots\geq n_k$ allora gli $n_i$ sono univocamente determinati.
\end{proposition}
\begin{proof}
Supponiamo di avere un'altra scrittura $E\cong \Oc(m_1)\oplus\cdots\oplus \Oc(m_k)$ con $m_1\geq\cdots\geq m_{k}$ (il numero di addendi \`e lo stesso perch\'e \`e pari al rango del fibrato). Senza perdita di generalit\`a supponiamo $m_1\geq n_1$ e tensorizziamo per $\Oc(-m_1)$ per ottenere
\[\Oc(0)^a\oplus \under{E'}{\Oc(n_2')\oplus\cdots\oplus \Oc(n_k')}\overset\vp\cong \Oc(0)^b\oplus \under{F'}{\Oc(m_2')\oplus\cdots\oplus \Oc(m_k')}\]
dove $n_i'$ e $m_i'$ sono gli opportuni interi che codificano le differenze $n_j-m_1$ e $m_j-m_1$ dopo aver raccolto le copie di $\Oc(0)$. Per costruzione tutti gli addendi diretti sono fibrati lineari con sezioni globali nulle (eccetto per le copie  di $\Oc(0)$), quindi
\[\dim \Gamma(\Pj^1,\Oc(0)^a\oplus E')=a\cdot (0+1)+(k-a)\cdot 0=a,\quad \dim \Gamma(\Pj^1,\Oc(0)^b\oplus F')=b\]
e cio\`e $a=b$.

Scriviamo l'isomorfismo $\vp:\Oc(0)^a\oplus E'\to \Oc(0)^a\oplus F'$ in forma matriciale come
\[\vp=\mat{\al &\beta\\
\gamma&\delta}\]
(cio\`e $\al:\Oc(0)^a\to \Oc(0)^a$, $\beta:E'\to \Oc(0)^a$ eccetera). Notiamo che dare un morfismo $\Oc(0)\to F'$ \`e la stessa cosa di dare una sezione di $F'$, quindi $\gamma=0$, dunque $\vp\res{\Oc(0)^a}=\al:\Oc(0)^a\to \Oc(0)^a$ \`e un isomorfismo e quindi abbiamo un isomorfismo
\[E'=E/\Oc(0)^a\xrightarrow{\vp}E/\Oc(0)^a=F'\]
e per induzione la scrittura di questi due pezzi \`e unica. Ritensorizzando per $\Oc(m_1)$ troviamo che la scrittura originale era unica.
\end{proof}































\section{Fibrati con azione di gruppo}

\begin{definition}
    Se $X$ è una $G$-varietà, un \textbf{$G$-fibrato} su $X$ è un fibrato vettoriale $p\colon E \to X$ su $X$ per cui esiste un'azione lineare di $G$ su $E$ tale che $p$ è $G$-equivariante.
\end{definition}
\begin{remark}
In questo caso, $\Gamma(X,E)$ ha un'azione lineare di $G$, data da 
\[(g\cdot \sigma)(x)=g(\sigma(g^{-1}x)).\] 
\end{remark}

Si potrebbe dimostrare che $G$ agisce in modo regolare su $\Gamma(X,E)$, e quindi che $\Gamma(X,E)$ è una rappresentazione algebrica di $G$. (Non lo faremo.) 

\begin{definition}
    Dati due fibrati vettoriali $p_1\colon E_1 \to X$ e $p_2\colon E_2 \to X$ su $X$, un \textbf{morfismo di fibrati} è una mappa $\varphi\colon E_1 \to E_2$ che preserva la somma e il prodotto per scalare e tale che $p_2\circ\varphi=p_1$. Diciamo che $\varphi$ è \textbf{equivariante} se commuta con $G$.
\end{definition}

\begin{remark}
Per ogni $x_0$ in $X$, un morfismo $\varphi$ definisce una mappa lineare $\varphi_{x_0}\colon (E_1)_{x_0}\to (E_2)_{x_0}$. 
\end{remark}

\begin{proposition}\label{PrClassificazioneFibratiSuGruppoQuoziente}
Le classi di isomorfismo fibrati vettoriali $G$-equivarianti di rango $m$ su $X=G/H$ sono in corrispondenza con le classi di isomorfismo delle rappresentazioni $m$-dimensionali di $H$.
\end{proposition}
\begin{proof}
Costruiamo esplicitamente questa corrispondenza:
\smallskip

\noindent
Dato un fibrato vettoriale $G$-equivariante $p\colon E\to G/H$ e posto $x_0=H=e_{G/H}$, possiamo considerare la fibra $E_{x_0}$ e chiaramente $H$ agisce su $E_{x_0}$. 
\smallskip

\noindent
Data una rappresentazione $V$ di $H$, consideriamo $G\times V$ con l'azione di $H$ data da 
\[h(g,v)=(gh^{-1},hv).\]
Passando al quoziente in $H$ abbiamo
% https://q.uiver.app/#q=WzAsNCxbMCwwLCJFPVxcZGZyYWN7R1xcdGltZXMgVn17SH0iXSxbMCwxLCJbZyx2XSJdLFsxLDAsIlg9e0d9L3tIfSJdLFsxLDEsIltnXSJdLFswLDJdLFsxLDMsIiIsMCx7InN0eWxlIjp7InRhaWwiOnsibmFtZSI6Im1hcHMgdG8ifX19XSxbMSwwLCJcXGluIiwzLHsic3R5bGUiOnsiYm9keSI6eyJuYW1lIjoibm9uZSJ9LCJoZWFkIjp7Im5hbWUiOiJub25lIn19fV0sWzMsMiwiXFxpbiIsMyx7InN0eWxlIjp7ImJvZHkiOnsibmFtZSI6Im5vbmUifSwiaGVhZCI6eyJuYW1lIjoibm9uZSJ9fX1dXQ==
\[\begin{tikzcd}
	{E=\dfrac{G\times V}{H}} & {X={G}/{H}} \\
	{[g,v]} & {[g]}
	\arrow[from=1-1, to=1-2]
	\arrow["\in"{marking, allow upside down}, draw=none, from=2-1, to=1-1]
	\arrow[maps to, from=2-1, to=2-2]
	\arrow["\in"{marking, allow upside down}, draw=none, from=2-2, to=1-2]
\end{tikzcd}\]
Notiamo che 
\[[g,v]+[gh,u]=[g,v]+[g,hu]=[g,v+hu]\] 
e che 
\[\lambda[g,v]=[g,\lambda v],\]
quindi le fibre hanno una struttura vettoriale.

Per concludere andrebbe mostrato che $E$ è una varietà algebrica, che la mappa costruita è regolare e che abbiamo la trivializzazione locale.
\end{proof}



\begin{example}
Se vogliamo costruire fibrati $G$-equivarianti lineari (cioè di rango $1$) su $\GL(n)/B_n$, sappiamo che questi corrispondono alle rappresentazioni $1$-dimensionali di $B_n$ 
% https://q.uiver.app/#q=WzAsMyxbMCwwLCJCX24iXSxbMiwwLCJcXEteXFx0aW1lcyJdLFsxLDEsIlQ9Ql9uL1VfbiJdLFswLDJdLFsyLDFdLFswLDFdXQ==
\[\begin{tikzcd}
	{B_n} && {\K^\times} \\
	& {T=B_n/U_n}
	\arrow[from=1-1, to=1-3]
	\arrow[from=1-1, to=2-2]
	\arrow[from=2-2, to=1-3]
\end{tikzcd}\]
Quindi queste possono essere caratterizzate attraverso i caratteri del toro.
\end{example}