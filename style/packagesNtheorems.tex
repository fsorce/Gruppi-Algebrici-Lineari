\usepackage[utf8]{inputenc}
\usepackage{amsmath,amssymb,amsfonts,amsthm,stmaryrd}
\usepackage{mathrsfs} % per mathscr
\usepackage{dsfont} % per mathbb1
\usepackage{graphicx}% ruota freccia per le azioni
\usepackage{oldgerm} % Fractur Particolare
\usepackage{marvosym}% per il \Lightning
\usepackage{array}
\usepackage{faktor} %per gli insiemi quoziente
\usepackage{hyperref}
\usepackage{xparse} % Per nuovi comandi con tanti input opzionali
\usepackage{tikz-cd}
\usepackage{multicol}
\usepackage{multirow}
\usepackage{cancel}

\usepackage[italian]{babel}
\usepackage[position=top]{subfig}
\usepackage{setspace}
\usepackage{mdframed}
\usepackage{calc}
\usepackage{enumerate}
\usepackage{centernot}
\usepackage{comment}

\usepackage{ mathdots }

% Ambienti per teoremi =================================
% <name> 
% <space above> 
% <space below> 
% <body font> 
% <indent amount> 
% <Theorem head font> 
% <punctuation after theorem head> 
% <space after theorem head> (default .5em) 
% <Theorem head spec>

% I nuovi ambienti sono costruiti in modo da andare alla riga successiva

\newsavebox{\boiteencadrement}
\newenvironment{encadrement}[1][gray!15]
  {\par\addvspace{\topsep+10pt}%
   \def\couleurencadrement{#1}%
   \begin{lrbox}{\boiteencadrement}%
   \begin{minipage}{\linewidth-6\fboxsep-1pt}%
  }
  {\end{minipage}\end{lrbox}%
   \noindent\begin{tikzpicture}[
   inner sep=8pt,
   line width=0.75pt]
   \node[rounded corners = 5pt,
   draw,
   color=black,
   fill=\couleurencadrement] {\usebox{\boiteencadrement}};
   \end{tikzpicture}%
   \par\addvspace{\topsep+10pt}%
  }

%Equazioni
\numberwithin{equation}{section} 

%Teoremi
\theoremstyle{plain}

%\newtheorem{lem}[thm]{Lemma}
%\newtheorem{exer}[thm]{Esercizio}
%\newtheorem{cor}[thm]{Corollary}

\theoremstyle{definition}

\newcounter{theorem}
\counterwithin{theorem}{chapter}

    %thm
\newtheorem{pretrm}[theorem]{Teorema}
\newenvironment{theorem}{\begin{encadrement}\begin{pretrm}}{\end{pretrm}\end{encadrement}}

\newtheorem{prefct}[theorem]{Fatto}
\newenvironment{fact}{\begin{encadrement}\begin{prefct}}{\end{prefct}\end{encadrement}}

%\newtheorem{definition}[theorem]{Definition}
%\newcounter{definition}
    %def
\newtheorem{predef}[theorem]{Definizione}
\newenvironment{definition}{\begin{encadrement}\begin{predef}}{\end{predef}\end{encadrement}}
    %prop
\newtheorem{preprop}[theorem]{Proposizione}
\newenvironment{proposition}{\begin{encadrement}\begin{preprop}}{\end{preprop}\end{encadrement}}
    %cor
\newtheorem{precor}[theorem]{Corollario}
\newenvironment{corollary}{\begin{encadrement}\begin{precor}}
{\end{precor}\end{encadrement}}
    %lem
\newtheorem{prelem}[theorem]{Lemma}
\newenvironment{lemma}{\begin{encadrement}\begin{prelem}}{\end{prelem}\end{encadrement}}
%
\newtheorem{conjecture}[theorem]{Congettura}
\newtheorem{example}[theorem]{Esempio}
\newtheorem{exercise}[theorem]{Esercizio}
\newtheorem{remark}[theorem]{Osservazione}
\newtheorem*{notation}{Notazione}

\makeatletter
\renewenvironment{proof}[1][\proofname]
{
    \par
    \pushQED{\qed}
    \normalfont \topsep6\p@\@plus6\p@\relax
    \trivlist
    \item[\hskip\labelsep\itshape#1\@addpunct{.}]\mbox{}\\*
}
{
    \popQED\endtrivlist\@endpefalse
}
\makeatother



%========= Preambolo per quiver ================
% quiver e' uno strumento che uso spesso per
% disegnare diagrammi. L'interfaccia sul loro sito
% permette di creare in modo visivo il diagramma e
% poi esportarlo come codice LaTeX da inserire nel
% documento. Il sito e' https://q.uiver.app/ 

%-----------------------------------------------
% *** quiver ***
% A package for drawing commutative diagrams exported from https://q.uiver.app.
%
% This package is currently a wrapper around the `tikz-cd` package, importing necessary TikZ
% libraries, and defining a new TikZ style for curves of a fixed height.
%
% Version: 1.2.1
% Authors:
% - varkor (https://github.com/varkor)
% - Andr\e'C (https://tex.stackexchange.com/users/138900/andr%C3%A9c)

\NeedsTeXFormat{LaTeX2e}
%\ProvidesPackage{quiver}[2021/01/11 quiver]

% `tikz-cd` is necessary to draw commutative diagrams.
\RequirePackage{tikz-cd}
% `amssymb` is necessary for `\lrcorner` and `\ulcorner`.
\RequirePackage{amssymb}
% `calc` is necessary to draw curved arrows.
\usetikzlibrary{calc}
% `pathmorphing` is necessary to draw squiggly arrows.
\usetikzlibrary{decorations.pathmorphing}

% A TikZ style for curved arrows of a fixed height, due to Andr\e'C.
\tikzset{curve/.style={settings={#1},to path={(\tikztostart)
    .. controls ($(\tikztostart)!\pv{pos}!(\tikztotarget)!\pv{height}!270:(\tikztotarget)$)
    and ($(\tikztostart)!1-\pv{pos}!(\tikztotarget)!\pv{height}!270:(\tikztotarget)$)
    .. (\tikztotarget)\tikztonodes}},
    settings/.code={\tikzset{quiver/.cd,#1}
        \def\pv##1{\pgfkeysvalueof{/tikz/quiver/##1}}},
    quiver/.cd,pos/.initial=0.35,height/.initial=0}

% TikZ arrowhead/tail styles.
\tikzset{tail reversed/.code={\pgfsetarrowsstart{tikzcd to}}}
\tikzset{2tail/.code={\pgfsetarrowsstart{Implies[reversed]}}}
\tikzset{2tail reversed/.code={\pgfsetarrowsstart{Implies}}}
% TikZ arrow styles.
\tikzset{no body/.style={/tikz/dash pattern=on 0 off 1mm}}
%=================================================
