\chapter{Geometria Algebrica}
In questa sezione, assumeremo sempre $\K$ algebricamente chiuso.

\section{Variet\`a affini immerse}
\begin{definition}
    Un sottoinsieme $X$ di $\K^n$ si dice una \textbf{varità algebrica affine} (\textbf{immersa}) se esistono $f_1,\dots,f_h$ in $\K[x_1,\dots,x_n]$ tali che 
	\[X=\cpa{v\in \K^n \mid f_1(v)=\ldots=f_h(v)=0}.\]
\end{definition}

\begin{notation}
Data una varità algebrica affine immersa $X$ poniamo
\[I(X)\doteqdot\{f\in \K[x_1,\dots,x_n] \mid f(v)=0 \text{ per ogni } v\in X  \}.\]
\end{notation}
\begin{remark}
L'insieme $I(X)$ risulta un ideale di $\K[x_1,\dots,x_n]$.
\end{remark}

\begin{notation}
Se $J$ è un ideale di $ \K[x_1,\dots,x_n]$ scriviamo \[V(J)\doteqdot\left\{v\in \K^n \mid f(v)=0 \text{ per ogni } f\in J\right\}.\]
\end{notation}

Richiamiamo inoltre il classico Teorema degli Zeri di Hilbert. 
\begin{notation}
    Nel seguito, denoteremo con $P$ l'anello $\K[x_1,\dots,x_n]$.
\end{notation}

\begin{theorem}[Nullstellensatz]
Se $\K$ \`e un campo algebricamente chiuso e $I,J$ sono ideali di $P$ allora 
\[V(I)=V(J)\coimplies \sqrt I=\sqrt J.\]
\end{theorem}
In particolare abbiamo una corrispondenza biunivoca \[\left\{\text{varietà algebriche affini immerse in $\K^n$} \right\} \longleftrightarrow \left\{\text{ideali radicali di $\K[x_1,\dots,x_n]$}\right\}.\]

%(controesempio classico di $X^2+1$ in $\R[X]$.)
\begin{example} Vediamo qualche controesempio classico.
\begin{itemize}
    \item Se $\K$ non è algebricamente chiuso, il Nullstellensatz non vale: ad esempio per $\K=\R$, il polinomio $X^2+1$ in $\R[X]$ genera un ideale proprio (massimale) $J$ tale che $V(J)=\emptyset$.
    \item La corrispondenza precedente non vale per gli ideali in generale: $\K=\C$ e $n=1$, l'ideale $J=(x^2)$ è tale che $V(J)=\{0\}$, ma $I(V(J))=(x)\ne J$.
\end{itemize}
\end{example}

\medskip 
\noindent
Vediamo ora alcune conseguenze del Nullstellensatz.
\setlength{\leftmargini}{0cm}
\begin{enumerate}
\item Gli ideali massimali di $P$ sono tutti e solo quelli della forma 
\[m_\alpha=(x_1-\alpha_1,\dots,x_n-\alpha_n),\] 
con $\alpha=(\alpha_1,\dots,\alpha_n)$ in $\K^n$.
\begin{proof}
    Che i precedenti siano tutti ideali massimali è evidente, mostriamo che sono i soli. Sia $\mf$ un ideale massimale di $P$. Allora $V(\mf)\ne\emptyset$ (in quanto altrimenti avremmo $P=I(\emptyset)=\sqrt{\mf}=\mf$), dunque esiste un $\alpha$ in $\K^n$ appartenente a $V(\mf)$. Ma allora $\mf_\alpha\subseteq \mf$ e per massimalità di $\mf_\alpha$ si ottiene l'uguaglianza. 
\end{proof}
\item Sia $I\subseteq P$ e sia $\alpha$ in $V(I)$. Allora $\sqrt{I}\subseteq \mf_\alpha$ e in generale \[V(I)=\{\alpha\in \K^n \mid \mf_\alpha \supseteq I\}.\]    
\end{enumerate}
\setlength{\leftmargini}{0.5cm}
Ricordiamo inoltre che se $I$ è un ideale di $P$, allora si ha 
\[\sqrt{I}=\bigcap_{\substack{\mf \in \operatorname{Max}(P) \\ I\subseteq \mf}} \mf .\]


\begin{definition}
    Sia $X$ una varietà algebrica affine immersa in $\K^n$. Si definisce l'\textbf{anello delle coordinate} di $X$ come il quoziente \[\K[X]= P/I(X)=\left\{\left.f\right|_{X} \colon f\in P\right\}.\] %delle funzioni regolari su $X$.
\end{definition}

Ricordiamo che su $\K^n$ è definita una topologia, detta \textbf{topologia di Zariski}, in cui i chiusi sono tutti e soli gli insiemi $V(I)$ al variare degli ideali $I$ in $P$. Ricordiamo infatti che, se $I,J$ sono ideali di $P$, allora \begin{itemize}
    \item $V(I)\cup V(J)=V(I\cap J)$;
    \item $\bigcap_{i\in I} V(I_i)=V\left(\sum_{i\in I}I_i\right)$.
\end{itemize}

\begin{example}
    Nel caso $\K=\C$ e $n=1$, i chiusi sono gli insiemi con un numero finito di punti, oppure tutto $\C$.
\end{example}

\begin{definition}
    Sia $f$ un elemento di $P$. Si definisce l'\textbf{aperto principale} relativo a $f$ come
	\[\left(\K^n\right)_f\doteqdot \K^n\setminus V(f)=\left\{\alpha\in \K^n \colon f(\alpha)\ne 0\right\}.\] 
\end{definition}
\begin{remark}
Se $U=\K^n\setminus V(I)$ è un aperto e $\alpha$ è in $U$, allora esiste $f$ in $P$ tale che $\alpha\in\left(\K^n\right)_f\subseteq U$. Infatti $I=(f_1,\dots,f_h)$, $V(I)\subseteq V(f_i)$ e $\left(\K^n\right)_{f_i}=\K^n\setminus V(f_i) \subseteq U$.
\end{remark}

\noindent Dotiamo ogni varietà algebrica affine immersa in $\K^n$ della topologia di sottospazio. 

\begin{notation}
Poniamo $X_f\doteqdot\left(\K^n\right)_f\cap X=\{\alpha\in X \colon f(\alpha)\ne 0\}$.
\end{notation} 

\begin{remark}
    L'insieme \[\left\{\frac{g}{f^n} \colon g\in P\right\}\] è ben definito su $\left(\K^n\right)_f$.
\end{remark}

\begin{definition}[Funzioni regolari]
    Sia $U$ un aperto di $\K^n$. Si definisce l'insieme delle \textbf{funzioni regolari} su $U$ come \[\Oc_X(U)=\left\{f\colon U \to \K \mid \forall \alpha \in U \ \exists\,g,h\in P\colon g(\alpha)\ne0 \text{ e } f=h/g \text{ su } U\cap X_g\right\}.\]
\end{definition}


\begin{lemma}
Sia $X$ variet\`a affine immersa, allora
\begin{enumerate}
	\item $\Oc(X)=\K[X]$
	\item Se $g\in\K[X]$ allora $\Oc(X_g)=\K[X]_g$
\end{enumerate}
\end{lemma}
\begin{proof}
L'affermazione 2. implica l'affermazione 1. scegliendo $g\in\K\nz$, quindi basta mostrare $2$.

Una funzione della forma $f/g^n$ appartiene a $\Oc(X_g)$, quindi basta mostrare l'altra inclusione. Sia $\vp\in \Oc(X_g)$. Per definizione, per ogni $\al\in X_g$ esistono $h$ e $k$ tali che $\vp=k/h$ in $X_g\cap X_h=X_{gh}$.

Abbiamo dunque un ricoprimento $X_g=\bigcup X_{gh_i}$ dove $\vp=k_i/h_i$ su $X_{gh_i}$. Sia $I=(h_i)$ l'ideale in $\K[X]_g$ generato dagli $h_i$. Se per assurdo $I\neq \K[X_g]$ allora esiste un massimale $\mf$ che contine $I$. Un massimale corrisponde ad un punto $\al\in X_g\subseteq X$ ma, poich\'e $X_g$ \`e ricoperto dagli $X_{gh_i}$, esiste un indice $i_0$ tale che $\al\in X_{gh_{i_0}}$ e questo \`e assurdo perch\'e vorrebbe dire 
\[h_{i_0}\notin \mf_\al\supseteq I\ni h_{i_0}.\]

Questo mostra che $I=\K[X_g]$. Possiamo dunque scrivere\footnote{nota che la somma \`e finita per definizione di ideale generato.} $1=\sum \al_i h_i$ per opportuni $\al_i$. Affermiamo che $\vp=\sum \al_i h_i\vp=\sum \al_i k_i$, il quale \`e un elemento di $\K[X]_g$. L'identit\`a vale perch\'e dove $h_i\neq 0$ abbiamo $\vp=k_i/h_i$ e quindi $h_i\vp=k_i$, se invece $h_i(x)=0$ si ha che $h_i(x)\vp(x)=0=k_i(0)$, perch\'e se cos\`i non fosse, poich\'e $x\in X_{h_j}$ per qualche $j\neq i$, si ha che \[k_j/h_j=k_i/h_i\coimplies h_ik_j=k_ih_j\] su $X_{h_ih_j}$ e quindi valutando in $x$ abbiamo $k_i(x)h_j(x)=0$ con $h_j(x)\neq 0$.
\end{proof}

\section{Variet\`a algebriche e morfismi}
\begin{definition}
Una \textbf{variet\`a algebrica} su $\K$ \`e uno spazio topologico compatto $X$ tale che per ogni aperto $U$ abbiamo un insieme di funzioni $\Oc_X(U)\subseteq C^0(X\to \K)$ tale che
\begin{enumerate}
	\item Se $U\subseteq V$ e $f\in \Oc_X(V)$ allora $f\res U\in \Oc_X(U)$.
	\item Se $V=\bigcup U_\al$ e $f:V\to \K$ \`e tale che $f\res{U_\al}\in \Oc_X(U_\al)$ per ogni $\al$ allora $f\in \Oc_X(V)$.
	\item $(X,\Oc_X)$ \`e localmente isomorfo ad una variet\`a affine immersa, cio\`e per ogni $x\in X$ esiste un intorno $U$ e un omeomorfismo $\vp:U\to Y$ con $Y$ affine tale che per ogni $V$ aperto di $U$ si ha $\vp(V)=W$ aperto e $\vp^\#:\Oc_Y(W)\to\Oc_X(V)$ \`e un isomorfismo.
\end{enumerate}
\end{definition}

\begin{remark}
Quello che abbiamo appena definito \`e pi\`u propriamente detto \textit{schema ridotto di tipo finito su $\K$}.
\end{remark}


\begin{example}
Sia $X=\Pj^1$. Come spazio topologico, i chiusi sono $\Pj^1,\emptyset$ e sottoinsiemi finiti di punti. Per ogni $U$ aperto di $\Pj^1$ poniamo
\[\Oc_{\Pj^1}(U)=\cpa{f:U\to \K\mid f\circ \vp_1\in\Oc(\vp_1\ii(U)),f\circ \vp_2\in\Oc(\vp_2\ii(U))}\]
dove
\[\vp_1:\funcDef{\K}{\Pj^1}{t}{[t:1]},\quad \vp_2:\funcDef{\K}{\Pj^1}{t}{[1:t]}.\]
\end{example}

\begin{definition}[Funzione regolare]
Se $(X,\Oc_X),(Y,\Oc_Y)$ sono variet\`a, la mappa $\vp:X\to Y$ \`e \textbf{regolare} se
\begin{enumerate}
	\item $\vp$ \`e continua
	\item Per ogni $W$ aperto di $Y$, se $f\in\Oc_Y(W)$ allora $f\circ \vp\in \Oc_X(\vp\ii(W))$
\end{enumerate}
\end{definition}

\begin{definition}[Variet\`a affine]
$(X,\Oc_X)$ \`e una \textbf{variet\`a affine} se \`e isomorfa ad una variet\`a affine immersa.
\end{definition}


\begin{proposition}
Se $X$ \`e una variet\`a affine immersa in $\K^N$, $X_f$ \`e isomorfa alla variet\`a affine immersa $Y$ data da 
\[V((\cpa{f(x),f(x)t-1}_{f\in I(X)}))\subseteq \K^N_x\times \K_t.\]
\end{proposition}
\begin{proof}
ESERCIZIO.
\end{proof}
\begin{corollary}[]
$\GL(n)$ \`e una variet\`a affine.
\end{corollary}
\begin{proof}
Sia $X=\Mc(n,\K)= \K^{n^2}$. Basta notare che il determinante $\det$ \`e una funzione regolare su $V$ e che $\GL(n)=X_{\det}$.
\end{proof}


\begin{proposition}[]
Se $X$ \`e una variet\`a affine immersa
\[\Hom(X,\K^n)\cong\Hom(\K[x_1,\cdots, x_n],\K[X]).\]
\end{proposition}
\begin{proof}
Un morfismo $\vp:X\to \K^n$ induce un omomorfismo di anelli
\[\vp^\ast:\funcDef{\K[x_1,\cdots, x_n]}{\K[X]}{f}{f\circ \vp}\]
in particolare possiamo definire $f_i=\vp^\ast(x_i)=x_i\circ \vp$ in $\K[X]$ tali che $\vp(x)=(f_1(x),\cdots,f_n(x))$ per definizione.

Viceversa, dati $f_1,\cdots, f_n\in\K[X]$ si ha che $\vp(x)=(f_1(x),\cdots, f_n(x))$ \`e una mappa regolare.
\begin{itemize}
	\item[$\boxed{\text{Cont.}}$] Ovvio.
	\item[$\boxed{\text{Pullback}}$] Sia $U=D(g)\subseteq \K^n$. Notiamo che
	\[\vp\ii(U)=\cpa{x\in X\mid h(x)\doteqdot g(f_1(x),\cdots, f_n(x))\neq 0}=X_h\]
	Se $\al:U\to \K$ \`e regolare, $\al=\ell/g^n$ con $\ell\in\K[x_1,\cdots, x_n]$, quindi
	\[\al\circ \vp=\frac{\ell(f_1,\cdots,f_n)}{h^n}\]
	che \`e un elemento di $\Oc(X_h)$ come voluto.
\end{itemize}
\end{proof}


\begin{proposition}[Morfismi verso affine]
Se $X$ e $Y$ sono variet\`a affini allora un morfismo $\vp:X\to Y$ \`e univocamente determinato dall'omomorfismo $\vp^\ast:\K[Y]\to\K[X]$.
\end{proposition}
\begin{proof}
Sia $\psi:X\to \K^n$ tale che 
	% https://q.uiver.app/#q=WzAsMyxbMCwwLCJYIl0sWzEsMCwiWSJdLFsxLDEsIlxcS15uIl0sWzAsMiwiXFxwc2kiLDJdLFswLDEsIlxcdnAiXSxbMSwyLCJcXHN1YnNldGVxIl1d
	\[\begin{tikzcd}
		X & Y \\
		& {\K^n}
		\arrow["\vp", from=1-1, to=1-2]
		\arrow["\psi"', from=1-1, to=2-2]
		\arrow["\subseteq", from=1-2, to=2-2]
	\end{tikzcd}\]
Notiamo che $\vp$ \`e morfismo se e solo se $\psi$ lo \`e. Se $\psi(X)\subseteq Y$ e $f\in I(Y)$ allora $\psi^\ast(f)=f\circ \vp=0$, quindi abbiamo un diagramma di algebre.

Viceversa se un morfismo di algebre sollevo e bla bla bla trovo morfismo di variet\`a.
\end{proof}


\begin{definition}[Prodotto]
Siano $X,Y$ variet\`a affini immerse in $\K^\ell_x$ e $\K^m_y$ rispettivamente, allora $X\times Y\subseteq \K^{\ell+m}_{(x,y)}$ \`e una variet\`a, data da
\[X\times Y=V((\cpa{f(x),g(y)}_{f\in I(X),g\in I(Y)}))\subseteq \K^\ell\times\K^m.\]
\end{definition}


\begin{proposition}
$\K[X\times Y]\cong \K[X]\otimes\K[Y]$
\end{proposition}
\begin{proof}
L'anello delle coordinate del prodotto di due variet\`a affini \`e dato da
\[\K[X\times Y]=\frac{\K[x_1,\cdots, x_\ell,y_1,\cdots, y_m]}{(f(x),g(y))}\cong \K[X]\otimes \K[Y]\]
Infatti abbiamo un morfismo bilineare da $\K[X]\times\K[Y]\to \K[X\times Y]$ dato dal prodotto dei polinomi. Il morfismo indotto $\Phi $\`e surgettivo perch\'e $x_i=\Phi(x_i\times 1)$ e $y_i=\Phi(1\otimes y_i)$, quindi abbiamo i generatori. Per l'iniettivit\`a procediamo per casi
\begin{itemize}
	\item Supponiamo $X=\K^\ell$ e $Y=\K^m$, allora abbiamo una inversa di $\Phi$ data da
	\[x_i\mapsto x_i\otimes 1,\qquad y_i\mapsto 1\otimes y_i\]
	\item Scrivendo $\K[X\times Y]$ come $\frac{\K[x_1,\cdots, y_m]}{(f(x),g(y))}$ abbiamo un morfismo dal numeratore verso $\K[X]\otimes \K[Y]$ ben definito e notiamo che il nucleo \`e esattamente $(f(x),g(y))$. (dimostrare che protto tensore di $\K$-alegbre ridotte finitamente generate \`e ridotta.)
\end{itemize}
\end{proof}

\begin{definition}[Prodotto fibrato]
Se $X,Y,Z$ variet\`a affini e morfismi $f:X\to Z$ e $g:Y\to Z$, definiamo $W=\cpa{(x,y)\mid f(x)=g(y)}\subseteq X\times Y$. 
% https://q.uiver.app/#q=WzAsNCxbMSwxLCJaIl0sWzAsMSwiWCJdLFsxLDAsIlkiXSxbMCwwLCJXIl0sWzEsMCwiZiIsMl0sWzIsMCwiZyJdLFszLDEsIiIsMix7InN0eWxlIjp7ImJvZHkiOnsibmFtZSI6ImRhc2hlZCJ9fX1dLFszLDIsIiIsMCx7InN0eWxlIjp7ImJvZHkiOnsibmFtZSI6ImRhc2hlZCJ9fX1dLFszLDAsIiIsMSx7InN0eWxlIjp7Im5hbWUiOiJjb3JuZXIifX1dXQ==
\[\begin{tikzcd}
	W & Y \\
	X & Z
	\arrow[dashed, from=1-1, to=1-2]
	\arrow[dashed, from=1-1, to=2-1]
	\arrow["\lrcorner"{anchor=center, pos=0.125}, draw=none, from=1-1, to=2-2]
	\arrow["g", from=1-2, to=2-2]
	\arrow["f"', from=2-1, to=2-2]
\end{tikzcd}\]
\end{definition}

\begin{remark}
Notiamo che $W$ \`e un chiuso, infatti \`e luogo di zeri di $f(x)-g(y)$ in $X\times Y$. Osserviamo per\`o che queste equazioni non sono ridotte a priori\footnote{ed \`e naturale che sia cos\`i. Intersezioni sono in particolare prodotti fibrati e quando intersechiamo due variet\`a \`e naturale trovare delle molteplicit\`a, per esempio $V(x)\cap V(y-x^2)$ in $\A^2$ \`e il punto $(0,0)$ ma con molteplicit\`a 2.}.
\end{remark}


Concludiamo la sezione citando il seguente teorema che useremo ripetutamente

\begin{theorem}[Chevalley]\label{ThChevalley}
    Se $\varphi\colon X \to Y$ è un morfismo tra due varietà, allora $\varphi(X)$ contiene un aperto di $\overline{\varphi(X)}$.
\end{theorem}

\section{Connessione e Irriducibilit\`a}
Sia $X$ uno spazio topologico connesso. Allora
\begin{enumerate}
    \item Se $Y\subset X$ è connesso, allora $\overline{Y}$ è connesso.
    \item Se $\varphi\colon X \to Y$ è continua, allora $\varphi(X)$ è connesso.
\end{enumerate}

\begin{exercise}
    Se $X$ è una varietà algebrica affine, allora \[X \text{ sconnesso } \iff \K[X]=A\times B.\]
\end{exercise}

\begin{exercise}
    Siano $X,Y$ due varietà affini connesse. Allora $X\times Y$ è connessa. [Attenzione: la topologia di $X\times Y$ non è la topologia prodotto!]
\end{exercise}

\begin{exercise}
    Se $X$ è una varietà, allora $X$ ha un numero finito di componenti connesse. In particolare le componenti connesse sono chiuse e aperte.
\end{exercise}

\begin{definition}
    Uno spazio topologico $X$ si dice \textbf{riducibile} se esistono due sottospazi chiusi propri $Z$ e $W$ di $X$ tali che $X=Z\cup W$. Lo spazio $X$ si dice \textbf{irriducibile} se non è riducibile.
\end{definition}

\begin{proposition}
    Se $X$ è uno spazio di Hausdorff irriducibile, allora $X$ è un punto.
\end{proposition}
\begin{proof}
    Se $x,y$ sono punti distinti di $X$, allora esistono due intorni disgiunti $U_x$ e $U_y$ di $x$ e $y$ rispettivamente. Posti $Z=X\setminus U_y$ e $W=X\setminus U_x$, si ottiene $X=Z\cup W$.
\end{proof}

\begin{proposition}
    Sia $X$ una varietà affine. Allora $X$ è irriducibile se e solo se l'anello delle coordinate $\K[X]$ è un dominio d'integrità.
\end{proposition}

\begin{proof}
    Siano $f,g$ in $\K[X]$ tali che $f g=0$. Consideriamo $Z=V(f)$ e $W=V(g)$. Allora $Z$ e $W$ sono due chiusi tali che $X=Z\cup W$, dunque $X=Z$ oppure $X=W$, cioè $f=0$ oppure $g=0$.
\end{proof}

\begin{example}
    La variet\`a $X=\cpa{(x,y)\in \K^2 \mid xy=0}$ non \`e irriducibile.
\end{example}

\begin{remark}
    Se $Y$ è un sottospazio irriducibile di una varietà affine $X$, allora $\overline{Y}$ è irriducibile. Se inoltre $\varphi\colon Y \to X$ è una mappa continua, allora $\varphi(Y)$ è irriducibile. Infine, se $X$ e $Y$ sono spazi topologici irriducibili, allora $X\times Y$ è uno spazio topologico irriducibile.
\end{remark}

Sia $X$ una varietà affine. Ricordiamo che esiste una bigezione tra gli ideali radicali di $\K[X]$ e le sottovarietà chiuse di $X$, data da $I \mapsto V(I)$ e viceversa $Y\mapsto I(Y)$. 

\begin{remark}
    Sia $I$ un ideale di $\K[X]$. Allora \[V(I) \text{ è irriducibile} \quad \iff \quad I \text{ è un ideale primo.}\]
\end{remark}

Ricordiamo che, se $I$ è un ideale radicale allora $V(I)$, come spazio topologico, è omeomorfo alla varietà affine avente come anello di coordinate $\K[X]/I$. L'insieme $V(I)$ è in corrispondenza biunivoca con l'insieme degli ideali massimali di $\K[X]$ contenenti $I$. D'altra parte, gli ideali massimali di $\K[X]/I$ sono gli ideali massimali di $\K[X]$ contenenti $I$. Se $\pi\colon \K[X]\to \K[X]/I$ è la proiezione canonica, si ha una corrispondenza biunivoca 
\[\correspDef{\cpa{\text{Massimali di $\K[X]$ che contengono $I$}}}{\cpa{\text{Massimali di $\K[X]/I$}}}{\mf}{\mf/I}{\pi\ii(M)}{M}\]


\begin{definition}
    Uno spazio topologico $X$ si dice \textbf{Noetheriano} se ogni successione  \[Z_1\supseteq Z_2 \supseteq Z_3 \supseteq \ldots\]
    di sottospazi chiusi di $X$ stabilizza.
\end{definition}

\begin{remark}
    Ogni varietà affine è uno spazio topologico Noetheriano.
\end{remark}
\begin{proof}
    Se $Z_1\supseteq Z_2 \supseteq \ldots \supseteq Z_n \supseteq\ldots$ è una successione di sottospazi chiusi di $X$, allora si ha una successione  \[I(Z_1)\subseteq I(Z_2) \subseteq \ldots \subseteq I(Z_n)\subseteq \ldots\]  di ideali di $\K[X]$ che stabilizza perché $\K[X]$ è Noetheriano. Quindi anche la successione $Z_1\supseteq Z_2 \supseteq \ldots \supseteq Z_n \supseteq\ldots$ stabilizza.
\end{proof}

\begin{remark}
    Ogni varietà è uno spazio topologico Noetheriano.
\end{remark}
\begin{proof}
Scrivendo $X=X_1\cup\ldots\cup X_n$ con $X_i$ aperti affini e data una successione di chiusi $Z_1\supseteq Z_2 \supseteq \ldots \supseteq Z_n \supseteq\ldots$, si ha una successione $Z_1\cap X_i\supseteq Z_2 \cap X_i\supseteq \ldots \supseteq Z_n  \cap X_i\supseteq\ldots$  che stabilizza per ogni $i$. Allora esiste un $N$ tale che $Z_n\cap X_i= Z_N \cap X_i$ per ogni $n\ge N$ e per ogni $i$, da cui $Z_n=Z_N$ per ogni $n\ge N$.
\end{proof}

\begin{proposition}
    Sia $X$ uno spazio topologico Noetheriano. Allora esistono dei sottospazi $Y_1,\ldots,Y_n$ chiusi e irriducibili di $X$ tali che $X=Y_1\cup\ldots\cup Y_n$. Inoltre, se $Y_i\not\subseteq Y_j$ per ogni $i\ne j$, una tale decomposizione di $X$ è unica, e i sottospazi $Y_i$ si dicono le \textbf{componenti irriducibili} di $X$.
\end{proposition}

\begin{proof}
    Consideriamo la famiglia $\mathcal{F}$ dei chiusi di $X$ che non possono essere scritti come unione di un numero finito di chiusi irriducibili. Mostriamo che $\mathcal{F}$ è vuota.

    Supponiamo per assurdo che $\mathcal{F}$ sia non vuota. Poiché ogni catena in $\mathcal{F}$ ammette un minimo, per il Lemma di Zorn esiste un elemento minimale $Z$ in $\mathcal{F}$. Per costruzione, $Z$ è riducibile, quindi possiamo scrivere $Z=C\cup D$ con $C,D$ sottospazi chiusi e propri di $Z$. Poiché $Z$ è un elemento minimale, $C,D$ non sono in $\mathcal{F}$, quindi ammettono una decomposizione finita in chiusi irriducibili, ma allora anche $Z$ ammette una tale decomposizione. 

    Supponiamo ora che $X$ ammetta due decomposizioni in chiusi irriducibili \[X=Y_1\cup\ldots\cup Y_m=Z_1\cup\ldots\cup Z_n\] tali che $Y_i\not\subseteq Y_j$ e $Z_i\not\subseteq Z_j$ per ogni $i\ne j$. Poiché \[Y_i=Y_i\cap (Z_1\cup\ldots\cup Z_n)=Y_i\cap Z_1\cup\ldots\cup Y_i\cap Z_n\] e poiché $Y_i$ è irriducibile, si ha $Y_i\subseteq Z_{\alpha(i)}$. Analogamente, $Z_j\subseteq Y_{\beta(i)}$. Allora $\beta\circ\alpha=id$, infatti \[Y_i\subseteq Z_{\alpha(i)}\subseteq Y_{\beta(\alpha(i))}=Y_i.\]
    Analogamente $\alpha\circ\beta=id$ e ciò conclude.
\end{proof}

\begin{remark}[Componenti irriducibili corrispondono a primi minimali]
    Siano $I,J_\alpha$ ideali di $\K[x_1,\ldots,x_n]$ tali che i $J_\alpha$ siano primi contenti $I$. Siano $X=V(I)$ e $Y_\alpha=V(J_\alpha)$. Allora 
    \[X=Y_1\cup \ldots \cup Y_n,\] 
    dove $J_1,\ldots,J_n$ sono i primi minimali contenenti $I$. Sappiamo dal corso di Algebra 2 che i primi minimali sono in numero finito.
\end{remark}

\begin{example}
    $I=(zx,zy)$. $p_1=(z)$ è il piano, $p_2=(x,y)$ è la retta, $X=V(p_1)\cup V(p_2)$.
\end{example}


\section{Dimensione di una varietà algebrica}

\begin{definition}
    Sia $X$ uno spazio topologico. Definiamo la \textbf{dimensione (di Krull)} di $X$ come 
    \[\sup\left\{n\in\N \colon \text{ esiste una catena di chiusi irriducibili } Z_0\subsetneq Z_1 \subsetneq \ldots \subsetneq Z_n\right\}.\]
    Denotiamo questo numero $\dim X$.
\end{definition}

\begin{remark}
    Se $X$ è una varietà affine, per quanto già visto si ha 
    \[\dim(X)=\sup\left\{n\in\N\mid\text{ esistono primi $P_0,\cdots, P_n$ di $\K[X]$ : } P_0\supsetneq P_1 \supsetneq \ldots \supsetneq P_n\right\}.\] 
\end{remark}

\begin{theorem}
    Ogni varietà affine ha dimensione finita.
\end{theorem}



\begin{definition}[Altezza]
Se $\pf$ è un ideale primo di $\K[X]$, la sua \textbf{altezza} è definita come 
\[\operatorname{ht}(\pf)\doteqdot \sup\left\{n\in\N \mid \text{ esistono primi $\pf_0,\cdots, \pf_n$ di $\K[X]$ : } \pf_0\subsetneq \cdots \subsetneq \pf_n=\pf\right\}\]
\end{definition}



\begin{definition}[Campo delle funzioni razionali]
Se $X$ \`e una variet\`a affine irriducibile definiamo il suo \textbf{campo delle funzioni razionali} $K_X$ come il campo dei quozienti\footnote{la definizione \`e ben posta perch\'e se $X$ \`e irriducible allora $\K[X]$ \`e un dominio} di $\K[X]$.
\end{definition}

\begin{theorem}
    Sia $X$ una varietà affine irriducibile. Allora \begin{enumerate}
        \item $\dim(X)=\operatorname{tr\,deg}_\K(K_X)$. 
        \item Se $\pf$ è un ideale primo di $K[X]$ e $Y=V(\pf)$, allora \[\dim(X)=\dim(Y)+\operatorname{ht}(\pf).\]
        \item Se $f$ è un elemento non nullo e non invertibile di $\K[X]$, allora $V(f)$ è un chiuso di $X$. Se la sua decomposizione in componenti irriducibili è data da $V(f)=Y_1\cup\ldots\cup Y_h$, allora \[\dim(Y_i)=\dim(X)-1.\]
    \end{enumerate}
\end{theorem}
















\section{Variet\`a liscie}
\begin{notation}
Sia $X$ una varietà (affine) e sia $x$ un punto di $X$. In questa sezione poniamo $A=\K[X]_{\mf_x}$ e $\mf_x=\{f\in A\colon f(x)=0\}$.
\end{notation}


\begin{definition}
    Lo \textbf{spazio cotangente} di $X$ nel punto $x$ è il quoziente 
    \[T^\ast_x X:= \mf_x/\mf_x^2.\]
    Dato un elemento $f$ in $A$, definiamo il \textbf{differenziale} di $f$ come la classe 
    \[df:= [f-f(x)]\in T^\ast_x X.\] 
    Lo \textbf{spazio tangente} di $X$ nel punto $x$ è il duale 
    \[T_x X=(\mf_x/\mf_x^2)^\ast\]
    dello spazio cotangente di $X$ in $x$.
\end{definition}

\begin{definition}
    Una \textbf{derivazione} su $A$ in $x$ è una funzione $\partial\colon A \to \K$ lineare che soddisfa la seguente proprietà \[\partial(fg)=f(x)\partial g + g(x)\partial f.\]
    Indichiamo con $\Der_x(A,\K(x))$ l'insieme delle derivazioni su $A$.
\end{definition}

\begin{exercise}
    La mappa $T_xX \to \Der_x(A,\K)$ definita da $\varphi\mapsto \widetilde{\varphi}$, dove \[\widetilde{\varphi}(\la+\mu)=\varphi(\mu)\] definisce una bigezione (stiamo usando la decomposizione $A=\K\oplus \mf_x$).
\end{exercise}

\begin{example}
    Se $X=\K$ e $p$ è un punto di $\K$, allora $\mf=\mf_p=(x-p)$ e $\mf/\mf^2=\K(x-p)=\K d_p x$.
\end{example}

\begin{example}
    Se $X=\K^n$ e $p=(p_1,\ldots,p_n)$, allora $\mf=\mf_p=(x_1-p_1,\ldots,x_n-p_n)$ e si ha \[\Der_p(\K[x_1,\ldots,x_n],\K)=\bigoplus_{i=1}^n \K \left.\frac{\partial}{\partial x_i}\right\lvert_{p}\]
\end{example}

\begin{example}
    Se $X=V(I)$ è contenuta in $\K^n$ e $p$ è un punto di $X$, allora $I\subseteq \mf=(x_1-p_1,\ldots,x_n-p_n)$. Posto $\overline{\mf}=\mf/I$, si ha \[\frac{\overline{\mf}}{\overline{\mf}^2}=\frac{\mf}{\mf^2+I} \] e si ha un'immersione 
    \[\Der_p(A/I,\K)=\left\{\partial\in \Der_p(A,\K) \colon \partial(I)=0\right\}\hookrightarrow \Der_p(A,\K) \]
    data da $\partial\mapsto \partial\circ \pi$, dove $\pi\colon A \to A/I$ è la proiezione canonica.
\end{example}

\begin{remark}
    Se $\varphi\colon X \to Y$ è un morfismo di varietà e $p$ è un punto di $X$, posto $q=\varphi(p)$, si ha una mappa \[d_p\varphi \colon  T_qX \to  T_p Y\] definita tramite
    \[\begin{array}{ccc}
         \Der_p(A,\K) &\to & \Der_q(B,\K)  \\
        \partial &\mapsto & \partial\circ \varphi
    \end{array}\]
    dove $B=\K[Y]$. 
\end{remark}

\begin{example}
Se $Y=\K$ e $\varphi(p)=q$, la mappa $d_p\varphi\colon T_p X \to T_q \K=\K \left.\frac{\partial}{\partial x}\right\lvert_q$ è definita da $\partial \mapsto \widetilde{\partial}$, dove 
\[\widetilde{\partial}(f)=\partial(f\circ \varphi).\] 
Se $\widetilde{\partial}(x)=\lambda$, allora $\widetilde{\partial}=\lambda\left.\frac{\partial}{\partial x}\right\lvert_q $.
\end{example}

\begin{remark}
Se $X$ è affine e $\varphi$ è in $A$, allora $\varphi$ definisce una mappa $\K[x]\to A$ data da $g\mapsto g\circ \varphi$. In particolare $x\mapsto \varphi$. Quindi \[\widetilde{\partial}(x)=\partial(x\circ \varphi)=\partial(\varphi).\]
\end{remark}


\begin{remark}
Nel caso di $\varphi=(\varphi_1,\ldots,\varphi_n)\colon X \to \K^n$, si ha \[d_p\vp:\funcDef{T_pX}{T_q \K^n}{\del}{\sum_{i=1}^n \partial(\varphi_i)\left.\frac{\partial}{\partial x_i}\right\lvert_q}\]
Nel caso in cui $X$ sia $\K^m$ (con coordinata $y$), si ha
\[d_p\vp:\funcDef{T_p\K^m}{T_q \K^n}{\left.\frac{\partial}{\partial y_j}\right\lvert_p}{\displaystyle\sum_{i=1}^n \left.\left.\frac{\partial \vp_i}{\partial y_j} \right\lvert_p\frac{\partial}{\partial x_i}\right\lvert_q}\]
cio\`e $d_p\vp=\Dc_p\vp$ a meno di rinominare i vettori base.
\end{remark}



\begin{theorem}
    Se $X$ è una varietà irriducibile di dimensione $n$, allora \[\dim(\mf/\mf^2)=\dim T_pX \ge n.\]
\end{theorem}

\begin{definition}
    Sia $X$ una varietà di dimensione $n$. Un punto $p$ di $X$ si dice \textbf{liscio} se $\dim \mf_p/\mf_p^2 =n$.
\end{definition}

\begin{remark}
Se $X$ è una varietà irriducibile di dimensione $d$ contenuta in $\K^n$, consideriamo il punto $p=0$ in $X$. Vogliamo capire se $0$ è liscio. Scriviamo $X=V(I)$, con $I=(f_1,\ldots,f_r)$. In questo caso $\mf=\mf_0=(x_1,\ldots,x_n)$. I generatori $f_i$ di $I$ si scrivono come $f_i=\ell_i+f_{1,i}$ con $\ell_i$ lineare omogeneo e $f_{1,i}$ di grado superiore a $\deg(\ell_i)$. Quindi $\mf=\mf^2+\K x_1\oplus \ldots \oplus \K x_n$, da cui \[\frac{\overline{\mf}}{\overline{\mf}^2}=\frac{\mf}{\mf^2+I}=\frac{\mf}{\mf^2+\langle
\ell_1,\ldots,\ell_r\rangle}=\frac{\K x_1\oplus \ldots \oplus \K x_n}{\langle
\ell_1,\ldots,\ell_r\rangle}.\] Dunque $X$ è liscio in $0$ se $\dim_\K\langle
\ell_1,\ldots,\ell_r\rangle=n-d$.
\end{remark}

\begin{example}
    Consideriamo $X$ la curva affine definita dall'equazione $y^2=x^3$ (notiamo che $\dim X=1$). Posto $f=y^2-x^3$, si ha\footnote{$y^2-x^3$ non ha parte lineare}
    \[\frac{\mf_0}{\mf_0^2}=\frac{\K x\oplus \K y}{0},\] 
    che ha dimensione $2$, quindi $X$ non è liscia in $(0,0)$. Se invece consideriamo $X$ definita da $y^2=x^3+x$, allora 
    \[\frac{\mf_0}{\mf_0^2}=\frac{\K x\oplus \K y}{x}\]
    che ha dimensione $1$, quindi $X$ è liscia in $0$.
\end{example}

\begin{example}
    Se $X=V(I)$ in $\K^n$, con $I=(f_1,\ldots,f_r)$, consideriamo la mappa \[f=(f_1,\ldots,f_r)\colon \K^n \to \K^r.\] Allora 
    \[d_0f:\funcDef{T_0 \K^n_x}{T_0 \K^r_y}{\del}{\sum_j \left.\partial(f_j)\frac{\partial}{\partial x_j}\right\lvert_0}\]
    e \[T_0 X=\{\partial\colon \partial(I)=0\}=\{\partial\colon \partial(f_j)=0 \text{ per ogni }j=1,\ldots,r\}=\ker d_0f.\]
\end{example}

\begin{theorem}
    Se $X$ è una varietà, allora esiste un aperto $U$ in $X$ costituito da punti lisci.
\end{theorem}



\begin{theorem}[Zariski]\label{ThZariski}
Supponiamo $\cha(\K)=0$. Se $X$ \`e irriducibile e $Y$ \`e liscia, ogni morfismo $\varphi\colon X\to Y$ bigettivo è un isomorfismo.    
\end{theorem}

In caratteristica $p$ il teorema va enunciato in modo diverso: $\K=\overline{\F_p}$, $\varphi\colon \K \to \K$ definito da $x\mapsto x^p$.


\begin{definition}
    Sia $\varphi\colon X\to Y$ un morfismo di varietà lisce. Diciamo che $\varphi$ è \textbf{liscio}\footnote{In geometria differenziale sarebbe una sommersione.} in un punto $p$ di $X$ se la mappa $d_p\varphi\colon T_pX\to T_{\varphi(p)}Y$ è surgettiva.
\end{definition}

La definizione di morfismo liscio \`e il seguente risultato
\begin{theorem}\label{ThFibrePerMappeLiscie}
Siano $X$ e $Y$ variet\`a lisce e $\vp:X\to Y$ una mappa liscia, allora
\begin{enumerate}
	\item $X_y=\vp\ii(y)$ \`e una variet\`a liscia di dimensione $\dim X-\dim Y$
	\item Se $x\in X_y$ allora
	\[0\to T_x X_y\to T_x X\xrightarrow{d\vp_x} T_y Y\to 0\]
    cio\`e $T_x (\vp\ii(y))=\ker d\vp_x$.
\end{enumerate}
\end{theorem}


Enunciamo altri teoremi utili sulle mappe lisce:
\begin{theorem}\label{ThMorfismoDominanteTraLisceHaRestrizioneLisciaSuAperto}
    Assumiamo $\cha(\K)=0$. Se $\varphi\colon X\to Y$ è un morfismo \textbf{dominante} di varietà lisce (cioè $\overline{\varphi(X)}=Y$), allora esiste un aperto non vuoto $U$ di $Y$ tale che la restrizione \[\varphi\res{\varphi^{-1}(U)}\colon \varphi^{-1}(U) \to U\] è liscia.
\end{theorem}

\begin{theorem}\label{ThMappaLisciaEAperta}
    Ogni mappa $\varphi\colon X \to Y$ liscia è aperta. Inoltre, per ogni varietà $Z$ (non necessariamente liscia), la mappa \[(\varphi,id)\colon X\times Z \to Y\times Z\] è aperta\footnote{in quanto ``liscia" per una definizione pi\`u generale. Per pi\`u dettagli cercare ``essere un morfismo liscio \`e stabile per cambiamento di base".}.
\end{theorem}






\section{Variet\`a separate e complete}

Introdiciamo gli analoghi di Hausdorff e Compatto in geometria algebrica

\begin{definition}[Variet\`a separata]
$X$ \`e \textbf{separata} se la diagonale $X\to X\times X$ \`e un morfismo chiuso.
\end{definition}
\begin{center}
	\textbf{Supporremo che sia tutto separato.}
\end{center}

\begin{definition}[Variet\`a completa]
$X$ \`e \textbf{completa} se per ogni $Z$ variet\`a, $\pi:X\times Z\to Z$ \`e un morfismo chiuso.
\end{definition}

\begin{proposition}\label{PrRegolariSuCompleteIrriducibiliSonoCostanti}
Se $X$ \`e irriducibile e completa allora $\Oc_X(X)\cong \K$.
\end{proposition}
\begin{proof}
Sia $f:X\to \K$ regolare e consideriamo $\Gamma(f)\subseteq X\times \K\to \K$. Questa proiezione \`e chiusa quindi $\pi(\Gamma(f))=f(X)$ \`e un chiuso irriducibile (immagine di irriducibile) di $\K$. Ma i chiusi irriducibili di $\K$ non vuoti sono o tutto $\K$ o solo un punto.

Consideriamo ora $W=\cpa{(x,y)\in X\times \K\mid yf(x)=1}$. Se $f(X)=\K$ allora l'immagine di $W$ tramite la proiezione $X\times \K\to \K$ \`e $\K^\times$, che non \`e chiuso in $\K$, assurdo.

Quindi $f(X)$ \`e un solo punto, cio\`e $f$ \`e costante.
\end{proof}
\begin{corollary}
Se $X$ \`e completa, affine e connessa allora $\K[X]=\K$ significa che $X$ \`e un solo punto $\cpa{(0)}$.

In generale le affini complete sono un numero finito di punti.
\end{corollary}

\begin{theorem}[I proiettivi sono completi]\label{ThProiettiviSonoCompleti}
Se $V$ \`e uno spazio vettoriale di dimensione finita, $\Pj(V)$ \`e completa.
\end{theorem}


\begin{remark}
Se $X$ \`e completa e $Z\subseteq X$ chiuso allora $Z$ \`e completo\footnote{Questo corrisponde a ``chiuso in un compatto \`e compatto" in senso classico.}.
\end{remark}

\begin{remark}
Se $\vp:X\to Y$ \`e surgettiva e $X$ \`e completa allora $Y$ \`e completa\footnote{Questo corrisponde a ``immagine di un compatto \`e compatta" in senso classico.}.
\end{remark}