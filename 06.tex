\chapter{Quozienti}
\section{Costruzione dei quozienti}
\begin{lemma}\label{LmSottogruppoEStabilizzatoreDiUnaRettaInQualcheRappresentazione}
Sia $G$ un gruppo algebrico e $H$ sottogruppo di $G$, allora
\begin{enumerate}
    \item Esistono una rappresentazione di dimensione finita $V$ di $G$ e una retta $L\subseteq V$ tali che $H=\stab_G L=\cpa{g\in G\mid g(L)=L}$
    \item Se $H$ \`e normale allora $V$ si pu\`o scegliere in modo che sia somma dei $V_\al$ per $\al\in X(H)$ e $V_\al=\cpa{v\in V\mid h\cdot v=\al(h)v}$.
\end{enumerate}
\end{lemma}
\begin{proof}
Mostriamo le due affermazioni
\setlength{\leftmargini}{0cm}
\begin{enumerate}
\item Sia $I_H\subseteq \K[G]$ l'ideale che definisce $H$, allora
\[H=\cpa{g\in G\mid g(I_H)=I_H}\]
\setlength{\leftmargini}{0cm}
\begin{itemize}
\item[$\boxed{\subseteq}$] Se $g\in H$ e $f\in I_H$ allora $gf(k)=f(g\ii k)$, quindi se $k\in H$ allora $g\ii k\in H$ e quindi questa funzione vale $0$, cio\`e $gf\in I_H$. L'altra inclusione segue dallo stesso ragionamento fatto su $g\ii$.
\item[$\boxed{\supseteq}$] Se $g\ii(I_H)\subseteq I_H$ allora per ogni $f\in I_H$
\[f(g)=\under{\in I_H}{(g\ii f)}(e)=0\]
cio\`e $g\in H$.
\end{itemize}
\setlength{\leftmargini}{0.5cm}
Consideriamo ora dei generatori $f_1,\cdots, f_m$ per $I_H$ e sia $V_0\subseteq \K[G]$ una $G$-sotto-rappresentazione di dimensione finita che contiene ogni $f_i$. Consideriamo il sottospazio vettoriale $W_0=I_H\cap V_0$ e notiamo che
\[H=\cpa{g\in G\mid g(W_0)= W_0}.\]
\setlength{\leftmargini}{0cm}
\begin{itemize}
\item[$\boxed{\subseteq}$] Ovvio per quanto detto sopra.
\item[$\boxed{\supseteq}$] Se $g(W_0)= W_0$ allora $g(I_H)=I_H$, infatti $g(I_H)\subseteq I_H$ ovvio per costruzione di $W_0$, l'altra inclusione segue dal fatto che $g(W_0)=W_0\coimplies g\ii(W_0)=W_0$.
\end{itemize}
\setlength{\leftmargini}{0.5cm}
Sia $\dim W_0=m$. Poniamo
\[V=\bw^m V_0,\quad L=\bw^m W_0\subseteq \bw^m V_0.\]
Per concludere basta mostrare che
\[H=\cpa{g\in G\mid g(L)=L}\]
\setlength{\leftmargini}{0cm}
\begin{itemize}
\item[$\boxed{\subseteq}$] Se $g(W_0)=W_0$ allora chiaramente 
\[g(L)=g\pa{\bw^mW_0}=\bw^m g(W_0)=\bw^m W_0=L.\] 
\item[$\boxed{\supseteq}$] Notiamo che se $u_1,\cdots, u_m$ \`e una base di $U\subseteq V$ sottospazio vettoriale allora 
\[U=\cpa{u\in V\mid u\wedge u_1\wedge\cdots\wedge u_m=0}.\]
Fissiamo una base $w_1,\cdots, w_m$ di $W_0$ e osserviamo che $g w_1,\cdots, gw_m$ \`e una base di $g(W_0)$. Se $g(L)=L$ allora per definizione
\[\ps{g w_1\wedge\cdots\wedge gw_m}=\ps{w_1\wedge\cdots\wedge w_m},\] 
quindi per il criterio appena citato si ha che
\begin{align*}
    W_0=&\cpa{v\in V\mid v\wedge w_1\wedge\cdots\wedge w_m=0}=\\
    =&\cpa{v\in V\mid u\wedge gw_1\wedge\cdots\wedge gw_m=0}=g(W_0).
\end{align*}
\end{itemize}
\setlength{\leftmargini}{0.5cm}
\item Sia $V'=\bigoplus_{\al\in X(H)} V_\al\subseteq V$ con $V$ di prima. Mostriamo che $V'$ \`e $G$-invariante:
Se $v_\al\in V_\al$ per $\al\in X(H)$, $h\in H$ e $g\in G$ allora
\[h\cdot(g v_\al)=(gg\ii h g)\cdot v_\al=g(\al(g\ii hg)v_\al)=\al(g\ii hg)gv_\al,\]
cio\`e $gv_\al$ \`e un autovettore per l'azione di $h$ per un qualsiasi $g\in G$ e $h\in H$, ovvero $V'$ \`e $G$-invariante. 

Per concludere \`e dunque sufficiente mostrare che $L\subseteq V'$, ma abbiamo gi\`a visto che 
\[g(w_1\wedge\cdots \wedge w_m)=gw_1\wedge\cdots \wedge gw_m=\la(g) w_1\wedge\cdots\wedge w_m\] 
per qualche $\la(g)$ per ogni $g\in H$, quindi $L\subseteq V_\la$.
\end{enumerate}
\setlength{\leftmargini}{0.5cm}
\end{proof}

Siano $H< G$ e $L,V$ come nel lemma. Allora $L\in \Pj(V)$ per definizione di spazio proiettivo. Definiamo la variet\`a proiettiva
\[Y=\ol{G\cdot L}\subseteq \Pj(V)\]
e scriviamo $X=G\cdot L$. Mostriamo che $X$ \`e aperto: data la mappa
\[\funcDef{G}{Y}{g}{gL},\]
per Chevalley (\ref{ThChevalley}) si ha che $X$ contiene un aperto $U$ di $Y$. Se $x\in U\subseteq X$ allora $gx\in gU\subseteq X$ quindi
\[X=\bigcup_{g\in G}gU\text{ \`e aperto.}\]
Insiemisticamente si ha $X=G\cdot L=G/\stab_G(L)=G/H$, quindi prendere la chiusura \`e in un qualche modo il minimo indispensabile per rendere $G/H$ una variet\`a.


\begin{exercise}
Sia $G=\GL(2)$ e $H=B(2)=\cpa{\mat{\ast&\ast\\0&\ast}}$. Sia $V=\K^2$ e $L=\K e_1$, allora effettivamente $H=\stab_G(L)$ e $G\cdot L=\Pj(V)$ (ogni retta si ottiene da $L$ applicando una trasformazione lineare), quindi $G/H\leftrightarrow \Pj(V)$.
\end{exercise}

\begin{exercise}
Sia $G=\GL(n,\C)$ e $H=O(n,\C)$. Sia $V=\Sym(n,\C)$ lo spazio delle matrici simmetriche. Se $A\in V$ e $g\in G$ agisce su $A$ tramite 
\[g\cdot A=gA g^\top\]
allora $H=\stab_G(I_n)$ (stabilizzatore della matrice identit\`a).
\[X=G\cdot I_n=\cpa{gg^\top\mid g\in \GL(n)}=\cpa{A\in \Sym(n)\mid \det(A)\neq 0}\subseteq\ol{G\cdot I_n}=Y\subseteq \Pj(V).\]
%Siano $V'=V\oplus \K$ e $L=\ps{(I_n,1)}$, allora \[\stab_G(L)=H\text{ e }\Pj(V')=V\oplus \Pj(V).\]
\end{exercise}

\begin{proposition}
Se $H$ \`e normale, $G/H$ \`e un gruppo algebrico affine.
\end{proposition}
\begin{proof}
Costruiamo $L$ e $V=\bigoplus V_\al$ come nel punto 2. del lemma (\ref{LmSottogruppoEStabilizzatoreDiUnaRettaInQualcheRappresentazione}). Sia
\[W=\cpa{T:V\to V\mid \forall \al\in X(H),\ T(V_\al)\subseteq V_\al}\]
e notiamo che $G$ agisce su $W$ come $gT=g\circ T\circ g\ii$. Per rendere valido quanto detto dobbiamo verificare che $g T g\ii(V_\al)=V_\al$, ma questo segue dal fatto che se $g\ii(V_\al)=V_\beta$ allora
\[g T g\ii(V_\al)=gT(V_\beta)\subseteq gV_\beta=V_\al.\]
Notiamo ora che
\[\cpa{g\in G\mid g\res W=id_W}=\cpa{g\in G\mid g\res{V_\al}=\la_\al id_{V_\al}\ \forall \al\in X(H)}=H,\]
la seconda uguaglianza segue dalla definizione di carattere mentre la prima si ricava osservando le matrici associate agli elementi di $g$ visti come automorfismi di $V$: gli elementi di $W$ sono diagonali a blocchi e ci\`o che commuta\footnote{$g\res W=id_W$ significa che per ogni $T\in W$ $gTg\ii=T$, cio\`e $gT=Tg$.} con tutte le diagonali a blocchi sono le cose che sono multiplo di identit\`a su ogni blocco.

Abbiamo dunque costruito un omomorfismo di gruppi algebrici $\vp:G\to \GL(W)$ il cui nucleo \`e $H$. Poich\'e $\vp(G)$ \`e un sottogruppo chiuso di $\GL(W)$ per Chevalley (\ref{PrMorfismoTraGruppiAlgebriciAffiniHaImmagineChiusa}) si ha che $G/H\cong \vp(G)$ eredita la struttura di gruppo algebrico lineare da $\vp(G)$.
\end{proof}

















\begin{remark}
    Se $H$ è un sottogruppo di $G$ e $\pi\colon G \to X=G\cdot L$, allora $\pi$ è $G$-equivariante e induce una bigezione tra $G/H$ e $X$.
\end{remark}


Nel seguito supporremo $\cha(\K)=0$.

\begin{proposition} Se $H$ è un sottogruppo di $G$ e $\pi\colon G \to X$ una mappa $H$-equivariante, valgono le seguenti proprietà.
    \begin{enumerate}
        \item Per ogni varietà $Z$, la mappa \[G\times Z \xrightarrow{(\pi,id)}X\times Z\] è aperta.
        \item Per ogni aperto $U$ di $X$, si ha un isomorfismo \[\pi^\ast:\Oc_X(U)\xrightarrow{\sim}(\Oc_G(\pi^{-1}(U)))^H.\]
    \end{enumerate}
\end{proposition}

\begin{proof}
Notiamo che $\pi\colon G \to X$ induce una bigezione insiemistica tra $G/H$ e $X$. Poiché per ogni varietà $Z$ la mappa $(\pi,id)\colon G\times Z \to X\times Z$ è liscia ($\pi$ è liscia), tale mappa è anche aperta (\ref{ThMappaLisciaEAperta}).

Sia $U$ un aperto di $X$, poniamo $V=\pi\ii(U)$, e consideriamo la mappa
\[\pi^\ast:\Oc_X(U)\to \Oc_G(V)^H.\]
Osserviamo che tale mappa è iniettiva, infatti se $f(\pi(x))=0$ per ogni $x$ in $V=\pi\ii (U)$, si ha $f(y)=0$ per ogni $y$ in $U$, per cui $f=0$. 

Mostriamo che la mappa $\pi^{-1}(U)\to U$ è surgettiva.
Posso ridurmi al caso $U$ irriducibile:

In $G$ le componenti connesse coincidono con le componenti irriducibili (\ref{ReInGruppoComponentiConnesseSonoIrriducibili}). Lo stesso vale per $X$, infatti se $G^0$ \`e la componente connessa di $1_G$ allora da $X=G/H$ troviamo che $X^0=G^0/H\cap G^0$ \`e aperto e $X$ \`e unione finita disgiunta dei traslati di $X^0$ ($G$ ha finite componenti irriducibili per Noetherianit\`a).

Se $U$ \`e un aperto di $X$ allora 
\[X=X^0\sqcup g_1X^0\sqcup\cdots\sqcup g_nX^0\implies U=U\cap X^0\sqcup \cdots\sqcup U\cap g_n X^0.\]
Poich\'e $X^0$ irriducibile i suoi aperti sono irriducibili, quindi per localit\`a della verifica di surgettivit\`a sui pullback per aperti di $X^0$, ci siamo quindi ricondotti al caso $X$ e $U$ irriducibili.

Sia $f:V\to \K$ regolare $H$-equivariante e consideriamo il grafico $\Gamma(f)\subseteq V\times \K$. Sia $g$ la fattorizzazione 
% https://q.uiver.app/#q=WzAsMyxbMCwwLCJWIl0sWzIsMCwiVT1WL0giXSxbMCwyLCJcXEsiXSxbMCwyLCJmIiwyXSxbMCwxLCJcXHBpIl0sWzEsMiwiZyIsMCx7InN0eWxlIjp7ImJvZHkiOnsibmFtZSI6ImRhc2hlZCJ9fX1dXQ==
\[\begin{tikzcd}
	V && {U=V/H} \\
	\\
	\K
	\arrow["\pi", from=1-1, to=1-3]
	\arrow["f"', from=1-1, to=3-1]
	\arrow["g", dashed, from=1-3, to=3-1]
\end{tikzcd}\]
Decomponiamo $V$ in irriducibili $V=V_1\sqcup\cdots\sqcup V_n$ e per irriducibilit\`a di $U$ deve essere il caso che $H$ agisce transitivamente su $\cpa{V_1,\cdots, V_n}$.

Per il punto 1., $(\pi,id):G\times \K\to X\times\K$ \`e una mappa aperta, quindi la restrizione $\psi:V\times \K\to U\times \K$ resta aperta perch\'e $V\times \K$ \`e un aperto di $G\times \K$. Segue che\footnote{Se $(v,f(v))\in\Gamma(f)$ allora $\psi(v,f(v))=(\pi(v),f(v))=(\pi(v),g(\pi(v)))$.} $\psi(\Gamma(f))=\Gamma(g)$ \`e un chiuso di $U\times \K$. In realt\`a notando che $\Gamma(g)$ \`e $H$ invariante e che esso \`e immagine di $\Gamma(f)=\bigsqcup \Gamma(f)\cap V_i$ si ha che in realt\`a possiamo scrivere $\Gamma(g)$ solo come immagine di un singolo $\Gamma(f)\cap V_i$, in particolare \`e irriducible (altrimenti potremmo decomporre $V$ ulteriormente).

Consideriamo ora il diagramma
% https://q.uiver.app/#q=WzAsMyxbMCwwLCJcXEdhbW1hKGcpIl0sWzIsMCwiVVxcdGltZXMgXFxLIl0sWzIsMiwiVSJdLFswLDEsIiIsMCx7InN0eWxlIjp7InRhaWwiOnsibmFtZSI6Imhvb2siLCJzaWRlIjoidG9wIn19fV0sWzEsMiwicF9VIl0sWzAsMiwicSIsMl1d
\[\begin{tikzcd}
	{\Gamma(g)} && {U\times \K} \\
	\\
	&& U
	\arrow[hook, from=1-1, to=1-3]
	\arrow["q"', from=1-1, to=3-3]
	\arrow["{p_U}", from=1-3, to=3-3]
\end{tikzcd}\]
dove $p_U$ \`e la proiezione su $U$ e $q$ \`e la restrizione a $\Gamma(g)$. Chiaramente $q$ \`e regolare in quanto composizione di regolari. $q$ \`e anche bigettiva perch\'e $(u,g(u))\mapsto u$ pu\`o essere facilmente invertita. Poich\'e $U$ \`e liscio e $\Gamma(g)$ irriducibile, per il teorema di Zariski (\ref{ThZariski}) si ha che $q$ \`e un isomorfismo, quindi la mappa $u\mapsto (u,g(u))$ un morfismo e in particolare $g$ stesso \`e un morfismo. Questo mostra che $\pi^\ast$ effettivamente \`e surgettiva perch\'e abbiamo trovato $g\in \Oc_X(U)$ tale che $\pi^\ast(g)=g\circ \pi=f$.
\end{proof}


\begin{theorem}
    Per ogni $G$-varietà $Y$ e per ogni $y_0$ in $Y$ tale che $H$ è contenuto in $\stab_G(y_0)$, vale la seguente proprietà: se $\varphi\colon G \to Y$ è la mappa definita da $\varphi(g)=gy_0$, allora esiste un'unica $\psi\colon X \to Y$  tale che $\psi\circ\pi=\varphi$.
    % https://q.uiver.app/#q=WzAsMyxbMCwwLCJHIl0sWzIsMCwiWCJdLFsxLDEsIlkiXSxbMCwxLCJcXHBpIl0sWzAsMiwiXFx2YXJwaGkiLDJdLFsxLDIsIlxcZXhpc3RzICFcXHBzaSIsMCx7InN0eWxlIjp7ImJvZHkiOnsibmFtZSI6ImRhc2hlZCJ9fX1dXQ==
\[\begin{tikzcd}
	G && X \\
	& Y
	\arrow["\pi", from=1-1, to=1-3]
	\arrow["\varphi"', from=1-1, to=2-2]
	\arrow["{\exists !\psi}", dashed, from=1-3, to=2-2]
\end{tikzcd}\]
\end{theorem}

\begin{proof}
    Insiemisticamente, la mappa $\psi\colon gH \mapsto gy_0$ è definita. Inoltre $\psi$ è continua: se $U$ è un aperto di $Y$, allora \[\psi^{-1}(U)=\pi(\pi^{-1}(\psi^{-1}(U)))=\pi(\varphi^{-1}(U)),\] che è aperto. Dato un aperto $U$ di $Y$, verifichiamo che l'immagine della mappa 
    \[\psi^\ast:\Oc_Y(U)\to\Hom_{(\mathrm{Set})}(\psi^{-1}(U),\K)\]
    è contenuta nell'insieme delle funzioni regolari su $X$. Consideriamo una funzione $f\colon U\to \K$. Allora $\wt{f}:= f\circ \psi\circ\pi =f \circ \varphi$ è in $\Oc_G(\varphi^{-1}(U))$ in quanto $\varphi$ è un morfismo di varietà. Mostriamo che in realtà è in $\Oc_G(\varphi^{-1}(U))^H$. Sia $\lambda$ in $\Oc_G(\varphi^{-1}(U))$ e sia $h$ in $H$. Allora per ogni $x$ in $G$ si ha \[(h\widetilde{f})(x)=\widetilde{f}(xh)=f(\varphi(xh))=f(xhy_0)=f(x y_0)=f(\varphi(x))=\widetilde{f}(x).\]
    %\[(g\cdo \lambda)(x)=\lamb\]
\end{proof}

\begin{definition}
Una variet\`a \`e \textbf{omogenea} rispetto al gruppo $G$ se l'azione di $G$ su $X$ \`e transitiva.
\end{definition}

\begin{corollary}
    Se $X$ è una varietà omogenea rispetto a $G$, allora $X$ è liscia.
\end{corollary}
\begin{proof}
    Poiché $X$ ha un aperto $U$ di punti lisci, la tesi segue dal fatto che $X=\bigcup_{g\in G} gU$.
\end{proof}


\begin{corollary}\label{CorEquivarianteTraVarietaOmogeneeELiscia}
    Se $X$ e $Y$ sono varietà omogenee per $G$ e $\varphi\colon X \to Y$ è $G$-equivariante, allora $\varphi$ è liscia.
\end{corollary}
\begin{proof}
    Sicuramente $\varphi$ è surgettiva, perché $X$ e $Y$ sono varietà omogenee per $G$ e $\varphi$ è $G$-equivariante. Per il teorema (\ref{ThMorfismoDominanteTraLisceHaRestrizioneLisciaSuAperto}) esiste un aperto non vuoto $U$ di $Y$ tale che $\varphi\res{\varphi^{-1}(U)}\colon \varphi^{-1}(U) \to U$ è liscia. Allora si ha un diagramma commutativo % https://q.uiver.app/#q=WzAsNCxbMCwwLCJcXHZhcnBoaV57LTF9KGdVKSJdLFsyLDAsImdVIl0sWzAsMiwiXFx2YXJwaGleey0xfShVKSJdLFsyLDIsIlUiXSxbMCwxLCJcXHZhcnBoaVxccmVze2dcXHZhcnBoaV57LTF9KFUpfSJdLFsyLDMsIlxcdmFycGhpXFxyZXN7XFx2YXJwaGleey0xfShVKX0iXSxbMiwwLCJcXHNpbSJdLFszLDEsIlxcc2ltIiwyXV0=
\[\begin{tikzcd}
	{\varphi^{-1}(gU)} && gU \\
	\\
	{\varphi^{-1}(U)} && U
	\arrow["{\varphi\res{g\varphi^{-1}(U)}}", from=1-1, to=1-3]
	\arrow["\sim", from=3-1, to=1-1]
	\arrow["{\varphi\res{\varphi^{-1}(U)}}", from=3-1, to=3-3]
	\arrow["\sim"', from=3-3, to=1-3]
\end{tikzcd}\] 
Poiché $Y=\bigcup gU$, $\varphi$ è liscia in ogni punto e quindi è liscia.

%\[\varphi\res{g\varphi^{-1}(U)}\colon \varphi^{-1}(gU) \to gU\]
\end{proof}









\section{Sottogruppi generati e contenimenti}
\begin{lemma}
Sia $G$ un gruppo algebrico e siano $X_i$ delle variet\`a irriducibili. Siano $\vp_i:X_i\to G$ tali che $1_G\in \imm\vp_i$ per ogni $i$. Poniamo $Y_i=\vp_i(X_i)$.

Sia $H=\ps{\cpa{Y_i}_i}$ il sottogruppo generato dalle immagini delle $\vp_i$. Allora

\begin{enumerate}
	\item $H$ \`e chiuso
	\item esistono $i_1,\cdots, i_n$, $\e_1,\cdots, \e_n$ tali che
	\[H=Y_{i_1}^{\e_1}\cdots Y_{i_n}^{\e_n},\qquad \e_j\in\cpa{1,-1}\]
\end{enumerate}
\end{lemma}
\begin{proof}
Supponiamo $X_i=Y_i\subseteq G$ e supponiamo che tra le $X_i$ compaiano anche le $X_i\ii$ (cos\`i evitiamo gli $\e$).

Definiamo iterativamente
\begin{align*}
	Z_1&=X_1, &W_1&=\ol{Z_1},\\
	Z_2&=X_1\cdot X_2, &W_2&=\ol{Z_2},\\ 
	&\;\;\vdots &&\;\;\vdots\\	
	Z_n&=X_1\cdots X_n, &W_n&=\ol{Z_n},\\
	Z_{n+1}&=X_1\cdots X_nX_1, &W_{n+1}&=\ol{Z_{n+1}}\\
	&\;\;\vdots &&\;\;\vdots
\end{align*}
Notiamo che $Z_i$ \`e l'immagine di $X_1\times\cdots\times X_i\to G$, quindi $Z_i$ \`e irriducible per ogni $i$, dunque anche $W_i=\ol{Z_i}$ \`e irriducibile.

Notiamo ora che $W_1\subseteq W_2\subseteq\cdots\subseteq G$ \`e una catena di chiusi, ma dato che $G$ ha dimensione finita essa stabilizza, cio\`e esiste $N$ tale che $W_N=W_{N+1}$, ovvero $W_N\cdot X_i\subseteq W_N$ per ogni $i$.

Allora $W_NZ_N\subseteq W_N$ cio\`e $W_N\cdot W_N\subseteq W_N$ perch\'e $W_n$ \`e chiuso. Quindi\footnote{$Z_N\ii W_N\subseteq W_N\implies Z_N\ii\subseteq W_N\implies \ol{Z_N\ii}\subseteq W_N$ e $\ol{Z_N\ii}=\ol{Z_N}\ii=W_N\ii$.} $W_N\ii\subseteq W_N$ e in particolare $Z_N\ii\subseteq W_N$, ma $Z_N\ii=X_N\ii\cdots X_1\ii=X_{i_1}\cdots X_{i_N}$, cio\`e mettendo tutto insieme $H\subseteq W_N$

Mostriamo che $W_N=Z_N\cdot Z_N$: abbiamo una mappa
\[\under{irrid.}{X_1\times\cdots\times X_N}\to Z_N\to W_N,\]
quindi per Chevalley (\ref{ThChevalley}) $Z_N\supseteq U$ per $U$ aperto non vuoto di $W_N$, dunque $U\cdot U=W_N$ perch\'e $W_N$ \`e irriducibile.

Questo mostra che $W_N\subseteq H$ ma ci sono tutti gli elementi di $H$ quindi effettivamente $H=W_N$.
\end{proof}


\begin{remark}[Dimensione del quoziente]\label{ReDimensioneDelQuoziente}
Sia $\vp:G\to G/H$ e ricordiamo che una mappa di questo tipo \`e liscia per omogeneit\`a. Allora per il teorema (\ref{ThFibrePerMappeLiscie}) abbiamo una successione esatta
\[0\to T_{e}H\to T_{e}G\xrightarrow{d\vp_{e}}T_{e}(G/H)\to 0\]
e sappiamo che $\dim (G/H)=\dim G=\dim H$.
\end{remark}

\begin{proposition}\label{PrContenimentoTraSottogruppiSiControllaSuiTangenti}
Supponiamo $\cha\K=0$. Siano $H,K\subseteq G$ sottogruppi connessi, allora
\[H\subseteq K\coimplies T_e H\subseteq T_eK\text{ in }T_eG.\]
\end{proposition}
\begin{proof}
L'implicazione $\implies$ \`e ovvia quindi basta mostrare l'altra.

Sia $X=H\cdot K\subseteq G$ e notiamo che \`e omogeneo rispetto all'azione di $H\times K$ data da $(h,k)\cdot g=hgk\ii$.
Osserviamo che $X$ \`e aperto in $\ol X\subseteq G$ per Chevalley (\ref{ThChevalley}) (contiene un aperto della chiusura ed \`e omogeneo in quanto orbita di $e$).
Si ha
\[X\cong \frac{H\times K}{\stab_{H\times K}(e)}.\]
Studiando lo stabilizzatore notiamo che
\[\stab_{H\times K}(e)=\cpa{(h,k)\mid hk\ii=e}=\cpa{(h,k)\mid h=k}\cong H\cap K,\]
dunque per il teorema (\ref{ThFibrePerMappeLiscie}) abbiamo $\dim X=\dim H+\dim K-\dim H\cap K$ e una successione esatta
\[0\to T_e(H\cap K)\to T_e(H\times K)\to T_e X\to 0\]
da cui segue $\dim T_e X=\dim T_eH+\dim T_eK-\dim T_e(H\cap K)$.
\medskip

Se $T_e X=T_eH+T_eK$ allora 
\[\dim (T_e X)=\dim (T_eH+T_eK)\overset{T_e H\subseteq T_eK}=\dim T_eK,\]
dunque 
\[\dim X=\dim T_eX=\dim T_eK=\dim K\]
e quindi $K\subseteq X\subseteq \ol X$ con $K$ e $\ol X$ irriducibili della stessa dimensione, dunque $K=X=\ol X$, e poich\'e $X=H\cdot K$ questo mostra 
\[K=X=H\cdot K\implies H\subseteq K\]
che \`e la tesi.

Per concludere basta dunque dimostrare che $T_e X=T_eH+T_eK$.
Consideriamo la composizione
\[\begin{array}{ccccc}
	H\times K &\xrightarrow{}& H\times K & \to &X \\
	(h,k) & \mapsto & (h,k\ii) &  & \\
	       &        & (h',k') &\mapsto & h'(k'{\ii}) 
\end{array}\] 
che \`e la mappa di moltiplicazione $\mu\res{H\times K}$. Dal teorema (\ref{ThFibrePerMappeLiscie}) ricaviamo una successione esatta
\[0\to T_e(\stab_{H\times K}(e))\to \under{T_eH\times T_eK}{T_e(H\times K)}\to T_eX\to 0\]
In particolare $d\pa{\mu\res{H\times K}}_{(e,e)}$ \`e surgettiva, quindi per concludere basta mostrare che 
\[d\pa{\mu\res{H\times K}}_{(e,e)}(\al,\beta)=\al+\beta.\]
Dato che $G\subseteq \GL(n)$ \`e sufficiente che la tesi valga per $\GL(n)$ e $d\mu_{(e,e)}$. Notiamo che $\Mat_{n\times n}=T_I\GL(n)$ dove l'identificazione \`e data da
\[(a_{i,j})\longleftrightarrow \pa{\rbar{a_{i,j}\pp{x_{ij}}{}}_{I}}\]
Traduciamo allora $d\mu_{(e,e)}$ in queste coordinate: se
\[d\mu_{(e,e)}\funcDef{\Mat_{n\times n}\times \Mat_{n\times n}}{\Mat_{n\times n}}{((a_{i,j}),(b_{i,j}))}{(c_{i,j})}\]
allora
\begin{align*}
	c_{i,j}=&\rbar{\pp{x_{i,j}}{}(x_{i,j}\circ \mu)}_I=\rbar{\pp{x_{i,j}}{}\pa{\sum_\ell y_{i,\ell}z_{\ell,j}}}_I=\\
	=&\sum_\ell \rbar{\pp{x_{i,\ell}}{}(y_{i,\ell})z_{\ell,j}}_I + \sum_\ell y_{i,\ell}\rbar{\pp{x_{i,j}}{}(z_{\ell,j})}_I\overset{(\star)}=\\
	=&\sum_\ell \rbar{\pp{y_{i,j}}{}(y_{i,\ell})}_I\delta_{\ell,j} + \sum_\ell \delta_{i,\ell}\rbar{\pp{z_{i,j}}{}(z_{\ell,j})}_I=\\
	=&\rbar{\pp{y_{i,j}}{}(y_{i,j})}_I+\rbar{\pp{z_{i,j}}{}(z_{i,j})}_I=\\
	=&a_{i,j}+b_{i,j}
\end{align*}
dove il passaggio $(\star)$ segue perch\'e stiamo interpretando quei differenziali come derivazioni.
\end{proof}


\section{Variet\`a complete}

\begin{definition}[Variet\`a separata]
$X$ \`e \textbf{separata} se la diagonale $X\to X\times X$ \`e un morfismo chiuso.
\end{definition}
\begin{center}
	\textbf{Supporremo che sia tutto separato.}
\end{center}

\begin{definition}[Variet\`a completa]
$X$ \`e \textbf{completa} se per ogni $Z$ variet\`a, $\pi:X\times Z\to Z$ \`e un morfismo chiuso.
\end{definition}

\begin{remark}\label{ReRegolariSuCompleteIrriducibiliSonoCostanti}
Se $X$ \`e irriducibile e completa allora $\Oc_X(X)\cong \K$.
\end{remark}
\begin{proof}
Sia $f:X\to \K$ regolare e consideriamo $\Gamma(f)\subseteq X\times \K\to \K$. Questa proiezione \`e chiusa quindi $\pi(\Gamma(f))=f(X)$ \`e un chiuso irriducibile (immagine di irriducibile) di $\K$. Ma i chiusi irriducibili di $\K$ non vuoti sono o tutto $\K$ o solo un punto.

Consideriamo ora $W=\cpa{(x,y)\in X\times \K\mid yf(x)=1}$. Se $f(X)=\K$ allora l'immagine di $W$ tramite la proiezione $X\times \K\to \K$ \`e $\K^\times$, che non \`e chiuso in $\K$, assurdo.

Quindi $f(X)$ \`e un solo punto, cio\`e $f$ \`e costante.
\end{proof}
\begin{corollary}
Se $X$ \`e completa, affine e connessa allora $\K[X]=\K$ significa che $X$ \`e un solo punto $\cpa{(0)}$.

In generale le affini complete sono un numero finito di punti.
\end{corollary}

\begin{theorem}[I proiettivi sono completi]\label{ThProiettiviSonoCompleti}
Se $V$ \`e uno spazio vettoriale di dimensione finita, $\Pj(V)$ \`e completa.
\end{theorem}

\begin{center}
	Moralmente ``completo=compatto" in senso classico, per esempio valgono:
\end{center}

\begin{remark}
Se $X$ \`e completa e $Z\subseteq X$ chiuso allora $Z$ \`e completo.
\end{remark}

\begin{remark}
Se $\vp:X\to Y$ \`e surgettiva e $X$ \`e completa allora $Y$ \`e completa
\end{remark}

\subsection{Punto fisso di Borel}
\begin{proposition}[Esiste orbita chiusa]\label{PrEsisteOrbitaChiusa}
Se $K$ \`e un gruppo che agisce su una variet\`a $Y$ allora $K$ ha un'orbita chiusa in $Y$.
\end{proposition}
\begin{proof}
Consideriamo un'orbita che ha dimensione minima $Z=G\cdot y$. Notiamo che $Gy$ \`e un aperto denso di $\ol Z$ per Chevalley (\ref{ThChevalley}) (contiene un aperto ed \`e omogeneo), quindi $\dim \ol Z\bs Z<\dim Z$, ma se questa differenza \`e non vuota allora $G$ agisce su questa differenza e quindi esiste un'orbita di dimensione pi\`u piccola.

Dunque $\ol Z\bs Z=\emptyset$, cio\`e $\ol Z=Z$.
\end{proof}

\begin{theorem}[Punto fisso di Borel]\label{ThPuntoFissoBorel}
Se $G$ \`e un gruppo risolubile connesso che agisce su una variet\`a completa allora $G$ ha un punto fisso.
\end{theorem}
\begin{proof}
Se $\dim G=0$ allora $G$ \`e un singolo punto e quindi \`e l'indetit\`a e agisce banalmente.


Supponiamo $\dim G>0$. Sia $H=[G,G]$ il sottogruppo dei commutatori. Questo \`e un sottogruppo algebrico connesso. Dato che $G$ \`e risolubile, $H\subsetneq G$, quindi per ipotesi induttiva $X^H\neq \emptyset$.

Notiamo ora che su $X^H$ agisce $A=G/H$, quindi su $X^H$ abbiamo un'orbita chiusa (\ref{PrEsisteOrbitaChiusa}) $A\cdot x\subseteq X^H\subseteq X$. Notiamo che $A\cdot x$ \`e completa perch\'e chiuso di $X$ completa, ma $A\cdot x=A/\stab_A x$, quindi \`e anche una variet\`a affine. $A\cdot x$ \`e anche connessa perch\'e $A$ \`e connesso in quanto $G$ lo \`e.

$A\cdot x$ \`e affine, completa e connessa, quindi \`e un punto, cio\`e $x$ \`e un punto fisso per $A$ in $X^H$, ma allora $x$ \`e un punto fisso per $G$.
\end{proof}


\begin{example}
Consideriamo $\C^\ast\acts \Pj^n$ come segue:
\[\la\cdot [x_0:\cdots:x_n]=[x_0:\la x_1:\cdots:\la^n x_n]\]
Questa azione ha come punti fissi quelli della forma $[0:\cdots:0:1:0:\cdots:0]$.
\end{example}



Ricordiamo che $B_n$ è il sottogruppo di $\GL(n)$ costituito dalle matrici triangolari superiori.

\begin{corollary}\label{CorCaratterizzazioneConnessiRisolubili}
    Se $G$ è connesso, sono tra loro equivalenti \begin{enumerate}
        \item $G$ è risolubile.
        \item Le uniche rappresentazioni irriducibili di $G$ sono di dimensione $1$.
        \item Esiste un intero positivo $n$ tale che $G$ è contenuto in $B_n$. 
    \end{enumerate}    
\end{corollary}
\begin{proof}
    Supponiamo $G$ risolubile e mostriamo il punto $1$. Sia $V$ una rappresentazione irriducibile di $G$. Allora $\Pj(V)^G$ è non vuoto e dunque per il teorema del punto fisso (\ref{ThPuntoFissoBorel}) esiste una retta $\ell$ in $V$ tale che $G\cdot\ell=\ell$. Quindi $\ell=V$, da cui $\dim V=1$.

    Mostriamo l'implicazione $2\Rightarrow 3$. Assumiamo $G$ contenuto in $\GL(V)$ per qualche $V$. Sia $F_1$ una sottorappresentazione irriducibile di dimensione $1$. Allora $G$ agisce su $V/F_1$. Allora esiste una retta $F_2/F_1$ in $V/F_1$ tale che $G F_2/F_1 \subseteq F_2/F_1$. Procedendo induttivamente, troviamo una bandiera \[0\subseteq F_1\subseteq F_2 \subseteq \ldots \subseteq F_n =V\] tale che $G F_i=F_i$ con $\dim F_i=i$. Se $v_1,\ldots,v_n$ è una base compatibile con tale bandiera (cioè $v_i\in F_i$ per ogni indice $i$), allora abbiamo un'immersione $G \hookrightarrow B_n$ definita da $g\mapsto [g]_{\underline{v}}^{\underline{v}}$. 

    Infine, l'implicazione $3 \Rightarrow 1$ è immediata perché $B_n$ è risolubile.
\end{proof}



\subsection{Sottogruppi parabolici e di Borel}
\begin{definition}
Sia $G$ un gruppo algebrico e $P\subseteq G$ sottogruppo chiuso.
\begin{enumerate}
	\item $P$ si dice \textbf{parabolico} se $G/P$ \`e completo
	\item $B\subseteq G$ si dice \textbf{sottogruppo di Borel} se \`e un sottogruppo risolubile connesso massimale.
\end{enumerate}
\end{definition}

\begin{remark}
Se $P$ \`e un sottogruppo parabolico, $G/P\subseteq \Pj(V)$ per qualche $V$.
\end{remark}
\begin{proof}
Ricorda (\ref{LmSottogruppoEStabilizzatoreDiUnaRettaInQualcheRappresentazione}) che esistono una rappresentazione $V$ di $G$ e una retta $L\subseteq V$ tali che $\stab_G L=P$. Per questa rappresentazione $G/P=G\cdot L\subseteq \Pj(V)$.

Notiamo che l'immagine di $G/P$ in $\Pj(V)$ \`e chiusa, infatti se $\Gamma$ \`e il grafico di $G/P\inj \Pj(V)$ allora esso \`e contenuto in $G/P \times \Pj(V)\to \Pj(V)$ e per definizione di sottogruppo parabolico $G/P$ \`e completa.

Questo mostra che $G/P$ \`e una sottovariet\`a di $\Pj(V)$.
\end{proof}

\begin{example}
Sia $G=\GL(n)$, $B\subseteq G$ le matrici triangolari superiori, diagonale inclusa.
\[G/B=\cpa{0\subseteq F_1\subseteq \cdots\subseteq F_n=\K^n\mid \dim F_i=i}\]
\end{example}

\begin{lemma}\label{LmConnessoRisolubileSSENoSottogruppiParabolici}
$G$ \`e un gruppo connesso e risolubile se e solo se $G$ non ha sottogruppi parabolici propri.
\end{lemma}
\begin{proof}
Diamo le due implicazioni
\setlength{\leftmargini}{0cm}
\begin{itemize}
\item[$\boxed{\implies}$] Sia $P\subseteq G$ parabolico. Per il teorema del punto fisso di Borel (\ref{ThPuntoFissoBorel}) si ha che $G/P$ ha un punto fisso per $G$, ma per omogeneit\`a questo significa che $G/P$ consiste di un solo punto, cio\`e $P=G$.
\item[$\boxed{\impliedby}$] Sia $P=G^0$ e mostriamo che $G/G^0$ \`e completo, questo basta perch\'e in tal caso $G^0$ \`e parabolico e quindi per ipotesi $G^0=G$.

Per il corollario (\ref{CorCaratterizzazioneConnessiRisolubili}) basta dimostrare che le rappresentazioni irriducibili hanno dimensione 1: se $V$ irriducibile, $G$ agisce su $\Pj(V)$ e ha un'orbita chiusa $G\cdot \ell\subseteq\Pj(V)$ per (\ref{PrEsisteOrbitaChiusa}). Sia $P=\stab_G(\ell)$, questo gruppo deve essere parabolico perch\'e il quoziente \`e $G\ell$ orbita chiusa ma allora $P=G$ e per irriducibilit\`a questo mostra $V=\ell$.
\end{itemize}
\setlength{\leftmargini}{0.5cm}
\end{proof}


\begin{theorem}\label{ThProprietaSottogruppiDiBorel}
Sia $G$ un gruppo algebrico. 
\begin{enumerate}
    \item Se $B$ è un sottogruppo di Borel di $G$, allora $B$ è parabolico.
    \item Se $B$ è un sottogruppo di Borel di $G$ e $P$ è un sottogruppo parabolico di $G$, allora esiste $g$ in $G$ tale che $gBg^{-1}$ è contenuto in $P$.
    \item Tutti i sottogruppi di Borel sono coniugati.
\end{enumerate}
\end{theorem}

\begin{proof}
Mostriamo contemporaneamente il primo e il secondo punto. 

Osserviamo che è sufficiente mostrare che $G^0/B$ è completo, infatti, se $G=\coprod_i g_i G^0$, allora $G/B=\coprod_i g_i G^0/B$. Possiamo quindi assumere che $G$ sia connesso. 

Se $G$ è risolubile, allora per per definizione di sottogruppo di Borel $G=B$ e si conclude. Se $G$ non è risolubile, allora per il lemma (\ref{LmConnessoRisolubileSSENoSottogruppiParabolici}) esiste un sottogruppo parabolico proprio $P$ di $G$. Possiamo quindi considerare l'azione di $B$ su $G/P$, la quale ha un punto fisso (\ref{ThPuntoFissoBorel}), cioè esiste un $g$ in $G$ tale che $gBg^{-1}P=P$, ovvero $gBg^{-1}\subseteq P$. Ciò in particolare mostra il secondo punto. 

Possiamo assumere $B\subseteq P \subsetneq G$, per cui $B$ è un sottogruppo di Borel di $P$. A questo punto procediamo per induzione sulla dimensione di $G$. Poiché $\dim P <\dim G$, abbiamo che $B$ è un sottogruppo parabolico di $P$ per ipotesi induttiva, quindi $P/B$ è completa e $G/P$ è completa. Avendo mostrato che $B$ \`e parabolico in $P$ e che $P$ \`e parabolico in $G$, per il lemma (\ref{LmSottogruppoParabolicoDiParabolicoEParabolico}) si ha che $B$ è parabolico in $G$.

Infine, per il terzo punto, siano $B,B'$ due sottogruppi di Borel di $G$. Allora esiste $g$ in $G$ tale che $gBg^{-1} \subseteq B'$ per il punto 2. Poiché $gBg^{-1}$ e $B'$ sono sottogruppi connessi, risolubili e massimali, concludiamo che $gBg^{-1} = B'$.
\end{proof}

\begin{corollary}\label{CorPrabolicoRisolubileConnessoEBorel}
Se $B\subseteq G$ \`e parabolico, connesso e risolubile allora \`e di Borel.
\end{corollary}
\begin{proof}
Per il punto 2. del teorema (\ref{ThProprietaSottogruppiDiBorel}) si ha che $B$ contiene un sottogruppo di Borel, ma essendo connesso e risolubile deve essere gi\`a un sottogruppo di Borel per massimalit\`a.
\end{proof}

\begin{lemma}\label{LmSottogruppoParabolicoDiParabolicoEParabolico}
    Siano $G$ un gruppo algebrico, $P\subseteq Q$ due sottogruppi tali che $Q$ è parabolico in $G$ e $P$ è parabolico in $Q$. Allora $P$ è parabolico in $G$.
\end{lemma}

\begin{proof}
    Data una varietà arbitraria $Z$, mostriamo che la mappa $\pi_Z\colon G/P\times Z \to Z$ è chiusa. Sia $A$ un chiuso in $G/P\times Z$. Sia $A'$ l'immagine inversa di $A$ rispetto la proiezione $\pi_1\colon G\times Z \to G/P\times Z$. Allora $A'$ è chiuso e $A' \cdot P=A'$. Se mostriamo che $A'\cdot Q=A'$, abbiamo concluso, infatti, considerando la composizione \[\begin{array}{ccccc}
         G\times Z &\xrightarrow{}& G/Q \times Z& \to &Z \\
          A' & \to & A'' & \to & \pi_Z(A) 
    \end{array}\] 
    l'immagine di $A'$ tramite la prima mappa è un certo $A''$, che viene mandato in $\pi_Z(A)$. \\
    Sia $A'''=A'\cdot Q$. Mostriamo che $A'''$ è chiuso (ciò permette di concludere per quanto appena detto). Consideriamo la mappa 
	\[\vp:\funcDef{Q\times G\times X}{G}{(q,g,x)}{(gq,x)}\] 
    Allora $A^{IV}\coloneqq \varphi^{-1}(A')$ è chiuso ed è stabile per l'azione a destra di $P$ su $Q$, cioè se $(q,g,x)$ è in $A^{IV}$, allora $(gp,g,x)$ è in $A^{IV}$. Infatti se $(gq,x)$ è in $A'$, allora anche $(gqp,x)$ è in $A'$. Ragionando come prima, abbiamo 
	\[\begin{array}{ccccc}
         Q\times G\times Z &\xrightarrow{}& Q/P \times G \times Z& \to &G\times Z \\
          A^{IV} & \to & \text{imm chiusa} & \to & A^V \text{chiuso} 
    \end{array}\] 
    dove 
	\[A^V=\left\{(g,x)\in G\times X \colon \, \exists\,q\in Q, \ (gq,x)\in A'\right\}=A'\cdot Q.\] 
	A questo punto $A' \cdot Q$ è chiuso e concludiamo come prima.
\end{proof}

Con tecniche simili, è possibile mostrare il seguente risultato.

\begin{theorem}\label{ThNormalizzatoreDiBorelEDiParabolico}
    Se $G$ è un gruppo algebrico connesso, $B$ un sottogruppo di Borel di $G$ e $P$ un sottogruppo parabolico di $G$, allora \begin{enumerate}
        \item $N_G(B)=B$ e $N_G(P)=P$.
        \item $P$ è connesso.
    \end{enumerate}
\end{theorem}

Sia $\mathscr{B}$ la famiglia dei sottogruppi di Borel di un gruppo algebrico $G$. Allora $G$ agisce per coniugio (e transitivamente) su $\mathscr{B}$. Per il Teorema (\ref{ThNormalizzatoreDiBorelEDiParabolico}) abbiamo $\stab_G(B)=N_G(B)=B$, da cui deduciamo il seguente risultato.
\begin{corollary}
    Sia $B$ un sottogruppo di Borel di un gruppo algebrico $G$. Esiste una corrispondenza biunivoca tra la famiglia $\Bs$ e il quoziente $G/B$, data da \[\begin{array}{ccc}
        G/B &\longrightarrow& \mathscr{B} \\
         gB & \longmapsto & gBg^{-1}.
    \end{array}\]
\end{corollary}





\section{Esempi di sottogruppi di Borel e Parabolici}
Citiamo il seguente teorema per la sua utilit\`a in alcuni degli esempi:

\begin{theorem}[Witt]\label{ThWitt}
Se $V$ spazio vettoriale, $b$ forma simmetrica non degenere, $W_1,W_2$ sottospazi di $V$ e $\vp:W_1\to W_2$ isomorfismo che preserva $b$, allora $\vp$ si estende a $V$ e continua a preservare $b$.
\end{theorem}

\subsection{Matrici invertibili}
Fissiamo $G=\GL(n)$. 
\begin{example}[Borel di $\GL(n)$ sono le matrici triangolari]
Consideriamo il sottogruppo $B_n$ di $G$. Sappiamo che $B_n$ è connesso e risolubile. Mostriamo che è massimale (e quindi un sottogruppo di Borel). 

Sia $H$ un sottogruppo risolubile connesso contenente $B_n$. Essendo risolubile e connesso, $H$ si triangolarizza rispetto a una bandiera $F_1\subset \ldots \subset F_n$. Poiché $H\cdot F_i=F_i$ per ogni $i=1,\ldots,n$, si ha anche $B\cdot F_i=F_i$ per ogni indice $i$. Poiché $B_n$ è lo stabilizzatore in $G$ della bandiera $F_1\subset \ldots \subset F_n$, si ottiene $H=B_n$. 
\end{example}

\begin{example}[Sottogruppi parabolici di $\GL(n)$]
Abbiamo visto che se $P$ è un sottogruppo di $G$ contenente $B_n$, allora $P$ è un sottogruppo parabolico di $G$.  Se $P$ è il sottogruppo costituito dalle matrici triangolari superiori con blocchi sulla diagonale di tipo $h\times h$ e $k\times k$, sappiamo che $G/P$ è in corrispondenza biunivoca con $\operatorname{Gr}(h,\K^n)$, sulla quale $G$ agisce transitivamente e per cui $\stab_G \langle e_1,\ldots,e_h\rangle=P$. Quindi $\operatorname{Gr}(h,\K^n)$ è una varietà proiettiva. 
Vogliamo descrivere una sua immersione esplicita in $\Pj(V)$ per qualche spazio vettoriale $V$. Per farlo, dobbiamo trovare uno spazio vettoriale $V$ contenente una retta $L$ fissata da $P$. Consideriamo $V=\bigwedge^h(\K^n)$ e $L=\bigwedge^h W$, dove $W$ è il sottospazio generato da $e_1,\ldots,e_h$. Abbiamo già visto che, fissato $g$ in $G$, si ha $gL=L$ se e solo se $gW=W$. Quindi $\stab_G L = \stab_G W=P$. Dunque % https://q.uiver.app/#q=WzAsMTEsWzEsMSwiXFxvcGVyYXRvcm5hbWV7R3J9KGgsS15uKSJdLFszLDEsIlxcUChWKSJdLFsxLDJdLFsxLDMsIkcvUCJdLFszLDMsIlxcUCh2KSJdLFsxLDRdLFswLDQsImciXSxbMCwxLCJnVyJdLFszLDQsImdMPVxcbGVmdChnXFxiaWd3ZWRnZV5oV1xccmlnaHQpPVxcbGVmdChcXGJpZ3dlZGdlXmggZ1dcXHJpZ2h0KSJdLFsxLDAsIlUiXSxbMywwLCJcXGJpZ3dlZGdlXmggVSJdLFswLDMsIiIsMSx7InN0eWxlIjp7InRhaWwiOnsibmFtZSI6ImFycm93aGVhZCJ9fX1dLFswLDFdLFs2LDcsIiIsMSx7InN0eWxlIjp7InRhaWwiOnsibmFtZSI6Im1hcHMgdG8ifX19XSxbNiw4XSxbMyw0XSxbOSwxMF1d
\[\begin{tikzcd}
	& U && {\bigwedge^h U} \\
	gW & {\operatorname{Gr}(h,\K^n)} && {\Pj(V)} \\
	& {} \\
	& {G/P} && {\Pj(v)} \\
	g & {} && {gL=\left(g\bigwedge^hW\right)=\left(\bigwedge^h gW\right)}
	\arrow[from=1-2, to=1-4]
	\arrow[from=2-2, to=2-4]
	\arrow[tail reversed, from=2-2, to=4-2]
	\arrow[from=4-2, to=4-4]
	\arrow[maps to, from=5-1, to=2-1]
	\arrow[from=5-1, to=5-4]
\end{tikzcd}\]

Descriviamo questa mappa in coordinate. Presa \[U=\begin{pmatrix}
    v_1 \lvert \ldots \lvert v_h
\end{pmatrix} \longmapsto v_1\wedge \ldots \wedge v_h =\sum_{i_1<\ldots<i_h} p_{i_1,\ldots,i_h}e_{i_1}\wedge\ldots\wedge e_{i_h}\]

dove $p_{i_1,\ldots,i_h}$ è il determinante del minore di $(v_1 \lvert \ldots \lvert v_h)$ identificato dalle righe $i_1,\ldots,i_h$  e 
\[v_1\wedge \ldots \wedge v_h = \det(v_1 \lvert \ldots \lvert v_h) e_1\wedge \ldots \wedge e_h.\] 
Quindi si ha 
\[U=\langle v_1,\ldots,v_h\rangle \longrightarrow [p_I(v_1,\ldots,v_h)]_{I=\{i_1<\ldots<i_h\}}\]
Se $U=\langle w_1,\ldots,w_h\rangle$, allora esiste una matrice invertibile $A$ di taglia $h$ tale che $MA=N$, dove $N=(w_1\lvert\ldots\lvert w_h)$. Si ha \begin{align*}
    p_I(M)&=\det (M_{i_1,\ldots,i_h}) \\
    p_I(N)&=\det(N_{i_1,\ldots,i_h})=\det (M_{i_1,\ldots,i_h}\cdot A)=p_I(M) \det(A).
\end{align*}

Infine $(v)\longrightarrow [p_I(M)]$ e $(w)\longrightarrow [p_I(N)]=[p_I(M)\det (A)]=[p_I(M)]$.
\end{example}


\subsection{Matrici ortogonali speciali}

Ora consideriamo $G=\SO(n)$. Cerchiamo di capire come \`e definito questo gruppo
\setlength{\leftmargini}{0cm}
\begin{itemize}
\item[$\boxed{n=1}$] $G=\{1\}$.
\item[$\boxed{n=2}$] Scegliamo la forma quadratica $q(x,y)=xy$. Lo stabilizzatore di tale forma è dato da 
\[\left\{A\in \SO(2)\colon A \begin{pmatrix}
    0 & 1 \\ 1& 0
\end{pmatrix}A^t=\begin{pmatrix}
    0 & 1 \\ 1& 0
\end{pmatrix}\right\}.\] 
Scriviamo 
\[A=\begin{pmatrix}
    a & b \\ c& d
\end{pmatrix}.\] 
Imponendo che $A$ sia nello stabilizzatore della forma $q$, si trova $b=c=0$. Poiché $\det(A)=1$, troviamo $d=a^{-1}$ e quindi $\SO(2)$ è isomorfo a $\K^\times$.
\item[$\boxed{n=3}$] Mostriamo che 
\[\SO(3)\cong \SL(2)/\{\pm I\}.\]
\end{itemize}
\setlength{\leftmargini}{0.5cm}

\begin{proposition}
$\SO(3)\cong \SL(2)/\{\pm I\}.$
\end{proposition}
\begin{proof}
Sappiamo che $\SL(2)$ agisce su $W\coloneqq \K^2$. Consideriamo la rappresentazione $W\otimes W=S^2 W \oplus \bigwedge^2 V$. 
Consideriamo la forma bilineare e simmetrica 
\[b(u\otimes u', v \otimes v')=\det(u,v)\det(u',v').\] 
Tale forma è $\SL(2)$-invariante, infatti, se $g$ è in $\SL(2)$, allora $\det(g)=1$ e quindi
\begin{align*}
	b(gu\otimes gu', gv \otimes gv')=&\det(gu,gv)\det(gu',gv')=\\
	=&\det(g)^2b(u\otimes u', v \otimes v')=\\
	=&b(u\otimes u', v \otimes v').
\end{align*}
Fissiamo una base $e_{11}=e_1\otimes e_1$, $e_{12}=e_1\otimes e_2$, $e_{21}=e_2\otimes e_1$, $e_{22}=e_2\otimes e_2$. Allora \[[b]_{e_{ij}}=\begin{pmatrix}
    0 & 0 & 0 & 1 \\
    0 &0 & -1 & 0 \\
    0 & -1 & 0&0 \\
    1 &0&0&0
\end{pmatrix}.\]

In particolare $b$ è non degenere. Inoltre, anche la restrizione $\beta \coloneqq b\res{S^2W}$ è non degenere, in quanto $S^2W=\langle e_{11},e_{22},e_{12}+e_{21}\rangle$. Abbiamo quindi definito una mappa 
\[\SL(2)\to \SO(\beta).\] 
Imponendo $g e_{11}=e_{11}$ e $g e_{22}=e_{22}$, troviamo $g=\pm I$, quindi abbiamo un'immersione $\SL(2)/\{\pm I\}\hookrightarrow \SO(\beta)$. Mostriamo che è suriettiva. Osserviamo che $\GL(2)$ è isomorfo (come varietà) a $\SL(2)\times \K^\times$ tramite la mappa $(A,\lambda)\mapsto A \begin{pmatrix}
    \lambda & 0 \\ 0 & 1
\end{pmatrix}$. Quindi $\dim\SL(2)=3$. Calcoliamo la dimensione di $\SO(\beta)$ (e in generale di $\SO(n)$). Consideriamo la mappa 
\[\vp:\funcDef{\Mat_{n\times n}}{\Mat_{n\times n}}{A}{A^t A}.\] 
Allora $\operatorname{O}(n)=\varphi^{-1}(I)$. Quindi abbiamo una mappa tra gli spazi tangenti a $I$ data da 
\[T:\funcDef{T_I\Mat_{n\times n}}{T_I\Mat_{n\times n}}{B}{B+B^t}\]
Quindi $T_I\operatorname{O}(n)=\ker T$. Quindi $\dim\SO(n)=\binom{n}{2}$ e in particolare $\dim\SO(3)=3=\dim\SL(2)$.  Inoltre, il quoziente $\SL(2)/\{\pm I\}$ coincide con la componente connessa di $\SO(\beta)$, quindi è sufficiente mostrare che $\SO(\beta)$ (e più in generale $\SO(n)$) è connesso. 

Procediamo per induzione su $n\ge 1$. Per $n=1$ e $n=2$ sappiamo già che ciò è vero. Assumiamo $n>2$. Consideriamo la forma quadratica standard rappresentata dalla matrice identità, consideriamo $e_1$ in $\K^n$ e 
\[Y\coloneqq \SO(n)\cdot e_1=\left\{ \begin{pmatrix}
        x_1 \\ \vdots \\ x_n 
    \end{pmatrix} \colon x_1^2+\ldots+x_n^2=1\right\}\] 
Poiché il polinomio $x_1^2+\ldots+x_n^2-1$ è irriducibile, la varietà $Y$ è connessa. Inoltre si ha $Y\cong \SO(n)/\SO(n-1)$, che, in quanto connesso, ci permette di concludere.
\end{proof}

 





















Ricordiamo che $\SO(3)\cong \SL(2)/\{\pm I\}$.
Sia 
\[B=\cpa{\mat{a&b\\0&a\ii}}\subseteq \SL(2)\]
Notiamo che $\SL(2)$ agisce su $\Pj^1$ transitivamente e che lo stabilizzatore di $[e_1]$ \`e $B$.


Pi\`u precisamente, sia $V=\K^2$ e $L=[e_1]\in\Pj(V)$. Poniamo
\[W=S^n\K^2=S^nV\]
e, notando che $e_1^n\in W$, si ha $\stab_G[e_1^n]=B$.

Possiamo descrivere il quoziente anche come
\[\Pj(V)=\Pj^1\cong G/B=G[e_1^n]\subseteq \Pj(W)=\Pj(S^n V)\]
dove l'immersione \`e data da $v\mapsto v^n$. Questa mappa \`e un morfismo equivariante e una immersione chiusa $\Pj(V)\inj \Pj(W)$.





\begin{remark}
Se $n=2$ allora
\[\SO(3)/B_{\SO(3)}=\frac{\SL(2)/\pm id}{B_{\SL}/\pm id}=\frac{\SL(2)}{B_{\SL}}.\]
Lo spazio $V=\K^2$ non \`e una rappresentazione di $\SO(3)$ perch\'e $-id$ non agisce banalmente, ma $S^2V$ \`e una rappresentazione di $\SO(3)$.
\end{remark}



\begin{example}
Consideriamo la forma quadratica $x_1x_3=x_2^2$, lo spazio $V=\K^3$ e la retta $L=\K(1\ 0\ 0)^\top$. In $\Pj(V)$ si ha che $G\cdot L$ \`e la quadrica $Q:x_1x_3=x_2^2$ per il teorema di Witt (\ref{ThWitt}).

Notiamo anche che $\stab(L)=B$, quindi abbiamo una mappa
\[\funcDef{\Pj^1}{Q}{[a:b]}{[a^2:ab:b^2]}.\]
\end{example}







\subsection{Matrici del gruppo lineare speciale}

Sia $G=\SL(n)$ e ricordiamo che il sottogruppo delle matrici triangolari superiori $B$ \`e un sottogruppo di Borel.
\[G/B\cong\cpa{F_1\subseteq\cdots\subseteq F_n\mid \dim F_i=i}\]

\begin{definition}
Definiamo la \textbf{variet\`a delle bandiere} di taglia $n$ come
\[\GL(n)/B_{\GL(n)}=\SL(n)/B_{\SL(n)}=\Fs\ell=\cpa{F_1\subseteq\cdots\subseteq F_n\mid \dim F_i=i}.\]
\end{definition}

Vogliamo un risultato analogo per $\SO(n)$
\begin{remark}
Se consideriamo la forma quadratica $q=x_1^2+\cdots+x_n^2$ e interseco $\SO(n)=\cpa{A\mid A^\top A=I,\det A=1}$ con $B_{\SL(n)}$ troviamo
\[T^\top T=I\quad e \quad T\text{ triangolare superiore}\implies T\text{ diagonale con entrate }\pm1.\]
\end{remark}


Consideriamo la forma quadratica associata a
\[J=\mat{
0&\cdots&0&1\\
\vdots&0&\iddots&0\\
0 &\iddots& 0&\vdots\\
1&0&\cdots&0
}\]
da cui $SO(n)=\cpa{A\mid AJ A^\top=J,\ \det A=1}$. 

\begin{proposition}
Il sottogruppo di questo $\SO(n)$ dato da
\[B=B_{\SO}=\cpa{\mat{A&B\\0&D}\mid D=(A^{at})\ii,\ (BA^{at})^{at}=-(BA^{at}),\ A\text{ triang. sup.}},\]
dove $M^{at}$ indica la trasposta di $M$ rispetto all'altra diagonale, \`e un sottogruppo di Borel.
\end{proposition}
\begin{proof}
Assumiamo $n=2m$, il caso dispari \`e simile.

Notiamo che $Jg^\top J=g^{at}$, cio\`e la trasposta di $g$ rispetto all'altra diagonale. Scrivendo $g$ triangolare superiore evidenziando blocchi di dimensione $m$ troviamo
\[g=\mat{A&B\\0&D}.\]
Poich\'e $g\in SO(n)$ significa che $gg^{at}=id$ con determinante 1, $g$ triangolare superiore appartiene a $\SO(n)$ se
\[\mat{I_m &0\\0&I_m}=\mat{A&B\\0&D}\mat{D^{at}&B^{at}\\0&A^{at}}=\mat{AD^{at}&AB^{at}+BA^{at}\\0&DA^{at}}\]
ovvero se $D=(A^{at})\ii$ e $(BA^{at})^{at}=-(BA^{at})$. Il determinante di $g$ \`e automaticamente $1$ perch\'e\footnote{$\det M^{at}=\det(JM^\top J)=1\cdot \det M^\top\cdot 1=\det M$.} $\det D=(\det A)\ii$.
Dunque l'intersezione tra $\SO(n)$ e le triangolari superiori \`e
\[B=B_{\SO}=\cpa{\mat{A&B\\0&D}\mid D=(A^{at})\ii,\ (BA^{at})^{at}=-(BA^{at}),\ A\text{ triang. sup.}}.\]
Notiamo che $B_{\SO}$ \`e risolubile e connesso\footnote{connesso perch\'e immagine di $\cpa{\text{triang.sup.invertibili}}\times \cpa{\text{antisimmetriche per l'altra diagonale}}$}. Se dimostriamo che $B$ \`e parabolico allora \`e un sottogruppo di Borel per (\ref{CorPrabolicoRisolubileConnessoEBorel}). 



Sia $\Es$ la bandiera data dai vettori di base 
\[E_1=\ps{e_1}\subseteq E_2=\ps{e_1,e_2}\subseteq\cdots\subseteq E_n=\ps{e_1,\cdots, e_{2m}}=\K^n.\] 
Notiamo che $\stab_{\GL(n)}(\Es)=B_{\GL(n)}$ e che $\stab_{O(n)}(\Es)=B_{\GL(n)}\cap O(n)=B_{O(n)}=B_{\SO(n)}$, dove l'ultima uguaglianza segue perch\'e tutte le matrici ortogonali triangolari superiori per la forma associata a $J$ hanno gi\`a determinante 1.
Notiamo che 
\[O(n)\Es\subseteq \cpa{0=F_0\subseteq\cdots\subseteq F_{2m}\mid \dim F_i=i,\ F_{m-i}^\perp=F_{m+i}\ \forall i\in\cpa{0,\cdots,m}}\subseteq \Fs\ell.\]
Per il teorema di Estensione di Witt (\ref{ThWitt}) si ha che $O(n)\Es$ \`e esattamente lo spazio delle bandiere di quella forma, che chiamiamo $\Fs_O$ e notiamo che \`e un chiuso di $\Fs\ell$.

Vogliamo dimostrare che $\Fs_{\SO}$ \`e chiuso. Sia
\[\Gs_O=\cpa{g\in \GL(n)\mid gB_{\GL}\in \Fs_O}\]
Poich\'e $\Fs\ell=\GL(n)/B_{\GL}$ basta mostrare che $\Gs_O$ \`e chiuso. Sia $g=\pa{v_1\mid\cdots\mid v_{2m}}$ e notiamo che $g\in\Gs_O$ se $b(v_i,v_j)=0$ per $i,j\leq m$, $b(v_{m+\al},v_i)=0$ per $i\leq m-\al$ e $b(v_i,v_j)=0$ se $i+j\leq 2m$. Queste sono condizioni chiuse quindi $\Gs_O$ \`e chiuso. Questo mostra che
\[O(n)/B_{\SO(n)}\]
\`e proiettivo.
\end{proof}
\begin{remark}
Con le notazioni di sopra,
\[O(n)=\frac{\SO(n)\sqcup g\SO(n)}{B_{\SO(n)}}\]
per una qualsiasi $g$ tale che $\det g=-1$, $g\in O(n)$, dunque
\[\frac{O(n)}{B_{\SO}}=\frac{\SO(n)}{B_{\SO}}\sqcup g\frac{\SO(n)}{B_{\SO}}\]
e quindi $\frac{\SO(n)}{B_{\SO}}$ \`e proiettivo.
\end{remark}


\begin{example}[Parabolici di $\GL(n)$]
Sia $P\supseteq B$ con $P$ parabolico e $B$ di Borel, $P$ connesso.

Guardiamo $T_e P$ e notiamo che $T_e P\supseteq T_e B=\cpa{\text{triang.sup.}}$. Sia $A=(a_{i,j})\in T_e P$. Si ha che $B$ agisce per coniugio su $P$. In particolare se fissiamo $b\in B$ allora definiamo
\[AD_b:\funcDef{P}{P}{p}{bpb\ii}\quad\leadsto\quad Ad_b:\funcDef{T_eP}{E_eP}{C}{bCb\ii}\]
quindi $T_e P\subseteq T_e\GL(n)=\Mat_{n\times n}$ \`e stabile per $Ad_b$ per ogni $b\in B$. Se $b=diag(\la_1,\cdots,\la_n)$ allora
\[Ad_b(A)=\pa{\la_i\la_j\ii a_{i,j}}\]
se $a_{i,j}\neq 0$ allora $E_{i,j}\in T_eP$ 
\end{example}