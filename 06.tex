\chapter{Quozienti}
\section{Costruzione dei quozienti}
\begin{lemma}\label{LmSottogruppoEStabilizzatoreDiUnaRettaInQualcheRappresentazione}
Sia $G$ un gruppo algebrico e $H$ sottogruppo di $G$, allora
\begin{enumerate}
    \item Esistono una rappresentazione di dimensione finita $V$ di $G$ e una retta $L\subseteq V$ tali che $H=\stab_G L=\cpa{g\in G\mid g(L)=L}$
    \item Se $H$ \`e normale allora $V$ si pu\`o scegliere in modo che sia somma dei $V_\al$ per $\al\in X(H)$ e $V_\al=\cpa{v\in V\mid h\cdot v=\al(h)v}$.
\end{enumerate}
\end{lemma}
\begin{proof}
Mostriamo le due affermazioni
\setlength{\leftmargini}{0cm}
\begin{enumerate}
\item Sia $I_H\subseteq \K[G]$ l'ideale che definisce $H$, allora
\[H=\cpa{g\in G\mid g(I_H)=I_H}\]
\setlength{\leftmargini}{0cm}
\begin{itemize}
\item[$\boxed{\subseteq}$] Se $g\in H$ e $f\in I_H$ allora $gf(k)=f(g\ii k)$, quindi se $k\in H$ allora $g\ii k\in H$ e quindi questa funzione vale $0$, cio\`e $gf\in I_H$. L'altra inclusione segue dallo stesso ragionamento fatto su $g\ii$.
\item[$\boxed{\supseteq}$] Se $g\ii(I_H)\subseteq I_H$ allora per ogni $f\in I_H$
\[f(g)=\under{\in I_H}{(g\ii f)}(e)=0\]
cio\`e $g\in H$.
\end{itemize}
\setlength{\leftmargini}{0.5cm}
Consideriamo ora dei generatori $f_1,\cdots, f_m$ per $I_H$ e sia $V_0\subseteq \K[G]$ una $G$-sotto-rappresentazione di dimensione finita che contiene ogni $f_i$. Consideriamo il sottospazio vettoriale $W_0=I_H\cap V_0$ e notiamo che
\[H=\cpa{g\in G\mid g(W_0)= W_0}.\]
\setlength{\leftmargini}{0cm}
\begin{itemize}
\item[$\boxed{\subseteq}$] Ovvio per quanto detto sopra.
\item[$\boxed{\supseteq}$] Se $g(W_0)= W_0$ allora $g(I_H)=I_H$, infatti $g(I_H)\subseteq I_H$ ovvio per costruzione di $W_0$, l'altra inclusione segue dal fatto che $g(W_0)=W_0\coimplies g\ii(W_0)=W_0$.
\end{itemize}
\setlength{\leftmargini}{0.5cm}
Sia $\dim W_0=m$. Poniamo
\[V=\bw^m V_0,\quad L=\bw^m W_0\subseteq \bw^m V_0.\]
Per concludere basta mostrare che
\[H=\cpa{g\in G\mid g(L)=L}\]
\setlength{\leftmargini}{0cm}
\begin{itemize}
\item[$\boxed{\subseteq}$] Se $g(W_0)=W_0$ allora chiaramente 
\[g(L)=g\pa{\bw^mW_0}=\bw^m g(W_0)=\bw^m W_0=L.\] 
\item[$\boxed{\supseteq}$] Notiamo che se $u_1,\cdots, u_m$ \`e una base di $U\subseteq V$ sottospazio vettoriale allora 
\[U=\cpa{u\in V\mid u\wedge u_1\wedge\cdots\wedge u_m=0}.\]
Fissiamo una base $w_1,\cdots, w_m$ di $W_0$ e osserviamo che $g w_1,\cdots, gw_m$ \`e una base di $g(W_0)$. Se $g(L)=L$ allora per definizione
\[\ps{g w_1\wedge\cdots\wedge gw_m}=\ps{w_1\wedge\cdots\wedge w_m},\] 
quindi per il criterio appena citato si ha che
\begin{align*}
    W_0=&\cpa{v\in V\mid v\wedge w_1\wedge\cdots\wedge w_m=0}=\\
    =&\cpa{v\in V\mid u\wedge gw_1\wedge\cdots\wedge gw_m=0}=g(W_0).
\end{align*}
\end{itemize}
\setlength{\leftmargini}{0.5cm}
\item Sia $V'=\bigoplus_{\al\in X(H)} V_\al\subseteq V$ con $V$ di prima. Mostriamo che $V'$ \`e $G$-invariante:
Se $v_\al\in V_\al$ per $\al\in X(H)$, $h\in H$ e $g\in G$ allora
\[h\cdot(g v_\al)=(gg\ii h g)\cdot v_\al=g(\al(g\ii hg)v_\al)=\al(g\ii hg)gv_\al,\]
cio\`e $gv_\al$ \`e un autovettore per l'azione di $h$ per un qualsiasi $g\in G$ e $h\in H$, ovvero $V'$ \`e $G$-invariante. 

Per concludere \`e dunque sufficiente mostrare che $L\subseteq V'$, ma abbiamo gi\`a visto che 
\[g(w_1\wedge\cdots \wedge w_m)=gw_1\wedge\cdots \wedge gw_m=\la(g) w_1\wedge\cdots\wedge w_m\] 
per qualche $\la(g)$ per ogni $g\in H$, quindi $L\subseteq V_\la$.
\end{enumerate}
\setlength{\leftmargini}{0.5cm}
\end{proof}

Siano $H< G$ e $L,V$ come nel lemma. Allora $L\in \Pj(V)$ per definizione di spazio proiettivo. Definiamo la variet\`a proiettiva
\[Y=\ol{G\cdot L}\subseteq \Pj(V)\]
e scriviamo $X=G\cdot L$. Mostriamo che $X$ \`e aperto: data la mappa
\[\funcDef{G}{Y}{g}{gL},\]
per Chevalley (\ref{ThChevalley}) si ha che $X$ contiene un aperto $U$ di $Y$. Se $x\in U\subseteq X$ allora $gx\in gU\subseteq X$ quindi
\[X=\bigcup_{g\in G}gU\text{ \`e aperto.}\]
Insiemisticamente si ha $X=G\cdot L=G/\stab_G(L)=G/H$, quindi prendere la chiusura \`e in un qualche modo il minimo indispensabile per rendere $G/H$ una variet\`a.


\begin{exercise}
Sia $G=\GL(2)$ e $H=B(2)=\cpa{\mat{\ast&\ast\\0&\ast}}$. Sia $V=\K^2$ e $L=\K e_1$, allora effettivamente $H=\stab_G(L)$ e $G\cdot L=\Pj(V)$ (ogni retta si ottiene da $L$ applicando una trasformazione lineare), quindi $G/H\leftrightarrow \Pj(V)$.
\end{exercise}

\begin{exercise}
Sia $G=\GL(n,\C)$ e $H=O(n,\C)$. Sia $V=\Sym(n,\C)$ lo spazio delle matrici simmetriche. Se $A\in V$ e $g\in G$ agisce su $A$ tramite 
\[g\cdot A=gA g^\top\]
allora $H=\stab_G(I_n)$ (stabilizzatore della matrice identit\`a).
\[X=G\cdot I_n=\cpa{gg^\top\mid g\in \GL(n)}=\cpa{A\in \Sym(n)\mid \det(A)\neq 0}\subseteq\ol{G\cdot I_n}=Y\subseteq \Pj(V).\]
%Siano $V'=V\oplus \K$ e $L=\ps{(I_n,1)}$, allora \[\stab_G(L)=H\text{ e }\Pj(V')=V\oplus \Pj(V).\]
\end{exercise}

\begin{proposition}
Se $H$ \`e normale, $G/H$ \`e un gruppo algebrico affine.
\end{proposition}
\begin{proof}
Costruiamo $L$ e $V=\bigoplus V_\al$ come nel punto 2. del lemma (\ref{LmSottogruppoEStabilizzatoreDiUnaRettaInQualcheRappresentazione}). Sia
\[W=\cpa{T:V\to V\mid \forall \al\in X(H),\ T(V_\al)\subseteq V_\al}\]
e notiamo che $G$ agisce su $W$ come $gT=g\circ T\circ g\ii$. Per rendere valido quanto detto dobbiamo verificare che $g T g\ii(V_\al)=V_\al$, ma questo segue dal fatto che se $g\ii(V_\al)=V_\beta$ allora
\[g T g\ii(V_\al)=gT(V_\beta)\subseteq gV_\beta=V_\al.\]
Notiamo ora che
\[\cpa{g\in G\mid g\res W=id_W}=\cpa{g\in G\mid g\res{V_\al}=\la_\al id_{V_\al}\ \forall \al\in X(H)}=H,\]
la seconda uguaglianza segue dalla definizione di carattere mentre la prima si ricava osservando le matrici associate agli elementi di $g$ visti come automorfismi di $V$: gli elementi di $W$ sono diagonali a blocchi e ci\`o che commuta\footnote{$g\res W=id_W$ significa che per ogni $T\in W$ $gTg\ii=T$, cio\`e $gT=Tg$.} con tutte le diagonali a blocchi sono le cose che sono multiplo di identit\`a su ogni blocco.

Abbiamo dunque costruito un omomorfismo di gruppi algebrici $\vp:G\to \GL(W)$ il cui nucleo \`e $H$. Poich\'e $\vp(G)$ \`e un sottogruppo chiuso di $\GL(W)$ per Chevalley (\ref{PrMorfismoTraGruppiAlgebriciAffiniHaImmagineChiusa}) si ha che $G/H\cong \vp(G)$ eredita la struttura di gruppo algebrico lineare da $\vp(G)$.
\end{proof}

















\begin{remark}
    Se $H$ è un sottogruppo di $G$ e $\pi\colon G \to X=G\cdot L$, allora $\pi$ è $G$-equivariante e induce una bigezione tra $G/H$ e $X$.
\end{remark}


Nel seguito supporremo $\cha(\K)=0$.

\begin{proposition} Se $H$ è un sottogruppo di $G$ e $\pi\colon G \to X$ una mappa $H$-equivariante, valgono le seguenti proprietà.
    \begin{enumerate}
        \item Per ogni varietà $Z$, la mappa \[G\times Z \xrightarrow{(\pi,id)}X\times Z\] è aperta.
        \item Per ogni aperto $U$ di $X$, si ha un isomorfismo \[\pi^\ast:\Oc_X(U)\xrightarrow{\sim}(\Oc_G(\pi^{-1}(U)))^H.\]
    \end{enumerate}
\end{proposition}

\begin{proof}
Notiamo che $\pi\colon G \to X$ induce una bigezione insiemistica tra $G/H$ e $X$. Poiché per ogni varietà $Z$ la mappa $(\pi,id)\colon G\times Z \to X\times Z$ è liscia ($\pi$ è liscia), tale mappa è anche aperta (\ref{ThMappaLisciaEAperta}).

Sia $U$ un aperto di $X$, poniamo $V=\pi\ii(U)$, e consideriamo la mappa
\[\pi^\ast:\Oc_X(U)\to \Oc_G(V)^H.\]
Osserviamo che tale mappa è iniettiva, infatti se $f(\pi(x))=0$ per ogni $x$ in $V=\pi\ii (U)$, si ha $f(y)=0$ per ogni $y$ in $U$, per cui $f=0$. 

Mostriamo che la mappa $\pi^{-1}(U)\to U$ è surgettiva.
Posso ridurmi al caso $U$ irriducibile:

In $G$ le componenti connesse coincidono con le componenti irriducibili (\ref{ReInGruppoComponentiConnesseSonoIrriducibili}). Lo stesso vale per $X$, infatti se $G^0$ \`e la componente connessa di $1_G$ allora da $X=G/H$ troviamo che $X^0=G^0/H\cap G^0$ \`e aperto e $X$ \`e unione finita disgiunta dei traslati di $X^0$ ($G$ ha finite componenti irriducibili per Noetherianit\`a).

Se $U$ \`e un aperto di $X$ allora 
\[X=X^0\sqcup g_1X^0\sqcup\cdots\sqcup g_nX^0\implies U=U\cap X^0\sqcup \cdots\sqcup U\cap g_n X^0.\]
Poich\'e $X^0$ irriducibile i suoi aperti sono irriducibili, quindi per localit\`a della verifica di surgettivit\`a sui pullback per aperti di $X^0$, ci siamo quindi ricondotti al caso $X$ e $U$ irriducibili.

Sia $f:V\to \K$ regolare $H$-equivariante e consideriamo il grafico $\Gamma(f)\subseteq V\times \K$. Sia $g$ la fattorizzazione 
% https://q.uiver.app/#q=WzAsMyxbMCwwLCJWIl0sWzIsMCwiVT1WL0giXSxbMCwyLCJcXEsiXSxbMCwyLCJmIiwyXSxbMCwxLCJcXHBpIl0sWzEsMiwiZyIsMCx7InN0eWxlIjp7ImJvZHkiOnsibmFtZSI6ImRhc2hlZCJ9fX1dXQ==
\[\begin{tikzcd}
	V && {U=V/H} \\
	\\
	\K
	\arrow["\pi", from=1-1, to=1-3]
	\arrow["f"', from=1-1, to=3-1]
	\arrow["g", dashed, from=1-3, to=3-1]
\end{tikzcd}\]
Decomponiamo $V$ in irriducibili $V=V_1\sqcup\cdots\sqcup V_n$ e per irriducibilit\`a di $U$ deve essere il caso che $H$ agisce transitivamente su $\cpa{V_1,\cdots, V_n}$.

Per il punto 1., $(\pi,id):G\times \K\to X\times\K$ \`e una mappa aperta, quindi la restrizione $\psi:V\times \K\to U\times \K$ resta aperta perch\'e $V\times \K$ \`e un aperto di $G\times \K$. Segue che\footnote{Se $(v,f(v))\in\Gamma(f)$ allora $\psi(v,f(v))=(\pi(v),f(v))=(\pi(v),g(\pi(v)))$.} $\psi(\Gamma(f))=\Gamma(g)$ \`e un chiuso di $U\times \K$. In realt\`a notando che $\Gamma(g)$ \`e $H$ invariante e che esso \`e immagine di $\Gamma(f)=\bigsqcup \Gamma(f)\cap V_i$ si ha che in realt\`a possiamo scrivere $\Gamma(g)$ solo come immagine di un singolo $\Gamma(f)\cap V_i$, in particolare \`e irriducible (altrimenti potremmo decomporre $V$ ulteriormente).

Consideriamo ora il diagramma
% https://q.uiver.app/#q=WzAsMyxbMCwwLCJcXEdhbW1hKGcpIl0sWzIsMCwiVVxcdGltZXMgXFxLIl0sWzIsMiwiVSJdLFswLDEsIiIsMCx7InN0eWxlIjp7InRhaWwiOnsibmFtZSI6Imhvb2siLCJzaWRlIjoidG9wIn19fV0sWzEsMiwicF9VIl0sWzAsMiwicSIsMl1d
\[\begin{tikzcd}
	{\Gamma(g)} && {U\times \K} \\
	\\
	&& U
	\arrow[hook, from=1-1, to=1-3]
	\arrow["q"', from=1-1, to=3-3]
	\arrow["{p_U}", from=1-3, to=3-3]
\end{tikzcd}\]
dove $p_U$ \`e la proiezione su $U$ e $q$ \`e la restrizione a $\Gamma(g)$. Chiaramente $q$ \`e regolare in quanto composizione di regolari. $q$ \`e anche bigettiva perch\'e $(u,g(u))\mapsto u$ pu\`o essere facilmente invertita. Poich\'e $U$ \`e liscio e $\Gamma(g)$ irriducibile, per il teorema di Zariski (\ref{ThZariski}) si ha che $q$ \`e un isomorfismo, quindi la mappa $u\mapsto (u,g(u))$ un morfismo e in particolare $g$ stesso \`e un morfismo. Questo mostra che $\pi^\ast$ effettivamente \`e surgettiva perch\'e abbiamo trovato $g\in \Oc_X(U)$ tale che $\pi^\ast(g)=g\circ \pi=f$.
\end{proof}


\begin{theorem}
    Per ogni $G$-varietà $Y$ e per ogni $y_0$ in $Y$ tale che $H$ è contenuto in $\stab_G(y_0)$, vale la seguente proprietà: se $\varphi\colon G \to Y$ è la mappa definita da $\varphi(g)=gy_0$, allora esiste un'unica $\psi\colon X \to Y$  tale che $\psi\circ\pi=\varphi$.
    % https://q.uiver.app/#q=WzAsMyxbMCwwLCJHIl0sWzIsMCwiWCJdLFsxLDEsIlkiXSxbMCwxLCJcXHBpIl0sWzAsMiwiXFx2YXJwaGkiLDJdLFsxLDIsIlxcZXhpc3RzICFcXHBzaSIsMCx7InN0eWxlIjp7ImJvZHkiOnsibmFtZSI6ImRhc2hlZCJ9fX1dXQ==
\[\begin{tikzcd}
	G && X \\
	& Y
	\arrow["\pi", from=1-1, to=1-3]
	\arrow["\varphi"', from=1-1, to=2-2]
	\arrow["{\exists !\psi}", dashed, from=1-3, to=2-2]
\end{tikzcd}\]
\end{theorem}

\begin{proof}
    Insiemisticamente, la mappa $\psi\colon gH \mapsto gy_0$ è definita. Inoltre $\psi$ è continua: se $U$ è un aperto di $Y$, allora \[\psi^{-1}(U)=\pi(\pi^{-1}(\psi^{-1}(U)))=\pi(\varphi^{-1}(U)),\] che è aperto. Dato un aperto $U$ di $Y$, verifichiamo che l'immagine della mappa 
    \[\psi^\ast:\Oc_Y(U)\to\Hom_{(\mathrm{Set})}(\psi^{-1}(U),\K)\]
    è contenuta nell'insieme delle funzioni regolari su $X$. Consideriamo una funzione $f\colon U\to \K$. Allora $\wt{f}:= f\circ \psi\circ\pi =f \circ \varphi$ è in $\Oc_G(\varphi^{-1}(U))$ in quanto $\varphi$ è un morfismo di varietà. Mostriamo che in realtà è in $\Oc_G(\varphi^{-1}(U))^H$. Sia $\lambda$ in $\Oc_G(\varphi^{-1}(U))$ e sia $h$ in $H$. Allora per ogni $x$ in $G$ si ha \[(h\widetilde{f})(x)=\widetilde{f}(xh)=f(\varphi(xh))=f(xhy_0)=f(x y_0)=f(\varphi(x))=\widetilde{f}(x).\]
    %\[(g\cdo \lambda)(x)=\lamb\]
\end{proof}

\begin{definition}
Una variet\`a \`e \textbf{omogenea} rispetto al gruppo $G$ se l'azione di $G$ su $X$ \`e transitiva.
\end{definition}

\begin{corollary}
    Se $X$ è una varietà omogenea rispetto a $G$, allora $X$ è liscia.
\end{corollary}
\begin{proof}
    Poiché $X$ ha un aperto $U$ di punti lisci, la tesi segue dal fatto che $X=\bigcup_{g\in G} gU$.
\end{proof}


\begin{corollary}\label{CorEquivarianteTraVarietaOmogeneeELiscia}
    Se $X$ e $Y$ sono varietà omogenee per $G$ e $\varphi\colon X \to Y$ è $G$-equivariante, allora $\varphi$ è liscia.
\end{corollary}
\begin{proof}
    Sicuramente $\varphi$ è surgettiva, perché $X$ e $Y$ sono varietà omogenee per $G$ e $\varphi$ è $G$-equivariante. Per il teorema (\ref{ThMorfismoDominanteTraLisceHaRestrizioneLisciaSuAperto}) esiste un aperto non vuoto $U$ di $Y$ tale che $\varphi\res{\varphi^{-1}(U)}\colon \varphi^{-1}(U) \to U$ è liscia. Allora si ha un diagramma commutativo % https://q.uiver.app/#q=WzAsNCxbMCwwLCJcXHZhcnBoaV57LTF9KGdVKSJdLFsyLDAsImdVIl0sWzAsMiwiXFx2YXJwaGleey0xfShVKSJdLFsyLDIsIlUiXSxbMCwxLCJcXHZhcnBoaVxccmVze2dcXHZhcnBoaV57LTF9KFUpfSJdLFsyLDMsIlxcdmFycGhpXFxyZXN7XFx2YXJwaGleey0xfShVKX0iXSxbMiwwLCJcXHNpbSJdLFszLDEsIlxcc2ltIiwyXV0=
\[\begin{tikzcd}
	{\varphi^{-1}(gU)} && gU \\
	\\
	{\varphi^{-1}(U)} && U
	\arrow["{\varphi\res{g\varphi^{-1}(U)}}", from=1-1, to=1-3]
	\arrow["\sim", from=3-1, to=1-1]
	\arrow["{\varphi\res{\varphi^{-1}(U)}}", from=3-1, to=3-3]
	\arrow["\sim"', from=3-3, to=1-3]
\end{tikzcd}\] 
Poiché $Y=\bigcup gU$, $\varphi$ è liscia in ogni punto e quindi è liscia.

%\[\varphi\res{g\varphi^{-1}(U)}\colon \varphi^{-1}(gU) \to gU\]
\end{proof}









\section{Sottogruppo generato}
\begin{lemma}
Sia $G$ un gruppo algebrico e siano $X_i$ delle variet\`a irriducibili. Siano $\vp_i:X_i\to G$ tali che $1_G\in \imm\vp_i$ per ogni $i$. Poniamo $Y_i=\vp_i(X_i)$.

Sia $H=\ps{\cpa{Y_i}_i}$ il sottogruppo generato dalle immagini delle $\vp_i$. Allora

\begin{enumerate}
	\item $H$ \`e chiuso
	\item esistono $i_1,\cdots, i_n$, $\e_1,\cdots, \e_n$ tali che
	\[H=Y_{i_1}^{\e_1}\cdots Y_{i_n}^{\e_n},\qquad \e_j\in\cpa{1,-1}\]
\end{enumerate}
\end{lemma}
\begin{proof}
Supponiamo $X_i=Y_i\subseteq G$ e supponiamo che tra le $X_i$ compaiano anche le $X_i\ii$ (cos\`i evitiamo gli $\e$).

Definiamo iterativamente
\begin{align*}
	Z_1&=X_1, &W_1&=\ol{Z_1},\\
	Z_2&=X_1\cdot X_2, &W_2&=\ol{Z_2},\\ 
	&\;\;\vdots &&\;\;\vdots\\	
	Z_n&=X_1\cdots X_n, &W_n&=\ol{Z_n},\\
	Z_{n+1}&=X_1\cdots X_nX_1, &W_{n+1}&=\ol{Z_{n+1}}\\
	&\;\;\vdots &&\;\;\vdots
\end{align*}
Notiamo che $Z_i$ \`e l'immagine di $X_1\times\cdots\times X_i\to G$, quindi $Z_i$ \`e irriducible per ogni $i$, dunque anche $W_i=\ol{Z_i}$ \`e irriducibile.

Notiamo ora che $W_1\subseteq W_2\subseteq\cdots\subseteq G$ \`e una catena di chiusi, ma dato che $G$ ha dimensione finita essa stabilizza, cio\`e esiste $N$ tale che $W_N=W_{N+1}$, ovvero $W_N\cdot X_i\subseteq W_N$ per ogni $i$.

Allora $W_NZ_N\subseteq W_N$ cio\`e $W_N\cdot W_N\subseteq W_N$ perch\'e $W_n$ \`e chiuso. Quindi\footnote{$Z_N\ii W_N\subseteq W_N\implies Z_N\ii\subseteq W_N\implies \ol{Z_N\ii}\subseteq W_N$ e $\ol{Z_N\ii}=\ol{Z_N}\ii=W_N\ii$.} $W_N\ii\subseteq W_N$ e in particolare $Z_N\ii\subseteq W_N$, ma $Z_N\ii=X_N\ii\cdots X_1\ii=X_{i_1}\cdots X_{i_N}$, cio\`e mettendo tutto insieme $H\subseteq W_N$

Mostriamo che $W_N=Z_N\cdot Z_N$: abbiamo una mappa
\[\under{irrid.}{X_1\times\cdots\times X_N}\to Z_N\to W_N,\]
quindi per Chevalley (\ref{ThChevalley}) $Z_N\supseteq U$ per $U$ aperto non vuoto di $W_N$, dunque $U\cdot U=W_N$ perch\'e $W_N$ \`e irriducibile.

Questo mostra che $W_N\subseteq H$ ma ci sono tutti gli elementi di $H$ quindi effettivamente $H=W_N$.
\end{proof}


\section{Variet\`a complete}

\begin{definition}[Variet\`a separata]
$X$ \`e \textbf{separata} se la diagonale $X\to X\times X$ \`e un morfismo chiuso.
\end{definition}
\begin{center}
	\textbf{Supporremo che sia tutto separato.}
\end{center}

\begin{definition}[Variet\`a completa]
$X$ \`e \textbf{completa} se per ogni $Z$ variet\`a, $\pi:X\times Z\to Z$ \`e un morfismo chiuso.
\end{definition}

\begin{remark}\label{ReRegolariSuCompleteIrriducibiliSonoCostanti}
Se $X$ \`e irriducibile e completa allora $\Oc_X(X)\cong \K$.
\end{remark}
\begin{proof}
Sia $f:X\to \K$ regolare e consideriamo $\Gamma(f)\subseteq X\times \K\to \K$. Questa proiezione \`e chiusa quindi $\pi(\Gamma(f))=f(X)$ \`e un chiuso irriducibile (immagine di irriducibile) di $\K$. Ma i chiusi irriducibili di $\K$ non vuoti sono o tutto $\K$ o solo un punto.

Consideriamo ora $W=\cpa{(x,y)\in X\times \K\mid yf(x)=1}$. Se $f(X)=\K$ allora l'immagine di $W$ tramite la proiezione $X\times \K\to \K$ \`e $\K^\times$, che non \`e chiuso in $\K$, assurdo.

Quindi $f(X)$ \`e un solo punto, cio\`e $f$ \`e costante.
\end{proof}
\begin{corollary}
Se $X$ \`e completa, affine e connessa allora $\K[X]=\K$ significa che $X$ \`e un solo punto $\cpa{(0)}$.

In generale le affini complete sono un numero finito di punti.
\end{corollary}

\begin{theorem}[I proiettivi sono completi]\label{ThProiettiviSonoCompleti}
Se $V$ \`e uno spazio vettoriale di dimensione finita, $\Pj(V)$ \`e completa.
\end{theorem}

\begin{center}
	Moralmente ``completo=compatto" in senso classico, per esempio valgono:
\end{center}

\begin{remark}
Se $X$ \`e completa e $Z\subseteq X$ chiuso allora $Z$ \`e completo.
\end{remark}

\begin{remark}
Se $\vp:X\to Y$ \`e surgettiva e $X$ \`e completa allora $Y$ \`e completa
\end{remark}

\subsection{Punto fisso di Borel}
\begin{proposition}[Esiste orbita chiusa]\label{PrEsisteOrbitaChiusa}
Se $K$ \`e un gruppo che agisce su una variet\`a $Y$ allora $K$ ha un'orbita chiusa in $Y$.
\end{proposition}
\begin{proof}
Consideriamo un'orbita che ha dimensione minima $Z=G\cdot y$. Notiamo che $Gy$ \`e un aperto denso di $\ol Z$ per Chevalley (\ref{ThChevalley}) (contiene un aperto ed \`e omogeneo), quindi $\dim \ol Z\bs Z<\dim Z$, ma se questa differenza \`e non vuota allora $G$ agisce su questa differenza e quindi esiste un'orbita di dimensione pi\`u piccola.

Dunque $\ol Z\bs Z=\emptyset$, cio\`e $\ol Z=Z$.
\end{proof}

\begin{theorem}[Punto fisso di Borel]\label{ThPuntoFissoBorel}
Se $G$ \`e un gruppo risolubile connesso che agisce su una variet\`a completa allora $G$ ha un punto fisso.
\end{theorem}
\begin{proof}
Se $\dim G=0$ allora $G$ \`e un singolo punto e quindi \`e l'indetit\`a e agisce banalmente.


Supponiamo $\dim G>0$. Sia $H=[G,G]$ il sottogruppo dei commutatori. Questo \`e un sottogruppo algebrico connesso. Dato che $G$ \`e risolubile, $H\subsetneq G$, quindi per ipotesi induttiva $X^H\neq \emptyset$.

Notiamo ora che su $X^H$ agisce $A=G/H$, quindi su $X^H$ abbiamo un'orbita chiusa (\ref{PrEsisteOrbitaChiusa}) $A\cdot x\subseteq X^H\subseteq X$. Notiamo che $A\cdot x$ \`e completa perch\'e chiuso di $X$ completa, ma $A\cdot x=A/\stab_A x$, quindi \`e anche una variet\`a affine. $A\cdot x$ \`e anche connessa perch\'e $A$ \`e connesso in quanto $G$ lo \`e.

$A\cdot x$ \`e affine, completa e connessa, quindi \`e un punto, cio\`e $x$ \`e un punto fisso per $A$ in $X^H$, ma allora $x$ \`e un punto fisso per $G$.
\end{proof}
\begin{corollary}\label{CorCaratterizzazioneConnessiRisolubili}
Sia $G$ un gruppo conneso, sono equivalenti
\begin{enumerate}
	\item $G$ \`e risolubile
	\item le uniche rappresentazioni irriducibili di $G$ hanno dimesione 1
	\item $G\subseteq B_n$ matrici triangolari superiori di qualche taglia $n$.
\end{enumerate}
\end{corollary}


\begin{example}
Consideriamo $\C^\ast\acts \Pj^n$ come segue:
\[\la\cdot [x_0:\cdots:x_n]=[x_0:\la x_1:\cdots:\la^n x_n]\]
Questa azione ha come punti fissi quelli della forma $[0:\cdots:0:1:0:\cdots:0]$.
\end{example}

\subsection{Sottogruppi parabolici e di Borel}
\begin{definition}[Sottogruppi parabolici e di Borel]
Sia $G$ un gruppo algebrico e $P\subseteq G$ sottogruppo chiuso.
\begin{enumerate}
	\item $P$ si dice \textbf{parabolico} se $G/P$ \`e completo
	\item $B\subseteq G$ si dice \textbf{sottogruppo di Borel} se \`e un sottogruppo risolvibile connesso massimale.
\end{enumerate}
\end{definition}

\begin{remark}
Se $P$ \`e un sottogruppo parabolico, $G/P\subseteq \Pj(V)$ per qualche $V$.
\end{remark}
\begin{proof}
Ricorda (\ref{LmSottogruppoEStabilizzatoreDiUnaRettaInQualcheRappresentazione}) che esistono una rappresentazione $V$ di $G$ e una retta $L\subseteq V$ tali che $\stab_G L=P$. Per questa rappresentazione $G/P=G\cdot L\subseteq \Pj(V)$.

Notiamo che l'immagine di $G/P$ in $\Pj(V)$ \`e chiusa, infatti se $\Gamma$ \`e il grafico di $G/P\inj \Pj(V)$ allora esso \`e contenuto in $G/P \times \Pj(V)\to \Pj(V)$ e per definizione di sottogruppo parabolico $G/P$ \`e completa.

Questo mostra che $G/P$ \`e una sottovariet\`a di $\Pj(V)$.
\end{proof}

\begin{example}
Sia $G=\GL(n)$, $B\subseteq G$ le matrici triangolari superiori, diagonale inclusa.

\[G/B=\cpa{0\subseteq F_1\subseteq \cdots\subseteq F_n=\K^n\mid \dim F_1=1}\]
\end{example}

\begin{lemma}\label{LmConnessoRisolubileSSENoSottogruppiParabolici}
$G$ \`e un gruppo connesso e risolubile se e solo se $G$ non ha sottogruppi parabolici propri.
\end{lemma}
\begin{proof}
Diamo le due implicazioni
\setlength{\leftmargini}{0cm}
\begin{itemize}
\item[$\boxed{\implies}$] Sia $P\subseteq G$ parabolico. Per il teorema del punto fisso di Borel (\ref{ThPuntoFissoBorel}) si ha che $G/P$ ha un punto fisso per $G$, ma per omogeneit\`a questo significa che $G/P$ consiste di un solo punto, cio\`e $P=G$.
\item[$\boxed{\impliedby}$] Sia $P=G^0$ e mostriamo che $G/G^0$ \`e completo, questo basta perch\'e in tal caso $G^0$ \`e parabolico e quindi per ipotesi $G^0=G$.

Per il corollario (\ref{CorCaratterizzazioneConnessiRisolubili}) basta dimostrare che le rappresentazioni irriducibili hanno dimensione 1: se $V$ irriducibile, $G$ agisce su $\Pj(V)$ e ha un'orbita chiusa $G\cdot \ell\subseteq\Pj(V)$ per (\ref{PrEsisteOrbitaChiusa}). Sia $P=\stab_G(\ell)$, questo gruppo deve essere parabolico perch\'e il quoziente \`e $G\ell$ orbita chiusa ma allora $P=G$ e per irriducibilit\`a questo mostra $V=\ell$.
\end{itemize}
\setlength{\leftmargini}{0.5cm}
\end{proof}


