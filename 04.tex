\chapter{Semisemplice, Unipotente, Nilpotente, Completamente riducibile}

Avendo ricondotto lo studio dei gruppi algebrici affini allo studio di particolari gruppi di matrici, in questo capitolo cerchiamo di trasportare strumenti di algebra lineare al contesto dei gruppi algebrici.


\begin{remark}\label{ReRappresentazioneRegolareHaSottorappresentazioneSemplice}
Se $V$ \`e una rappresentazione regolare non nulla allora $V$ contiene una sottorappresentazione semplice, infatti basta prendere $W\subseteq V$ di dimensione finita non banale e poi una sottorappresentazione di $W$ di dimensione minima.
\end{remark}

\begin{lemma}[]\label{LmSottorappresentazioneCompletamenteRiducibileECompletamenteRiducibile}
Se $V$ \`e una rappresentazione completamente riducibile e $W\subseteq V$ \`e una sottorappresentazione allora anche $W$ e $V/W$ sono completamente riducibili.
\end{lemma}
\begin{proof}
Per completa riducibilit\`a $V=W\oplus U$ con $U\cong V/W$, quindi basta mostrarlo per $W$. Sia $X\subseteq W$ una sottorappresentazione. Poich\'e, sempre per completa riducibilit\`a di $V$, $V=X\oplus Y$ si ha $W=X\oplus Y\cap W$.
\end{proof}

\begin{proposition}\label{PrSemisempliceEquivaleCompletamenteRiducibile}
Una rappresentazione regolare di $G$ \`e completamente riducibile se e solo se \`e semisemplice.
\end{proposition}
\begin{proof}
Sappiamo gi\`a che semisemplice implica completamente riducibile (\ref{PrSommaSempliciESommaDirettaSemplici}), basta dunque mostrare il contrario. Sia
\[\Fc=\cpa{\bigoplus_{i\in I}S_i\subseteq V}\]
con $S_i$ tutti semplici e consideriamo l'ordine su $\Fc$ dato da
\[\bigoplus_{i\in I}S_i\preceq \bigoplus_{j\in J}T_j\coimplies I\subseteq J\text{ e }S_i=T_i\text{ per $i\in I$}.\]
Ogni catena ammette maggiorante dato sommando sull'unione degli indici. Sia allora $W=\bigoplus S_i\subseteq V$ massimale e scriviamo $V=W\oplus U$. Se $U\neq (0)$ allora (\ref{ReRappresentazioneRegolareHaSottorappresentazioneSemplice}) esiste una sottorappresentazione semplice $S\subseteq U$ e quindi $W'=W\oplus S$ sarebbe maggiore di $W$, assurdo.
\end{proof}

Vorremmo capire per quali gruppi $G$ le rappresentazioni regolari sono semisemplici.

\section{Elementi semisemplici, unipotenti e nilpotenti}
Prima di studiare le rappresentazioni, cerchiamo di capire come un elemento di $G$ pu\`o agire su una rappresentazione.

\begin{definition}[Elementi unipotenti, nilpotenti e semisemplici]
Sia $V$ uno spazio vettoriale di dimensione finita e sia $\vp\in\End(V)$. Affermiamo che $\vp$ \`e
\begin{itemize}
    \item \textbf{semisemplice} se \`e diagonalizzabile
    \item \textbf{nilpotente} se $\vp^n=0$ per qualche $n\in\N$
    \item \textbf{unipotente} se $(\vp-id_V)^n=0$ per qualche $n\in\N$
\end{itemize}
\end{definition}

\begin{remark}
Se $\vp$ \`e invertibile allora $\Z$ agisce su $V$ come
\[n\cdot v=\vp^nv,\]
quindi la definizione di semisemplice si sposa bene con quella gi\`a data in quanto $\vp$ induce una decomposizione in autospazi $\vp$-invarianti.
\end{remark}

Diamo delle definizioni valide anche per rappresentazioni di dimensione infinita

\begin{definition}[Endomorfismo localmente finito]
Siano $V$ uno spazio vettoriale e $\vp\in \End(V)$. $\vp$ \`e \textbf{localmente finito} se per ogni $v\in V$ esiste $W$ di dimensione finita $\vp$-stabile con $v\in W$. In tal caso diciamo che
\begin{itemize}
    \item $\vp$ \`e \textbf{semisemplice} se $\vp\res W$ \`e semisemplice per ogni $W\subseteq V$ $\vp$-stabile di dimensione finita.
    \item $\vp$ \`e \textbf{nilpotente} se $\vp\res W$ \`e nilpotente per ogni $W\subseteq V$ $\vp$-stabile di dimensione finita.
    \item $\vp$ \`e \textbf{unipotente} se $\vp\res W$ \`e unipotente per ogni $W\subseteq V$ $\vp$-stabile di dimensione finita.
\end{itemize}
\end{definition}

\begin{definition}[Semisemiplice, unipotente e nilpotente per gruppi algebrici]
Se $G$ \`e un gruppo algebrico e $g\in G$ allora $g$ \`e
\begin{itemize}
    \item \textbf{semisemplice} se l'azione di $g$ su ogni rappresentazione regolare \`e semisemplice, cio\`e l'azione di $g$ su ogni rappresentazione di dimensione finita \`e semisemplice
    \item \textbf{unipotente} se l'azione di $g$ su ogni rappresentazione regolare \`e unipotente, cio\`e l'azione di $g$ su ogni rappresentazione di dimensione finita \`e unipotente
    \item \textbf{nilpotente} se l'azione di $g$ su ogni rappresentazione regolare \`e nilpotente, cio\`e l'azione di $g$ su ogni rappresentazione di dimensione finita \`e nilpotente.
\end{itemize}
\end{definition}


\begin{lemma}\label{LmPassaggioSemisempliceUnipotenteNilpotenteAdOperazioniVettoriali}
Siano $g\in\End(V)$ e $h\in\End(W)$ con $V$ e $W$ localmente finite. Allora
\begin{itemize}
    \item Se $g$ e $h$ semisemplici allora
    \begin{align*}
    &g\oplus h:V\oplus W\to V\oplus W\text{ \`e semisemplice}\\
    &g\otimes h:V\otimes W\to V\otimes W\text{ \`e semisemplice}
    \end{align*}
    \item Se $g$ \`e semisemplice e $U\subseteq V$ \`e $g$-stabile allora 
    \[g\res U:U\to U\quad\text{ e }\quad \ol g:V/U\to V/U\] 
    sono semisemplici.
    \item Se $g$ \`e semisemplice allora l'azione di $g$ su $SV$ \`e semisemplice
    \item Se $\dim V$ \`e finita e $g$ \`e semisemplice allora $V^\ast$ \`e semisemplice.
\end{itemize}
Valgono anche gli analoghi per unipotente e nilpotente.
\end{lemma}
\begin{proof}
Tante verifiche noiose, riportiamo solo quelle per $g\otimes h$ per dare una idea:

Se $V$ e $W$ hanno dimensione finita allora esistono basi di autovettori per $g$ e $h$
\[gv_i=\la_iv_i,\quad hw_i=\mu_iv_i.\]
Dunque
\[(g\otimes h)(v_i\otimes w_j)=\la_i\mu_j v_i\otimes w_j,\]
e quindi $v_i\otimes w_j$ \`e ancora base di autovettori.
Consideriamo ora $V$ e $W$ generali e sia $U\subseteq V\otimes W$ di dimensione finita. Scegliendo una base $\cpa{u_j}$ di $U$ si ha che $u_j=\sum_{i} a_{ij}v_{ij}\otimes w_{ij}$. Siano $\wt V=\Span(\orb_{\ps{g}}(v_{ij}))$ e $\wt W=\Span(\orb_{\ps{h}}(w_{ij}))$. Per costruzione $\wt V$ e $\wt W$ hanno dimensione finita, $U\subseteq \wt V\otimes \wt W$ e questo prodotto tensore \`e $g\otimes h$-stabile. Usando il caso di dimensione finita troviamo una base di autovettori per $\wt V\otimes \wt W$ e dato che $U$ \`e invariante troviamo una base di autovettori per $U$ scartando qualche elemento da quella di $\wt V\otimes \wt W$. 
\end{proof}


\begin{lemma}[Rappresentazioni finite si immergono in ${\K[G]^n}$]\label{LmIniezioneRappresentazioniFiniteInPotenzaAnelloCoordinateDiG}
Sia $V$ una rappresentazione di dimensione finita di $G$, allora abbiamo una iniezione
\[f:V\inj \K[G]^n\]
$G$-equivariante.
\end{lemma}
\begin{proof}
Sia $v_1,\cdots, v_n$ una base di $V$ e sia $\vp_1,\cdots, \vp_n$ la base duale. Definiamo
\[\psi:\funcDef{V}{\K[G]^n}{v}{(\vp_1\otimes v,\cdots, \vp_n\otimes v)}\]
dove se $\vp\in V^\ast$ e $v\in V$ poniamo
\[(\vp\otimes v)(g)=\vp(g\ii v).\]
\setlength{\leftmargini}{0cm}
\begin{itemize}
\item[$\boxed{G\text{-equivariante}}$] Vogliamo $\psi(gv)=g\psi(v)$, cio\`e $g(\vp\otimes v)=\vp\otimes gv$. Allora calcoliamo
\[(g(\vp\otimes v))(h)=(\vp\otimes v)(g\ii h)=\vp(h\ii g v)=(\vp\otimes gv)(h)\]
\item[$\boxed{\text{iniettiva}}$] Supponiamo $\psi(v)=0$, allora per ogni $i$
\[0=(\vp_i\otimes v)(e)=\vp_i(v)\]
ma $\vp_i$ era una base del duale, quindi $v=0$.
\end{itemize}
\setlength{\leftmargini}{0.5cm}
\end{proof}

\begin{remark}
Se $gv=(\al_{i,j}(g))_{1\leq i,j\leq n} v$ allora $\vp_i\otimes v_j=\al_{i,j}$.
\end{remark}


\begin{remark}[Non parte del corso]
Stiamo dicendo che possiamo trovare un fibrato vettoriale banale sopra $G$ che ammette un frame globale dato da sezioni che si comportano come una base di $V$. Questo sar\`a un $G$-fibrato e questo ci permette di ricostruire l'azione di $G$ su $V$ guardando come $G$ agisce su combinazioni lineari di questo frame.
\end{remark}


\begin{corollary}\label{CorControlloSemisempliceSuAnelloCoordinate}
Se $g\in G$ allora
\begin{itemize}
    \item $g$ \`e semisemplice se e solo se l'azione di $g$ su $\K[G]$ \`e semisemplice
    \item $g$ \`e unipotente se e solo se l'azione di $g$ su $\K[G]$ \`e unipotente
\end{itemize}
\end{corollary}
\begin{proof}
Facciamo il caso semisemplice
\setlength{\leftmargini}{0cm}
\begin{itemize}
\item[$\boxed{\implies}$] Ovvio
\item[$\boxed{\impliedby}$] Dobbiamo verificare che l'azione di $g$ su ogni rappresentazione di dimensione finita $V$ \`e semisemplice. Per il lemma (\ref{LmIniezioneRappresentazioniFiniteInPotenzaAnelloCoordinateDiG}) abbiamo che $V\subseteq \K[G]^n$ e questo \`e semisemplice quindi anche $g$ lo \`e per il secondo punto del lemma (\ref{LmPassaggioSemisempliceUnipotenteNilpotenteAdOperazioniVettoriali}) 
\end{itemize}
\setlength{\leftmargini}{0.5cm}
\end{proof}



\begin{lemma}[Criterio per semisemplice/unipotente in gruppi lineari]\label{LmCriterioSemisempliceUnipotentePerGruppiLineari} Sia $V$ uno spazio vettoriale di dimensione finita e sia $G\subseteq GL(V)$ un sottogruppo chiuso. Allora
\begin{itemize}
    \item $g\in G$ \`e semisemplice se e solo se l'azione di $g$ su $V$ \`e semisemplice.
    \item $g\in G$ \`e unipotente se e solo se l'azione di $g$ su $V$ \`e unipotente.
\end{itemize}
\end{lemma}
\begin{proof}
Diamo le implicazioni per il caso semisemplice
\setlength{\leftmargini}{0cm}
\begin{itemize}
\item[$\boxed{\implies}$] Ovvio
\item[$\boxed{\impliedby}$] Verifichiamo che l'azione su $\K[G]$ \`e semisemplice. Osserviamo che 
\[\K[\GL(V)]\onto \K[G]\]
\`e surgettivo e $G$-equivariante quindi basta far vedere che $g$ agisce in modo semisemplice su $\K[\GL(V)]$.
\[\K[\GL(V)]=\K[\End(V)]\spa{{{\det}^{-1}}}\]
Verifichiamo che $g$ agisce in modo semisemplice su $\K[\End(V)]=S(\End(V)^\ast)$. Per il lemma (\ref{LmPassaggioSemisempliceUnipotenteNilpotenteAdOperazioniVettoriali}) basta verificare che $g$ agisce in modo semisemplice su $\End(V)^\ast$ o equivalentemente su $\End(V)$ per lo stesso lemma. Ricordiamo che $g$ agisce tramite la moltiplicazione a sinistra.

Se $\dim V=n$ allora $\End(V)$ con l'azione di moltiplicazione a sinistra di $g$ \`e uguale a considerare l'azione di $g$ su $V^{\oplus n}$ dove la corrispondenza \`e data dal fatto che l'azione per moltiplicazione a sinistra agisce sulle colonne della matrice a destra per restituire le colonne della matrice risultato.

Poich\'e $g$ agiva in modo semisemplice su $V$, agisce in modo semisemplice anche sulla somma che abbiamo considerato, quindi mettendo tutto insieme abbiamo mostrato che $g$ agisce in modo semisemplice su $\K[\End(V)]$.

Sia ora $W\subseteq \K[\End(V)][{\det}\ii]$ un sottospazio di dimensione finita, in particolare
\[W\subseteq \frac1{{\det}^N}\K[\End(V)]\quad \text{per qualche }N\]
Consideriamo allora l'azione di $g$ su $\K[\End(V)]\otimes \K$ dove sulla copia di $\K$ abbiamo $g\la=(\det g)^{-N}\la$. Abbiamo una mappa $G$-equivariante surgettiva
\[\funcDef{\K[\End(V)]\otimes \K}{\frac1{{\det}^N}\K[\End(V)]}{f\otimes \la}{\frac\la{(\det)^N}f}\]
quindi, poich\'e $g$ agisce in modo semisemplice su $\K[\End(V)]$ e su $\K$, si ha che agisce in modo semisemplice su $\frac1{{\det}^N}\K[\End(V)]$ e quindi su $W$.


Mettendo tutto insieme, abbiamo mostrato che $g$ agisce in modo semisemplice su $\K[G]$ e questo conclude per il corollario (\ref{CorControlloSemisempliceSuAnelloCoordinate}).
\end{itemize}
\setlength{\leftmargini}{0.5cm}
\end{proof}
















\section{Decomposizione di Jordan}
\begin{proposition}\label{PrDecomposizioneSemisempliceNilpotente}
    Sia $V$ un $\K$-spazio vettoriale di dimensione finita e sia $T$ un endomorfismo di $V$. Allora \begin{enumerate}
        \item Esistono e sono unici $S$ semisemplice e $N$ nilpotente in $\End(V)$ tali che $T=S+N$ e $SN=NS$.
    \end{enumerate}
    Gli endomorfismi $S$ e $N$ si dicono \textbf{parte semisemplice} e \textbf{parte nilpotente} di $T$ e li denoteremo $T_s$ e $T_n$ rispettivamente. 
    \begin{enumerate}
        \item[2.] Esistono $f,g$ in $\K[x]$, con $f(0)=g(0)=0$, tali che $S=f(T)$ e $N=g(T)$.
        \item[3.] Se $W$ è un sottospazio $T$-stabile di $V$, allora $W$ è $S$-stabile e $N$-stabile. Inoltre \[\left(\left.T\right|_{W}\right)_s=\left.T_s\right|_{W} \quad \text{ e } \quad \left(\left.T\right|_{W}\right)_n=\left.T_n\right|_{W}.\]
        \item[4.] Se $V'$ è un $\K$-spazio vettoriale di dimensione finita, $T'$ è un endomorfismo di $V'$ e $L\colon V \to V'$ è un'applicazione lineare, allora \[L\circ T_s = T'_s\circ L \quad \text{ e } \quad L\circ T_n = T'_n\circ L.\]
    \end{enumerate}
\end{proposition}

\begin{proof}
Dimostriamo i vari punti. 
\setlength{\leftmargini}{0cm}
\begin{enumerate}
    \item Scriviamo $T$ in forma di Jordan e poniamo $S$ la parte diagonale di $T$ (che è quindi semisemplice). Posto $N=T-S$, si ha che $N$ è nilpotente e $SN=NS$.
    
    Mostriamo ora l'unicità: se $S'$ e $N'$ sono tali che $T=S+N=S'+N'$ e $S'N'=N'S'$, allora $S,S',N,N'$ commutano con $T$, quindi $S',N'$ commutano con $S,N$. Osservando che $S-S'=N'-N$, dove il primo membro è diagonale e il secondo è nilpotente, troviamo $S=S'$ e $N=N'$. 
    \item Consideriamo il polinomio caratteristico $p_T$ di $T$ e scriviamolo nella forma \[p_T(t)=\prod_{i=1}^r(t-\lambda_i)^{n_i},\] dove i $\lambda_i$ sono distinti. Cerchiamo un polinomio $f(t)$ in $\K[t]$ tale che 
    \[\begin{cases}
    f(t)\equiv 0 &\pmod{(t)}\\
    f(t)\equiv \la_i &\pmod{(t-\lambda_i)^{n_i}} \quad \forall i\in\cpa{1,\cdots, r}
    \end{cases}\]
    Tale polinomio esiste per il teorema cinese del resto\footnote{se $\la_i=0$ per qualche $i$ la condizione $f(t)\equiv 0 \pmod{(t)}$ viene implicata da $f(t)\equiv 0 \pmod{(t)^{n_i}}$ e quindi le congruenze continuano ad avere moduli coprimi.}. Inoltre soddisfa $f(T)=S$. Infatti, sul singolo blocco di Jordan $J_i$ relativo all'autovalore $\lambda_i$ (di taglia $m_i\le n_i$), si ha che $f(T)=\lambda_i I$. Poiché $N=T-S$, posto $g(t)=t-f(t)$, si ha $N=g(T)$. 
    \item Poiché $S=f(T)$ e $N=g(T)$, se $W$ è $T$-stabile allora \`e chiaramente anche $T_s$- e $T_n$-stabile. Siano ora $t=\left.T\right|_{W}$, $s=\left.T_s\right|_{W}$ e $n=\left.T_n\right|_{W}$. Allora $t=s+n$, $sn=ns$ e, per il lemma (\ref{LmPassaggioSemisempliceUnipotenteNilpotenteAdOperazioniVettoriali}), $s$ è semisemplice e $n$ è nilpotente. Quindi la tesi discende dall'unicità della decomposizione.
    \item Consideriamo il diagramma:
    % https://q.uiver.app/#q=WzAsNixbMCwwLCJWIl0sWzIsMCwiVlxcb3BsdXMgViciXSxbNCwwLCJWJyJdLFswLDEsIlYiXSxbMiwxLCJWXFxvcGx1cyBWJyJdLFs0LDEsIlYnIl0sWzIsNSwiVCciXSxbMSw0LCJUJyc9VFxcb3BsdXMgVCciXSxbMCwzLCJUIl0sWzAsMSwiTF8xIl0sWzEsMiwiTF8yIl0sWzMsNF0sWzQsNV1d
\[\begin{tikzcd}
	V && {V\oplus V'} && {V'} \\
	V && {V\oplus V'} && {V'}
	\arrow["{L_1}", from=1-1, to=1-3]
	\arrow["T", from=1-1, to=2-1]
	\arrow["{L_2}", from=1-3, to=1-5]
	\arrow["{T''=T\oplus T'}", from=1-3, to=2-3]
	\arrow["{T'}", from=1-5, to=2-5]
	\arrow[from=2-1, to=2-3]
	\arrow[from=2-3, to=2-5]
\end{tikzcd}\]    dove $L_1 \colon v \mapsto (v,L(v))$ e $L_2\colon (v,w)\mapsto w$. Per definizione abbiamo che $L_1\circ T_s=T''_s\circ L_1$ e $L_2\circ T''_s=T'_s\circ L_2$. Deduciamo quindi che \[L\circ T_s =L_2\circ L_1 \circ T_s= L_2 \circ T''_s\circ L_1= T'_s\circ L_2 \circ L_1= T'_s\circ L.\]
\end{enumerate}
\setlength{\leftmargini}{0.5cm}
    Per il caso nilpotente la dimostrazione è analoga.
\end{proof}

Vale una decomposizione analoga nel caso moltiplicativo, sostituendo elementi nilpotenti con elementi unipotenti.

\begin{proposition}\label{PrDecomposizioneSemisempliceUnipotente}
    Sia $V$ un $\K$-spazio vettoriale di dimensione finita e sia $T$ in $\GL(V)$. Allora \begin{enumerate}
        \item Esistono e sono uniche $S$ semisemplice e $U$ unipotente tali che $T=SU=US$, date da $S=T_s$ e $U=T_u$
        \item Esistono $f,g$ in $\K[x]$ tali che $S=f(T)$ e $U=g(T)$.
        \item Se $W$ è un sottospazio $T$-stabile di $V$, allora $W$ è $S$-stabile e $U$-stabile. Inoltre \[\left(\left.T\right|_{W}\right)_s=\left.T_s\right|_{W} \quad \text{ e } \quad \left(\left.T\right|_{W}\right)_u=\left.T_u\right|_{W}.\]
        \item Se $V'$ è un $\K$-spazio vettoriale di dimensione finita, $T'$ è un endomorfismo di $V'$ e $L\colon V \to V'$ è un'applicazione lineare, allora \[L\circ T_s = T'_s\circ L \quad \text{ e } \quad L\circ T_u = T'_u\circ L.\]
    \end{enumerate}
\end{proposition}

\begin{proof}
    Per analogia con la proposizione precedente, ci limitiamo a dimostrare il primo punto. Partendo dalla decomposizione additiva, si ha \[T=S+N=S(I+S^{-1}N).\] Poiché $S^{-1}N$ è nilpotente, l'elemento $U=I+S^{-1}N$ è unipotente. Mostriamo l'unicità: se $S'$ e $U'$ sono tali che $T=S'U'$, posto $U'=I+M$ con $M$ nilpotente, si ha $T=S'+S'M$. Quindi l'unicità discende da quella del caso additivo.
\end{proof}
Vogliamo ora estendere quanto fatto al caso localmente finito. Ricordiamo che un'applicazione lineare $T\colon V\to V$ è \emph{localmente finita} se per ogni $v$ in $V$ esiste un sottospazio $W$ di $V$ di dimensione finita contenente $v$ e $T$-stabile.
\begin{theorem}
    Sia $V$ un $\K$-spazio vettoriale e sia $T\colon V\to V$ lineare, invertibile e localmente finita. Allora esistono e sono unici $S$ semisemplice e $U$ unipotente tali che $T=SU=US$.
\end{theorem}
\begin{proof}
Mostriamo dapprima l'esistenza. Per ogni $v$ in $V$, sia $W$ un sottospazio di $V$ di dimensione finita contenente $v$ e $T$-stabile. Consideriamo la restrizione $\left.T\right|_W$ in $\GL(W)$. Consideriamo allora gli elementi $S_W=(\left.T\right|_W)_s$ e $U_W=(\left.T\right|_W)_u$ dati dalla Proposizione (\ref{PrDecomposizioneSemisempliceUnipotente}) e definiamo $S(v)=S_W(v)$. Osserviamo che $S(v)$ non dipende dalla scelta di $W$. Infatti, se $W$ è contenuto in un $W'$, allora $\left.S_{W'}\right|_{W}=S_W$. Procedendo in modo analogo per $U$, otteniamo l'esistenza. 

Osserviamo che $S$ e $U$ costruite soddisfano le proprietà 3 e 4 dell'enunciato precedente. 
\begin{mdframed}[topline=false,rightline=false,bottomline=false]  
    Verifichiamo ad esempio la 4. Mostriamo che, con la notazione della Proposizione (\ref{PrDecomposizioneSemisempliceUnipotente}), si ha $LS=S'L$. Per ogni $v$ in $V$, consideriamo un sottospazio $W$ di dimensione finita, contenente $v$ e $T$-stabile. Sia $W'=L(W)$. Allora \[S'(W')=S'L(W)=LS(W)\subseteq L(W)=W'.\]
    È quindi sufficiente mostrare che \[\left.L\right|_{W}\circ \left.S\right|_{W} (v)=\left.S'\right|_{W'}\circ \left.L\right|_{W} (v),\] ma ciò segue dal caso di dimensione finita.
\end{mdframed}
Mostriamo ora l'unicità. Assumiamo $T=su=us$ e mostriamo che $S=s$ e $U=u$. Osserviamo che $S,U$ commutano con $s,u$. Infatti, per il punto 4, sappiamo che $LS=SL$ e $LU=UL$. Allora dal diagramma % https://q.uiver.app/#q=WzAsNCxbMCwwLCJWIl0sWzIsMCwiViJdLFswLDEsIlYiXSxbMiwxLCJWIl0sWzEsMywiVCJdLFswLDIsIlQiXSxbMCwxLCJMPXMiXSxbMiwzLCJMPXMiLDJdXQ==
\[\begin{tikzcd}
	V && V \\
	V && V
	\arrow["{L=s}", from=1-1, to=1-3]
	\arrow["T", from=1-1, to=2-1]
	\arrow["T", from=1-3, to=2-3]
	\arrow["{L=s}"', from=2-1, to=2-3]
\end{tikzcd}\]
ricaviamo $Ss=sS$ e $Us=sU$. Analogamente troviamo $Uu=uU$ e $Su=uS$. Scriviamo ora $V=\bigoplus_\lambda V_\lambda(S)$. Poiché $V_\lambda(S)$ è stabile per $U,s,u$, lo spazio \[V_{\lambda,\mu}(S,s)=\{v\in V\colon Sv=\lambda v, \ sv=\mu v\}\] è stabile per $U$ e $u$. Supponiamo per assurdo che esistano $\lambda, \mu$ distinti tali che $V_{\lambda,\mu}\ne 0$. Allora, poiché $U$ e $u$ sono unipotenti, si ha $(V_{\lambda,\mu}(S,s))^U\ne 0$ e, poiché $U$ e $u$ commutano, si ha anche $((V_{\lambda,\mu}(S,s))^U)^u\ne 0$. Scegliendo $v$ non nullo in $((V_{\lambda,\mu}(S,s))^U)^u$, si ottiene $T(v)=SU(v)=S(v)=\lambda v$, ma anche $T(v)=su(v)=s(v)=\mu v$, contro l'ipotesi che $\lambda$ e $\mu$ sono distinti.
\end{proof}

\begin{theorem}[Decomposizione di Jordan] Sia $G$ un gruppo algebrico e sia $g$ un elemento di $G$. Allora esistono e sono unici $s,u$ in $G$ tali che $s$ è semisemplice, $u$ è unipotente e $g=su=us$.
\end{theorem}
\begin{proof}
    Assumiamo che $G$ sia un sottogruppo chiuso di $\GL(W)$. $G=V(I)$ con $I\subseteq \K[\GL(W)]$. Mostriamo l'esistenza di $s$ e $u$. Poiché $g$ è in $\GL(W)$, possiamo scrivere (\ref{PrDecomposizioneSemisempliceUnipotente}) $g=su=us$ con $u$ unipotente e $s$ semisemplice in $\GL(W)$ (cio\`e $s$ agisce in modo semisemplice su $\K[\GL(W)]$ e $u$ agisce in modo unipotente su $\K[\GL(W)]$). In particolare $g=su$ è la decomposizione di Jordan per l'azione di $g$ su $\K[\GL(W)]$.
    Mostriamo che effettivamente $u$ e $s$ sono in $G$ verificando ogni elemento $f$ di $I$ si annulla su essi. Osserviamo che $I$ è stabile per l'azione di $g$ (e in generale per l'azione di $G$), infatti, per ogni $h$ in $G$ si ha $(gf)(h)=f(g^{-1}h)=0$. Poiché $s$ è la parte semisemplice di $g$, $s(I)$ è contenuto in $I$ e se $f$ è un elemento di $I$, si ha $sf$ è $0$ su $G$. D'altro canto, valutando sull'elemento neutro $e$ di $G$, troviamo \[0=(sf)(e)=f(s^{-1}),\] dunque $s^{-1}$ (e quindi $s$) appartiene a $G$. L'unicità segue direttamente dal fatto che la decomposizione è unica in $\GL(W)$.
\end{proof}
\begin{notation}
    Denotiamo $g_s=s$ e $g_u=u$.
\end{notation}
\begin{exercise}
    Se $\varphi\colon G \to G'$ è un morfismo di gruppi algebrici, allora $\varphi(g_s)=\varphi(g)_s$ e $\varphi(g_u)=\varphi(g)_u$.
\end{exercise}

\begin{exercise}
    Sia $\K$ un campo perfetto (assumiamo per l'esercizio $\K=\R$). Consideriamo l'inclusione \[\GL(n,\R)\subset \GL(n,\C).\]
    Sia $G$ un sottogruppo di $\GL(n,\C)$ definito da un'equazione a coefficienti in $\R$. Sia $G_\R=G\cap \GL(n,\R)$. Consideriamo un elemento $g$ in $G_\R$. Mostrare che $g_s$ e $g_u$ sono in $G_\R$.
\end{exercise}


\section{Gruppi unipotenti}
\begin{definition}
    Un gruppo $G$ si dice \textbf{unipotente} se ogni suo elemento è unipotente.
\end{definition}

\begin{lemma}\label{LmUnicaRappresentazioneIrriducibileBanaleImplicaUpperTriangularCon1SuDiagonale}
Se $G$ \`e tale che l'unica rappresentazione irriducibile di $G$ \`e banale allora $G$ si immerge nel gruppo delle matrici triangolari superiori aventi $1$ sulla diagonale, cio\`e
\[G\subseteq U(n)=\cpa{\mat{1 & \ast & \ast\\ &\ddots &\ast\\ &&1}}\]
\end{lemma}
\begin{proof}
    Sia $V$ un $\K$-spazio vettoriale di dimensione finita per cui $G$ è un sottogruppo di $\GL(V)$. Sia $W$ una sottorappresentazione di $V$ irriducibile non nulla. Allora $W=\K v_1$ e $g v_1=v_1$ per ogni $g$ in $G$. Se $V_1=0$ abbiamo la tesi, altrimenti consideriamo il quoziente $V_1=V/\langle v_1\rangle$. Anche $V_1$ \`e una rappresentazione di $G$, quindi scegliendo una sottorappresentazione irriducibile non nulla troviamo $v_2$ in $V$ tale che per ogni $g$ in $G$ si ha $gv_2\equiv v_2 \pmod{\K v_1}$. Procedendo in questo modo, possiamo scegliere una base $v_1,\ldots,v_n$ di $V$, rispetto alla quale risulta chiaramente $G\subseteq U(n)$.
\end{proof}

\begin{theorem}\label{ThSottoalgebraAssociativaDiEndomorfismiDiSempliceETuttiGliEndomorfismi}
Sia $\K$ algebricamente chiuso, $V$ spazio vettoriale di dimensione finita, $A\subseteq \End_\K(V)$ che sia una $\K$-algebra associativa. Se $V$ \`e un $A$-modulo semplice allora $A=\End_\K(V)$.
\end{theorem}
\begin{proof}
Sia $v_1,\cdots, v_n$ una base di $V$. Se $v=(v_1,\cdots, v_n)\in V^n$ allora vogliamo mostrare che $\cpa{a(v)\mid a\in A}=V^n\equiv \End_\K(V)$.

Poich\'e $V$ \`e semplice, $V^n$ \`e somma di rappresentazioni semplici, quindi per la proposizione (\ref{PrSommaSempliciESommaDirettaSemplici}) si ha che $V^n$ \`e semisemplice e quindi completamente riducibile. Possiamo allora scrivere $V^n=Av\oplus P$ per $P$ un $A$-sottomodulo. Sia $\pi_P:V^n\to V^n$ la proiezione su $P$ e notiamo che essa ammette una decomposizione a blocchi
\[\pi_P=(\al_{ij})_{i,j}\quad\text{ per degli endomorfismi }\al_{ij}:V\to V\]
dove il dominio di $\al_{ij}$ \`e la $j$-esima copia di $V$ in $V^n$ e il codominio \`e l'$i$-esima copia.

Quindi $\al_{ij}\in \End_A(V)$ con $V$ spazio vettoriale su $\K$ algebricamente chiuso. Per il lemma di Schur (\ref{LmSchur}) questi endomorfismi sono le costanti.

Ricordando che $V^n=Av\oplus P$, si ha necessariamente $\pi_P(v)=0$ per definizione di $\pi_P$, quindi per ogni $i$ abbiamo $\sum \al_{ij}v_j=0$. Poich\'e gli $v_j$ sono una base e gli endomorfismi $\al_{ij}$ sono costanti, per indipendenza lineare questo significa che per ogni $i$ e ogni $j$ si ha $\al_{ij}=0$. Segue dunque che $\pi_P=0$ e quindi $P=\imm \pi_P=\cpa{0}$, cio\`e $V^n=Av$.
\end{proof}

\begin{exercise}
Trova un controesempio per $\K$ non algebricamente chiuso.
\end{exercise}



\begin{theorem}\label{ThUnipotenteSeESoloSeUnicaRappresentazioneIrriducibileEBanale}
    Un gruppo $G$ è unipotente se e solo se l'unica rappresentazione irriducibile di $G$ è quella banale.
\end{theorem}
\begin{proof}
Diamo le due implicazioni
\setlength{\leftmargini}{0cm}
\begin{itemize}
\item[$\boxed{\impliedby}$] Se $G$ ha questa propriet\`a allora per il lemma (\ref{LmUnicaRappresentazioneIrriducibileBanaleImplicaUpperTriangularCon1SuDiagonale}) abbiamo
\[G\subseteq U(n)=\cpa{\mat{1 & \ast & \ast\\ &\ddots &\ast\\ &&1}}\]
e chiaramente un gruppo di questa forma \`e unipotente.
\item[$\boxed{\implies}$] Sia $V$ una rappresentazione semplice di $G$ di dimensione $n$. Allora\footnote{tiriamo in gioco la traccia perch\'e svolger\`a il ruolo di un prodotto scalare definito positivo su $\End_\K(V)$, che mostriamo essere $\Span(G)$ grazie al teorema.} $\tr(g_V)=n$ per ogni $g\in G$ (perch\'e si immerge nelle triangolari superiori), quindi
\[\forall g_V,h_V\in G,\qquad \tr((g_V-1)h_V)=\tr(g_Vh_V-h_V)=0.\]
Se $A$ \`e il $\K$-spazio vettoriale generato da $G$, $A\subseteq \End_\K(V)$ ed \`e un'algebra associativa. Poich\'e $V$ \`e semplice per $A$ (in quanto semplice per $G$ e $A=\Span_\K(G)$), si ha per il teorema (\ref{ThSottoalgebraAssociativaDiEndomorfismiDiSempliceETuttiGliEndomorfismi}) che $A=\End_\K(V)$.

Segue per linearit\`a della traccia che $\tr((g_V-1)a)=0$ per ogni $a\in \End_\K(V)=A$, da cui $(g_v-1)=0$, cio\`e $G$ agisce banalmente.
\end{itemize}
\setlength{\leftmargini}{0.5cm}
\end{proof}

\begin{corollary}
    Se $G$ è unipotente, allora è contenuto nel gruppo $U_n$ delle matrici triangolari superiori aventi $1$ sulla diagonale principale.
\end{corollary}

\begin{corollary}\label{CorFissatoDaUnipotenteNonEBanale}
    Se $V$ \`e una rappresentazione di $G$ non nulla allora $V^G\neq 0$.
\end{corollary}

\begin{corollary}
    Se $G$ \`e unipotente allora $G$ \`e nilpotente come gruppo, cio\`e se definiamo iterativamente 
    \begin{align*}
    G^{(0)}\;\;\;\,&=G \\
    G^{(k+1)}&=[G^{(k)},G] 
    \end{align*} 
    allora esiste $n$ tale che $G^{(n)}=\cpa{id_G}$.
\end{corollary}
\begin{proof}
Basta immergere $G$ in $U(n)$ e notare che sottogruppi di triangolari superiori con 1 sulla diagonale hanno questa propriet\`a.
\end{proof}

\begin{example}
Consideriamo il gruppo $(\C,+)$. Questo è un gruppo unipotente perché possiamo vederlo in $\GL(2)$ tramite la rappresentazione 
\[x\mapsto \begin{pmatrix}
1 & x \\ 0 & 1 
\end{pmatrix}\]
\end{example}

\begin{example}
Il gruppo $G=\znz p\cong \F_p$ \`e un gruppo unipotente se $\cha \K=p$, infatti si pu\`o scrivere come
\[G=\cpa{\mat{1&\al\\0&1}\sep \al\in\F_p\subseteq\K}\]
\end{example}
\begin{remark}
Se $G=U(n)$ con $\cha \K=p$ allora $g^{p^n}=id_G$.
\end{remark}

\subsection{Esponenziale e logaritmo}

\begin{notation}
Definiamo lo spazio vettoriale
\[N(n)=\cpa{\mat{0&\ast&\ast\\&\ddots&\ast\\&&0}}.\]
\end{notation}

\begin{definition}[Esponenziale]
Definiamo la mappa \textbf{esponenziale}
\[\exp:\funcDef{N(n)}{U(n)}{M}{\sum_{i=0}^n\frac1{i!}M^i}.\]
\end{definition}
\begin{remark}
La mappa esponenziale appena definita \`e algebrica. Inoltre rispetta le usuali propriet\`a:
\begin{itemize}
    \item $\exp((\la+\mu)M)=\exp(\la M)\exp(\mu M)$
    \item Se $M_1M_2=M_2M_1$ allora $\exp(M_1+M_2)=\exp(M_1)\exp(M_2)$.
\end{itemize}
\end{remark}

\begin{definition}[Logaritmo]
Definiamo la mappa \textbf{logaritmo}
\[\log:\funcDef{U(n)}{N(n)}{B}{\sum_{i=1}^n\frac{(-1)^{i+1}}{i}(M-I_n)^i}.\]
\end{definition}

\begin{remark}
$\log$ e $\exp$ sono mappe algebriche e inverse. Sono anche in realt\`a la definizione usuale, solo che per matrici in $N(n)$ e $U(n)$ queste somme finite coincidono con la definizione in serie.
\end{remark}

\begin{notation}
Se $G\subseteq U(n)$ definiamo $X=\log(G)$ e notiamo che $X$ \`e isomorfo a $G$ come variet\`a.
\end{notation}

\begin{proposition}\label{PrChiusuraDelGeneratoDaPotenzeDiElemento}
Se $g\in G\bs\cpa{id_G}$ \`e unipotente e $\cha \K=0$ allora, ponendo
\[\ol{\cpa{g^n\mid n\in\Z}}=H\subseteq G,\]
si ha $H\cong \G_a=(\K,+)$.
\end{proposition}
\begin{proof}
Poich\'e $g\neq id_G$, $\log g=x\neq 0$, inoltre $\log(g^n)=nx$. Notiamo dunque che da $nx\neq 0$ per ogni $n$ ricaviamo $g^n\neq id_G$ per ogni $n$. Se $Y=\K x$ allora
\[\log(H)=\log(\ol{\cpa{g^n}})=\ol{\cpa{nx}}\subseteq Y\]
Poich\'e $Y$ \`e una retta e $\cpa{nx}$ sono infiniti, $\ol{\cpa{nx}}= Y$ per come sono fatti i chiusi di Zariski di $\A^1$, quindi 
\[\funcDef{\K}{H=\exp(Y)}{\la}{\exp(\la x)}\]
\`e un isomorfismo di gruppi tra $\K$ e $H$.
\end{proof}

\begin{exercise}
Se $\cha \K=0$ e $G$ unipotente abeliano allora $G\cong \K^n$.
\end{exercise}

\begin{fact}
Se $\cha \K=p$ e $g^p=id$ per ogni $g\in G$ abeliano connesso allora $G$ \`e unipotente e $G\cong \K^n$.
\end{fact}

\begin{example}
Se $\cha \K=p$ poniamo $\wt \al_i=\frac1p\binom pi$ per $i\in\cpa{0,\cdots, p-1}$ e definiamo $\al_i$ come l'immagine di $\wt \al_i$ in $\K$.

Definiamo\footnote{moralmente $c(a,b)=\frac{(a+b)^p-a^p-b^p}p$, che non potremmo fare direttamente in $\K$ per l'identit\`a del binomio ingenuo}
\[c:\funcDef{\K\times \K}{\K}{(a,b)}{\sum_{i=1}^{p-1} \al_i a^ib^{p-i}}\]
e notiamo che
\[c(a,b)c(a+b,c)=c(b,c)c(a,b+c)\implies c(a,0)=c(a,b)=0,\]
quindi se $G=\K\times\K$ con prodotto
\[(a,b)\cdot(a',b')=(a+a',b+b'+c(a,a'))\]
allora $G$ \`e unipotente, abeliano e connesso ma non \`e isomorfo a $(\K^2,+)$ come gruppo a priori.
Infatti $(1,0)^p=(0,c(1,1)+c(1,2)+\cdots+c(1,p-1))\neq (0,0)$ in generale.
\end{example}

\section{Gruppi completamente riducibili}
\begin{definition}
Un gruppo \`e \textbf{completamente riducibile} se ogni sua rappresentazione regolare \`e semisemplice.
\end{definition}
\begin{remark}
Basta anche chiedere ``ogni rappresentazione regolare \emph{finita} \`e semisemplice".
\end{remark}

\begin{proposition}[Criterio per completa riduciblit\`a]
$G$ \`e completamente riducibile se e solo se $\K[G]$ \`e semisemplice.
\end{proposition}
\begin{proof}
$\K[G]$ \`e una rappresentazione regolare di $G$ quindi una implicazione \`e ovvia.
Se $V$ ha dimensione $n$ e $\K[G]$ \`e semisemplice allora per l'immersione (\ref{LmIniezioneRappresentazioniFiniteInPotenzaAnelloCoordinateDiG}) $V\inj \K[G]^m$ si ha che $V$ \`e semisemplice (\ref{LmPassaggioSemisempliceUnipotenteNilpotenteAdOperazioniVettoriali}).
\end{proof}


\begin{lemma}
    Se $G\subseteq \GL(V)$ allora $G$ \`e completamente riducibile se e solo se $V^{\otimes n}$ \`e semisemplice per ogni $n$
\end{lemma}
\begin{proof}
Diamo le implicazioni
\setlength{\leftmargini}{0cm}
\begin{itemize}
\item[$\boxed{\implies}$] Ovvio
\item[$\boxed{\impliedby}$] La dimostrazione \`e del tutto analoga a quella esposta per il lemma (\ref{LmCriterioSemisempliceUnipotentePerGruppiLineari}). Dimostriamo che $\K[G]$ \`e semisemplice. Dato il morfismo
$\K[\GL(V)]\onto \K[G]$
basta mostrare che $\K[\GL(V)]$ \`e semisemplice. Scriviamo \[\K[\GL(V)]=\K[\End(V)][{\det}\ii].\]
$\K[\End(V)]$ \`e un quoziente di somme di rappresentazioni della forma $(V^\ast)^{\otimes m}$ e quindi \`e quoziente di $(V^\ast\oplus\cdots, \oplus V^\ast)^{\otimes m}$ e dato che $V^\ast$ \`e semisemplice ho finito.
\end{itemize}
\setlength{\leftmargini}{0.5cm}
\end{proof}

\begin{corollary}
    Se $\K=\C$ e $G\subseteq \GL(n,\C)$ \`e tale che se $g\in G$ allora $g^\dagger\in G$, allora\footnote{se vedo $g$ come matrice, $g^\dagger$ \`e la matrice trasposta coniugata di $g$.} $G$ \`e completamente riducibile.
\end{corollary}
\begin{proof}
Sia $V=\C^n$ e dimostriamo che $V^{\otimes m}$ \`e semisemplice per ogni $m$, cio\`e che per ogni $U\subseteq V^{\otimes m}$ che sia $G$-stabile esiste $W$ $G$-stabile tale che $V^{\otimes m}=U\oplus W$.

Consideriamo il caso $m=1$. Definiamo la forma hermitiana\footnote{Definiamo un prodotto scalare in modo da definire il supplementare come l'ortogonale a $U$ rispetto a questa forma. La $G$-stabilit\`a deriva dall'ipotesi sul coniugio complesso.}
\[h((x_i)(y_i))=\sum_{i=1}^n \ol{x_i}y_i.\]
Se $U\subseteq V$, poniamo $W=U^\perp$ rispetto a questa forma. Chiaramente $V=U\oplus U^\perp$, quindi vogliamo mostrare che $U$ $G$-stabile implica $U^\perp$ $G$-stabile.
\[h(gw,u)=\ol w^\top \ol g^\top u=h(w,\overset{\in U}{\overbrace{\under{\in G}{\ol g^\top}u}})\pasgnlmath={w\in U^\perp}0.\]


Per il caso generale l'idea \`e la stessa ma usiamo questa forma hermitiana
\[h_m(v_1\otimes\cdots\otimes v_m,u_1\otimes \cdots\otimes u_m)=\prod_{i=1}^m h(v_i,u_i).\]
Si conclude usando la tesi per $h$.
\end{proof}

\begin{example}
Sia $G\in\cpa{\GL(n,\C), \SL(n), O(n)}$, allora $G$ \`e completamente riducibile. Per $\GL(n,\C)$ e $\SL(n,\C)$ questo \`e ovvio. Per $O(n)=\cpa{g^\top g=id}$ basta mostrare che se $g^\top g=id$ allora $\ol g \ol g^\top=id$, ma questo \`e chiaro.


Anche $SO(n)$ e $S_p(2n)$ hanno questa propriet\`a, dove
\[S_p(2n)=\cpa{g\mid gJg^\top=J},\qquad J=\mat{0 &I\\ -I &0}.\]
\end{example}



\begin{example}
Sia $\cha \K=p=2$ e consideriamo $G=\SL(2,\K)$. Esso ammette una rappresentazione semplice $V=\K^2$. Sia $x,y$ una base di $V$ e consideriamo l'azione data da
\[\mat{a&b\\c&d}x=ax+by,\qquad \mat{a&b\\c&d}y=cx+dy.\]
Consideriamo ora $S^2V=\ps{x^2,xy,y^2}$. Per questioni di caratteristica 2, $gx^2=(gx)^2$. Notiamo allora che $W=\ps{x^2,y^2}$ \`e una sottorappresentazione di $S^2V$ che non ammette un complementare.
\end{example}

\begin{exercise}
Se $\cha \K=p$ allora $\SL(n,\K)$ e $\GL(n,\K)$ non sono completamente riducibili.
\end{exercise}




\begin{proposition}\label{PrCompletamenteRiducibileNonHaSottogruppiNormaliUnipotenti}
Se $G$ \`e completamente riducibile allora $G$ non ha sotto-gruppi unipotenti normali non banali.
\end{proposition}
\begin{proof}
Sia $W$ tale che $G\inj \GL(W)$ (definizione di gruppo algebrico lineare).

Sia $V$ una rappresentazione irriducibile di $G$ e sia $U\subseteq G$ un sottogruppo normale unipotente. Poich\'e $V\neq (0)$ e $U$ unipotente, $V^U\neq 0$ per il corollario (\ref{CorFissatoDaUnipotenteNonEBanale}). Poich\'e $U$ \`e normale, $V^U$ \`e stabile per $G$ e quindi $V^U=V$, cio\`e $U$ agisce banalmente su tutte le rappresentazioni irriducibili. Questo mostra che per la mappa $G\to \GL(W)$ si ha che $U$ finisce in $\cpa{id_W}$ perch\'e $G$ \`e completamente riducibile, ma questa mappa \`e iniettiva e quindi $U=\cpa{id_G}$.
\end{proof}

\subsection{Caratteri, Tori e Gruppi abeliani connessi Semisemplici}


Uno degli esempi pi\`u importanti di gruppi completamente riducibili sono i tori algebrici:
\begin{definition}
Un gruppo $G$ \`e un \textbf{toro (algebrico)} se $G\cong (\G_m)^n$.
\end{definition}

\begin{remark}
Ricordando che $\K[\G_m]=\K[t^{\pm1}]$ notiamo che
\[\K[\G_m^n]=\K[t_1^{\pm1},\cdots,t_n^{\pm1}].\]
Questo spazio ha una base data da $t^\al=t_1^{\al_1}\cdots t_n^{\al_n}$ per $\al\in \Z^n$.
\end{remark}

\begin{proposition}[Tori algebrici sono semisemplici]\label{PrToriAlgebriciSonoSemisemplici}
Se $G$ toro algebrico allora $G$ \`e semisemplice.
\end{proposition}
\begin{proof}
Se $g=(\la_1,\cdots, \la_n)$ e $h=(\mu_1,\cdots, \mu_n)$ sono elementi di $\G_m^n=(\K^\times)^n$ si ha che
\begin{align*}
(gt^\al)(h)=&t^\al(g\ii h)=t^\al(\la_1\ii \mu_1,\cdots, \la_n\ii\mu_n)=\\
=&(\la_1\ii\mu_1)^{\al_1}\cdots (\la_n\ii\mu_n)^{\al_n}=\\
=&\la^{-\al}\mu^\al=\\
=&(\la^{-\al}t^\al)h.
\end{align*}
Segue che $gt^\al=\la^{-\al}t^\al$ e che quindi i $t^\al$ sono autovettori per ogni $g\in G$. Poich\'e
\[\K[G]=\bigoplus_{\al\in\Z^n} \K t^\al\]
e ogni $\K t^\al$ \`e semisemplice, segue (\ref{LmPassaggioSemisempliceUnipotenteNilpotenteAdOperazioniVettoriali}) che $\K[G]$ \`e semisemplice.

Quindi le rappresentazioni semplici di $G$ sono di dimensione 1 e sono date da $\K_\al=\K t^{-\al}$ con $gz=t^\al(g)z$
\end{proof}


\begin{definition}[Caratteri di un gruppo]
Dato un gruppo algebrico $G$ definiamo un \textbf{carattere} di $G$ come un omomorfismo di gruppi
\[\al:G\to \GL(1)=\K^\times.\]
L'insieme dei caratteri $X(G)$ forma un gruppo abeliano:
\[(\al\beta)(g)=\al(g)\beta(g)=\beta(g)\al(g)=(\beta\al)(g).\]
\end{definition}


\begin{theorem}\label{ThAbelianoConnessoCompletamenteRiducibileImplicaToro}
Sia $G$ un gruppo abeliano connesso completamente riducibile, allora $G$ \`e un toro algebrico.
\end{theorem}
\begin{proof}
Notiamo che ogni elemento di $G$ \`e semisemplice: se $g\in G$ allora per la decomposizione di Jordan (\ref{PrDecomposizioneSemisempliceUnipotente}) si ha $g=su$ con $u$ unipotente in $G$. Poich\'e $G$ \`e abeliano, $\ps{u}$ \`e un suo sottogruppo normale, quindi per la proposizione sopra (\ref{PrCompletamenteRiducibileNonHaSottogruppiNormaliUnipotenti}) si ha che $\ps{u}=\cpa{1_G}$, cio\`e $u=1_G$. Dunque $g=s1_G=s$, cio\`e $g$ \`e semisemplice.

Poich\'e $G$ \`e abeliano, gli elementi commutano. Dato che ogni elemento \`e semisemplice (cio\`e in ogni rappresentazione \`e diagonalizzabile), si ha che per ogni rappresentazione esiste una base di autovettori dove ogni elemento di $G$ \`e \emph{simultaneamente} diagonalizzabile. In particolare le rappresentazioni irriducibili hanno dimensione $1$.
\smallskip

Consideriamo allora una decomposizione di $\K[G]$ che rende ogni elemento di $G$ diagonalizzabile
\[\K[G]=\bigoplus \K^{n_\al}_\al\]
dove $\K_\al^{n_\al}$ sono le funzioni regolari $f$ tali che $gf=\al(g)f$. Notiamo che 
\[\al(gh)f=(gh)f=g(hf)=g(\al(h)f)=\al(h)gf=\al(h)\al(g)f\]
quindi $\al(gh)=\al(g)\al(h)$. Questo ci permette di identificare questa decomposizione con
\[\K[G]=\bigoplus_{\al\in X(G)}V_\al,\qquad\text{dove }V_\al=\cpa{h\in \K[G]\mid gh=\al(g)h}.\]

Per ogni carattere $\al\in X(G)$ definiamo $f_\al=\al\ii\in \Hom(G,\K^\times)\subseteq\Hom(G,\K)=\K[G]$. Sfruttando il fatto che $\al$ \`e un omomorfismo si ha $f_\al\in V_\al$, infatti
\begin{align*}
    (gf_\al)(x)=&f_\al(g\ii x)=\al\ii(g\ii x)=(\al(g\ii x))\ii=\\
    =&(\al(g)\ii \al(x))\ii=(\al(x))\ii\al(g)=\\
    =&\al(g) \al\ii(x)=\\
    =&\al(g)f_\al(x).
\end{align*}
Se $h$ ha carattere $\al$, cio\`e $gh=\al(g)h$, allora per ogni $g\in G$
\[\frac{h(1_G)}{f_\al(1_G)}=\frac{\al(g\ii)h(1_G)}{\al(g\ii)f_\al(1_G)}=\frac{g\ii h(1_G)}{g\ii f_\al(1_G)}=\frac{h(g1_G)}{f_\al(g1_G)}=\frac{h(g)}{f_\al(g)},\]
cio\`e $h$ \`e un multiplo di $f_\al$ (in particolare $n_\al=1$ nella scrittura sopra).
\smallskip

Mostriamo che $X(G)\cong \Z^n$ per qualche $n$ mostrando che \`e un gruppo abeliano finitamente generato libero da torsione:
\setlength{\leftmargini}{0cm}
\begin{itemize}
\item[$\boxed{\text{fin.gen.}}$] Sappiamo che $\K[G]$ \`e una $\K$-algebra finitamente generata quindi consideriamo dei generatori $f_{\al_1},\cdots, f_{\al_m}$. Come $\K$-spazio vettoriale, $\K[G]$ \`e generato da elementi della forma
\[f_{\al_1}^{n_1}\cdots f_{\al_m}^{n_m},\]
il quale ha carattere $\prod \al_i^{n_i}$. Questo mostra che i caratteri $\al_1,\cdots, \al_m$ sono dei generatori di $X(G)$.
\item[$\boxed{\text{tor.free}}$] Per assurdo supponiamo esista $\al\neq 1$ tale che $\al^N=1$, cio\`e $\al(g)^N=1$ per ogni $g\in G$. Notiamo che
\[G=\coprod_{\smat{\omega\in \K^\times\ t.c.\\ \omega^N=1}}\cpa{g\mid \al(g)=\omega},\]
ma poich\'e $G$ \`e connesso, questa unione disgiunta di chiusi deve consistere di un solo termine, mostrando che $\al$ assume solo il valore $1$ contraddicendo le ipotesi.
\end{itemize}
\setlength{\leftmargini}{0.5cm}
\medskip

Abbiamo quindi mostrato che $X(G)\cong \Z^n$. Sia $\al_1,\cdots, \al_n$ una sua base. Se scriviamo $x_i=f_{\al_i}$ troviamo
\[\K[G]=\bigoplus \ps{f_\al}_\K=\bigoplus \ps{x_1^{m_1}\cdots x_n^{m_n}}_\K\]
e in questa decomposizione il prodotto \`e esattamente quello che ci aspetteremmo.
Se scriviamo $\al=\al_1^{m_1}\cdots\al_n^{m_n}$ allora $f_\al=x_1^{m_1}\cdots x_n^{m_n}$, dunque abbiamo proprio mostrato che
\[\K[G]=\K[x_1^{\pm1},\cdots, x_n^{\pm 1}]=\K[(\K^\times)^n].\]
Essendo sia $G$ che $(\K^\times)^n$ affini questo mostra che sono isomorfi.
\end{proof}

\begin{remark}
Il gruppo dei caratteri di un toro \`e dato da un reticolo $\Z^n$.
\end{remark}

